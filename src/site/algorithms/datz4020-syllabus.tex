\documentclass[11pt]{article}
\usepackage{ucs}
\usepackage[utf8x]{inputenc}
\usepackage{changepage}
\usepackage{graphicx}
\usepackage{amsmath}
\usepackage{gensymb}
\usepackage{amssymb}
\usepackage{enumerate}
\usepackage{tabularx}
\usepackage{lipsum}

\oddsidemargin 0.0in
\evensidemargin 0.0in
\textwidth 6.27in
\headheight 1.0in
\topmargin -0.1in
\headheight 0.0in
\headsep 0.0in
%\textheight 9.69in
\textheight 9.50in

\setlength\parindent{0pt}

\newenvironment{myenv}{\begin{adjustwidth}{0.4in}{0.4in}}{\end{adjustwidth}}
\renewcommand{\abstractname}{Anotācija}
\renewcommand\refname{Atsauces}

\newenvironment{uzdevums}[1][\unskip]{%
\vspace{3mm}
\noindent
\textbf{#1:}
\noindent}
{}

\newcommand{\subf}[2]{%
  {\small\begin{tabular}[t]{@{}c@{}}
  #1\\#2
  \end{tabular}}%
}



\newcounter{alphnum}
\newenvironment{alphlist}{\begin{list}{(\Alph{alphnum})}{\usecounter{alphnum}\setlength{\leftmargin}{2.5em}} \rm}{\end{list}}


\makeatletter
\let\saved@bibitem\@bibitem
\makeatother

\usepackage{bibentry}
%\usepackage{hyperref}


\begin{document}

\thispagestyle{empty}

{\Large \bf Lietišķie algoritmi: Syllabus (2019.g. rudens)}

\begin{uzdevums}[P1]
Dots pirmskaitlis $p \geq 2$.
Eduardo and Fernando spēlē sekojošu spēli, pārmaiņus
izdarot gājienus: Katrā gājienā spēlētājs izvēlas 
indeksu $i$ no kopas 
$\{0,1,\ldots,p-1\}$, 
kuru neviens no viņiem vēl nav izvēlējies, un 
tad izvēlas elementu $a_i$ no kopas
$\{0, 1, 2, 3, 4, 5, 6, 7, 8, 9\}$. 
Spēli sāk Eduardo. Spēle beidzas tad, kad visi 
indeksi $i \in \{0,1,\ldots,p-1\}$ ir izvēlēti. 
Tad izrēķina skaitli: 
$$M = a_0 + 10 \cdot a_1 + \cdots + 
10^{p-1} \cdot a_{p-1} =
\sum_{j=0}^{p-1} a_j \cdot 10^j.$$
Eduardo mērķis ir padarīt skaitli $M$ dalāmu ar $p$, 
bet Fernando mērķis ir to nepieļaut.  
Pierādīt, ka Eduardo ir uzvaroša stratēģija -- viņš
vienmēr var sasniegt savu mērķi.
\end{uzdevums}

\begin{uzdevums}[P2]
Dots pirmskaitlis $p>3$, kuram $p \equiv 3\;(\mbox{mod}\,4)$. 
Dotam naturālam $n$
skaitlim $a_0$ virkni $a_0, a_1,\ldots$ definē kā 
$a_n = a_{n-1}^{2^n}$ visiem $n = 1, 2,\ldots$. 
Pierādīt, ka $a_0$ var izvēlēties
tā, ka apakšvirkne $a_N, a_{N+1}, a_{N+2},\ldots$ nav konstanta 
pēc moduļa $p$ nevienam naturālam $N$.
\end{uzdevums}

\begin{uzdevums}[P3]
Ar $n > 1$ apzīmēts kāds naturāls skaitlis. Pierādīt, ka
bezgalīgi daudzi locekļi virknei 
$a_k=\left\lfloor\frac{n^k}{k}\right\rfloor$
ir nepāru skaitļi. ($\lfloor x\rfloor$ apzīmē 
lielāko veselo skaitli, kas nepārsniedz $x$.)
\end{uzdevums}

\begin{uzdevums}[P4]
Pierādīt, ka jebkuram naturālam $n$ atradīsies
$n$ pēc kārtas sekojoši naturāli skaitļi, no kuriem neviens
nav pirmskaitļa pakāpe, ieskaitot pirmo. 
\end{uzdevums}

\begin{uzdevums}[P5]
Dots naturāls skaitlis $n$ un $a_1, a_2, a_3, \ldots, a_k$ ($k \geq 2$) 
ir dažādi veseli skaitļi no kopas $\{1, 2, \ldots , n\}$ ka $n$ dala $a_i (a_{i+1} - 1)$
pie $i = 1, 2,\ldots,k-1$. Pierādīt, ka $n$ nedala $a_k(a_1-1)$.
\end{uzdevums}



\begin{uzdevums}[P6]
Par {\em aromātisku} sauksim tādu naturālu skaitļu kopu, 
kas sastāv no vismaz diviem elementiem un katram no tās
elementiem ir vismaz viens kopīgs pirmreizinātājs ar 
vismaz vienu no pārējiem elementiem. Apzīmēsim 
$P(n)=n^2+n+1$. Kāda ir mazākā iespējamā naturālā skaitļa 
$b$ vērtība, pie nosacījuma, ka eksistē tāds nenegatīvs 
vesels skaitlis $a$, kuram kopa\\ 
$\{P(a+1),P(a+2),\ldots,P(a+b)\}$ ir {\em aromātiska}?
\end{uzdevums}

\begin{uzdevums}[P7]
Skaitļa decimālpieraksts satur $3^{2013}$ ciparus "3"; 
citu ciparu skaitļa pierakstā nav. Atrast augstāko skaitļa $3$ pakāpi, 
kas ir šī skaitļa dalītājs.
\end{uzdevums}

\begin{uzdevums}[P8]
Ar $P(n)$ apzīmējam lielāko pirmskaitli, ar ko dalās $n$. Atrast
visus naturālos skaitļus $n \geq 2$, kam
$$P(n) + \lfloor \sqrt{n} \rfloor = P(n+1) + \lfloor \sqrt{n+1} \rfloor.$$
\end{uzdevums}

\begin{uzdevums}[P9]
Vai eksistē naturāls skaitlis $n$, ka tam ir tieši $2000$ 
dalītāji, kas ir pirmskaitļi, un $2^n + 1$ dalās ar $n$?
\end{uzdevums}

\begin{uzdevums}[P10]
Atrast visas sirjektīvās funkcijas 
$f : \mathbb{N} \rightarrow \mathbb{N}$, ka
visiem $m, n \in \mathbb{N}$ un katram pirmskaitlim $p$, skaitlis
$f(m+n)$ dalās ar $p$ tad un tikai tad, ja 
$f(m)+f(n)$ dalās ar $p$.\\  
{\em Piezīme.} Sirjektīvās funkcijas pieņem visas vērtības no $\mathbb{N}$.
\end{uzdevums}

\begin{uzdevums}[P11]
Noskaidrojiet, vai eksistē tāds naturāls skaitlis $n$, ka 
skaitlis $n \cdot 2^{2016} - 7$ ir
naturāla skaitļa kvadrāts.
\end{uzdevums}

\begin{uzdevums}[P12]
Dots nekonstants polinoms $P(x)$ ar veseliem koeficientiem. 
Pierādīt, ka nevar atrast tādu $m$, ka visi $P(m+1),P(m+2),P(m+3),\ldots$ būtu bezkvadrātu ({\em square free}),  
t.i. visi to pirmreizinātāji būtu dažādi. 
\end{uzdevums}

\end{document}
