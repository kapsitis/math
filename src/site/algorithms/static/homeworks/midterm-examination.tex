\documentclass[11pt]{article}
\usepackage{ucs}
\usepackage[utf8x]{inputenc}
\usepackage{changepage}
\usepackage{graphicx}
\usepackage{amsmath}
\usepackage{gensymb}
\usepackage{amssymb}
\usepackage{enumerate}
\usepackage{tabularx}
\usepackage{lipsum}
\usepackage{hyperref}

\oddsidemargin 0.0in
\evensidemargin 0.0in
\textwidth 6.27in
\headheight 1.0in
\topmargin -0.1in
\headheight 0.0in
\headsep 0.0in
\textheight 9.50in

%\setlength\parindent{0pt}

\newenvironment{myenv}{\begin{adjustwidth}{0.4in}{0.4in}}{\end{adjustwidth}}
\renewcommand{\abstractname}{Anotācija}
\renewcommand\refname{Atsauces}



\newcounter{alphnum}
\newenvironment{alphlist}{\begin{list}{(\Alph{alphnum})}{\usecounter{alphnum}\setlength{\leftmargin}{2.5em}} \rm}{\end{list}}

\makeatletter
\let\saved@bibitem\@bibitem
\makeatother

\usepackage{bibentry}
%\usepackage{hyperref}


\begin{document}

\thispagestyle{empty}

{\Large Lietišķie algoritmi \textendash{} Semestra vidus eksāmens (Midterm)}

\begin{enumerate}
\item 
Spēlētājs $X$ ar vienādām varbūtībām var izķeksēt no urnas kartiņu, uz kuras rakstīts
kāds no burtiem {\tt a}, {\tt b} vai {\tt c}. Vienā gājienā viņš izķeksē no urnas trīs kartiņas, sakārto 
tās burtu alfabētiskā secībā. Pēc tam viņš nosūta spēlētājam $Y$ kā ziņojumu to burtu, kurš šajā sakārtojumā 
bija pirmais. (Piemēram, ja izķeksētie burti ir {\tt c,b,c}, tad pēc sakārtošanas tie būs {\tt "bcc"} un 
tiks nosūtīts ziņojums {\tt "b"}.) 
\begin{enumerate}
\item Kāds ir informācijas saturs ziņojuma {\tt a} nosūtīšanai?
\item Informācijas saturs ziņojumam {\tt b}? 
\item Informācijas saturs ziņojumam {\tt c}?
\item Kāda ir entropija vienam ziņojumam, ko $X$ nosūta $Y$ saskaņā ar augšminēto procedūru?
\end{enumerate}
\item 
Hafmana kodu sauksim par kanonisku, ja tas apmierina šīs īpašības: ... un ...
Ir $5$ ziņojumu kopa $\{ a, b, c, d, e \}$, kur visu piecu ziņojumu atbilstošās varbūtības ir ... 
Atrast šai ziņojumu kopai atbilstošo kanonisko Hafmana koda koku.



\item 
LZ78 kodējums - izveidot tabulu un nokodēt sekojošu ziņojumu: ... 

\item 
Veikt inverso Berouza-Vīlera transformāciju virknītei: ....  




\end{document}



