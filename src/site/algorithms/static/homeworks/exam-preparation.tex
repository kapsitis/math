\documentclass[11pt]{article}
\usepackage{ucs}
\usepackage[utf8x]{inputenc}
\usepackage{changepage}
\usepackage{graphicx}
\usepackage{amsmath}
\usepackage{gensymb}
\usepackage{amssymb}
\usepackage{enumerate}
\usepackage{tabularx}
\usepackage{lipsum}
\usepackage{hyperref}

\oddsidemargin 0.0in
\evensidemargin 0.0in
\textwidth 6.27in
\headheight 1.0in
\topmargin -0.1in
\headheight 0.0in
\headsep 0.0in
\textheight 9.0in

\setlength\parindent{0pt}

\newenvironment{myenv}{\begin{adjustwidth}{0.4in}{0.4in}}{\end{adjustwidth}}
\renewcommand{\abstractname}{Anotācija}
\renewcommand\refname{Atsauces}



\newcounter{alphnum}
\newenvironment{alphlist}{\begin{list}{(\Alph{alphnum})}{\usecounter{alphnum}\setlength{\leftmargin}{2.5em}} \rm}{\end{list}}

\makeatletter
\let\saved@bibitem\@bibitem
\makeatother

\usepackage{bibentry}
%\usepackage{hyperref}


\begin{document}

\thispagestyle{empty}

% Vai gribi uzzināt, kā ar Hafmana koku saspiest saldējumā esošo rozīņu skaitu? 
% 


{\Large Lietišķie algoritmi \textendash{} Eksāmena sagatavošanās vingrinājumi}

\noindent
{\bf 1.uzdevums: Diskrēts gadījumlielums.}
M.Bendiks reizēm pērk Rūjienas saldējumu ar rozīnēm (\url{https://bit.ly/2OO1xdn}).
Pieņemsim, ka rozīņu skaits vienā saldējumā pakļaujas Puasona sadalījumam 
ar vidējo vērtību $\lambda = 5$. Aprēķinu vienkāršošanai var pieņemt, ka
rozīņu skaits nevienā saldējumā nepārsniedz $100$ (t.i.\ gadījumlielums
$X \in \{ 0,1,\ldots,99,100 \}$), tie saldējumi, kuros tas izrādījās lielāks, 
pārdošanā nenonāca.\\
Atrasto rozīņu skaitus M.Bendiks vēlas nosūtīt, izmantojot Hafmana kodus. 
\begin{enumerate}[(a)]
\item Cik garš ir visīsākais Hafmana kods; kuram rozīņu skaitam tas atbilst?
\item Cik garš Hafmana kods atbilst vērtībai $X=0$?
\item Cik garš Hafmana kods atbilst vērtībai $X=100$?
\item Kāda ir saldējumā atrodamā rozīņu skaita entropija jeb caurmēra ziņojumā
esošās informācijas saspiežamības robeža? {\em (Noapaļot rezultātu līdz 3 cipariem 
aiz komata.)}
\end{enumerate}

\vspace{6pt}
{\bf 2.uzdevums Aritmētiskais kods:}
We have the







\end{document}



