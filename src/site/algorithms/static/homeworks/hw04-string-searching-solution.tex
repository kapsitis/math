\documentclass[11pt]{article}
\usepackage{ucs}
\usepackage[utf8x]{inputenc}
\usepackage{changepage}
\usepackage{graphicx}
\usepackage{amsmath}
\usepackage{gensymb}
\usepackage{amssymb}
\usepackage{enumerate}
\usepackage{tabularx}
\usepackage{lipsum}
\usepackage{hyperref}
\usepackage{enumitem}


\oddsidemargin 0.0in
\evensidemargin 0.0in
\textwidth 6.27in
\headheight 1.0in
\topmargin -0.1in
\headheight 0.0in
\headsep 0.0in
\textheight 9.00in

\setlength\parindent{0pt}


\newenvironment{myenv}{\begin{adjustwidth}{0.4in}{0.4in}}{\end{adjustwidth}}
\renewcommand{\abstractname}{Anotācija}
\renewcommand\refname{Atsauces}

\setlist{nosep}

\newcounter{alphnum}
\newenvironment{alphlist}{\begin{list}{(\Alph{alphnum})}{\usecounter{alphnum}\setlength{\leftmargin}{2.5em}} \rm}{\end{list}}


\usepackage{verbatim}
\newlength\myverbindent
\setlength\myverbindent{1in} % change this to change indentation
\makeatletter
\def\verbatim@processline{%
  \hspace{\myverbindent}\the\verbatim@line\par}
\makeatother

\makeatletter
\let\saved@bibitem\@bibitem
\makeatother

\usepackage{bibentry}



\begin{document}

\thispagestyle{empty}

{\Large Lietišķie algoritmi \textendash{} 4. mājas darbs: Atrisinājumi}


{\footnotesize
\vspace{2ex}
{\bf 1.uzdevums: RSA algoritms.}
Bobs vēlas izveidot savu privātās/publiskās
atslēgas pāri $(n,e)$, kur $n = p \cdot q$ ir reizinājums diviem
mazākajiem $17$-ciparu pirmskaitļiem (un $p<q$), bet kā
publisko kāpinātāju viņš grib izvēlēties skaitli $e = 2^{16} + 1$.
\begin{enumerate}[label=(\alph*)]
\item Kādi ir pirmskaitļi $p$ un $q$ un to reizinājums $n$?
(Lielu pirmskaitļu pārbaudīšanai var izmantot esošas bibliotēkas, kas
implementē Rabina-Millera varbūtisko pirmskaitļu pārbaudi,
piemēram Python funkciju {\tt sympy.isprime}.)
\item Alise grib nosūtīt ziņojumu $m = 100$, izmantojot
publisko RSA kriptoatslēgu - pāri $(n,e)$. Kāds ir viņas
iešifrētais ziņojums?
\item Cik reizināšanas darbības pēc $n$ moduļa Alisei jāveic, lai
iešifrētu $m$?
\item Kāda ir Boba izmantotā privātā kriptoatslēga $d$, kurai
ir spēkā $e \cdot d \equiv 1$ pēc $\varphi(n)$ moduļa?
\item Cik reizināšanas darbības pēc $n$ moduļa Bobam jāveic, lai
atšifrētu Alises iešifrēto ziņojumu?
\end{enumerate}
}

\vspace{2ex}
{\bf (a)} Mazākais $17$-ciparu skaitlis ir $10^{16}$ (viens vieninieks
un sešpadsmit nulles). Pitona scenārijs $p,q$ (mazāko 17-ciparu pirmskaitļu atrašanai):
\setlength\myverbindent{.5in}
\begin{verbatim}
import sympy
primes = list()
n = 10**16
while len(primes) < 2:
    n += 1
    if sympy.isprime(n):
        primes.append(n)
p,q = primes[0],primes[1]
print('(p,q)=({},{})'.format(p,q))
\end{verbatim}
Iegūstam, ka $p=10000000000000061$ un $q=10000000000000069$. To reizinājums
$$n = pq = 100000000000001300000000000004209.$$


{\bf (b)} Alise nosūta skaitli $m^e\;(\text{mod}\,n)$. Mūsu gadījumā $m=100$,
$e = 2^{16} + 1$, $n = pq$. Lai kāpinātu pakāpē $e$, izmantojam nevis atkārtotu reizināšanu
ciklā (kas prasītu ļoti daudz laika), bet gan sešpadsmit reizes kāpinām skaitli $m=100$
kvadrātā un pēc tam vēlreiz piereizinām ar $m$. To panāk ar šādu scenārija turpinājumu:
\begin{verbatim}
n = p*q
m = 100
encrypt = m
# Raise to the power 2^16 - repeat 16 times
for _ in range(16):
    encrypt = (encrypt*encrypt) % n
# Raise power 2^16 to one higher: 2^16+1
encrypt = (encrypt*m) % n
print('Alice sends {}'.format(encrypt))
\end{verbatim}
Iegūtais kriptoteksts ir šāds:
$$m^e\;(\text{mod}\,n) = 14901635321531082019468095932167.$$

{\bf (c)} Iepriekš aprakstītajā aprēķinā Alise veica $16+1 = 17$ reizināšanas darbības:
16 reizes reizināja skaitli pašu ar sevi (un atrada atlikumu), pēdējā solī vēlreiz
piereizināja ar $m=100$.

{\bf (d)} Privāta kriptoatslēga, ko izmanto Bobs ir $d = e^{-1}\;(\text{mod}\,(p-1)(q-1))$.
Šeit $e = 2^{16} + 1$; tātad jāatrod skaitlis $d$, kuru piereizinot ar $e$ iegūsim atlikumu $1$,
dalot ar $(p-1)(q-1)$.
\begin{verbatim}
def egcd(a, b):
    if a == 0:
        return (b, 0, 1)
    else:
        g, y, x = egcd(b % a, a)
        return (g, x - (b // a) * y, y)

def modinv(a, m):
    g, x, y = egcd(a, m)
    if g != 1:
        raise Exception('modular inverse does not exist')
    else:
        return x % m

e = 2**16 + 1
priv = (p-1)*(q-1)
d = modinv(e,priv)
print('d = {}'.format(d))
\end{verbatim}
Iegūstam šādu atslēgas vērtību:
$$d = 93617345926729611259593817236833$$
Atšifrēšanas pareizuma pārbaudei var izmantot Pitona iebūvēto funkciju {\tt pow(x,y,m)}, kas
kāpina $x$ pakāpē $y$ pēc $m$ moduļa:
\begin{verbatim}
decrypt = pow(encrypt,d,n)
print('decrypted = {}'.format(decrypt))
\end{verbatim}


{\bf (e)} Iepriekšējā solī atšifrēšana (kāpināšana lielajā pakāpē $d$)
notika ar iebūvētu Pitona funkciju {\tt pow(x,y,m)}. Ja mums pašiem tā
būtu efektīvi jāuzprogrammē,
kāpinātāju $d$ pārveidojam divnieku skaitīšanas sistēmā (var izmantot Pitona
iebūvēto funkciju {\tt bin(...)}, bet bināro pierakstu var ātri iegūt arī, atkārtoti dalot ar divnieku):
\begin{verbatim}
print('d_bin = {}'.format(bin(d)))
\end{verbatim}
Iegūtais skaitlis ir (skaitļa pierakstā ir $107$ binārie cipari, no tiem $55$ ir vieninieki).
$$d = (1001001110\ldots{}0101100001)_2.$$
Kāpināšanai šādā pakāpē, izmantojot {\em Exponentiation by Squaring} metodi,
vajag $107-1 = 106$ reizes kāpināt kvadrātā un arī $55-1 = 54$ reizes
piereizināt kāpināmo skaitli. Pavisam tātad $160$ reizināšanas.

{\bf Piezīme.} Ievērosim, ka Alisei kāpinot ļoti optimāli izraudzītajā pakāpē $e = 2^{16}+1$ vajadzēja tikai $17$
reizināšanas, bet atšifrēšanai vajag $160$ reizināšanas.
Asimetriskiem algoritmiem bieži gadās, ka
iešifrēšana un atšifrēšana atšķiras laika sarežģītības ziņā. Mūsu gadījumā
publiskā atslēga $e = 2^{16}+1$ tika
izraudzīta tā, lai iešifrēt varētu īpaši ātri.


{\footnotesize
\vspace{10ex}
{\bf 2.uzdevums: Afīnās mērogošanas metode LP uzdevumā.}
Dots LP uzdevums: Maksimizēt $2x_1 + 3x_2$, kur
$$\left\{ \begin{array}{l}
x_1 - 2x_2 \leq 4,\\
x_1 + x_2 \leq 18,\\
x_2 \leq 10,\\
x_1,x_2 \geq 0.
\end{array} \right.$$
Aprēķināt un attēlot koordinātu plaknē šī LP uzdevuma pirmos $2$ tuvinājumus
$X(1)$ un $X(2)$ kā $2$-dimensionālus vektorus, izmantojot
afīnās skalēšanas metodi.\\
Izvēlētais sākumpunkts $X(0) = (5,5)$ (t.i. sākumpunkta
koordinātes ir $x_1 = 5$ un $x_2 = 5$). Gan $X(1)$, gan $X(2)$ abas koordinātes
atbildē noapaļot līdz $5$ cipariem aiz komata. Soļa garums abos
gadījumos: $\beta = 0.96$. (Vektoru un matricu operācijām var izmantot Pitona bibliotēkas.)
}






{\footnotesize
\vspace{10ex}
{\bf 3.uzdevums: KMP un BM algoritmi.}
% https://www.csie.ntu.edu.tw/~hsinmu/courses/_media/dsa_17spring/dsa_2017_hw2_sol_1.pdf
Virknē {\tt 947892879487} meklējam apakšstringu {\tt 9487}.
\begin{enumerate}[label=(\alph*)]
\item Atrast Knuta-Morisa-Prata algoritmam vajadzīgo prefiksu funkciju $\pi$.
\item Atrast Bojera-Mūra algoritmam vajadzīgo labo sufiksu tabulu un sliktā simbola tabulu.
\end{enumerate}
}

\vspace{2ex}
{\bf (a)} Izveidojam tabuliņu prefiksu funkcijai:

\begin{tabular}{|r||c|c|c|c|} \hline
$j$ & 1 & 2 & 3 & 4 \\ \hline
$\pi(j)$ & 0 & 0 & 0 & 0 \\ \hline
\end{tabular}

{\bf (b)} Izveidojam sliktā simbola tabulu \textendash{} atzīmējam tabuliņā pēdējo indeksu,
ar kuru simbols ietilpst paraugā $P = \mathtt{9487}$ vai $-1$, ja simbola tur nav vispār.
Šeit $\mathtt{\ast}$ apzīmē visus citus simbolus, izņemot tabulā ierakstītos.

\begin{tabular}{|r||c|c|c|c|c|} \hline
$x$ & $\mathtt{4}$ & $\mathtt{7}$ & $\mathtt{8}$ & $\mathtt{9}$ & $\mathtt{\ast}$ \\ \hline
$\lambda(x)$ & 1 & 3 & 2 & 0 & -1 \\ \hline
\end{tabular}



{\footnotesize
\vspace{10ex}
{\bf 4.uzdevums: Bojera-Mūra algoritms.}
Dots teksts $T = \mathtt{abcabbcabcbcababababcbcab}$ un meklējamais paraugs
$P = \mathtt{abcbcab}$.
\begin{enumerate}[label=(\alph*)]
\item Uzrakstīt Bojera-Mūra algoritmā lietotās tabulas apakšstringam $P$.
\item Nodemonstrēt Bojera-Mūra darbību pa soļiem, meklējot paraugu $P$ dotajā tekstā $T$.
\end{enumerate}
}

\vspace{2ex}
{\bf (a)} Sliktā simbola tabula (tikai $3$ burti, jo citu tekstā un paraugā nav):

\begin{tabular}{|r||c|c|c|} \hline
$x$ & {\tt a} & {\tt b} & {\tt c} \\ \hline
$\lambda(x)$ & 5 & 6 & 4 \\ \hline
\end{tabular}

Labā sufiksa tabulu aprēķināsim, vispirms atrodot prefiksu funkcijas $\pi$ 
un $\pi'$ paraugam $P = \mathtt{abcbcab}$ un reversajam paraugam 
$P' = \mathtt{bacbcba}$ (no otra gala uzrakstītam paraugam).

\begin{tabular}{|r||c|c|c|c|c|c|c|} \hline
$j$ & 1 & 2 & 3 & 4 & 5 & 6 & 7 \\ \hline
$\pi(j)$ & 0 & 0 & 0 & 0 & 0 & 1 & 2 \\ \hline
\end{tabular}

\begin{tabular}{|r||c|c|c|c|c|c|c|} \hline
$j$ & 1 & 2 & 3 & 4 & 5 & 6 & 7 \\ \hline
$\pi'(j)$ & 0 & 0 & 0 & 1 & 0 & 1 & 2 \\ \hline
\end{tabular}


Pēc tam pielabojam šīs vērtības: 
\begin{enumerate}[(a)]
\item Vispirms piešķiram sākumvērtības (kuras vēlāk pielabosim):\\
$\gamma(j) = $ (kur $m = 7$ ir parauga garums). 
\item Apstaigājam visus indeksus 
$j_{\ell}=m−\pi[\ell]$ ($\ell \in \{ 1,\ldots,m \}$) - 
mūsu gadījumā tie ir $j_{\ell}= 7,7,7,6,7,6,5$.
\item Katram $j_{\ell}$, ja $\ell - \pi'[\ell] > \gamma[j_\ell]$, 
aizstājam 
\item 


\begin{tabular}{|r||c|c|c|c|c|c| \hline
$j$         & 7 & 6 & 5 & 4 & 3 & 2 & 1 \\ \hline
$m - \pi(j)$ & 5 & 6 & 7 & 7 & 7 & 7 & 7 \\ \hline



$j$         & 1 & 2 & 3 & 4 & 5 & 6 & 7 \\ \hline
$m - \pi(j)$ & 7 & 7 & 7 & 7 & 7 & 6 & 5 \\ \hline
\end{tabular}








{\footnotesize
\vspace{10ex}
{\bf 5.uzdevums: I-iespēja (atzīmei 10).}
Vispārināt Rabina-Karpa algoritmu, lai atrastu kvadrātveida paraugu $m \times m$
divdimensionālā simbolu masīvā ar izmēru $n \times n$, kur $n > m$.
(Meklējamo paraugu var bīdīt pa horizontāli un vertikāli, bet to nedrīkst pagriezt.)
\begin{enumerate}[label=(\alph*)]
\item
Aprakstīt algoritmu (ar skaidri definētiem soļiem),
kas atrod visas parauga atrašanās vietas divdimensionālajā
$n \times n$ masīvā kā pozīciju pārus $(s_x,s_y)$, kur $s_x$ ir nobīde pa horizontāli
un $s_y$ ir nobīde pa vertikāli.
\item
Pamatot, ka Jūsu algoritms atrod izvada visas vietas, kur paraugs atrodams.
\item
Atrast mazāko laika sarežģītību visu atrašanās vietu izvadei.
\end{enumerate}
}

\vspace{2ex}
{\bf (a)} Apzīmējam meklējamā kvadrātveida parauga burtus ar
divdimensiju masīva elementiem: $P[i][j]$, kur $0 \leq i,j < m$ un
$P[i][j]$ ir simbols, kas atrodas $i$-tajā rindiņā un $j$-tajā kolonnā.
Gan rindiņas, gan kolonnas numurējam ar skaitļiem no $0$ līdz $m-1$.
Divdimensiju tekstā, kurā veicam meklēšanu, tieši tāpat apzīmējam
burtus ar $T[i][j]$, kur $i,j$ mainās no $0$ līdz $n-1$.

\begin{itemize}
\item
Izvēlamies konstantes $d$ un $q$ - polinomā ievietojamo vērtību $x = d$ un atlikumu moduli $q$.

\item
Katrā parauga $P$ rindiņā izrēķina polinoma atlikumu, ja polinomā ievieto dalot ar konstantu moduli $q$:
$$R[i] = P[i][0]\cdot d^{n-1} + P[i][1]\cdot d^{n-2} + P[i][2]\cdot d^{n-3} + \ldots + P[i][m-2] \cdot d^1 + P[i][m-1] =$$
$$=(\ldots{} (\ldots{}((P[i][0] \cdot d + P[i][1]) \cdot d + P[i][2]) \cdot d + P[i][3] )  \ldots{} ) \cdot d + P[i][m-1].$$
Šeit izmantojam Hornera shēmu - katrā solī pieskaitām kārtējo $P$ vērtību un piereizinām ar $d$. Turklāt
gan saskaitīšanu, gan reizināšanu šajos polinomu vērtību aprēķinos veicam pēc $q$ moduļa; tādējādi
izvairāmies no ļoti lieliem skaitļiem.
\item
No vērtībām $R[i]$ būvē nākamo polinomu (arī to rēķina ar Hornera shēmu tāpat kā iepriekš):
$$р^{\ast} = R[0]\cdot d^{n-1} + R[1]\cdot d^{n-2} + R[2]\cdot d^{n-3} + \ldots + R[m-2] \cdot d^1 + R[m-1]$$
\end{itemize}


Katrā matricas $n \times n$ rindiņā aprēķinām



\end{document}



