\documentclass[jou]{apa6}

\usepackage[american]{babel}

\usepackage{csquotes}
\usepackage[style=apa,sortcites=true,sorting=nyt,backend=biber]{biblatex}
\DeclareLanguageMapping{american}{american-apa}
\addbibresource{bibliography.bib}


%%%%%%%%%%%%%%%%%%%%%%%%%%%%%%%%%%%%%%%%
%% Discrete Structures
%% The start of RBS stuff
%%%%%%%%%%%%%%%%%%%%%%%%%%%%%%%%%%%%%%%%

% Working internal and external links in PDF
\usepackage{hyperref}
% Extra math symbols in LaTeX
\usepackage{amsmath}
\usepackage{gensymb}
\usepackage{amssymb}
% Enumerations with (a), (b), etc.
\usepackage{enumerate}
\usepackage[framemethod=TikZ]{mdframed}
\usepackage{xcolor}
\usepackage{graphicx}
\usepackage[justification=centering]{caption}
\usepackage{fancyvrb}
%\usepackage{changepage}
\usepackage{caption}% http://ctan.org/pkg/caption
\captionsetup[table]{format=plain,labelformat=simple,labelsep=period}%

\def\changemargin#1#2{\list{}{\rightmargin#2\leftmargin#1}\item[]}
\let\endchangemargin=\endlist 


\let\OLDitemize\itemize
\renewcommand\itemize{\OLDitemize\addtolength{\itemsep}{-6pt}}

\usepackage{etoolbox}
\makeatletter
\preto{\@verbatim}{\topsep=3pt \partopsep=3pt }
\makeatother

% These sizes redefine APA for A4 paper size
\oddsidemargin 0.0in
\evensidemargin 0.0in
\textwidth 6.27in
\headheight 1.0in
%\topmargin -24pt
\topmargin -32pt
\headheight 12pt
\headsep 12pt
%\textheight 9.19in
\textheight 9.35in


\title{Sample Quiz 8}
\author{Discrete Structures, Spring 2020}
\affiliation{RBS}

\leftheader{Discrete Sample Quiz 8}

\abstract{%
}

%\keywords{}

\setlength\parindent{0pt}

\begin{document}

\twocolumn
\thispagestyle{empty}

\begin{center}
{\Large Alternative Homework 4:}\\
{\Large String Search}
\end{center}

% https://ocw.mit.edu/courses/electrical-engineering-and-computer-science/6-441-information-theory-spring-2016/assignments/

\begin{changemargin}{10pt}{10pt}
{\footnotesize
{\em Note.} This is a parody of MIT OCW content.\\
See \url{https://ocw.mit.edu/terms/}. 
The original assignments and related materials can be retrieved from 
TBD and TBD\\
}
\end{changemargin}



%https://ocw.mit.edu/courses/electrical-engineering-and-computer-science/6-046j-design-and-analysis-of-algorithms-spring-2015/assignments/MIT6_046JS15_pset2.pdf

{\bf Question 1.}
Suppose you are given a source string $T[0..n-1]$ of length $n$, consisting of symbols $a$ and $b$.
Suppose further that you are given a pattern string $P[0..m-1]$ of length $m<<n$, consisting of
symbols $a$, $b$, and $\ast$, representing a pattern to be found in string $S$. The symbol $\ast$ is a ``wild card''
symbol, which matches a single symbol, either $a$ or $b$. The other symbols must match exactly.
The problem is to output a sorted list $M$ of valid ``match positions'', which are positions $j$ in $S$
such that pattern $P$ matches the substring $S[j..j+|P|-1]$. For example, if $S =\mathtt{ababbab}$ and
$P=\mathtt{ab\ast}$, then the output $M$ should be $[0,2]$.

{\bf (A)} KMP equivalent. 
{\bf (B)} Prefix function for pattern STH.
{\bf (C)} Compare with the polynomial algorithm. 




See \url{https://bit.ly/2XKX5AB}. 







%% String encoding
% https://ocw.mit.edu/courses/electrical-engineering-and-computer-science/6-851-advanced-data-structures-spring-2012/index.htm
%% Problemset 9 - a nice, but complicated exercise on suffix tries.

%% https://ocw.mit.edu/courses/electrical-engineering-and-computer-science/6-851-advanced-data-structures-spring-2012/lecture-videos/session-16-strings/
%% Document retrieval. 






% https://ocw.mit.edu/courses/electrical-engineering-and-computer-science/6-006-introduction-to-algorithms-fall-2011/lecture-videos/lecture-9-table-doubling-karp-rabin/
%% Karp-Rabin string search and rolling hash.

%% https://ocw.mit.edu/courses/electrical-engineering-and-computer-science/6-851-advanced-data-structures-spring-2012/lecture-videos/session-16-strings/
%% Document retrieval. 



% https://ocw.mit.edu/courses/electrical-engineering-and-computer-science/6-042j-mathematics-for-computer-science-fall-2010/index.htm
% MIT Mathematics for Computer Science course

% https://ocw.mit.edu/courses/electrical-engineering-and-computer-science/6-006-introduction-to-algorithms-fall-2011/index.htm
% MIT 6.006: Introduction to Algorithms course

% https://ocw.mit.edu/courses/electrical-engineering-and-computer-science/6-046j-design-and-analysis-of-algorithms-spring-2015/
% MIT Design and Analysis of Algorithms


\end{document}

