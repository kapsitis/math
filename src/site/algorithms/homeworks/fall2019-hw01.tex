\documentclass[11pt]{article}
\usepackage{ucs}
\usepackage[utf8x]{inputenc}
\usepackage{changepage}
\usepackage{graphicx}
\usepackage{amsmath}
\usepackage{gensymb}
\usepackage{amssymb}
\usepackage{enumerate}
\usepackage{tabularx}
\usepackage{lipsum}
\usepackage{hyperref}

\oddsidemargin 0.0in
\evensidemargin 0.0in
\textwidth 6.27in
\headheight 1.0in
\topmargin -0.1in
\headheight 0.0in
\headsep 0.0in
\textheight 9.50in

%\setlength\parindent{0pt}

\newenvironment{myenv}{\begin{adjustwidth}{0.4in}{0.4in}}{\end{adjustwidth}}
\renewcommand{\abstractname}{Anotācija}
\renewcommand\refname{Atsauces}



\newcounter{alphnum}
\newenvironment{alphlist}{\begin{list}{(\Alph{alphnum})}{\usecounter{alphnum}\setlength{\leftmargin}{2.5em}} \rm}{\end{list}}


%16.3-6

\makeatletter
\let\saved@bibitem\@bibitem
\makeatother

\usepackage{bibentry}
%\usepackage{hyperref}


\begin{document}

\thispagestyle{empty}

{\Large Lietišķie algoritmi \textendash{} 1. mājas darbs}

\noindent
{\em Termiņš: 2019.\ gada 30.\ septembris. Laiks 23:59:59 Eastern European Summer Time.}\\
{\em Iesūtīšanas veids: PDF uz epastu "kalvis.apsitis" domēnā "gmail.com".}\\
{\bf Ieteicamā lasāmviela.} {\bf [Blelloch2013]}, pp.16--19.\\
Sk. \url{http://linen-tracer-682.appspot.com/algorithms/references.html}

\begin{enumerate}
\item {\bf Aritmētiskais kods.} 
Dota ziņojumu kopa $S = \{ A,B,C,D \}$ ar attiecīgajām varbūtībām 
$\{ 0.2, 0.5, 0.2, 0.1 \}$.
\begin{enumerate}[(a)]
\item Parādīt, kā iegūt aritmētisko kodu $6$ ziņojumu virknei {\tt CBAABD} -- uzkonstruēt
tai atbilstošo intervālu $[l_6;l_6+s_6) \in [0;1]$ un atrast 
īsāko bitu virkni $d_1d_2\ldots{}d_{\ell}$ (visi $d_k \in \{ 0,1 \}$, 
kur pierakstot binārā pieraksta daļskaitlim $D = 0.d_1d_2\ldots{}d_{\ell}\ldots$ 
galā jebkuru turpinājumu ar cipariem $0$ vai $1$, iegūtais skaitlis $d+\varepsilon$ pieder
intervālam $[l_6;l_6+s_6)$.
\item Noteikt, kādu ziņojumu virkni alfabētā $S$ kodē skaitlis 
$D'' = 0.0011101011_2$.
\end{enumerate}
\item {\bf Lempela-Ziva algoritms.} 
\begin{enumerate}[(a)]
\item 
Ar LZ78 metodi nokodēt tekstu “abracadabra, abracadabra”.
\item Atkodēt ar LZ78 metodi nokodētu tekstu $a,b,c,d,2,5,a,6$, kur
$a$, $b$ un $c$ apzīmē atbilstošos burtus, bet skaitļi – vārdnīcas virkņu
numurus.
\item
Nokodēt (a) punkta tekstu “abracadabra, abracadabra” 
ar LZ77 metodi, kā logu lietojot visu nokodēto/atkodēto tekstu.
\end{enumerate}
\item {\bf Berouza-Vīlera transformācija.}
\begin{enumerate}[(a)]
\item 
Kāds ir rezultāts (transformētā simbolu virkne un sākotnējās virknes pozīcija), 
lietojot Berouza-Vīlera transformāciju 
$14$ simbolu virknei {\tt alusariirasula}?
\item Kāds ir iepriekšējā piemērā iegūtās transformētās simbolu virknes pieraksts,
izmantojot Move-to-Front kodēšanu?
\item Pēc BW transformācijas pielietošanas tika iegūta simbolu virkne 
{\tt mmmrvvauuuiibbbri}. Kāda bija simbolu virkne pirms 
transformācijas (ņemot 4.\ virkni
no atjaunotās tabulas)?
\end{enumerate}
\item {\bf I-iespēja (atzīmei 10).} 
Pieņemsim, ka ziņojumu kopai $S = \{ x_1, x_2, \ldots, x_n \}$ ar 
izveidots optimāls prefiksu kodējums. Šis kodējums jāpārraida,
izmantojot minimālu bitu skaitu.\\
Pierādīt vai apgāzt šādu 
apgalvojumu: Jebkuru optimālu prefiksu 
kodējumu šai $n$ ziņojumu kopai var nosūtīt, izmantojot ne vairāk kā 
${\displaystyle 2n - 1 + n \left\lceil log_2 n \right\rceil}$ bitus. 
Šeit $\lceil x \rceil$ apzīmē noapaļošanu uz augšu jeb 
mazāko veselo skaitli, kas nav mazāks par $x$.
Piemēram $\lceil 17 \rceil = 17$ un $\lceil 3.14 \rceil = 4$.\\
{\em Ieteikums.} Izmantojot $2n-1$ bitus, var attēlot kodējumu koka 
virsotņu apstaigāšanas secību.


\end{enumerate}


%Tie, kuriem patīk programmēt, var pirmos trīs pseidokoda uzdevumus (1.,2.,3.) aizstāt ar 
%kādu no tālākminētajiem programmēšanas uzdevumiem.
%\begin{enumerate}
%\item Implementēt objektu {\tt HuffmanMain} valodā Scala, kuru darbinot
%
%\end{enumerate}

%http://www.ccs.neu.edu/home/jaa/CS6800.11F/Homeworks/HW05.pdf


\end{document}



