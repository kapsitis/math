\documentclass[11pt]{article}
\usepackage{ucs}
\usepackage[utf8x]{inputenc}
\usepackage{changepage}
\usepackage{graphicx}
\usepackage{amsmath}
\usepackage{gensymb}
\usepackage{amssymb}
\usepackage{enumerate}
\usepackage{tabularx}
\usepackage{lipsum}
\usepackage{hyperref}
\usepackage{enumitem}


\oddsidemargin 0.0in
\evensidemargin 0.0in
\textwidth 6.27in
\headheight 1.0in
\topmargin -0.1in
\headheight 0.0in
\headsep 0.0in
\textheight 9.00in

\setlength\parindent{0pt}


\newenvironment{myenv}{\begin{adjustwidth}{0.4in}{0.4in}}{\end{adjustwidth}}
\renewcommand{\abstractname}{Anotācija}
\renewcommand\refname{Atsauces}

\setlist{nosep}

\newcounter{alphnum}
\newenvironment{alphlist}{\begin{list}{(\Alph{alphnum})}{\usecounter{alphnum}\setlength{\leftmargin}{2.5em}} \rm}{\end{list}}


%16.3-6

\makeatletter
\let\saved@bibitem\@bibitem
\makeatother

\usepackage{bibentry}



\begin{document}

\thispagestyle{empty}

{\Large Lietišķie algoritmi \textendash{} 4. mājas darbs}

{\footnotesize
{\bf Termiņš:} 2019.\ gada 9.\ decembris. Laiks 23:59:59 (UTC+2).\\
{\bf Lasāmviela:} \url{http://linen-tracer-682.appspot.com/algorithms/references.html}, tēmas 4,5,6.
}

\vspace{6pt}
{\bf 1.uzdevums: RSA algoritms.} 
Bobs vēlas izveidot savu privātās/publiskās 
atslēgas pāri $(n,e)$, kur $n = p \cdot q$ ir reizinājums diviem 
mazākajiem $17$-ciparu pirmskaitļiem (un $p<q$), bet kā 
publisko kāpinātāju viņš grib izvēlēties skaitli $e = 2^{16} + 1$.
\begin{enumerate}[label=(\alph*)]
\item Kādi ir pirmskaitļi $p$ un $q$ un to reizinājums $n$?
(Lielu pirmskaitļu pārbaudīšanai var izmantot esošas bibliotēkas, kas
implementē Rabina-Millera varbūtisko pirmskaitļu pārbaudi, 
piemēram Python funkciju {\tt sympy.isprime}.)
\item Alise grib nosūtīt ziņojumu $m = 100$, izmantojot 
publisko RSA kriptoatslēgu - pāri $(n,e)$. Kāds ir viņas 
iešifrētais ziņojums?
\item Cik reizināšanas darbības pēc $n$ moduļa Alisei jāveic, lai
iešifrētu $m$?
\item Kāda ir Boba izmantotā privātā kriptoatslēga $d$, kurai 
ir spēkā $e \cdot d \equiv 1$ pēc $\varphi(n)$ moduļa?
\item Cik reizināšanas darbības pēc $n$ moduļa Bobam jāveic, lai 
atšifrētu Alises iešifrēto ziņojumu?
\end{enumerate}


\vspace{6pt}
{\bf 2.uzdevums: Afīnās mērogošanas metode LP uzdevumā.}
Dots LP uzdevums: Maksimizēt $2x_1 + 3x_2$, kur
$$\left\{ \begin{array}{l}
x_1 - 2x_2 \leq 4,\\
x_1 + x_2 \leq 18,\\
x_2 \leq 10,\\
x_1,x_2 \geq 0.
\end{array} \right.$$
Aprēķināt un attēlot koordinātu plaknē šī LP uzdevuma pirmos $2$ tuvinājumus 
$X(1)$ un $X(2)$ kā $2$-dimensionālus vektorus, izmantojot 
afīnās skalēšanas metodi.\\ 
Izvēlētais sākumpunkts $X(0) = (5,5)$ (t.i. sākumpunkta
koordinātes ir $x_1 = 5$ un $x_2 = 5$). Gan $X(1)$, gan $X(2)$ abas koordinātes
atbildē noapaļot līdz $5$ cipariem aiz komata. Soļa garums abos 
gadījumos: $\beta = 0.96$. (Vektoru un matricu operācijām var izmantot Python bibliotēkas.)



\vspace{6pt}
{\bf 3.uzdevums: KMP un BM algoritmi.} 
% https://www.csie.ntu.edu.tw/~hsinmu/courses/_media/dsa_17spring/dsa_2017_hw2_sol_1.pdf
Virknē {\tt 947892879487} meklējam apakšstringu {\tt 9487}.
\begin{enumerate}[label=(\alph*)]
\item Atrast Knuta-Morisa-Prata algoritmam vajadzīgo prefiksu funkciju $\pi$. 
\item Atrast Bojera-Mūra algoritmam vajadzīgo labo sufiksu tabulu un sliktā simbola tabulu.
\end{enumerate}

\vspace{6pt}
{\bf 4.uzdevums: Bojera-Mūra algoritms.}
Dots teksts $T = \mathtt{abcabbcabcbcababababcbcab}$ un meklējamais paraugs
$P = \mathtt{abcbcab}$. 
\begin{enumerate}[label=(\alph*)]
\item Uzrakstīt Bojera-Mūra algoritmā lietotās tabulas apakšstringam $P$.
\item Nodemonstrēt Bojera-Mūra darbību pa soļiem, meklējot paraugu $P$ dotajā tekstā $T$.
\end{enumerate}



\vspace{6pt}
{\bf 5.uzdevums: I-iespēja (atzīmei 10).} 
Vispārināt Rabina-Karpa algoritmu, lai atrastu kvadrātveida paraugu $m \times m$
divdimensionālā simbolu masīvā ar izmēru $n \times n$, kur $n > m$.
(Meklējamo paraugu var bīdīt pa horizontāli un vertikāli, bet to nedrīkst pagriezt.) 
\begin{enumerate}[label=(\alph*)]
\item
Aprakstīt algoritmu (ar skaidri definētiem soļiem), 
kas atrod visas parauga atrašanās vietas divdimensionālajā
$n \times n$ masīvā kā pozīciju pārus $(s_x,s_y)$, kur $s_x$ ir nobīde pa horizontāli 
un $s_y$ ir nobīde pa vertikāli.
\item
Pamatot, ka Jūsu algoritms atrod izvada visas vietas, kur paraugs atrodams.
\item 
Atrast mazāko laika sarežģītību visu atrašanās vietu izvadei. 
\end{enumerate}



\end{document}



