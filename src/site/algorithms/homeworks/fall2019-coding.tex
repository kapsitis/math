\documentclass[11pt]{article}
\usepackage{ucs}
\usepackage[utf8x]{inputenc}
\usepackage{changepage}
\usepackage{graphicx}
\usepackage{amsmath}
\usepackage{gensymb}
\usepackage{amssymb}
\usepackage{enumerate}
\usepackage{tabularx}
\usepackage{lipsum}
\usepackage{hyperref}

\oddsidemargin 0.0in
\evensidemargin 0.0in
\textwidth 6.27in
\headheight 1.0in
\topmargin -0.1in
\headheight 0.0in
\headsep 0.0in
\textheight 9.0in

\setlength\parindent{0pt}

\newenvironment{myenv}{\begin{adjustwidth}{0.4in}{0.4in}}{\end{adjustwidth}}
\renewcommand{\abstractname}{Anotācija}
\renewcommand\refname{Atsauces}



\newcounter{alphnum}
\newenvironment{alphlist}{\begin{list}{(\Alph{alphnum})}{\usecounter{alphnum}\setlength{\leftmargin}{2.5em}} \rm}{\end{list}}

\makeatletter
\let\saved@bibitem\@bibitem
\makeatother

\usepackage{bibentry}
%\usepackage{hyperref}


\begin{document}

\thispagestyle{empty}

% Vai gribi uzzināt, kā ar Hafmana koku saspiest saldējumā esošo rozīņu skaitu? 
% 


{\Large Lietišķie algoritmi \textendash{} Eksāmena sagatavošanās vingrinājumi}

\noindent
{\bf 1.uzdevums. Diskrēts gadījumlielums.}
M.Bendiks reizēm pērk Rūjienas saldējumu ar rozīnēm (\url{https://bit.ly/2OO1xdn}).
Pieņemsim, ka rozīņu skaits vienā saldējumā pakļaujas Puasona sadalījumam 
ar vidējo vērtību $\lambda = 5$. Aprēķinu vienkāršošanai var pieņemt, ka
rozīņu skaits nevienā saldējumā nepārsniedz $20$ (t.i.\ gadījumlielums
$X \in \{ 0,1,\ldots,20 \}$); saldējumi, kuros $X > 20$, 
pārdošanā nenonāca.\\
Atrasto rozīņu skaitus M.Bendiks vēlas nosūtīt, izmantojot Hafmana kodus. 
\begin{enumerate}[(a)]
\item Cik garš ir visīsākais Hafmana kods; kuram rozīņu skaitam tas atbilst?
\item Cik garš Hafmana kods atbilst vērtībai $X=0$?
\item Cik garš Hafmana kods atbilst vērtībai $X=20$?
\item Kāda ir saldējumā atrodamā rozīņu skaita entropija jeb caurmēra ziņojumā
esošās informācijas saspiežamības robeža? {\em (Noapaļot rezultātu līdz 3 cipariem 
aiz komata.)}
\end{enumerate}

\vspace{6pt}
{\bf 2.uzdevums. Aritmētiskais kods.}
Aritmētiskais kods definēts, sākot ar 16.lappusi G.Bleloka konspektā: 
\url{https://www.cs.cmu.edu/~guyb/realworld/compression.pdf}. 
Dots apriori varbūtību sadalījums sekojošiem ziņojuma alfabēta 
burtiem: $\{ (\mathtt{A},6/10), (\mathtt{B},2/10), (\mathtt{C},1/10), 
(\mathtt{\$},1/10) \}$. Izveidot aritmētisko kodu sekojošai $8$ ziņojumu virknei:
$\mathtt{ABABACA\$}$.
Zināms, ka ziņojums beidzas ar "end-of-text" simbolu (dolāru), kam piedēvēta varbūtība $1/10$
(un nekur ziņojuma iekšienē šāda simbola nav).  
Kodēšana beidzas ar intervālu $[\ell_8,\ell_8+s_8)$. 
Aritmētiskais kods ir īsākais skaitļa $c$ binārais pieraksts ar $k$ (binārajiem) cipariem 
aiz komata, kuram $[c,c+2^{-k})$ ir intervāls, kurš pilnībā ietilpst
intervālā $[\ell_8,\ell_8+s_8)$.


% https://www.geeksforgeeks.org/move-front-data-transform-algorithm/
\vspace{6pt}
{\bf 3.uzdevums. Move to Front kods:} Pieņemsim, ka visi $26$ mazie 
latīņu alfabēta burti sākotnēji sakārtoti masīvā alfabētiskā secībā
($A[0]=\mathtt{a}$, $\ldots$, $A[25]=\mathtt{z}$). 
Izveidot "Move to Front" kodējumu sekojošam vārdam: $\mathtt{interchangeableness}$


\vspace{6pt}
{\bf 4.uzdevums. Markova ķēdes arhivācija:} 
Dota Markova ķēde no $6$ stāvokļiem - sk. zīmējumu 
\url{https://bit.ly/2LDJqo8}. 
Izveidot ar šo Markova ķēdi simbolu virknīti tieši miljons ciparu garumā
(katru ciparu attēlot kā ASCII simbolu "1", "2", "3", "4", "5", "6"). 
Arhivēt šo simbolu virkni ar WinZIP (vai Linux "gz") arhivatoru, lai iegūtu 
LZ78-veida arhīva garuma novērtējumu.\\
Saskaitīt, cik šajā virknē ir katra veida ciparu un izrēķināt entropiju (un 
teorētiski īsāko sasiestās virknes garumu). Salīdzināt, cik tuvu ir ZIP 
faila izmērs teorētiskajam minimumam. 


\vspace{6pt}
{\bf 5.uzdevums. Diskrētais kosinusu pārveidojums:} 
$$X_k = \sum_{n=0}^{N-1} x_n \cos \left[\frac{\pi}{N} \left(n+\frac{1}{2}\right) k \right] \quad \quad k = 0, \dots, N-1.$$
Diskrēto kosinusu pārveidojumu uzdod augšminētā formula, kur $N=100$. 
Iedomāsimies, ka $x_n$ ir virkne, kas definēta ar vienādojumu $x_n = \sin x$, kur $n=0,\ldots,99$. 
Atrast, kuri ir trīs lielākie no DCT koeficientiem $X_k$, un kurām $k$ vērtībām tie atbilst.








\end{document}



