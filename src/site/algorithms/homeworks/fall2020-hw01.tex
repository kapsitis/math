\documentclass[a4paper,12pt]{article}

\usepackage{amsmath,amssymb,amsthm,multicol,tikz,enumitem}
\usepackage{hyperref}
\usepackage{fancyvrb}
%\usepackage{enumerate}
\usepackage[margin=2cm]{geometry}
\usetikzlibrary{calc,arrows.meta}

\newcommand\N{\mathbf{N}}
\newcommand\Q{\mathbf{Q}}
\newcommand\R{\mathbf{R}}
\newcommand\Z{\mathbf{Z}}

\def\ojoin{\setbox0=\hbox{$\bowtie$}%
  \rule[-.02ex]{.25em}{.4pt}\llap{\rule[\ht0]{.25em}{.4pt}}}
\def\leftouterjoin{\mathbin{\ojoin\mkern-5.8mu\bowtie}}
\def\rightouterjoin{\mathbin{\bowtie\mkern-5.8mu\ojoin}}
\def\fullouterjoin{\mathbin{\ojoin\mkern-5.8mu\bowtie\mkern-5.8mu\ojoin}}

% Comment out one or the other

\ifdefined\mysolution
  \newcommand\answer[1]{\mbox{}\\[-15pt]{\color{blue}{#1}}\hfill{\color{blue}$\qed$}\mbox{}\\[-15pt]} 
  \newcommand\ans[1]{{\color{blue}{#1}}}
\else 
  \newcommand\answer[1]{}
  \newcommand\ans[1]{}
\fi


\setlength{\parindent}{0pt}

\begin{document}


\begin{center}
{\bf\Huge 1.\ mājasdarbs} \\[5pt]
Lietišķie algoritmi: Bezzudumu saspiešana \\
Termiņš: 2020-10-05\\[5pt]
{\em Atrisinājumus lūdzam pārveidot par vienu PDF.}
\end{center}

\hrule
\vspace{2pt}
\hrule
\vspace{12pt}

% https://ocw.mit.edu/courses/electrical-engineering-and-computer-science/6-441-information-theory-spring-2016/assignments/


%\begin{changemargin}{10pt}{10pt}
{\footnotesize
Vairāki uzdevumi šajā mājasdarbā iespaidojušies no MIT Open Courseware:
\url{https://bit.ly/37Aywtf} un \url{https://bit.ly/2BdK8GA}\\
}
%\end{changemargin}


\vspace{10pt}
{\bf 1.uzdevums (Gabalu garumu kodējums).}\\
Aplūkojam {\em Bernulli gadījumlielumu} $X$, kam
$$\left\{
\begin{array}{l}
P(X = 0) = \dfrac{255}{256},\\[6pt]
P(X = 1) = \dfrac{1}{256}.\\
\end{array} \right.$$
(Piemēram, tiek mesta asi\-met\-ris\-ka monēta, kam ``cipars''
({\em heads} jeb bits $1$) uzkrīt ar varbūtību $\frac{1}{256}$.)

Virknīti ar $n$ šī gadījumlieluma vērtībām ap\-zī\-mē ar $X^n$.
To saspiež, izmantojot {\em Gabalu garumu ko\-dē\-ju\-mu} ({\em run-length encoding}).
Katru vieninieku kodē atsevišķi, bet
nepārtrauktiem nuļļu gabaliem izvada tikai gabalu garumus.

{\bf Algoritma apraksts:}\\
Ievadei $X^n$ saspiesto rezultātu $f(X^n)$ iegūst,
atkārtoti izpildot šos 3 soļus, ko turpina līdzkamēr
ievade pilnībā nolasīta.

\begin{description}
\item[1.Solis] \hfill \\
Ja ievadē pirmais bits ir ``1'', tad to nolasa un
izvadē raksta astoņas nulles:
$\textcolor{blue}{\mathtt{00000000}}$.
\item[2.Solis] \hfill \\
Ja ievades sākumā ir $r \leq 255$ nulles, aiz kurām seko vieninieks,
tad visas šīs nulles nolasa un
izvadē raksta $8$-bitu virknīti \textendash{} skaitļa $r$ bināro pierakstu.
(Tā kā nuļļu skaits $r \neq 0$, tad šajā solī nerodas izvade $\textcolor{blue}{\mathtt{00000000}}$.)
\item[3.Solis] \hfill \\
Ja ievadē ir $r > 255$ pēc kārtas esošas nulles,
tad šajā solī nolasa tikai $255$ nulles (un izvadē raksta skaitli $255$ bināri, kā iepriekšējā punktā:
$\textcolor{blue}{\mathtt{11111111}}$). Turpina pildīt 3.soli (un izvadīt skaitļa
$255$ kodus) līdzkamēr iestājas 1.\ vai 2.\ soļa situācija.
\end{description}

Atrast saspiešanas attiecības robežu šim kodējumam:
\begin{equation}
\label{eq1}
\lim_{n \rightarrow \infty} \frac{1}{n} E(\ell(f(X^n)).
\end{equation}
Ar $E$ apzīmējam vidējo vērtību gadījumlielumam, bet
$\ell(f(X^n))$ apzīmē saspiestās virknītes garumu.

%\begin{enumerate}
%\item
%Atrast (\ref{eq1}) vērtību ar precizitāti līdz $10^{-8}$.
%\item
%Atrast {\em entropijas ātrumu} ({\em entropy rate}) Bernulli eksperimentu virknei:
%$\lim_{n \rightarrow \infty} \frac{1}{n} H(X^n)$ jeb teo\-rē\-tis\-ko robežu, cik labi
%šo virkni var saspiest. Izteikt šo lielumu ar formulu un arī aprēķināt to
%ar precizitāti līdz $10^{-8}$
%\end{enumerate}

(Šī ir parodija par 4.uzdevumu no \url{https://bit.ly/2Y6mUKa}.)


\vspace{10pt}
{\bf 2.uzdevums (Gadījuma analīze).}\\
Aplūkojam ziņojumu alfabētu ar četriem ziņojumiem:
$S = \{ \mathtt{a}, \mathtt{b}, \mathtt{c}, \mathtt{d} \}$,
kuru varbūtības ir attiecīgi $\{1/3, 1/3, 2/9, 1/9\}$.

\begin{enumerate}
\item Izmantot Hafmana algoritmu, lai atrastu šim ziņojumu alfabētam optimālu bezprefiksu ko\-dē\-ju\-mu.
\item Vai ar Hafmana algoritmu var konstruēt bū\-tis\-ki citādu optimālu bezprefiksu kodējumu
(t.i.\ tādu, kas simboliem $\mathtt{a}, \mathtt{b}, \mathtt{c}, \mathtt{d}$
piekārto citādus kodu garumus, nevis tikai kaut kur aizstāj
\textcolor{blue}{\tt 0} ar \textcolor{blue}{\tt 1} un otrādi).
\item Atrast bezprefiksu kodējumu, kurš arī ir optimāls, bet nevar būt radies Hafmana algoritma rezultātā.
\end{enumerate}

(Šis ir uzdevums 2.12. no \url{https://bit.ly/2Y4PKMe}, 55.lpp.)

\answer{

{\bf (A)} Saspiežot ar aritmētisko kodu, saspiestās virknes garums ir  
ziņojumu virknes garums reiz entropija $H(S)$.
Mūsu gadījumā, risinot Markova ķēdes stabilam stāvoklim 
atbilstošo lineāro vienādojumu sistēmu, iegūstam, ka
burta "a" varbūtība ir $1/2$, "b" un "c" varbūtības ir $1/4$. 
Tādēļ entropija $H(S) = 1.5$ (kods pagarinās par vienu bitu, 
ja iekodē "a", un par diviem bitiem, ja iekodē "b" vai "c"); 
vidēji vienam burtam tērējas $1.5$ biti.

Sākotnējās virknes garums ir $8000000$ biti (jeb $1000000$ baiti). 
Pēc saspiešanas tie ir $1.5 \cdot 1000000 = 1500000$ biti 
(jeb $187500$ baiti). 
Attiecība saspiestās virknes garumam baitos pret sākotnējā faila garumu (baitos) 
ir $0.1875$. Saspiestais fails aizņem $18.75\%$ 
no nesaspiesta faila izmēra.

{\em Piezīme.} Ja Jūsu pieņēmumi par sākotnējā faila kodējumu atšķīrās, tad
saspiešanas attiecība var iznākt arī cita. 
Nav grūti "a","b","c" kodēt pat ar diviem baitiem (daži 
Unikoda kodējumi tā dara) vai arī ļoti taupīgi (teiksim, ar 2 bitiem katru burtu, 
jo burtu ir tikai $3$.) 

{\bf (B)} Saspiežot ar WinZip (šādu eksperimentu apraksta
Aleksejs Jelisejevs) iznāk 115050 baiti. Citos 
mājasdarbos ir arī citi dati 132000 baiti utml. 
Kā redzams, vismaz dažos gadījumos WinZip (kas izmanto Lempela-Ziva
saimes algoritmus) var saspiest failus labāk par entropiju. 

{\bf (C)} Hafmana kodam (ņemot vērā to, ka 
burtu a,b,c varbūtības ir $1/2,1/4,1/4$ arī izdodas
sasniegt entropiju. T.i. saspiešanas attiecība ir $18.75\%$ 
(tāpat kā (A) piemērā). 

{\bf (D)} Burtu pārīšu varbūtības (iegūstam, apskatot veidus, kā tos
iegūt no automāta stāvokļiem):
$\mathtt{aa} \rightarrow 1/4$,\\
$\mathtt{ab} \rightarrow 1/4$,\\
$\mathtt{ba} \rightarrow 1/8$,\\
$\mathtt{bc} \rightarrow 1/8$,\\
$\mathtt{ca} \rightarrow 1/8$,\\
$\mathtt{cc} \rightarrow 1/8$.

Entropija šiem sešiem ziņojumiem ir $2.5$ (starp $2$ un $3$ biti 
uz vienu pārīti) jeb $1.25$ uz vienu burtu. 
Tātad saspiešanas attiecības robeža būs 
$1.25 \cdot 1000000 / 8000000 = 0.156250$. 
Redzam, ka simbolu pārīšiem entropijas kodi 
(Hafmans vai aritmētiskais) saspiež labāk nekā atsevišķiem 
simboliem. No otras puses, līdz WinZip līmenim šādi tik ir grūti.  

}


\vspace{10pt}
{\bf 3.uzdevums (Markova process).}\\
Pieņemsim, ka {\em Markova process} (Attēls~\ref{fig:markov-chain})
ir jau sasniedzis stabilu stāvokli
(jau ir notikusi pietiekami gara nejauša staigāšana pa šo varbūtisko grafu, lai
simbolu proporcijas un secību vairs nenoteiktu sā\-kum\-stā\-vok\-lis).


\begin{figure}[!htb]
\center{\includegraphics[width=2.6in]{fall2020-hw01/markov-chain.png}}
\caption{\label{fig:markov-chain} Markova process.}
\end{figure}

Aplūkojam stabilā Markova procesa ģenerētu simbolu virkni $X^n$, kur
$X$ pieņem vērtības no $\{\mathtt{a}, \mathtt{b}, \mathtt{c}\}$.

\begin{enumerate}
\item Atrast saspiešanas attiecības robežu, ja simbolu virkni $X^n$ saspiež ar
aritmētisko kodu.
\item Ar datorsimulāciju izveidot virkni $X^n$ garumā $n = 10^6$.
Atrast saspiešanas attiecību, ja izmanto WinZip arhivatoru
(WinZip parasti lie\-to {\em DEFLATE} algoritmu, kas ir modificēts LZ77 un
vēl arī Hafmana kods.)
\item Atrast saspiešanas attiecības robežu ($n \rightarrow \infty$),
ja Markova virkni $X^n$ saspiež ar Hafmana kodu,
piešķirot prefiksu kodus atsevišķiem burtiem $\{\mathtt{a}, \mathtt{b}, \mathtt{c}\}$.
\item Atrast saspiešanas attiecības robežu ($n \rightarrow \infty$),
ja Markova virkni $X^n$ saspiež ar Hafmana kodu, piešķirot prefiksu kodus
burtu pāriem
$$\{\mathtt{aa}, \mathtt{ab}, \ldots, \mathtt{cc}\}.$$
(Neiespējamie burtu pāri prefiksu kokā nav jāattēlo.)
\end{enumerate}

(Šī ir parodija par 2.30.uzdevumu no \url{https://bit.ly/2Y4PKMe}, 60.lpp.)



%\vspace{10pt}
%{\bf 4.uzdevums (Berouza-Vīlera transformācija).}\\
%Pieņemsim, ka Berouza-Vīlera transformāciju veic,
%kārtojot visas cikliskās permutācijas atpakaļejošā alfabētiskā secībā:
%\begin{figure}[!htb]
%\center{\includegraphics[width=3in]{fall2020-hw01/burrows-wheeler.png}}
%\caption{\label{fig:burrows-wheeler} Berouza-Vīlera transformācija.}
%\end{figure}
%Stringa $w$ transormācija $\text{BWT}(w)$ ir
%$(\mathtt{AB})^n$ jeb
%virknīte $\mathtt{AB}$ atkārtota $n$ reizes.
%Transformācija neatzīmē iekodējamā vārda beigas.
% Kuriem $n$ tas ir iespējams?
%https://www.cs.princeton.edu/courses/archive/spr02/cs493/lec3.pdf
%http://www.cs.tau.ac.il/~haimk/adv-ds-2007/homework2-2007.pdf



\vspace{10pt}
{\bf 4.uzdevums (LZ78 apakšējais novērtējums).}\\
Bobs vēlas izveidot bitu virknīti jeb stringu, uz kura 
LZ78 algoritms uzvedas vissliktāk (virknes garums pēc arhivēšanas sanāk visgarākais).
Šai nolūkā viņš izraksta visas netukšās galīgās bitu virknītes {\em Shortlex order} secībā:
vispirms leksikogrāfiski sakārtotas viena bita virknītes, tad leksikogrāfiski sakārtotas
divu bitu virknītes utml. Rezultātā viņam iznāk sekojoša virkne:
$$\mathtt{0.1.00.01.10.11.000.001.010.011.100.101.110.111}\ldots$$
Bobs šo virkni pārstāj rakstīt tad, kad ir izrakstītas visas bitu virknītes līdz garumam $k$ ieskaitot. 
(Piemērā dota virknīte tad, ja $k=3$.) Punkti ievietoti tikai uzskatāmības dēļ; LZ78 
ievade faktiski satur tikai simbolus ``0'' un ``1''.

Boba uzrakstīto bitu stringu (bez atdalītājpunktiem) apzīmēsim ar $W_k$. 
To arī izmantojam par LZ78 algoritma ``sliktāko ievadi'', 
kas cenšas panākt, lai Lempels-Zivs retāk
izmantotu virknes no vārdnīcas, lai kods sanāktu iespējami garš.

\begin{enumerate}
\item Pierādīt, ka Boba uzrakstītās virknītes garums $|W_k| = (k-1)2^{k+1} + 2$.
\item Ar $c(W_k)$ apzīmējam to kodu skaitu, kas tiek ievietoti vārdnīcā, ja iekodē $W_k$. 
Atrast, ar ko vienāds $c(W_k)$ (formula, kas atkarīga no $k$).
\item Uzrakstīt LZ78 kodu, kas rodas iekodējot $W_3$ (sk. slaidu ``LZ78 (biti par bitiem)''). 
Vai ir iespējams, ka LZ78 kods (bitu virkne) ir garāks par ievadē saņemto bitu virkni?
\end{enumerate}

{\bf Piezīme:} Šis uzdevums izmanto LZ78 variantu, kas sāk ar tukšu vārdnīcu. 
Piemēru sk. \url{https://bit.ly/3i7HEsU}. 
(Šo nevajag jaukt ar LZW algoritmu, kurš ir līdzīgs, bet sākumā ir jau sabāzis vārdnīcā 
visu ziņojumu alfabētu - katram alfabēta burtam ir jau unikāls kods.)


% http://www-math.mit.edu/~shor/PAM/lempel_ziv_notes.pdf


\vspace{10pt}
{\bf 5.uzdevums (I-iespēja).}\\
Teksts satur $n = 2^k$ dažādus simbolus:
$$S = \{ s_1,s_2,\ldots,s_n \}.$$
To
varbūtības izkārtotas dilstošā secībā:
$$p_1 \geq \ldots \geq p_n.$$

Zināms, ka ziņojumu kopai $S$
optimālu prefiksu ko\-dē\-ju\-mu veido visas
$2^k$ vienāda garuma $k$-bitu virk\-nī\-tes.

\begin{enumerate}
\item Vai noteikti ir spēkā apgalvojums: Populārākā ziņojuma varbūtība
apmierina nevienādību:
$${\displaystyle p_1 \leq \frac{2}{2^k + 1}}?$$
\item Vai noteikti ir spēkā apgalvojums: Populārākā ziņojuma varbūtība
nepārsniedz divu retāko ziņojumu varbūtību summu: $p_1 \leq p_{n-1} + p_{n}$?
\item Vai noteikti ir spēkā apgalvojums: Populārākā ziņojuma varbūtība
nepārsniedz divkāršotu visretākā ziņojuma varbūtību: $p_1 \leq 2p_n$?
\end{enumerate}

(Šī ir parodija par 16.3-8.uzdevumu no {\em Introduction to Algorithms. Third Edition.} Cormen, Leiserson, Rivest, 2009.)

% https://ocw.mit.edu/courses/electrical-engineering-and-computer-science/6-450-principles-of-digital-communications-i-fall-2006/index.htm


%https://ocw.mit.edu/courses/electrical-engineering-and-computer-science/6-441-information-theory-spring-2016/assignments/MIT6_441S16_problem_set3.pdf



% https://ocw.mit.edu/courses/electrical-engineering-and-computer-science/6-050j-information-and-entropy-spring-2008/
%% Entropy related theory.

% https://ocw.mit.edu/courses/electrical-engineering-and-computer-science/6-441-information-theory-spring-2016/lecture-notes/
%% Lossy compression


% https://ocw.mit.edu/courses/electrical-engineering-and-computer-science/6-004-computation-structures-spring-2017/c1/
%% Some very easy exercises on coding.


% https://ocw.mit.edu/courses/electrical-engineering-and-computer-science/6-02-introduction-to-eecs-ii-digital-communication-systems-fall-2012/lecture-slides/
%% LZW encoding and similar










\end{document}





