\documentclass[jou]{apa6}
%\documentclass[11pt]{article}
\usepackage{ucs}
\usepackage[utf8x]{inputenc}
\usepackage{changepage}
\usepackage{graphicx}
\usepackage{amsmath}
\usepackage{gensymb}
\usepackage{amssymb}
\usepackage{enumerate}
\usepackage{tabularx}
\usepackage{lipsum}
\usepackage{hyperref}

\oddsidemargin 0.0in
\evensidemargin 0.0in
\textwidth 6.27in
\headheight 1.0in
\topmargin -0.1in
\headheight 0.0in
\headsep 0.0in
\textheight 9.0in

\usepackage{xcolor}

\setlength\parindent{0pt}

\newenvironment{myenv}{\begin{adjustwidth}{0.4in}{0.4in}}{\end{adjustwidth}}
\renewcommand{\abstractname}{Anotācija}
\renewcommand\refname{Atsauces}



\newcounter{alphnum}
\newenvironment{alphlist}{\begin{list}{(\Alph{alphnum})}{\usecounter{alphnum}\setlength{\leftmargin}{2.5em}} \rm}{\end{list}}


%16.3-6

\makeatletter
\let\saved@bibitem\@bibitem
\makeatother

\usepackage{bibentry}
%\usepackage{hyperref}


\title{Homework 1: Grading Criteria}
\author{Kalvis}
\affiliation{RBS}



\begin{document}
\thispagestyle{empty}

\twocolumn
{\Large Homework 1: Grading Criteria}

If you quote an external source, you can use the results found on that page, but 
you still need to restate the statement (plus a short justification, why do you 
think the statement is credible). 

You are encouraged to write answers concisely \textendash{} the best solution would contain the 
minimal set of sentences that allow the reader to check, what you have done. 
On the other hand, grading does not penalize you for writing extra stuff.
The important thing is to avoid writing nonsense that casts doubt on your understanding. 
Also try to ensure that all the essential parts for your solution are present (writing too much 
stuff may cause forgetting something important).




\vspace{2ex}
{\bf Problem 1} The question in this problem from your textbook was this:
{\em ``What conclusions can you draw from these statements?''}
One can certainly draw various 
conclusions (verify, which statements
are mutually exclusive or imply each other; find how many are true 
or false; are they consistent; or even analyze all possible ways how they can 
be true or false and prove that there are no other possibilities).
\begin{enumerate}
\item {\bf 5 points} It is clearly said, if the statements are consistent at all
(is it possible to assign any truth values to them without getting a contradiction). 
And if so, then 
\item {\bf 4 points} 1 point subtracted, if there is an incorrect conclusion in {\bf (c)}.
For example, a statement that $49$ (or $50$) statements should be true.
\item {\bf 2 points} Deficiencies in reasoning for (a), (b). Unjustified claims about (c).
\end{enumerate}


\vspace{2ex}
{\bf Problem 2} Answer (in form of the question to ask the villager) has to be given; 
it should be analyzed for correctness. 
\begin{enumerate}
\item {\bf 1 point} Only the answer is given; no analysis. 
\item {\bf 2 points} Some attempt to write analysis, but it has serious deficiences; 
most cases are not analyzed. 
\end{enumerate}




\vspace{2ex}
{\bf Problem 3} 
Answer is not difficult to get in this case; justification of it might be somewhat
longer. 
\begin{itemize}
\item {\bf 1 point} Just an answer such as
$p \vee q = (p \downarrow q) \downarrow (p \downarrow q)$, but no justification.
Writing mere answers does not get much credit, since the homeworks
are meant for technical communication. Answer (without any procedure, proof, 
algorithm to obtain similar answers in the future) 
is not sufficient communication in computer science.
\item {\bf 2 points} An answer plus a link, but still no justification. 
\item {\bf 5 points} An answer plus a link, plus a short explanation (that the formula you quoted from 
an external source can be obtained via truth tables or similar). 
\item {\bf 5 points} Expression is derived by equivalences.
\item {\bf 4 points} $\neg p \vee q \equiv (p \downarrow p) \vee q$ directly goes to answer, 
without explaining, how the $\vee$ was transformed into $\downarrow$.
\item {\bf 4 points} Formulas for $p \wedge q$, $p \vee q$, $\neg p$ are all given, 
but $p \rightarrow q$ only mentions the answer, without any relation to the previous formulas.
\item {\bf 3 points} Same as above, but with some typos in the formulas; still no explanation 
why $p \rightarrow q$ should be equal the expression.
\end{itemize}




\vspace{2ex}
{\footnotesize
{\bf Problem 4 [39, p.114]} 
Let $S = x_1y_1 + x_2y_2 + \cdots + x_ny_n$, where $x_1,x_2,\ldots,x_n$
and $y_1,y_2,\ldots,y_n$ are orderings of two different sequences
of positive real numbers, each containing $n$ elements.
\begin{enumerate}[(a)]
\item Show that $S$ takes its maximum value over all orderings of the two 
sequences when both sequences are sorted (so that the elements in each 
sequence are in nondecreasing order). 
\item Show that $S$ takes its minimum value over all orderings of the two 
sequences when one sequence is sorted into nondecreasing order and the other 
is sorted into nonincreasing order.
\end{enumerate}
}

{\em (No grading guidelines.)}


\vspace{2ex}
{\bf Problem 5} 
Fermat's last theorem not really needed here; there are much easier ways to 
prove that $\sqrt[3]{4}$ cannot be expressed as a rational number $p/q$.
\begin{itemize}
\item Just the counterexample - 1 point.
\item A counterexample (and one proof that the cube is rational) - 2 points.
\end{itemize}



\vspace{2ex}
{\bf Problem 6}
\begin{itemize}
\item {\bf 4 points} If there is a counterexample, but cases with very small sizes for $A$
and $B$ (one or two elements) are not properly explained.
\end{itemize}

\vspace{2ex}
{\footnotesize
{\bf Problem 7 [78, p.164]}
Let $x$ be a real number. Show that 
$\lfloor 3x \rfloor = \lfloor x \rfloor + \lfloor x + \frac{1}{3} \rfloor +
\lfloor x + \frac{2}{3} \rfloor$.\\
{\em Note.} By $\lfloor x \rfloor$ we denote the 
largest integer number that does not exceed $x$. 
For example $\lfloor 3.14 \rfloor = 3$, $\lfloor 17 \rfloor = 17$, 
$\lfloor -4.5 \rfloor = -5$.
}

{\em (No grading guidelines.)}

\vspace{2ex}
{\footnotesize
{\bf Problem 8 [28, p.179]}
Let $a_n$ be the $n$-th term of the sequence $1, 2,2, 3,3,3, 4,4,4,4, 5,5,5,5,5, 6,6,6,6,6,6,\ldots$
constructed by including the integer $k$ exactly $k$ times. Show that 
${\displaystyle \left\lfloor \sqrt{2n} + \frac{1}{2} \right\rfloor}$. 
}

{\em (No grading guidelines.)}

\vspace{2ex}
{\footnotesize
{\bf Problem 9 [31, p.187]}
Show that $\mathbb{Z}^{+} \times \mathbb{Z}^{+}$ is countable by showing that
the polynomial function $f\,:\,\mathbb{Z}^{+} \times \mathbb{Z}^{+} \rightarrow \mathbb{Z}^{+}$
with ${\displaystyle f(m,n) = \frac{(m+n-2)(m+n-1)}{2} + m}$ is one-to-one and onto.
}

{\em (No grading guidelines.)}

\vspace{2ex}
{\bf Problem 10}
Subitem {\bf (a)} has 1 point max credit.
\begin{itemize}
\item {\bf 1 point.} Everybody who correctly quotes the source of the code and 
computes the numbers, gets max points for this.
\end{itemize}

Subitem {\bf (b)} has 4 points max credit.
\begin{itemize}
\item {\bf 4 points.} A correct explanation, why there should be infinitely many Ulam numbers. 
(The easiest proof is from the contradiction.)
\end{itemize}





\end{document}



