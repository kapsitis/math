\documentclass[jou]{apa6}
%\documentclass[11pt]{article}
\usepackage{ucs}
\usepackage[utf8x]{inputenc}
\usepackage{changepage}
\usepackage{graphicx}
\usepackage{amsmath}
\usepackage{gensymb}
\usepackage{amssymb}
\usepackage{enumerate}
\usepackage{tabularx}
\usepackage{lipsum}
\usepackage{hyperref}

\oddsidemargin 0.0in
\evensidemargin 0.0in
\textwidth 6.27in
\headheight 1.0in
\topmargin -0.1in
\headheight 0.0in
\headsep 0.0in
\textheight 9.0in

\usepackage{xcolor}

\setlength\parindent{0pt}

\newenvironment{myenv}{\begin{adjustwidth}{0.4in}{0.4in}}{\end{adjustwidth}}
\renewcommand{\abstractname}{Anotācija}
\renewcommand\refname{Atsauces}



\newcounter{alphnum}
\newenvironment{alphlist}{\begin{list}{(\Alph{alphnum})}{\usecounter{alphnum}\setlength{\leftmargin}{2.5em}} \rm}{\end{list}}


%16.3-6

\makeatletter
\let\saved@bibitem\@bibitem
\makeatother

\usepackage{bibentry}
%\usepackage{hyperref}


\title{Discrete Structures \textendash{} Homework 1}
\author{Kalvis}
\affiliation{RBS}



\begin{document}
\maketitle
\noindent
{\bf Due Date:} January 20, 2020. Submit PDF file to the "Homework 1" folder in ORTUS.

\vspace{2ex}
{\bf Problem 1 [53, p.17]} The $n$-th statement in a list of $100$ statements is
``Exactly $n$ of the statements in this list are false.'' 
\begin{enumerate}[(a)]
\item What conclusions can you draw from these statements?
\item Answer part (a), if the $n$-th statement is ``At least $n$ of 
the statements in this list are false.''
\item Answer part (b) assuming that the list contains $99$ statements.
\end{enumerate}

\vspace{2ex}
{\bf Problem 2 [19, p.24]} Each inhabitant of a remote village always tells the 
truth or always lies. A villager will give only a "Yes" or a "No" response
to a question a tourist asks. Suppose you are a tourist visiting this area and come
to a fork in the road. One branch leads to the ruins you want to visit; the
other branch leads deep into the jungle. A villager is standing at the fork in the road. 
What one question can you ask the villager to determine which branch to take?

\vspace{2ex}
{\bf Problem 3 [55, p.39]} 
Find a compound proposition logically equivalent to $p \rightarrow q$ using 
only the logical operator $\downarrow$.\\
{\em Note.} Operator $\downarrow$ is named {\bf Peirce arrow} (or NOR). 
Propositon $p \downarrow q$ is true when both $p$ and $q$ are false, and it is 
false otherwise. It is a shorthand: $p \downarrow q := \neg(p \vee q)$. 

\vspace{2ex}
{\bf Problem 4 [39, p.114]} 
Let $S = x_1y_1 + x_2y_2 + \cdots + x_ny_n$, where $x_1,x_2,\ldots,x_n$
and $y_1,y_2,\ldots,y_n$ are orderings of two different sequences
of positive real numbers, each containing $n$ elements.
\begin{enumerate}[(a)]
\item Show that $S$ takes its maximum value over all orderings of the two 
sequences when both sequences are sorted (so that the elements in each 
sequence are in nondecreasing order). 
\item Show that $S$ takes its minimum value over all orderings of the two 
sequences when one sequence is sorted into nondecreasing order and the other 
is sorted into nonincreasing order.
\end{enumerate}

\vspace{2ex}
{\bf Problem 5 [39, p.119]} 
Prove or disprove that if $x^2$ is irrational, then $x^3$ is irrational. 

\vspace{2ex}
{\bf Problem 6 [43, p.133]}
Prove or disprove that if $A$ and $B$ are sets, then 
$\mathcal{P}(A \times B) = \mathcal{P}(A) \times \mathcal{P}(B)$.

\vspace{2ex}
{\bf Problem 7 [78, p.164]}
Let $x$ be a real number. Show that 
$\lfloor 3x \rfloor = \lfloor x \rfloor + \lfloor x + \frac{1}{3} \rfloor +
\lfloor x + \frac{2}{3} \rfloor$.\\
{\color{red} {\em Note.} By $\lfloor x \rfloor$ we denote the 
largest integer number that does not exceed $x$. 
For example $\lfloor 3.14 \rfloor = 3$, $\lfloor 17 \rfloor = 17$, 
$\lfloor -4.5 \rfloor = -5$. }

\vspace{2ex}
{\bf Problem 8 [28, p.179]}
Let $a_n$ be the $n$-th term of the sequence $1, 2,2, 3,3,3, 4,4,4,4, 5,5,5,5,5, 6,6,6,6,6,6,\ldots$
constructed by including the integer $k$ exactly $k$ times. Show that 
${\displaystyle \left\lfloor \sqrt{2n} + \frac{1}{2} \right\rfloor}$. 


\vspace{2ex}
{\bf Problem 9 [31, p.187]}
Show that $\mathbb{Z}^{+} \times \mathbb{Z}^{+}$ is countable by showing that
the polynomial function $f\,:\,\mathbb{Z}^{+} \times \mathbb{Z}^{+} \rightarrow \mathbb{Z}^{+}$
with $\color{red} {\displaystyle f(m,n) = \frac{(m+n-2)(m+n-1)}{2} + m}$ is one-to-one and onto.

\vspace{2ex}
{\bf Problem 10 [28, p.198]}
We define the {\bf Ulam numbers} by setting $u_1 = 1$ and $u_2 = 2$. 
Furthermore, after determining whether the integers less than $n$ are 
Ulam numbers, we set $n$ equal to the next Ulam number, if it can be written uniquely as the
sum of two different Ulam numbers. Note that $u_3 = 3$, $u_4 = 4$, 
$u_5 = 6$, and $u_6 = 8$. 
\begin{enumerate}[(a)]
\item Find these five consecutive Ulam numbers: $u_{2020},u_{2021},u_{2022},u_{2023},u_{2024}$. 
\item Prove that there are infinitely many Ulam numbers.
\end{enumerate}






\end{document}



