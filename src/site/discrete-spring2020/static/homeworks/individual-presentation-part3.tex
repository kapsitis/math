\documentclass[jou]{apa6}
%\documentclass[11pt]{article}
\usepackage{ucs}
\usepackage[utf8x]{inputenc}
\usepackage{changepage}
\usepackage{graphicx}
\usepackage{amsmath}
\usepackage{gensymb}
\usepackage{amssymb}
\usepackage{enumerate}
\usepackage{tabularx}
\usepackage{lipsum}
\usepackage{hyperref}

\oddsidemargin 0.0in
\evensidemargin 0.0in
\textwidth 6.27in
\headheight 1.0in
\topmargin -0.1in
\headheight 0.0in
\headsep 0.0in
\textheight 9.0in

\usepackage{xcolor}

\setlength\parindent{0pt}

\newenvironment{myenv}{\begin{adjustwidth}{0.4in}{0.4in}}{\end{adjustwidth}}
\renewcommand{\abstractname}{Anotācija}
\renewcommand\refname{Atsauces}



\newcounter{alphnum}
\newenvironment{alphlist}{\begin{list}{(\Alph{alphnum})}{\usecounter{alphnum}\setlength{\leftmargin}{2.5em}} \rm}{\end{list}}


%16.3-6

\makeatletter
\let\saved@bibitem\@bibitem
\makeatother

\usepackage{bibentry}
%\usepackage{hyperref}


\title{Homework 1: Grading Criteria}
\author{Kalvis}
\affiliation{RBS}



\begin{document}
\thispagestyle{empty}

\twocolumn

\begin{center}
{\Large RBS: Discrete Structures}\\
{\Large RBS: Individual Presentations Part 3: Final Presentation}
\end{center}


\vspace{20pt}
{\large Section 1: Writing 4 Test Questions}

Once you know the objectives of what your presentation should have, 
you can start making tests about it. 

In general you can have three different types of performance objectives:
\begin{itemize}
\item {\bf Conceptual knowledge:} 
If you want to check that knowledge, you can use all the arsenal of mathematical testing: 
binary (true/false) questions, matching or multiple choice questions. Also consider
write-in answers (popular in language learning) or short-answer questions, where the answer is a number or
a short expression.
\item {\bf Covert/Implicit Procedural Knowledge:} Ability to perform decisions and procedures, which 
are not observable from the outside in much detail; they happen in the head of the person who performs the test. 
Typical examples are troubleshooting, or decision making activities.\\
If you want to check that knowledge, you can develop some scenarios, where you introduce
some situation. Then you ask the performer to choose an appropriate response in 
the given situation or to correct some mistake or fix some bug.
\item {\bf Explicit Procedural Knowledge:} Ability of your listeners to perform certain well-defined sequences of steps.\\
If you want to check that knowledge, you can also
consider behavior checklists or effectiveness checklists.
\end{itemize}

For the most part you might need conceptual knowledge, but in some cases you want the
learners to perform certain procedural steps as well. 
Below we discuss all the considerations as you do the test questions. 

{\bf (A) Binary (true/false) questions.} All questions with answers Yes/No or True/False are in this category. 
They are usually easy to understand for the test-taker. On the other hand, the questions should reflect
the material really well; too often it is possible to guess the right answer (just on basis of how the question 
is asked).\\
{\em Note.} To make such binary questions harder to guess, 
introduce 2-3 mutually related questions with Yes/No answers. I.e. they are not quite binary in this case; 
the full answer is actually 2-3 Yes/No values.

\vspace{6pt}
{\bf (B) Matching questions.} Ask the learners to match an item in one column with an item in another column. 
Another variety is \textendash{} ask the learners to order items according to some criterion. 
Usually matching/ordering of 3-5 items works best.\\
{\em Note.} These questions are especially useful when you have some order or pairing relationship between the
items. They might look artificial, if the knowledge that you test is not related to item pairing or ordering.

\vspace{6pt}
{\bf (C) Multiple choice single answer.} It is given that one (and only one) answer is true. 
The test taker has to pick this answer from 3-5 alternatives. The alternatives that are not correct are 
named {\em distractors}.\\
{\em Note.} Wording of your questions becomes very important in this case. Distractors should not be 
something trivially false (or things like "None of the above, etc."). Usually the alternatives would 
be shuffled; they should be independent. Having less than 3 alternatives is not acceptable (it is binary question 
in this case). Also having more than 5 alternatives is not good: Too much effort to read them all.

\vspace{6pt}
{\bf (D) Multiple choice multiple answers.} In this case you tell in advance how many responses are true, and 
ask the test taker to check ALL the correct responses. Once again you can have 3-5 alternatives (more than that
is difficult to read). And you can have 2-4 correct responses; the rest are distractors.\\
{\em Note.} Wording of your questions is very important in this case as well (a person who does not
understand the subject should not be able to eliminate the distractors).\\
On the other hand such questions are typically harder than MC single answer, since 
a correct answer in this case means that ALL items have been selected correctly.
For example, if you have 5 alternatives and 2 of them are true, then there are ${\displaystyle {5 \choose 2} = 10}$
different possibilities to pick the answer; only one of them gives a credit.

\vspace{6pt}
{\bf (E) Completion question.} You type a text, leave 1-3 blanks, and ask the test taker to fill in those
blanks. It is popular in language training (inserting correct word forms, endings, propositions, articles and so on), 
but there is no reason why it could not be used in mathematics as well. For example, an expression, where some
blanks have to be filled in.\\
{\em Note.} Completion questions are good in situations, where there are limited range of possible values
to enter (E.g. you can enter any English article - "a", "an", "the" or leave it blank. Or enter a single digit from 0 to 9 
or similar.) They are not good, if the omitted places can be filled in many different ways (because they become 
hard to grade).

\vspace{6pt}
{\bf (F) Short answer questions.} These are similar to the completion questions, but the blank is near
the end of the question. This style is extensively used by \url{https://www.khanacademy.org/}, and also 
by our weekly quizzes and commercial testing services like \url{https://www.uzdevumi.lv/}.\\
{\em Note.} These test questions are easy to create - a typical question in IT, Math, Physics or Chemistry has
a short answer. On the other hand, exam creators should be very careful to define the right syntax of the answers.\\
For example, if the answer is a real number, we should specify, how it should be rounded; 
if the answer is a line segment $AB$, can it also be written as $BA$, etc.


\vspace{6pt}
{\bf (G) Best response.} You may have questions with various (partial) credits \textendash{} in particular
if they are multiple choice. 
Some types of conceptual knowledge does not have black-and-white answers; there may be shades of gray. 
If you have 2-3 related situations with several answer alternatives (ranging from the best to the outright wrong), 
then you can consider a ``Best response'' question(s). They are frequantly used to compute psychology 
evaluations. In conceptual knowledge about IT (in particular, 
some real-life issues like IT security, auditing, software engineering practices), you might also have 
such questions.\\
{\em Note.} There may be many people (with good common-sense), who can answer
such questions with ease: They will understand precisely, what is expected from them. 
Meanwhile, they do not have the necessary hard skills to implement their good intentions and 
common-sense behaviors.


\vspace{6pt}
{\bf (H) Open-ended (essay) questions.} In this case the learner can respond with one text paragraph - several 
sentences explaining his understanding.\\ 
{\em Note.} This question has to be graded by humans (instructor in our case); Some subject areas need 
to include the learner's opinion, interpretation. It can explain all the ``why'' and ``how'' types of things, 
which cannot be tested by the above methods. 

\vspace{20pt}
{\large Section 2: Finalizing your Presentation}

\begin{itemize}
\item Review the recommendations: See ``Guidelines: Part 2'' on
\url{http://linen-tracer-682.appspot.com/discrete/assignments.html}. 
\item Pay attention to motivation, objectives near the end of your 
presentation (stuff that your quiz questions will test)
and summary near the end.
\item Reorder your slides so that they serve a single story flow. 
\item Review the instructor's feedback on your draft presentation. 
\item Add at least a few textual notes to all your slides; the notes
should provide some context to the images seen in the slide. 
The Notes are NOT the transcript for your speech.
\item Focus on 1-2 main slides or diagrams that explain the essence 
of your topic. In most presentations you can have the most important
content in very few slides; the remaining slides are just a ``framing'' 
for these essential slides.
\item After your presentation is ready, try to test it: Does it 
fit a 10 minute presentation? Should you refactor something; hide some
slides in order to make your story more focused?
\item Your final presentation can contain {\em scaffolding}: 
extra links to info sources, hidden slides or extra examples. 
There is no need to ``sweep it under the rug''. The main 
goal is to create sufficent presentation for a successful performance.
\end{itemize}




\end{document}



