\documentclass[jou]{apa6}

\usepackage[american]{babel}

\usepackage{csquotes}
\usepackage[style=apa,sortcites=true,sorting=nyt,backend=biber]{biblatex}
\DeclareLanguageMapping{american}{american-apa}
\addbibresource{bibliography.bib}


%%%%%%%%%%%%%%%%%%%%%%%%%%%%%%%%%%%%%%%%
%% Discrete Structures
%% The start of RBS stuff
%%%%%%%%%%%%%%%%%%%%%%%%%%%%%%%%%%%%%%%%

% Working internal and external links in PDF
\usepackage{hyperref}
% Extra math symbols in LaTeX
\usepackage{amsmath}
\usepackage{gensymb}
\usepackage{amssymb}
% Enumerations with (a), (b), etc.
\usepackage{enumerate}

\let\OLDitemize\itemize
\renewcommand\itemize{\OLDitemize\addtolength{\itemsep}{-6pt}}

\usepackage{etoolbox}
\makeatletter
\preto{\@verbatim}{\topsep=3pt \partopsep=3pt }
\makeatother

% These sizes redefine APA for A4 paper size
\oddsidemargin 0.0in
\evensidemargin 0.0in
\textwidth 6.27in
\headheight 1.0in
\topmargin -24pt
\headheight 12pt
\headsep 12pt
\textheight 9.19in



\title{Discrete Structures (W2): Sets and Predicates}
\author{Kalvis}
\affiliation{RBS}

\leftheader{Discrete Structures (W2)}

\abstract{
You will do more experiments with Boolean expressions and 
start to prove propositional tautologies in Coq.
You will also formulate simple "human-oriented" proofs
about integer arithmetic and other topics.
}

\keywords{Coq, Boolean expressions, Syntax trees, tautologies, proofs.}

\begin{document}
\maketitle


{\bf (W2.1)} Try Boolean short-circuit evaluation on Python: 
Define two functions: {\tt f(x)} and {\tt g(x)}. Each of them 
prints a message and then returns value {\tt False}. For example:
\begin{verbatim}
def f():
    print('Running f(x)')
	return False
\end{verbatim}
After that evaluate this statment:
\begin{verbatim}
if f(x) and g(x):
    print('Unreachable statement')
\end{verbatim}
There should be also eager 

%{\bf (W2.3)} Verify identities on set operations.
%
%{\bf (W2.4)} Indexed operations $\vee_{i=1}^{k}$, $\wedge_{i=1}^{k}$, 
%$\cup_{i=1}^{k}$, $\cap_{i=1}^{k}$, $\sum_{i=1}^{k}$, 
%$\prod_{i=1}^{k}$.

{\bf (W2.2)} Compare Boolean expressions in Coq, using prefix and infix notation.
\begin{verbatim}
Require Import Bool.
Eval compute in orb true false.
Eval compute in true || false.
Eval compute in andb true false.
Eval compute in true && false.
Eval compute in negb false.
Eval compute in if true then 3 else 4.
Definition a := true.
Eval compute in orb a (negb a).

Definition eMiddle (a:bool): bool :=
  orb a (negb a).
Eval compute in eMiddle true.

Definition Nor (a b:bool): bool :=
  negb (orb a b).
Eval compute in Nor true true.  
\end{verbatim}


{\bf (W2.3)} Use precedence and associativity to draw abstract syntax trees.
Use boolean commands "orb", "andb", "implb" and "negb" to 
compute in prefix notation this function: 
$$E(a,b,c) := a \vee \neg(b \wedge (a \rightarrow c)).$$
Draw the syntax tree of this operation. Its leaves
are variables $a,b,c$ (and also constants, if they are present in the expression). 
Inner nodes are all the Boolean operations with 1 or 2 variables. 
So, in this tree every inner node has 1 or 2 children. 

{\bf (W2.4)} Read a definition of a DNF (Disjunctive Normal Form). 
Create a DNF for a 3-argument Boolean function. 
You can pick the $8$ truth falues at random.

{\bf (W2.5)} 
Prove that for any odd integer $k$, the square $k^2$ 
gives remainder $1$, when divided by $8$.

{\bf (W2.6)} 
Prove that a positive integer $n$ has odd number of positive divisors 
(including $1$ and $n$ itself) if and only if $n$ is a full square - can be expressed as $k^2$.

{\bf (W2.7)} Prove that for any prime number $p$, the square root  
$\sqrt{p}$ is irrational.

{\bf (W2.8)}
Prove that for any real number $x$ its rounded value to the nearest tenth 
(rounding to one decimal place) is equal to $\frac{1}{10}\lfloor 10x + 0.5 \rfloor$.

{\bf (W2.9)}
Prove that there is a function  $f:(-\pi;\pi) \rightarrow \mathbb{R}$ 
mapping interval $(-\pi;\pi)$ to $\mathbb{R}$ such that every real number
$y$ has exactly one $x$ such that $f(x)=y$.

{\bf (W2.10)}
Prove that it is impossible to enumerate all subsets of natural numbers with natural numbers.
(Natural numbers are all integers $\geq 0$).

{\bf (W2.11)}
Try out the examples in \url{https://bit.ly/2sfIrUK}. 
Discuss, if you are unsure, how to write proofs. 
Visit tutorials \url{https://bit.ly/37Zjso0}, if you need more inspiration. 

{\bf (W2.12)}
Try out the Coq assignment on \url{https://bit.ly/37Zjso0}
(the link "Assignment about Coq (1 of 5)"). 

\newpage

\section{Some Answers}

{\bf (W2.1)}

Full Python program looks like this:
\begin{verbatim}
def f(): print('Runs f()'); return False
def g(): print('Runs g()'); return False
def h(): print('Runs h()'); return True

## Prints 'Runs f()':
if f() and g(): print('Unreachable')
## Prints 'Runs f()', 'Runs g()':
if f() & g(): print('Unreachable')
## Prints 'Runs h()' and 'Hi':
if h() or g(): print('Hi')
## Prints 'Runs h()', 'Runs g()' and 'Hi'
if h() | g(): print('Hi')
\end{verbatim}

{\bf (W2.5)} Assume that $n$ is odd. Represent it as 
$n = 2k+1$. Then 
$$n^2 = (2k+1)^2 = 4k^2 +4k + 1 = 4k(k+1) + 1.$$
Consider two cases:\\
(1) If $k$ is odd, then $k+1$ is even and 
$4\times (k+1)$ is divisible by $8$.\\
(2) If $k$ is even, then $4 \times k$ is divisible by 
$8$.\\
In either case $4k(k+1)$ is divisible by $8$. 
Therefore $4k(k+1) + 1$ always gives remainder $1$ when 
divided by $8$. 

{\bf (W2.6)} Biconditional (if-and-only-if) statement contains two 
implications:\\
(1) First, assume that $n$ is a full square: $n = k^2$. We need to 
prove that it has odd number of positive divisors.
For any divisor of $n$: $d < \sqrt{n}=k$ there exists another divisor 
$d' = n/d$ which is bigger than $\sqrt{n}$ \textendash{} you can 
divide all divisors in pairs. In addition, $n$ has a divisor $k = \sqrt{n}$
that is paired with itself.\\
Example: The number $36=6^2$ has these divisor pairs:
$$(1;36),\;(2;18),\;(3;12),\;(4;9),\;(6;6).$$
Since the divisor $\sqrt{n}$ is paired with itself, there is
an odd number of positive divisors.\\
(2) Secondly, assume that $n$ has odd number of divisors; we should prove
that it is a full square.\\
In fact, we will prove the counterpositive: If $n$ is {\bf not} a full square, 
then it has even number of divisors. To see this, we also arrange divisors
of $n$ in pairs \textendash{} as before, one of them is less than $\sqrt{n}$, another
one is bigger than $\sqrt{n}$. Only this time the number $\sqrt{n}$ itself
does not count as a divisor of $n$, because it is not an integer.\\
Example: The number $12 \approx (3.4641\ldots)^2$ has these divisor pairs:
$$(1;12),\;(2;6),\;(3;4).$$

\end{document}


