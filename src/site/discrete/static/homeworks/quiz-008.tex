\documentclass[jou]{apa6}

\usepackage[american]{babel}

\usepackage{csquotes}
\usepackage[style=apa,sortcites=true,sorting=nyt,backend=biber]{biblatex}
\DeclareLanguageMapping{american}{american-apa}
\addbibresource{bibliography.bib}


%%%%%%%%%%%%%%%%%%%%%%%%%%%%%%%%%%%%%%%%
%% Discrete Structures
%% The start of RBS stuff
%%%%%%%%%%%%%%%%%%%%%%%%%%%%%%%%%%%%%%%%

% Working internal and external links in PDF
\usepackage{hyperref}
% Extra math symbols in LaTeX
\usepackage{amsmath}
\usepackage{gensymb}
\usepackage{amssymb}
% Enumerations with (a), (b), etc.
\usepackage{enumerate}

\let\OLDitemize\itemize
\renewcommand\itemize{\OLDitemize\addtolength{\itemsep}{-6pt}}

\usepackage{etoolbox}
\makeatletter
\preto{\@verbatim}{\topsep=3pt \partopsep=3pt }
\makeatother

% These sizes redefine APA for A4 paper size
\oddsidemargin 0.0in
\evensidemargin 0.0in
\textwidth 6.27in
\headheight 1.0in
\topmargin -24pt
\headheight 12pt
\headsep 12pt
\textheight 9.19in



\title{Sample Quiz 8}
\author{Discrete Structures, Spring 2020}
\affiliation{RBS}

\leftheader{Discrete Sample Quiz 8}

\abstract{%
}

%\keywords{}

\setlength\parindent{0pt}

\begin{document}

\thispagestyle{empty}

\twocolumn
{\Large Discrete Quiz 8}

\vspace{10pt}
{\bf Question 1} 
Two chess queens are placed on two different places in a $4 \times 4$ chess-board.
Assume that all the ${16 \choose 2}$ possibilities how they are placed have equal probabilities.
Find a probability that one queen {\em attacks} the other. (Two queens attack each other, if
they are located on the same horizontal, vertical or diagonal).

Write your answer as a rational fraction: {\tt P/Q}

\vspace{10pt}
{\bf Question 2} Assume that you are generating 10-bit sequences (a string of 0's and 1's). 
All the $2^{10}$ sequences have equal probabilities. 
Find the probability of an event that the 10-bit sequence does NOT contain two consecutive 0's anywhere.

Write your answer as a rational fraction: {\tt P/Q}

\vspace{10pt}
{\bf Question 3}
Two players have 4 cubic dices. Instead of the usual numbers, their faces have the following numbers on their faces:\\
{\bf Dice A:} 4, 4, 4, 4, 0, 0;\\
{\bf Dice B:} 3, 3, 3, 3, 3, 3;\\
{\bf Dice C:} 6, 6, 2, 2, 2, 2;\\
{\bf Dice D:} 5, 5, 5, 1, 1, 1.

In a single round Players Alice and Bob randomly select two of the dices (with equal probabilities they can 
select any of the six pairs - $(A,B)$, or $(A,C)$, or $(A,D)$, or $(B,C)$, or $(B,D)$, or $(C,D)$).
There can be three outcomes:\\
{\bf Outcome 1:} They have selected "opposite dices" - pairs $(A,C)$ or $(B,D)$. In this case the payoff
is zero (nobody pays anything to the other).\\
{\bf Outcome 2:} The dices win in the "clockwise manner" ($A>B$ or $B>C$ or $C>D$ or $D>A$) - then Alice
wins 1 euro.\\
{\bf Outcome 3:} The dices win in the "counter-clockwise manner" ($B>A$ or $C>B$ or $D>C$ or $A>D$) - then Bob
wins 1 euro.\\
({\em Note.} The expression $A>B$ means that the number that rolled out on the the dice $A$ was larger than the number on dice $B$; 
but $B>A$ denotes the opposite event.)

Find the expected value - how much money Alice is expected to win in a single round of such a game.

Write your answer as a rational fraction: {\tt P/Q}\\
For example, if the expected win for Alice is 0.10 EUR, then write {\tt 1/10}. If Alice is expected on average to lose 0.10 EUR
per one round of this game, then write {\tt -1/10}. 




\vspace{10pt}
{\bf Question 4} For every year we count the number of Friday's that fall on the 13th date of some month
(such as Friday, March 13, 2020). Denote this count by $X$ \textendash{} it is your random variable.
Find the expected value and the variance of $X$. Round them to the nearest thousandth. 

Write your answer as two comma-separated numbers: {\tt D.DDD,D.DDD}.

\vspace{10pt}
{\bf Question 5} What is the
probability that a randomly chosen positive integer between $1$ and $600$ is not divisible 
by either $6$ or $10$?

Write your answer as a rational fraction: {\tt P/Q}


\vspace{10pt}
{\bf Question 6} A chip factory Intel adds one toy animal to every bag of the chips they produce.
There are three sorts of animals - Aligators, Bears or Cats (and they have equal probability of 1/3). 
Find the expected number of the chip bags one needs to purchase to collect all three animals. 

Write your answer as a rational fraction: {\tt P/Q}

\vspace{10pt}
{\bf Question 7} You create a random bit string of length five (all $32$ bit strings are equally probable). Consider
these events:\\
$E_1$: the bit string chosen begins with $1$;\\
$E_3$: the bit string chosen has exactly three $1$’s.\\
{\bf (A)} Find $p(E_1 \,\mid\, E_3)$.\\
{\bf (B)} Find $p(E_3 \,\mid\, E_1)$.

Write your answer as a comma-separated rational fractions {\tt P1/Q1,P2/Q2}


\end{document}

