\documentclass[jou]{apa6}
%\documentclass[11pt]{article}
\usepackage{ucs}
\usepackage[utf8x]{inputenc}
\usepackage{changepage}
\usepackage{graphicx}
\usepackage{amsmath}
\usepackage{gensymb}
\usepackage{amssymb}
\usepackage{enumerate}
\usepackage{tabularx}
\usepackage{lipsum}
\usepackage{hyperref}

\oddsidemargin 0.0in
\evensidemargin 0.0in
\textwidth 6.27in
\headheight 1.0in
\topmargin -0.1in
\headheight 0.0in
\headsep 0.0in
\textheight 9.0in

\usepackage{xcolor}

\setlength\parindent{0pt}

\newenvironment{myenv}{\begin{adjustwidth}{0.4in}{0.4in}}{\end{adjustwidth}}
\renewcommand{\abstractname}{Anotācija}
\renewcommand\refname{Atsauces}



\newcounter{alphnum}
\newenvironment{alphlist}{\begin{list}{(\Alph{alphnum})}{\usecounter{alphnum}\setlength{\leftmargin}{2.5em}} \rm}{\end{list}}


%16.3-6

\makeatletter
\let\saved@bibitem\@bibitem
\makeatother

\usepackage{bibentry}
%\usepackage{hyperref}


\title{Homework 1: Grading Criteria}
\author{Kalvis}
\affiliation{RBS}



\begin{document}
\thispagestyle{empty}

\twocolumn
{\Large Homework 2: Grading Criteria}


\vspace{2ex}
{\bf General Guidelines:}

\begin{enumerate}[(A)]
\item Every problem is graded with $0$, $1$, $2$, $3$, $4$ or $5$ points.
(We do not differentiate points by the ``difficulty'' of the problem. 
Sometimes difficulty is subjective. Moreover, 
solving a difficult problem and leaving and easy one unsolved is not any better
than doing things in the opposite way.)
\item  {\bf $0$ points} is typically given for unsolved problems or some statements that are very far from 
what is needed.\\
{\bf $1$ point} is typically given for some useful comments, or easy examples, which 
are helpful, but do not address the core difficulty of the problem.\\
{\bf $2$ points} are given for a completed part of the solution, which could be easier
than the omitted part.\\
{\bf $3$ points} are given for those cases, where the solved part is slightly harder than 
the omitted part.\\
{\bf $4$ points} are given for nearly complete solution with minor deficiencies.\\
{\bf $5$ points} are given for completely solved problems (in particular, all the necessary 
parts of the solution should be present \textendash{} depending on the problem's formulation. See {\bf (F)}).
\item Every student has a solution that is considered unique and is discussed individually with him or her.
We should never start ``comparisons'' (why one student's solution got $x$ points, but another 
student's solution got $y$ points).
\item If you refer to an external source:
\begin{itemize}
\item Add a bibliography reference or a Web URL (failing to add a reference is considered academic dishonesty).
\item Quote the complete statement that you need for your homework problem.
\item Add a short justification, why do you 
believe the statement is credible. For example, add a single sentence 
summary, how it was proven. You do not need to copy-paste the
full proof, since this is not a typesetting class. But you need to 
validate your sources. There are many false claims in the Internet; 
you need to develop your own methods to discern the truth. It can 
usually be done without repeating the work of other people.
\end{itemize}
\item You are encouraged to write answers concisely \textendash{} the best solution would contain the 
minimal set of sentences that allow the reader to check, what you have done. 
Grading does not penalize you for writing redundant stuff, but
please avoid writing false claims that cast doubt on the correctness of your solution.
\item Read the problem statement carefully; what is given and what has to be provided. 
The essential parts for your solution will depend on how the problem is stated.
Sometimes one counterexample is enough; in other cases there has to be a full proof about all 
situations. For some definitions you need to check multiple conditions. 
Or prove both directions of a biconditional statement.
\end{enumerate}




\vspace{2ex}
{\bf Problem 1} 
\begin{enumerate}
\item {\bf 1 point:} Some initial ideas (for example, that the possibility to cut into 
rectangles should depend on the location of the little square). Or a ``bare'' answer without
any justification. 
\item {\bf 2 points:} One example is shown (for example, how to cut, if the square $C3$ is removed).
\item {\bf 4 points:} A positive example and a coloring. But lacking 
\item {\bf 5 points:} A positive example and some negative one (with coloring or other justification). 
Symmetry is used to reduce the cases. 
\end{enumerate}


\vspace{2ex}
{\bf Problem 2} 
\begin{enumerate}
\item {\bf 2 points:} Example of a bijection $(0;1) \rightarrow \mathbb{R}$, but not proven that it is a bijection.
\item {\bf 3 points:} Example of a bijection $(0;1) \rightarrow \mathbb{R}$ with a proof.
\item {\bf 3 points:} Example of a bijection $[0;1) \rightarrow \mathbb{R}$ (if the construction of the sequence was described).
\item {\bf 3 points:} Injective constructions (instead of bijective ones) together with 
the theorem by Schr\"{o}der-Bernstein (bijection between $A$ and $B$ exists
iff there is an injection from $A$ to $B$ and also from $B$ to $A$). This is easier than to construct explicitly.
\item {\bf 1-2 points:} Example of a bijection $[0;1) \rightarrow \mathbb{R}$ written vaguely; no clear definition, which argument $x$ is mapped to which value.
\end{enumerate}

\vspace{2ex}
{\bf Problem 3} 
\begin{enumerate}
\item {\bf 2 points:} Incomplete constructions, where the continuity of real valued functions $f$ is not used. 
\item {\bf 4-5 points:} Correctly written constructions that encode $f$ on all rational numbers. (Sometimes 
special cases such as "9"-period are not handled or explained; but mostly the constructions were sufficient.)
\end{enumerate}


\vspace{2ex}
{\bf Problem 4} 
\begin{enumerate}
\item {\bf 5 points:} Reasoning that 3 nested loops (of length approximately $n$) would 
cause $O(n^3)$ worst-case time.
\end{enumerate}


\vspace{2ex}
{\bf Problem 5} 
\begin{enumerate}
\item {\bf 5 points:} Verbal algorithm or pseudocode that mentions different treatment for cases when $n$ is even and odd.
\item {\bf 4 points:} Verbal algorithm or code that solves the problem (finds min and max), but is not optimal as was required. 
For example, uses $2n$ or $2n-2$ comparisons.
\item {\bf 3 points:} Algorithms with some deficiencies (for example comparison switched to another side; computing maximum rather than minimum), etc. 
\end{enumerate}


\vspace{2ex}
{\bf Problem 5} 
\begin{enumerate}
\item {\bf 3 points:} Some justification, why we cannot do better than $3n/2-2$. 
\item {\bf 2 points:} Some justification, why ``expert'' (who knows everything about the coins) cannot demonstrate this faster than in $n-1$ steps.
\item {\bf minus 1 point:} Justification, why cannot do better is incomplete.
\item {\bf minus 1 more point:} Estimate for the ``expert'' (in part (B)) is incomplete.
\end{enumerate}




\vspace{2ex}
{\bf Problem 7} 
\begin{enumerate} 
\item {\bf 5 points:} Ternary search described; cases when $n$ is not divisible by $3$ and other non-standard
situations handled. No points subtracted, if the specified number of comparisons wastes a different constant. 
You can have $2\log_3 n$ or $4\log_3 n$ or similar. But it has to be correct and it has to 
equal $O(\log_2 n)$. 
\item {\bf 4 points:} $mid1$ and $mid2$ used, but not defined.
\end{enumerate}


\end{document}



