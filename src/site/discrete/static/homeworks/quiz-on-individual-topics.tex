\documentclass[jou]{apa6}

\usepackage[american]{babel}

\usepackage{csquotes}
\usepackage[style=apa,sortcites=true,sorting=nyt,backend=biber]{biblatex}
\DeclareLanguageMapping{american}{american-apa}
\addbibresource{bibliography.bib}


%%%%%%%%%%%%%%%%%%%%%%%%%%%%%%%%%%%%%%%%
%% Discrete Structures
%% The start of RBS stuff
%%%%%%%%%%%%%%%%%%%%%%%%%%%%%%%%%%%%%%%%

% Working internal and external links in PDF
\usepackage{hyperref}
% Extra math symbols in LaTeX
\usepackage{amsmath}
\usepackage{gensymb}
\usepackage{amssymb}
% Enumerations with (a), (b), etc.
\usepackage{enumerate}
\usepackage[framemethod=TikZ]{mdframed}
\usepackage{xcolor}
\usepackage{graphicx}
\usepackage[justification=centering]{caption}
\usepackage{fancyvrb}

\let\OLDitemize\itemize
\renewcommand\itemize{\OLDitemize\addtolength{\itemsep}{-6pt}}

\usepackage{etoolbox}
\makeatletter
\preto{\@verbatim}{\topsep=3pt \partopsep=3pt }
\makeatother

% These sizes redefine APA for A4 paper size
\oddsidemargin 0.0in
\evensidemargin 0.0in
\textwidth 6.27in
\headheight 1.0in
%\topmargin -24pt
\topmargin -32pt
\headheight 12pt
\headsep 12pt
%\textheight 9.19in
\textheight 9.35in


\title{Sample Quiz 8}
\author{Discrete Structures, Spring 2020}
\affiliation{RBS}

\leftheader{Discrete Sample Quiz 8}

\abstract{%
}

%\keywords{}

\setlength\parindent{0pt}

\begin{document}

\thispagestyle{empty}

\twocolumn
{\Large Quiz on Everyone's Individual Topics}

\vspace{4pt}
{\bf Question 1.} A hacker wants to 
use {\em Arithmetic coding}
that would send a virus to the client's computer and unpack itself there.
His favorite virus uses this RNS sequence: $\textcolor{blue}{\mathtt{GACGU\$}}$, 
where $\textcolor{blue}{\mathtt{A}},\textcolor{blue}{\mathtt{C}},\textcolor{blue}{\mathtt{G}},\textcolor{blue}{\mathtt{U}}$ are four nucleobases (the useful data payload), 
but the symbol $\textcolor{blue}{\$}$ is used just once to mark 
the end of the RNS string. 

He uses the following {\em a priori} frequencies for the symbols: 

\begin{tabular}{ccccc}
$\textcolor{blue}{\mathtt{A}}$ & $\textcolor{blue}{\mathtt{C}}$ & $\textcolor{blue}{\mathtt{G}}$ & $\textcolor{blue}{\mathtt{U}}$ & $\textcolor{blue}{\mathtt{\$}}$ \\ 
$30\%$ & $10\%$ & $30\%$ & $20\%$ & $10\%$ \\
\end{tabular}


\vspace{4pt}
He starts with the half-closed line segment $S_0 = [0;1)$ and at every step (for all $i = 0,1,2,3,4,5$) creates the 
next segment $S_{i+1}$ from $S_i$ by dividing $S_i$ into five parts of lengths proportional 
to the frequencies (the proportions of subdivisions are $3\!\!:\!\!1\!\!:\!\!3\!\!:\!\!2\!\!:\!\!1$). Then $S_{i+1}$ is the
subdivision of the previous $S_i$ corresponding to the newly encoded character.
After encoding all six characters in the virus message $\textcolor{blue}{\mathtt{GACGU\$}}$, the hacker
gets the segment $S_6$.

\begin{figure}[!htb]
\center{\includegraphics[width=3in]{arithmetic-coding.png}}
\caption{\label{fig:arithmetic-coding} Building arithmetic code}
\end{figure}


Find the binary fraction $\beta$ belonging to $S_6$.\\
\textcolor{teal}{\em Select one answer.}

{\bf (A)} $\beta = 0.011011101100011_2$\\
{\bf (B)} $\beta = 0.011011101100110_2$\\
{\bf (C)} $\beta = 0.011011101101001_2$\\
{\bf (D)} $\beta = 0.011011101101100_2$\\
{\bf (E)} $\beta = 0.011011101101111_2$\\


{\em Note.} In arithmetic coding any sequence of 
$n$ bits represents an interval of length $\frac{1}{2^n}$.
For example the bit sequence {\tt 010} stands for the segment of real numbers:
$[0.010_2; 0.011_2) = [0.25; 0.375)$ having length $\frac{1}{2^3} = \frac{1}{8}$. 
If the arithmetic coding yields a segment $[a;b)$ such that 
$[0.25; 0.375) \subseteq [a;b)$, then {\tt 010} is the 
result of the arithmetic coding.

The number $\beta = 0.010_2 \in [a;b)$, and the extra "0"
character ensures that the interval is not too long and it fits inside $[a;b)$. 
For example, {\tt 01} would represent a different interval $[0.25; 0.5) \neq [0.25; 0.375)$. 


\vspace{10pt}
{\bf Question 2.} We want to use a {\em regular expression} 
to find all phone numbers with the Latvian country code.
Assume that the phones can have one of the following formats (here 
symbol {\tt D} denotes any digit (0-9)). The space symbols
are used exactly as written (they are single space characters).

\begin{verbatim}
(+371) DDDDDDDD
+371 DDDDDDDD
\end{verbatim}

Pick a regular expression that would recognize just these 2 phone formats.\\
\textcolor{teal}{\em Select one answer.}


\begin{verbatim}
(A)   (+371|\(+371\)) \d{8}
(B)   (\+371|\(\+371\)) [0-9]{8}
(C)   (+371|\(+371\)) \d\d\d\d\d\d\d\d
(D)   \+371|(\+371\) [0-9]{8,8}
\end{verbatim}


{\em Note.} If you wish, you can use text editor such as Notepad++ search dialogue 
to verify which regex works. (Click {\bf Ctrl+F}, 
enter your regular expression, switch ``Search mode'' to Regular Expression, 
and click the button  ``Find All in All Opened Documents''): 


\begin{figure}[!htb]
\center{\includegraphics[width=2in]{notepad-regex.png}}
\caption{\label{fig:regex-search} Regex Search in Notepad++}
\end{figure}

If you are on a Linux machine, you can also 
test regex search with the command "grep" (and flag -E). 


\vspace{10pt}
{\bf Question 3.} About 70\% of the entries in a bit array of a {\em Bloom filter} are 
equal to $1$ as it is initialized with the words 
from some dictionary $D$. 
The Bloom filter computes $n=8$ independent hash functions to check, 
if some entry belongs to the dictionary $D$. 
What is an (approximate) chance $P$ to get a false positive?
A false positive is an event where one picks 
a random word $w \not\in D$,
and Bloom filter incorrectly states that $w$ belongs to $D$\\

\vspace{4pt}
\textcolor{teal}{\em Select one answer.}\\
{\bf (A)} $P = 30.00\%$,\\
{\bf (B)} $P = 5.76\%$,\\
{\bf (C)} $P = 3.75\%$,\\
{\bf (D)} $P = 0.66\%$.



\vspace{10pt}
{\bf Question 4.} 
A testing laboratory tests the same number of people daily, and on day $i$, the number
of people who tested positively for some health condition was $n_i$. The laboratory knows that 
the numbers $n_i$ are distributed according to {\em Poisson distribution} with the 
expected value $\lambda = 13.5$. 

By $P$ we denote the probability that on the given day 
$n_i < 5$ (i.e. less than $5$ people test 
positive for that condition). Which is the closest approximate value for this probability? 

{\bf (A)} $P = 0.26\%$,\\
{\bf (B)} $P = 0.51\%$,\\
{\bf (C)} $P = 5.78\%$,\\
{\bf (D)} $P = 10.89\%$.

{\em Note.} Another typical illustration of the same Poisson distribution: 
Imagine the ice cream ``R\={u}jienas sald\={e}jums'' with raisins. (In this case
the average number of raisins in one package is $\lambda = 13.5$; and you have to find the 
probability that a given ice cream package has at most $4$ raisins. 
See \url{https://bit.ly/2KanwIf}. 

\begin{figure}[!htb]
\center{\includegraphics[width=1in]{rujiena-icecream.png}}
\caption{\label{fig:rujiena-icecream} An Ice Cream Package with Raisins satisfying Poisson Distribution}
\end{figure}


% ff <- function(k) {  return((lam^k*exp(-lam))/factorial(k) }

\vspace{10pt}
{\bf Question 5.} Find the minimum number of colors to paint the $12$ vertices
from the graph shown in Figure~\ref{fig:graph-coloring} so
that any two vertices connected with an edge are having different colors. 
(This number $n$ is called the {\em chromatic number} for the graph $G$.)

\begin{figure}[!htb]
\center{\includegraphics[width=1.6in]{durer-graph.png}}
\caption{\label{fig:graph-coloring} Graph for Vertex Coloring}
\end{figure}




\vspace{10pt}
{\bf Question 6.} Assume that the ``World Wide Web'' contains only 
$4$ pages $A,B,C,D$ that link to each other as shown in the picture below.


\begin{figure}[!htb]
\center{\includegraphics[width=1in]{pageranks.png}}
\caption{\label{fig:pageranks} Four webpages with links.}
\end{figure}


% https://matrix.reshish.com/multiplication.php

In the Iteration $0$ initialize the page ranks with equal values 
$$\text{PR}_0(A) = \ldots = \text{PR}_0(D) = \frac{1}{4}.$$ 
Compute the first two iterations of these pageranks, using the formulas: 
$$\left\{ \begin{array}{l}
\text{PR}_{i+1}(A) = (1 - d) + d\left( \frac{\text{PR}_i(D)}{2} \right)\\
\text{PR}_{i+1}(B) = (1 - d) + d\left( \frac{\text{PR}_i(A)}{2} + \frac{\text{PR}_i(C)}{1} + \frac{\text{PR}_i(D)}{2}  \right)\\
\text{PR}_{i+1}(C) = (1 - d) + d\left( \frac{\text{PR}_i(B)}{2} \right)\\
\text{PR}_{i+1}(D) = (1 - d) + d\left( \frac{\text{PR}_i(A)}{2} + \frac{\text{PR}_i(B)}{2} \right)
\end{array} \right.$$
Set the value of the damping factor $d=0$.

Write the values of the second iteration for all the pages:
$\text{PR}_2(A),\ldots,\text{PR}_2(D).$.

\textcolor{teal}{\em Write $4$ comma-separated numbers; round them to 
the nearest thousandth.}





\vspace{4pt}
{\em Note.} You can also use vector algebra, if you are 
familiar with multiplying matrices with vectors
\textendash{} \url{https://bit.ly/2RJTNtC}.
$$\left( \begin{array}{c}
\text{PR}_2(A)\\
\text{PR}_2(B)\\
\text{PR}_2(C)\\
\text{PR}_2(D)
\end{array} \right) = \left( 
\begin{array}{cccc}
\textcolor{blue}{0}   & \textcolor{green}{0}   & \textcolor{red}{0} & \textcolor{purple}{1/2} \\
\textcolor{blue}{1/2} & \textcolor{green}{0}   & \textcolor{red}{1} & \textcolor{purple}{1/2} \\
\textcolor{blue}{0}   & \textcolor{green}{1/2} & \textcolor{red}{0} & \textcolor{purple}{0} \\
\textcolor{blue}{1/2} & \textcolor{green}{1/2} & \textcolor{red}{0} & \textcolor{purple}{0}
\end{array} \right)^2 \cdot \left( \begin{array}{c}
1/4 \\
1/4 \\
1/4 \\
1/4
\end{array} \right).$$
In this formula a square matrix $4 \times 4$ is twice multiplied to a $4 \times 1$
vector $(1/4, 1/4, 1/4, 1/4)$ from the left side. 
%See \url{https://bit.ly/2VkZHnh}. 


\vspace{4pt}
{\em Note.} \url{https://checkpagerank.net/check-page-rank.php}
shows that:\\
{\tt https://www.bitl.lv/} has PageRank $2/10$,\\
{\tt https://www.delfi.lv/} has PageRank $6/10$.\\
This does {\bf not} mean that there are three times more 
inbound links to Delfi than to BITL (or that these links are three times more ``valuable''). 
The value returned by this web resource is a {\em logarithmic measure}
of its iterative value. In fact, the difference between $2/10$ and $6/10$
means that the popularity of these pages differs by many orders of magnitude. 
See \url{https://bit.ly/2yrRMf7}.


\vspace{10pt}
{\bf Question 7.} As you probably know, {\em Karatsuba's algorithm} can express 
the multiplication of two numbers of length $n$ digits as 
three multiplications of numbers of length $n/2$ (i.e. the number of 
operations are three times larger, but the operands become two times shorter). 
This ultimately means that Karatsuba's algorithm requires
only $O(n^{1.585})$ operations to multiply numbers of length $n$.

Imagine that somebody has invented a new operation $a \otimes b$ for 
some objects $a,b$ (both $a,b$ have the same size $n$). 
Assume that s/he knows how to express 
$a \otimes b$ using $7$ operations $a_i \otimes b_i$ (where $i = 1,2,\ldots,7$, and
all $a_i,b_i$ have size $n/2$, i.e. half the size of the original operands $a,b$). 
({\em We do not care, what the operation $\otimes$ does; but we know that we 
can compute it for arguments $a,b$ of length $1$ in constant time; it is 
therefore easy for very short arguments.})

Find the best Big-O-Notation estimate for the time needed to compute $a \otimes b$, if $a,b$ are
both of size $n$. 

{\bf (A)} $O(n^2)$\\
{\bf (B)} $O(n^2 \log n)$\\
{\bf (C)} $O(n^{2.646})$\\
{\bf (D)} $O(n^{2.808})$\\
{\bf (E)} $O(n^3)$\\
{\bf (F)} $O(n^3 \log n)$\\
\textcolor{teal}{\em Select one answer.}

\vspace{4pt}
{\em Note 1.} One can use Master's theorem (Rosen2019, p.558) for 
this problem and also for the Karatsuba's algorithm.

\vspace{4pt}
{\em Note 2.} If the algorithm falls in multiple Big-O complexity classes, pick 
the one that shows the slowest growth. For example, if a speed of an algorithm is 
both in $O(n^2)$ and $O(n^3)$, 
then $O(n^2)$ would be a more precise and a more useful estimate.


\vspace{10pt}
{\bf Question 8.} Assume that two players $A$ and $B$ play a {\em matrix game}. 
They simultaneously guess one number each. Either player can guess 
one of these three numbers: $\{ 1,2,5 \}$. 
The payoff matrix is shown below. 
In each cell the first number is what is paid to player $A$, 
the second number is paid to player $B$. 

\begin{figure}[!htb]
\center{\includegraphics[width=2.4in]{matrix-game.png}}
\caption{\label{fig:matrix-game} Matrix game with payoffs.}
\end{figure}

Expressed in human language, the rules are as follows. 
Assume that the player $A$ just guessed a number $a$, and player $B$ guessed 
a number $b$. 
\begin{itemize}
\item If $a=b$, then it is a tie; nobody pays anything.
\item If $a>b$ (yet $a < 3b$), then $B$ pays to $A$ one euro. (And also, 
if $b>a$ yet $b < 3a$, then $A$ pays to $B$ one euro.)
\item If $a \geq 3b$, then $A$ pays to $B$ two euros. (And also, if $b \geq 3a$, 
then $B$ pays to $A$ two euros.)
\end{itemize}

{\em In this number guessing game one can win by guessing a number which is a little
bit larger than the other player's number. But one should not guess a number which is 
larger than the other by ``a lot'' (if you exceed the other player's number
three times or more, then you suffer a double loss.).}

Which can be {\em Nash equilibrium} for this number guessing game?
(You can assume that one of the answer variants is correct \textendash{} 
the same optimal strategy for both players. It is enough to find the 
one that beats all the other strategies.) 
Each strategy lists the probabilities for guessing the number $x$: 

{\bf (A)} $P(x = 1) = 1/3$, $P(x = 2) = 1/3$, $P(x = 5) = 1/3$.\\
{\bf (B)} $P(x = 1) = 0$, $P(x = 2) = 1/2$, $P(x = 5) = 1/2$.\\
{\bf (C)} $P(x = 1) = 1/2$, $P(x = 2) = 1/2$, $P(x = 5) = 0$.\\
{\bf (D)} $P(x = 1) = 1/4$, $P(x = 2) = 1/2$, $P(x = 5) = 1/4$.\\
{\bf (E)} $P(x = 1) = 2/6$, $P(x = 2) = 3/6$, $P(x = 5) = 1/6$.\\
\textcolor{teal}{\em Select one answer.}


\vspace{10pt}
{\bf Question 9.} The first iterations using Lindermayer system are given:\\
{\bf Iteration 0:} {\tt A}\\
{\bf Iteration 1:} {\tt AB}\\
{\bf Iteration 2:} {\tt ABBA}\\
{\bf Iteration 3:} {\tt ABBABAAB}\\
{\bf Iteration 4:} {\tt ABBABAABBAABABBA}

({\em To get Iteration 5: Take Iteration 4, 
change all A's into B's and
vice versa, and append such string to the end of Iteration 4.})
Find the correct set of rules to generate this L-system. 

{\bf (A)} ${\displaystyle \left\{ \begin{array}{l}
\mathtt{A} \rightarrow \mathtt{B}\\
\mathtt{B} \rightarrow \mathtt{BA}
\end{array} \right. }$\\
{\bf (B)} ${\displaystyle \left\{ \begin{array}{l}
\mathtt{A} \rightarrow \mathtt{AB}\\
\mathtt{B} \rightarrow \mathtt{AA}
\end{array} \right. }$\\
{\bf (C)} ${\displaystyle \left\{ \begin{array}{l}
\mathtt{A} \rightarrow \mathtt{AB}\\
\mathtt{B} \rightarrow \mathtt{BA}
\end{array} \right. }$\\
{\bf (D)} ${\displaystyle \left\{ \begin{array}{l}
\mathtt{A} \rightarrow \mathtt{AB}\\
\mathtt{B} \rightarrow \mathtt{AB}
\end{array} \right. }$

%{\em Note.} \url{https://bit.ly/2ypZ2rI} shows some 
%Lindenmayer images that can be created from such string iterations. 
\textcolor{teal}{\em Select one answer.}

{\em Note.} We can have a {\em turtle} that reads this 
sequence and performs actions:
\begin{itemize}
\item Letter $\mathtt{A}$: Step $1$ unit ahead, turn 
$60^{\circ}$ counterclockwise.
\item Letter $\mathtt{B}$: Turn $180^{\circ}$. 
\end{itemize}
In this case the iterations $0,2,4,\ldots$ would produce
a fragment of Koch snowflake (Figure~\ref{fig:lindenmayer-system}).

\begin{figure}[!htb]
\center{\includegraphics[width=2.5in]{lindenmayer-system.png}}
\caption{\label{fig:lindenmayer-system} Koch curve as an L-system}
\end{figure}





\vspace{10pt}
{\bf Question 10.} Assume that someone uses
a {\em secure hash} algorithm $h(x)$ that for any file $x$
outputs a hash value consisting of exactly $100$
bits. (The typical SHA-256 algorithm would return $256$ bits.)

Assume that we want to use brute force to find 
hash collision \textendash{} two different files $x_1,x_2$
such that $h(x_1) = h(x_2)$. 
You can estimate, how many hash values we need to compute before we 
get at least $50\%$ probability to find a hash collision. 
Estimation can be done using Square aproximation from 
the Birthday paradox: 
\begin{equation}
\label{eq:square-approximation}
p_{\text{collision}} \approx \frac{n^2}{2m},
\end{equation}
Formally: If a hash function $h(x)$ can take $m$ different values and
we randomly pick $n$ different integer numbers $x_1,\ldots,x_n$, then the probability 
that there is at least one collision ($h(x_i) = h(x_j)$ and $x_i \neq x_j$) is approximately expressed by
the formula (\ref{eq:square-approximation}). See \url{https://bit.ly/2RNjhGB}.

Assume that a single hash value $h(x)$ can be computed in one microsecond
($1\,\mu{}s = 10^{-6}\,s$). Estimate the number of years it would take to produce a
collision for a 100-bit secure hash algorithm with probability at least $50\%$. 

{\bf (A)} The expected time is $0.11$ years.\\
{\bf (B)} The expected time is $35.7$ years.\\
{\bf (C)} The expected time is $71.4$ years.\\
{\bf (D)} The expected time is $856$ years.\\
{\bf (E)} The expected time is $4.02\cdot 10^{16}$ years.\\
\textcolor{teal}{\em Select one answer.}


\vspace{10pt}
{\bf Question 11.} Consider the following 
problem solving strategies: 

{\footnotesize
{\bf (A)} {\bf Drawing a picture.} Can you 
write down all the things you need to consider on paper? 
Can you order them nicely in a list or a table? 
Can you show them in a two-dimensional or a three-dimensional drawing?\\
{\bf (B)} {\bf Getting hands dirty.} Can you start experimenting with the 
problem, plug in specific values, see where they lead you?\\
{\bf (C)} {\bf Going to the extremes.} Can you pick some ``borderline case''?
Is there the smallest or the largest item that is possible in the problem?\\
{\bf (D)} {\bf Lateral thinking.} Could it happen that your current solving approach 
is not applicable or is too inefficient? Can you pretend that you 
have not spent many years studying mathematics at school; 
can you apply lateral/divergent thinking out of the box to 
come up with something unexpected?\\
{\bf (E)} {\bf Looking for symmetries.} Can we switch two numbers or two letters in our 
notation? Can we inspect just one item and notice that many others are identical?\\
{\bf (F)} {\bf Making it easier.} Can we make a simpler version of this problem and
solve it first? Insert a smaller number? Solve only one particular case of it?\\
{\bf (G)} {\bf Penultimate step.} What precondition must take place before 
the final solution step is possible? Imagine, which result you would need 
in order to say that you are ``almost done''.\\
{\bf (H)} {\bf Wishful thinking.} Can you apply some outrageous simplification to your 
initial problem. Imagine for a while that you have already solved it: What would that imply?
% http://courses.cs.vt.edu/~cs4104/shaffer/Fall2010/PSintro.pdf
}

Now consider the following problem:

\begin{figure}[!htb]
\center{\includegraphics[width=2in]{tower-of-hanoi.png}}
\caption{\label{fig:tower-of-hanoi} Tower of Hanoi}
\end{figure}

\begin{mdframed}[roundcorner=6pt]
{\footnotesize
{\bf Problem.} A Tower of Hanoi (Figure~\ref{fig:tower-of-hanoi}) 
has three pegs ($A$, $B$, $C$) and
four disks initially on the peg $A$. The task is to move all the four disks to the peg $B$, where
the following rules apply:\\
Rule 1: Only one disk can be moved at a time.\\
Rule 2: Each move consists of taking the upper disk from any peg and moving it to 
another peg.\\
Rule 3: No larger disk may be placed on top of a smaller disk.

{\em The solver wants to come up with the sequence of moves. S/he has tried
a similar game with just three disks with some trial and error, but 
is not sure how to proceed in the case with four disks. Somebody suggests 
a few ``natural looking'' hints.}

{\bf Hint 1.} Find the disk that is the hardest to move anywhere or moved least frequently?\\
{\bf Hint 2.} To which peg all the other disks need to go before we move this disk?\\
{\bf Hint 3.} Assume that you know how to move three disks from the peg $A$ to the peg $B$. 
Can you move them between any other pegs? How?
}
\end{mdframed}


What kind of problem solving strategies are contained in the hints?

\textcolor{teal}{\em Select up to three relevant strategies (A-H).}



\begin{center}
\includegraphics[width=2in]{thinking-outside-the-box.png}\\
\textcopyright{} {\em Leo Cullum}, \url{https://www.newyorker.com/}
\end{center}

\vspace{10pt}
{\bf Question 12.} Someone wants to compute a MD5 checksum for the following file: 

\textcolor{blue}{
{\tt 95.211.48.179~~~bitl.lv}
}

The following is true: 
\begin{itemize}
\item File {\tt hosts.txt} is exactly $23$ bytes long.
\item The IP address is seperated from {\tt bitl.lv} by 
a single horizontal tab (byte in hexadecimal: {\tt 0x09}). 
\item The only line is ends with a Windows-style line ending (carriage return, line feed: 
bytes in hexadecimal: {\tt 0x0D}, {\tt 0x0A}). 
\end{itemize}

Figure~\ref{fig:hosts-file} shows file displayed by {\em Total Commander}; 
button {\bf F3}, then menu item {\bf Options $>$ Hex} (and also in the Notepad++ editor).

\begin{figure}[!htb]
\center{\includegraphics[width=3in]{hosts-file.png}}
\caption{\label{fig:hosts-file} Bytes in file {\tt hosts.txt}}
\end{figure}

\textcolor{teal}{\em Copy the whole MD5 checksum in your answer.}

{\em Note.} In this exercise it is important to have exactly the
same file content as shown in the picture. 
For example, replacing the {\bf TAB} character
by one or more spaces (or Windows-style line ending with a UNIX-style
line ending) would totally change MD5. 
(For secure hashes there is absolutely no string tokenization \textendash{}
unlike plagiarism detection they are very sensitive against
the smallest changes in their input.)


\vspace{10pt}
{\bf Question 13.} There are two people playing a game: Player $A$ (he is the Maximizer \textendash{} wants
to go down to a leaf with maximum payoff), and Player $B$ (he is the Minimizer \textendash{} wants
to minimize the $A$'s payoff). The current position is the root of the tree (Figure~\ref{fig:minimax}) and it is Player's $A$
turn to make the first move (to any of the root's children). After that Player $B$ moves (going down one more level) and so on
\textendash{} until they reach a leaf, which shows the payoff for Player $A$.\\
Find the maximum payoff for Player $A$ (you could use minimax algorithm with or without Alpha-Beta pruning 
speedup to find out). 

\textcolor{teal}{\em Write the payoff as a positive integer.}

\begin{figure}[!htb]
\center{\includegraphics[width=3in]{minimax.png}}
\caption{\label{fig:minimax} Game positions in a tree.}
\end{figure}


\vspace{10pt} 
{\bf Question 14.} If you verify the Conway's game of life the configuration $P_0$ 
of a straight line with $4$ live cells (Figure~\ref{fig:conway}), 
you need $N = 2$ steps until you reach ``periodic state'' $P_2$ that will 
return after period $T = 1$ (i.e.\ returning needs just one step $P_3 = P_2$, 
since ``beehive'' configuration is stable). So in this case $(N,T)=(2,1)$ \textendash{}
there are $N=2$ preliminary steps, and after that there is a period of length $T=1$. 

\begin{figure}[!htb]
\center{\includegraphics[width=2.2in]{conway.png}}
\caption{\label{fig:conway} Conway game for a line of $4$.}
\end{figure}

Now consider a different starting position $P_0$ that contains a straight line 
of $5$ live cells (Figure~\ref{fig:conway2}). Determine the number of steps $N$ needed
to reach the first position $P_N$ that would repeat infinitely often, and
the length of period $T$ with which the subsequent steps repeat, i.e. 
the smallest positive integer $T$ with the property:
$$\forall k \in \mathbb{Z}_{0+},\;(k \geq N)\;\rightarrow\; (P_{k+T} = P_k).$$

\begin{figure}[!htb]
\center{\includegraphics[width=0.6in]{conway2.png}}
\caption{\label{fig:conway2} Starting position $P_0$ for a line of $5$.}
\end{figure}

\textcolor{teal}{\em Write two comma-separated integers $N,T$.}

{\em Note.} In Conway's game there are some positions 
that are not periodic (glider guns that 
constantly create new stuff), but most simple positions eventually reach periodic state. 
Therefore the numbers $N$ and $T$ are defined in these cases.




\vspace{10pt} 
{\bf Question 15.} A mathematical theory $\mathcal{T}$ (in a similar way as Coq software) 
provides rules to prove various mathematical statements. 
Assume that in this theory $\mathcal{T}$ one can prove some statement $A$ and also the
statement $\neg A$. Which description is true for this theory:

{\bf (A)} $\mathcal{T}$ is consistent.\\
{\bf (B)} $\mathcal{T}$ is not consistent.\\
{\bf (C)} $\mathcal{T}$ is complete.\\
{\bf (D)} $\mathcal{T}$ is not complete.\\
{\bf (E)} $\mathcal{T}$ is effectively axiomatized.\\
{\bf (F)} $\mathcal{T}$ is not effectively axiomatized.\\
\textcolor{teal}{\em Select one answer.}


\vspace{10pt} 
{\bf Question 16.} Some banks can issue
19-digit credit card numbers (instead of the more typical 16-digit ones). 
Assume that there is a 19-digit number that satisfies the Luhn check (mod $10$):

$$\mathtt{557367054456450571\ast}.$$

Please find the digit that is written in the place of the last $\ast$ symbol.

\textcolor{teal}{\em Write a single digit.}


\vspace{10pt} 
{\bf Question 17.} 
Some text $T$ has been tokenized into a sequence of $N$ words ($w_0,\ldots,w_{N-1}$). 
Assume that you assigned
unique numbers to the stemmed words (each word $w_i$ in the text $T$ is replaced by 
a number $n(w_i)$); and then computed rolling hash values for 
five consecutive words in this text using this formula:\\

%\begin{align}
% & H(w_1,w_2,w_3,w_4,w_5) = \nonumber \\
%= & \left( n(w_1) \cdot a^4 + n(w_2) \cdot a^3 + n(w_3) \cdot a^2 + \right. \nonumber \\
%+ & \left. n(w_4) \cdot a^1 + n(w_5) \cdot a^0 \right)\;\; \text{mod}\;\; q. \nonumber
%\end{align}

{\footnotesize
\begin{align}
 & H(w_1,w_2,w_3,w_4,w_5) = \nonumber \\
= & \left( n(w_1) \cdot a^4 + n(w_2) \cdot a^3 + n(w_3) \cdot a^2 + n(w_4) \cdot a^1 + n(w_5) \right)\, \text{mod}\,q. \nonumber
\end{align}
}


This is a polynomial value for the argument $a$ followed by a remainder when dividing by $q$.
Parameters $a$ and $q$ are two large primes.

We compute all such hash values:
$$\left\{ \begin{array}{l}
v_0 = H(w_0,w_1,w_2,w_3,w_4),\\
v_1 = H(w_1,w_2,w_3,w_4,w_5),\\
\ldots\\
v_{N-5} = H(w_{N-5},w_{N-4},w_{N-3},w_{N-2},w_{N-1}).
\end{array} \right.$$

It turned out that $10\%$ of these hash values were found in an existing
hashtable $H$ (built from some existing texts using the same hash function) \textendash{}
about $5\%$ of the values in that hashtable are marked (the others are empty).
What is the most likely explanation for this? 

{\bf (A)} Text $T$ contains large chunks of text that is copy-pasted from other sources.\\
{\bf (B)} Overlaps of the size $10\%$ can happen by chance. On the other hand, overlaps
exceeding $1/5$ (the size of the rolling hash window) would be highly unusual and 
would require manual inspection.\\
{\bf (C)} This is not an effective way to detect copying and plagiarism.
Rolling hash should instead run on characters (not entire words),
since multiple authors may use the same words.\\
\textcolor{teal}{\em Select one answer.}





\vspace{10pt} 
{\bf Question 18.} Jane took an ordinary soccer ball made from an elastic material
(Figure~\ref{fig:icosahedron3d}).


\begin{figure}[!htb]
\center{\includegraphics[width=2in]{truncated-icosahedron-3d.png}
\caption{\label{fig:icosahedron3d} 3D Soccer Ball}
}
\end{figure}


She stretched one of its faces so that it became a planar graph
(Figure~\ref{fig:icosahedron2d}).
Then she marked a Hamiltonian cycle in this graph (not shown).


\begin{figure}[!htb]
\center{\includegraphics[width=2.4in]{truncated-icosahedron.png}}
\caption{\label{fig:icosahedron2d} Planar Soccer Ball}
\end{figure}

How many edges of this graph do {\bf not} belong to the Hamiltonian cycle?\\
\textcolor{teal}{\em Write a positive number.}




\end{document}

