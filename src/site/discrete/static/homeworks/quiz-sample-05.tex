\documentclass[jou]{apa6}

\usepackage[american]{babel}

\usepackage{csquotes}
\usepackage[style=apa,sortcites=true,sorting=nyt,backend=biber]{biblatex}
\DeclareLanguageMapping{american}{american-apa}
\addbibresource{bibliography.bib}


%%%%%%%%%%%%%%%%%%%%%%%%%%%%%%%%%%%%%%%%
%% Discrete Structures
%% The start of RBS stuff
%%%%%%%%%%%%%%%%%%%%%%%%%%%%%%%%%%%%%%%%

% Working internal and external links in PDF
\usepackage{hyperref}
% Extra math symbols in LaTeX
\usepackage{amsmath}
\usepackage{gensymb}
\usepackage{amssymb}
% Enumerations with (a), (b), etc.
\usepackage{enumerate}
\usepackage{xcolor}

\let\OLDitemize\itemize
\renewcommand\itemize{\OLDitemize\addtolength{\itemsep}{-6pt}}

\usepackage{etoolbox}
\makeatletter
\preto{\@verbatim}{\topsep=3pt \partopsep=3pt }
\makeatother

% These sizes redefine APA for A4 paper size
\oddsidemargin 0.0in
\evensidemargin 0.0in
\textwidth 6.27in
\headheight 1.0in
\topmargin -24pt
\headheight 12pt
\headsep 12pt
\textheight 9.19in

\setlength\parindent{0pt}

\title{Sample Quiz 5}
\author{Discrete Structures, Spring 2020}
\affiliation{RBS}

\leftheader{Discrete Sample Quiz 5}

\abstract{%
}

%\keywords{}

\begin{document}

\thispagestyle{empty}

\twocolumn
{\Large Discrete Sample Quiz 5}

\vspace{4pt}
{\bf Question 1.} Write at least one divisor (not equal to $1$ and to $N$) for the following numbers:\\
{\bf (A)} $N = 2^{48} + 1$\\
{\bf (B)} $N = 2^{77} - 1$\\
{\bf (C)} $N = 41^4 + 4$. 

Write your answer as three comma-separated numbers (or leave them as arithmetic expressions).

{\em Note.} You may want to use various algebraic identities to factorize:
\begin{align}
 & a^n - b^n = \nonumber \\
= & (a-b)\left(a^{n-1} + a^{n-2}b + \ldots + ab^{n-2} + b^{n-1}\right) \nonumber \\
 & a^{2n+1} + b^{2n+1} = \nonumber \\
= & (a+b)\left(a^{2n} - a^{2n-1}b + \ldots - ab^{2n-1} + b^{2n}\right) \nonumber \\
 & a^4 + 4b^4 = \nonumber \\
= & (a^2 + 2b^2 - 2ab) (a^2 + 2b^2 + 2ab) \nonumber
\end{align}



\vspace{10pt}
{\bf Question 2.} Are these statements true or false (`all integers'' include also negative numbers):
\begin{enumerate}[(A)]
\item For all integers $a,b,c$, if $a\,\mid\,b$ and $a\,\mid\,c$, then $a\,\mid\,\text{gcd}(b,c)$.
\item For all integers $a,b,c$, if $a\,\mid\,\text{gcd}(b,c)$, then $a|b$ and $a|c$.
\item For all integers $a,b,c,d$, if $a\,\mid\,b$ and $c\,\mid\,d$, then $ac\,\mid\,\text{lcm}(b,d)$.
\item For all integers $a,b,c$, $\text{gcd}(\text{gcd}(a,b),c) = \text{gcd}(a,\text{gcd}(b,c))$. 
\item For all integers $a,b,c$, $\text{lcm}(\text{gcd}(a,b),c) = \text{gcd}(\text{lcm}(a,c), \text{lcm}(b,c))$. 
\item For all primes $p>2$, $2^p +1$ is not a prime.
\end{enumerate}

Write your answer as a comma-separated string of T/F. For example, {\tt T,T,T,T,T,T}.\\
{\em Note.} Even though you only write the answers, 
make sure that you are able to justify your answer. For true statements you should be able to 
find a reasoning; for false ones \textendash{} a counterexample. 




\vspace{10pt}
{\bf Question 3.} Find $\text{gcd}(2160^{20},150^{30})$ 

Write your answer as a product of prime powers {\tt p\^{}a*q\^{}b*r\^{}c} or similar.
Numbers $p,q,r$ etc. should be in increasing order. All exponents (even those equal to $1$) should be written explicitly.



\vspace{10pt}
{\bf Question 4.}
Convert $(101\,0110\,0111)_2$ to base $16$, base $8$ and base $4$.

Write your answer as $3$ comma-separated numbers. 
For the hexadecimal notation use all digits and also capital letters 
A,B,C,D,E,F.



\vspace{10pt}
{\bf Question 5.} Find the sum and the product of these two integers written in ternary: 
$(110112)_3$, $(1000221)_3$.\\

Write your answer as two comma-separated numbers (both written in binary). 

{\em Note.} You may want to try the addition and multiplication algorithm directly in 
ternary system (without converting them into the decimal and back).



\vspace{10pt}
{\bf Question 6.}
Write the fraction $1/7$ as an infinite periodic binary fraction.

{\em Note.} One method to get, say, the first 16 digits of this fraction, you can 
multiply $1/7$ by $2^{16} = 65536$ and then express $65536/7 = 9362,\ldots$ in binary. 
A more efficient way is to use the regular division algorithm (``long division'', 
``dal\={\i}\v{s}ana stabi\c{n}\={a}''); this allows to generate a sequence of
binary digits of unlimited length. 

Write your answer as $\textcolor{blue}{\mathtt{0.(}\ldots\mathtt{)}}$ or 
$\textcolor{blue}{\mathtt{0.}\ldots\mathtt{(}\ldots\mathtt{)}}$.\\
(I.e. you start by the integer part, then write all digits preceding the period, 
then the perdiod itself in round parentheses.)



\vspace{10pt}
{\bf Question 7.} Write the first eight powers (with non-negative exponents) 
of number $5$ modulo $21$: $5^0,\,5^1,\,5^2,\,5^3,\,\ldots,5^{7}$.

Write your answer as a comma-separated list of eight remainders (mod $21$), \textendash{} all are numbers between $0$ and $20$.


\vspace{10pt}
{\bf Question 8.} Find the inverse values $1^{-1},\ldots,10^{-1}$ modulo $11$. (The inverse number of $x$ modulo $11$ 
is $x^{-1}$ such that $x^{-1}x \equiv 1\;(\text{mod}\,11)$.)

Write your answer as 10 comma-separated numbers.


\vspace{10pt}
{\bf Question 9.} Find the smallest three positive integer values of $x$ that
are solutions of the equation $55x + 21y$. 

Write your answer as three comma-separated integers. 

\end{document}

