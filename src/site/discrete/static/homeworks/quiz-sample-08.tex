\documentclass[jou]{apa6}

\usepackage[american]{babel}

\usepackage{csquotes}
\usepackage[style=apa,sortcites=true,sorting=nyt,backend=biber]{biblatex}
\DeclareLanguageMapping{american}{american-apa}
\addbibresource{bibliography.bib}


%%%%%%%%%%%%%%%%%%%%%%%%%%%%%%%%%%%%%%%%
%% Discrete Structures
%% The start of RBS stuff
%%%%%%%%%%%%%%%%%%%%%%%%%%%%%%%%%%%%%%%%

% Working internal and external links in PDF
\usepackage{hyperref}
% Extra math symbols in LaTeX
\usepackage{amsmath}
\usepackage{gensymb}
\usepackage{amssymb}
% Enumerations with (a), (b), etc.
\usepackage{enumerate}

\let\OLDitemize\itemize
\renewcommand\itemize{\OLDitemize\addtolength{\itemsep}{-6pt}}

\usepackage{etoolbox}
\makeatletter
\preto{\@verbatim}{\topsep=3pt \partopsep=3pt }
\makeatother

% These sizes redefine APA for A4 paper size
\oddsidemargin 0.0in
\evensidemargin 0.0in
\textwidth 6.27in
\headheight 1.0in
\topmargin -24pt
\headheight 12pt
\headsep 12pt
\textheight 9.19in



\title{Sample Quiz 8}
\author{Discrete Structures, Spring 2020}
\affiliation{RBS}

\leftheader{Discrete Sample Quiz 8}

\abstract{%
}

%\keywords{}

\setlength\parindent{0pt}

\begin{document}

\thispagestyle{empty}

\twocolumn
{\Large Discrete Sample Quiz 8}

\vspace{10pt}
{\bf Question 1} 
What is the probability that the sum of the numbers on two dice is even when they are rolled?\\
(Express your answer as $P/Q$.)

\vspace{10pt}
{\bf Question 2}\\
{\bf (A)} What is the probability to get sequence ``{\tt Heads}, {\tt Heads}, {\tt Heads}, 
{\tt Heads}, {\tt Heads}, {\tt Heads}'' when 
tossing a coin?\\
{\bf (B)} What is the probability to get sequence ``{\tt Heads, Tails, Heads, Tails, Heads, Tails, Heads, Tails}'' when 
tossing a coin?\\
{\bf (C)} What is the probability to get even number of "Heads" when tossing a coin 6 times?

\vspace{10pt}
{\bf Question 3}
Assume that there is the following gambling game involving dice (cube with six outcomes: $1,2,3,4,5,6$ with 
equal probabilities). Player $A$ guesses a number $n$ between $1$ and $6$ (for example $n = 3$). Then he rolls a dice three times. 
We have the following four outcomes (events):\\
$E_1$: If one of the dice rolls equals Player's $A$ number $n$, then Player $A$ wins 1 euro.\\
$E_2$: If two of the dice rolls equal Player's $A$ number $n$, then Player $A$ wins 2 euros.\\
$E_3$: If all three dice rolls equal Player's $A$ number $n$, then Player $A$ wins 3 euros.\\
$E_4$: If none of the dice rolls equals $n$, then Player $A$ loses 1 euro.\\
{\bf (A)} Find the probability for each event $E_1,E_2,E_3,E_4$.\\
{\bf (B)} Find the expected value of the money that Player $A$ wins or loses during a single round of this game. (One round = three 
dice rolls as described above). Use this formula \textendash{} a weighted sum of the euros multiplied by the respective probabilities:
$$P(E_1)\cdot 1 + P(E_2)\cdot 2 + P(E_3)\cdot 3 + P(E_4)\cdot (-1).$$


\vspace{10pt}
{\bf Question 4} Find a probability that choosing a random month during a leap year, it would have exactly $5$ Sundays.

\vspace{10pt}
{\bf Question 5} Suppose you and a friend each choose at random an integer between $1$ and $8$, inclusive. For example, some
possibilities are $(3,7)$, $(7,3)$, $(4,4)$, $(8,1)$, where your number is written first and your friend’s number second.
Find the following probabilities:\\
{\bf (A)} p(you pick 5 and your friend picks 8).\\
{\bf (B)} p(sum of the two numbers picked is < 4).\\
{\bf (C)} p(both numbers match).\\
{\bf (D)} p(the sum of the two numbers is a prime).\\
{\bf (E)} p(your number is greater than your friend’s number).

\vspace{10pt}
{\bf Question 6} In a card-game "Zol\={\i}te" there are $26$ cards: four Aces (worth 11 points each); 
four Kings (worth 4 points each); four Queens (worth 3 points each); four Jacks (worth 2 points each); 
four cards "10" (worth 10 points each); and also four "9", one "8" and one "7" (they are worth 0 points).\\
A single trick ({\em sti\c{k}is}) in this game consists of any three cards (you can assume that all the 
$C_{26}^3$ tricks have equal probabilities.)\\
{\bf (A)} Let $X$ be the random variable that expresses the number of points of a single card out 
of the 26 cards.  Find the expected value $E(X)$.\\
{\bf (B)} Find the variance $V(X)$.\\
{\bf (C)} Let $Y$ be the random variable that expresses the total number of points for all three
cards $c_1,c_2,c_3$ in a random trick. Find the expected value $E(Y)$.\\
{\bf (D)} Is $E(Y)$ exactly three times larger than $E(X)$?

\vspace{10pt}
{\bf Question 7} You create a random bit string of length five (all $32$ bit strings are equally probable). Consider
these events:\\
$E_1$: the bit string chosen begins with $1$;\\
$E_2$: the bit string chosen ends with $1$;\\
$E_3$: the bit string chosen has exactly three $1$’s.\\
{\bf (A)} Find $p(E_1 \,\mid\, E_3)$.\\
{\bf (B)} Find $p(E_3 \,\mid\, E_2)$.\\
{\bf (C)} Find $p(E_2 \,\mid\, E_3)$.


\vspace{10pt}
{\bf Question 8} A tiny ant travels in 3D space. It starts in the point $(0;0;0)$ and 
makes 9 steps altogether: With equal probability 
each step can be parallel to either $x$, $y$ or $z$ axes (incrementing the corresponding coordinate).\\ 
{\bf (A)} What is the probability that the 
ant has taken at least one step in the direction of each coordinate axis?\\
{\bf (B)} What is the probability that the ant has reached the point $(3;3;3)$?

\end{document}

