%%
%% This is file `./samples/shortsample.tex',
%% generated with the docstrip utility.
%%
%% The original source files were:
%%
%% apa6.dtx  (with options: `shortsample')
%% ----------------------------------------------------------------------
%% 
%% apa6 - A LaTeX class for formatting documents in compliance with the
%% American Psychological Association's Publication Manual, 6th edition
%% 
%% Copyright (C) 2011-2017 by Brian D. Beitzel <brian at beitzel.com>
%% 
%% This work may be distributed and/or modified under the
%% conditions of the LaTeX Project Public License (LPPL), either
%% version 1.3c of this license or (at your option) any later
%% version.  The latest version of this license is in the file:
%% 
%% http://www.latex-project.org/lppl.txt
%% 
%% Users may freely modify these files without permission, as long as the
%% copyright line and this statement are maintained intact.
%% 
%% This work is not endorsed by, affiliated with, or probably even known
%% by, the American Psychological Association.
%% 
%% ----------------------------------------------------------------------
%% 
\documentclass[jou]{apa6}

\usepackage[american]{babel}

\usepackage{csquotes}
\usepackage[style=apa,sortcites=true,sorting=nyt,backend=biber]{biblatex}
\DeclareLanguageMapping{american}{american-apa}
\addbibresource{bibliography.bib}


%%%%%%%%%%%%%%%%%%%%%%%%%%%%%%%%%%%%%%%%
%% Discrete Structures
%% The start of RBS stuff
%%%%%%%%%%%%%%%%%%%%%%%%%%%%%%%%%%%%%%%%

% Working internal and external links in PDF
\usepackage{hyperref}
% Extra math symbols in LaTeX
\usepackage{amsmath}
\usepackage{gensymb}
\usepackage{amssymb}
% Enumerations with (a), (b), etc.
\usepackage{enumerate}

\let\OLDitemize\itemize
\renewcommand\itemize{\OLDitemize\addtolength{\itemsep}{-6pt}}

\usepackage{etoolbox}
\makeatletter
\preto{\@verbatim}{\topsep=3pt \partopsep=3pt }
\makeatother

% These sizes redefine APA for A4 paper size
\oddsidemargin 0.0in
\evensidemargin 0.0in
\textwidth 6.27in
\headheight 1.0in
\topmargin -24pt
\headheight 12pt
\headsep 12pt
\textheight 9.19in



\title{Sample Quiz 3}
\author{Discrete Structures, Fall 2020}
\affiliation{RBS}

\leftheader{Discrete Sample Quiz 3}

\abstract{%
}

%\keywords{}

\begin{document}

\thispagestyle{empty}

\twocolumn
{\Large Discrete Sample Quiz 3}

\vspace{6pt}
{\bf Question 1.} {\em Definition.} A Boolean formula is a 
{\em Conjunctive Normal Form (CNF)}, 
if it is a conjunction of one or more clauses, 
where a clause is a disjunction of literals. Each literal 
is either a variable ($u,v,\ldots$) or its negation ($\neg u, \neg v, \ldots$). 

Identify, which is a CNF computing the following truth table for Boolean 
expression $E(a,b,c)$:\\
\begin{tabular}{ c | c | c | c }
$a$ & $b$ & $c$ & $E(a,b,c)$  \\ \hline
{\tt T} & {\tt T} & {\tt T} & {\tt T} \\ \hline
{\tt T} & {\tt T} & {\tt F} & {\tt F} \\ \hline
{\tt T} & {\tt F} & {\tt T} & {\tt F} \\ \hline
{\tt T} & {\tt F} & {\tt F} & {\tt F} \\ \hline
{\tt F} & {\tt T} & {\tt T} & {\tt T} \\ \hline
{\tt F} & {\tt T} & {\tt F} & {\tt F} \\ \hline
{\tt F} & {\tt F} & {\tt T} & {\tt F} \\ \hline
{\tt F} & {\tt F} & {\tt F} & {\tt T} \\ \hline
\end{tabular}

\noindent
In the test you will have to pick between several long expressions like this one:
$$(a \vee b \vee c) \wedge (a \vee b \vee \neg c) \wedge \ldots$$



\vspace{6pt}
{\bf Question 2.} The following CNF is given:
$$E = (a \vee \neg c) \wedge (b \vee \neg b).$$
Find $2$ Boolean expressions equivalent to $E$:\\
{\bf (A)} $a \rightarrow \neg b$,\\
{\bf (B)} $a \rightarrow c$,\\
{\bf (C)} $\neg c \rightarrow \neg a$,\\
{\bf (D)} $(a \wedge c) \rightarrow b$,\\
{\bf (E)} $c \rightarrow a$,\\
{\bf (F)} $a \rightarrow (c \rightarrow b)$,\\
{\bf (G)} $\neg a \rightarrow \neg c$.

\vspace{6pt}
{\bf Question 3.} We have six statements about Python programs $p \in \mathcal{P}$
that convert inputs $i \in \mathbb{Z}^{+}$ into results $r \in \mathbb{Z}^{+}$.  
A Python program may either loop indefinitely or halt (i.e.\ it eventually stops).

\begin{enumerate}
\item Some programs return the correct result for all possible inputs and 
they never loop indefinitely.
\item For any program one can find another program such that it returns
the same result for the same inputs as the first one (and also loops indefinitely, 
if the first program does the same).
\item There is a program that only loops indefinitely for at most finitely many inputs (or maybe none at all), 
but for all other inputs it produces the correct result.
\item There is at least one Python program that always halts, and for sufficiently large inputs it produces
the correct result, but it may err for some small-size inputs.
\item For a program to produce a correct result for some input $i$ it is strictly necessary to halt.\\
\item A Python program always produces exactly one result for the given input provided that it halts.
\end{enumerate}

We use these $3$ predicates:\\
$A(p_1,i_2,r_3)$ is true iff
Python program $p_1 \in \mathcal{P}$ receives input $i_2 \in \mathbb{Z}^{+}$ and outputs 
result $r_3 \in \mathbb{Z}^{+}$.\\
$H(p_1,i_2)$ is true iff program $p_1 \in \mathcal{P}$ receives input $i_2$ and halts (i.e.\ does not
loop indefinitely).\\
$C(i_1,r_2)$ is true iff for input $i_1$ the correct result is $r_2$. 

Please sort these answers to the five English sentences above.\\
{\bf Write your answer as a comma-separated list:} For example, 
{\tt F,E,D,C,B,A} tells that the 1st statement is {\bf (F)}, the 2nd one
is {\bf (E)}, $\ldots$, the 
last one is {\bf (A)}.


\begin{enumerate}[(A)]
%For a program to produce a correct result for some input $i$ it is strictly necessary to stop 
%(it cannot loop indefinitely).
\item ${\displaystyle \forall p \in \mathcal{P}\; \forall i \in \mathbb{Z}^+\; \forall r \in \mathbb{Z}^+,}$\\
${\displaystyle \left( A(p,i,r) \wedge C(i,r) \,\rightarrow\, H(p,i) \right) }$.
%For any program one can find another program such that it returns
%the same result for the same inputs as the first one (and also loops indefinitely, 
%if the first program does the same).
\item ${\displaystyle \forall p_1 \in \mathcal{P}\; \exists p_2 \in \mathcal{P}\;
\forall i \in \mathbb{Z}^{+}\; \exists r \in \mathbb{Z}^{+},}$\\
${\displaystyle \left( (\neg H(p_1,i) \wedge \neg H(p_2,i)) \vee 
(A(p_1,i,r) \leftrightarrow A(p_2,i,r)) \right)}$
%There is a program that only loops indefinitely for finitely many inputs
%and for other inputs it always produces the correct result.
\item ${\displaystyle \exists p \in \mathcal{P}\; \exists N \in \mathbb{Z}^{+}\; 
\forall i \in \mathbb{Z}^{+}\; \forall r \in \mathbb{Z}^{+},}$\\
${\displaystyle \left( ( i \leq N \wedge \neg H(p,i) ) \vee (A(p,i,r) \wedge C(i,r)) \right)}$.
%A Python program always produces exactly one result for the given input provided that it halts.
\item ${\displaystyle \forall p \in \mathcal{P}\; \forall i \in \mathbb{Z}^{+}\; \forall r_1 \in \mathbb{Z}^{+}\;
\forall r_2 \in \mathbb{Z}^{+},}$\\ 
${\displaystyle \left( H(p,i) \wedge A(p,i,r_1) \wedge  A(p,i,r_2) \,\rightarrow\, r_1 = r_2 \right)}$.
%There is at least one Python program that always halts, and for sufficiently large inputs it produces
%the correct result, but it may err for some small-size inputs.
\item ${\displaystyle \exists p \in \mathcal{P}\; \exists N \in \mathbb{Z}^{+}\; \forall i \in \mathbb{Z}^{+},}$\\
${\displaystyle \left( H(p,i) \,\wedge\, ( A(p,i,r) \wedge i > N \,\rightarrow\, C(i,r) ) \right)}$.
%Some programs return the correct result for all possible inputs and 
%they never loop indefinitely.
\item ${\displaystyle \exists p \in \mathcal{P}\; \forall i \in \mathbb{Z}^{+}\; \forall r \in \mathbb{Z}^{+},}$\\
${\displaystyle \left( H(p,i) \,\wedge\, (A(p,i,r) \rightarrow C(i,r) \right)}$. 
\end{enumerate}



\vspace{10pt}
{\bf Question 4.} There is a set of $4$ students $S = \{ s_1, s_2, s_3, s_4 \}$ and 
a set of $2$ chairs $C = \{ c_1, c_2 \}$. 
Find, how many such functions $f\,:\,S \rightarrow C$ exist, 
how many of them are injective, surjective and bijective.

{\em Fill in your answer:}

\begin{tabular}{ll} \hline
All functions $S \rightarrow C$ & $\ldots$ \\ \hline
Injective functions $S \rightarrow C$ & $\ldots$ \\ \hline
Surjective functions $S \rightarrow C$ & $\ldots$ \\ \hline
Bijective functions $S \rightarrow C$ & $\ldots$ \\\hline
\end{tabular}



\vspace{10pt}
{\bf Question 5.}
There is a predicate $S(x, y, z)$ defined for triplets
of positive integers, 
$S: \mathbb{Z}^{+} \times \mathbb{Z}^{+} \times \mathbb{Z}^{+} \rightarrow \{ \mathtt{T}, \mathtt{F} \}$. 
$S(x,y,z)$ is true iff $x \cdot y = z$.\\
Express these statements about positive integers
using only $S(x,y,z)$, Boolean operations and quantifiers. 

\begin{enumerate}[(a)] 
\item $x/y = z$,
\item $x = 1$,
\item $x = y$, 
\item $x$ is divisible by $y$ (i.e. $y \,\mid\, x$). 
\item $x$ has odd number of positive divisors.
\item $x$ is not a prime. 
\end{enumerate}


\section{Discussion on the Coq Lab}

{\bf Lemma 1:} For all propositions $a$, $\neg \neg a \rightarrow a$. 

{\bf Proof:}

\begin{itemize}
\item Assume that $\neg \neg a$ is true.
\item We sort two cases by ``classic'' axiom (Excluded Middle): either $a$ or $\neg a$ must be true. 
\item If $a$ is true, we are happy. 
\item Otherwise $\neg a$ is true (along with $\neg \neg a$ obtained before). This is a contradiction. 
\item Therefore $a$ must be true in all cases (it is either trivial, or a contradiction).
\end{itemize}

{\bf Lemma 2:} For all propositions $a$ and $b$, $\neg(a \rightarrow b) \rightarrow a$. 

{\bf Proof:}

\begin{itemize}
\item Assume that $\neg (a \rightarrow b)$ is true. 
\item We have to prove that $a$ is true. By Lemma 1, we will prove instead that $\neg neg a$ is true, 
then it will also imply $a$. 
\item We will assume that $\neg a$ is true, and attempt to get a contradiction (this means that $\neg \neg a$ must
be true). 
\item Let's prove now that $a \rightarrow b$ is true - this would be an immediate 
contradiction with $\neg (a \rightarrow b)$. 
\item To prove $a \rightarrow b$, assume that $a$ is true and let's prove $b$. But earlier
we assumed that $\neg a$. 
\item $a$ and $\neg a$ cannot be simultaneously true. This is a contradiction.
\end{itemize}

{\bf Peirce Lemma:} For all propositions $a$ and $b$, $((a \rightarrow b) \rightarrow a) \rightarrow a$.

{\bf Hint.} Just use Lemma 1 and 2 for this. And also the ``classic'' axiom: Sort 2 cases when $(a \rightarrow b)$
or $\neg (a \rightarrow b)$ are true.


{\bf Lemma 4:} For all propositions $a$ and $b$, $(\neg b \rightarrow \neg a) \rightarrow (a \rightarrow b)$.\\
This is the opposite direction from a well-known contrapositive ($\neg b \rightarrow a$ and $a \rightarrow b$
mean the same thing.)

{\bf Hint.} Use ``classic'' axiom (Excluded middle) on $b$. 

{\bf Lemma 5:} For all propositions $a,b,c,d,e$, 
$$((((a \rightarrow b) \rightarrow  (\neg c \rightarrow  \neg d)) \rightarrow  c) \rightarrow  e) \rightarrow \neg  a \rightarrow  (d -> e).$$

{\bf Hint.} Indeed, assume that $(((a \rightarrow  b) \rightarrow  \neg c \rightarrow  \neg d)\rightarrow  c) \rightarrow  e$; 
also assume $\neg a$ and $d$. Then you can prove $(((a \rightarrow b) \rightarrow \neg c \rightarrow \neg d) \rightarrow c)$ which is similar to what you need.

After all this, you can do Sample20 from the Coq lab.

\end{document}



%%%%%%%%%%%%%%%%%%%%%%%%%%%%%%%%%%%%%%%%
%% End of RBS stuff
%%%%%%%%%%%%%%%%%%%%%%%%%%%%%%%%%%%%%%%%


%% 
%% Copyright (C) 2011-2017 by Brian D. Beitzel <brian at beitzel.com>
%% 
%% This work may be distributed and/or modified under the
%% conditions of the LaTeX Project Public License (LPPL), either
%% version 1.3c of this license or (at your option) any later
%% version.  The latest version of this license is in the file:
%% 
%% http://www.latex-project.org/lppl.txt
%% 
%% Users may freely modify these files without permission, as long as the
%% copyright line and this statement are maintained intact.
%% 
%% This work is not endorsed by, affiliated with, or probably even known
%% by, the American Psychological Association.
%% 
%% 
%% This work is "maintained" (as per LPPL maintenance status) by
%% Brian D. Beitzel.
%% 
%% This work consists of the file  apa6.dtx
%% and the derived files           apa6.ins,
%%                                 apa6.cls,
%%                                 apa6.pdf,
%%                                 README,
%%                                 APAamerican.txt,
%%                                 APAbritish.txt,
%%                                 APAdutch.txt,
%%                                 APAenglish.txt,
%%                                 APAgerman.txt,
%%                                 APAngerman.txt,
%%                                 APAgreek.txt,
%%                                 APAczech.txt,
%%                                 APAturkish.txt,
%%                                 APAendfloat.cfg,
%%                                 apa6.ptex,
%%                                 TeX2WordForapa6.bas,
%%                                 Figure1.pdf,
%%                                 shortsample.tex,
%%                                 longsample.tex, and
%%                                 bibliography.bib.
%% 
%%
%% End of file `./samples/shortsample.tex'.
