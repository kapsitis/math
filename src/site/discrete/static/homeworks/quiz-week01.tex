%%
%% This is file `./samples/shortsample.tex',
%% generated with the docstrip utility.
%%
%% The original source files were:
%%
%% apa6.dtx  (with options: `shortsample')
%% ----------------------------------------------------------------------
%% 
%% apa6 - A LaTeX class for formatting documents in compliance with the
%% American Psychological Association's Publication Manual, 6th edition
%% 
%% Copyright (C) 2011-2017 by Brian D. Beitzel <brian at beitzel.com>
%% 
%% This work may be distributed and/or modified under the
%% conditions of the LaTeX Project Public License (LPPL), either
%% version 1.3c of this license or (at your option) any later
%% version.  The latest version of this license is in the file:
%% 
%% http://www.latex-project.org/lppl.txt
%% 
%% Users may freely modify these files without permission, as long as the
%% copyright line and this statement are maintained intact.
%% 
%% This work is not endorsed by, affiliated with, or probably even known
%% by, the American Psychological Association.
%% 
%% ----------------------------------------------------------------------
%% 
\documentclass[jou]{apa6}

\usepackage[american]{babel}

\usepackage{csquotes}
\usepackage[style=apa,sortcites=true,sorting=nyt,backend=biber]{biblatex}
\DeclareLanguageMapping{american}{american-apa}
\addbibresource{bibliography.bib}


%%%%%%%%%%%%%%%%%%%%%%%%%%%%%%%%%%%%%%%%
%% Discrete Structures
%% The start of RBS stuff
%%%%%%%%%%%%%%%%%%%%%%%%%%%%%%%%%%%%%%%%

% Working internal and external links in PDF
\usepackage{hyperref}
% Extra math symbols in LaTeX
\usepackage{amsmath}
\usepackage{gensymb}
\usepackage{amssymb}
% Enumerations with (a), (b), etc.
\usepackage{enumerate}

\let\OLDitemize\itemize
\renewcommand\itemize{\OLDitemize\addtolength{\itemsep}{-6pt}}

\usepackage{etoolbox}
\makeatletter
\preto{\@verbatim}{\topsep=3pt \partopsep=3pt }
\makeatother

% These sizes redefine APA for A4 paper size
\oddsidemargin 0.0in
\evensidemargin 0.0in
\textwidth 6.27in
\headheight 1.0in
\topmargin -24pt
\headheight 12pt
\headsep 12pt
\textheight 9.19in



\title{Discrete Structures (W1): Quiz}
\author{Kalvis}
\affiliation{RBS}

\leftheader{Discrete Structures (W1)}

\abstract{
}

%\keywords{}

\begin{document}
\maketitle

{\bf Question 1.} Fill in the missing entries in the truth table of this proposition:
$$E = \neg(r \rightarrow \neg q) \wedge (q \rightarrow r).$$

\vspace{3pt}
\noindent
{\em Fill in the $\ldots$:}

\begin{tabular}{ c | c | c | c }
$p$ & $q$ & $r$ & $E$ \\ \hline
{\tt T} & {\tt T} & {\tt T} & don't care \\ \hline
{\tt T} & {\tt T} & {\tt F} & don't care \\ \hline
{\tt T} & {\tt F} & {\tt T} & $\ldots$ \\ \hline
{\tt T} & {\tt F} & {\tt F} & $\ldots$ \\ \hline
{\tt F} & {\tt T} & {\tt T} & don't care \\ \hline
{\tt F} & {\tt T} & {\tt F} & don't care \\ \hline
{\tt F} & {\tt F} & {\tt T} & $\ldots$ \\ \hline
{\tt F} & {\tt F} & {\tt F} & $\ldots$ \\ \hline
\end{tabular}

\vspace{10pt}
{\bf Question 2.} Find the Boolean expression that has this truth table:

\begin{tabular}{ c | c | c }
$p$ & $q$ & ? \\ \hline
{\tt T} & {\tt T} & {\tt F} \\ \hline
{\tt T} & {\tt F} & {\tt F} \\ \hline
{\tt F} & {\tt T} & {\tt T} \\ \hline
{\tt F} & {\tt F} & {\tt F} \\ \hline
\end{tabular}

\noindent
\vspace{3pt}
{\em (Select 1 answer):}\\
{\bf (A)} $\neg (p \rightarrow q)$,\\
{\bf (B)} $\neg p \rightarrow \neg q$,\\
{\bf (C)} $\neg (q \rightarrow p)$,\\
{\bf (D)} $\neg q \rightarrow \neg p$.

\vspace{10pt}
{\bf Question 3.} Determine whether the following proposition is {\em satisfiable}:
$(\neg p \vee \neg q) \wedge (p \rightarrow q)$. If it is satisfiable, what are the 
truth values for $p$ and $q$ that makes it {\tt true}.\\


\vspace{3pt}
\noindent
{\em (Circle answer and fill in the $\ldots$, if appropriate.)}\\
Is the expression satisfiable: \hspace{5ex} YES \hspace{5ex} NO\\
If yes, what values can satisfy it (just 1 example):\\ 
$p = \ldots$ \hspace{5ex} $q = \ldots$ \hspace{5ex} $r = \ldots$. 



\vspace{10pt}
{\bf Question 4.}
Consider the following proposition: ``You cannot eat vegetables unless you also eat ice cream.''
Express it as a Boolean expression, if there are two atomic propositions:\\
$A$: ``Person $x$ can eat vegetables.''\\
$B$: ``Person $x$ eats ice cream.''

\vspace{3pt}
\noindent
{\em (Write your expression here)} $\ldots$


\vspace{10pt}
{\bf Question 5.} 
Determine whether the following two propositions are logically equivalent:
$E_1 =  p \vee \neg (q \vee r)$ and $E_2 = (p \wedge \neg q) \vee (p \wedge \neg r)$.\\
If they are not equivalent, find some values $p,q,r$ that makes $E_1$ different 
from $E_2$.



\vspace{3pt}
\noindent
{\em (Circle answer and fill in the $\ldots$, if appropriate.)}\\
Are both expressions equivalent: \hspace{5ex} YES \hspace{5ex} NO\\
If not, which truth values make them different:\\ 
$p = \ldots$ \hspace{5ex} $q = \ldots$ \hspace{5ex} $r = \ldots$. 



\vspace{10pt}
{\bf Question 6.}
Translate the given statement into propositional logic using the propositions provided:
\begin{quote}
``In Riga a person can receive {\em low income status}, 
if he or she lives in a family where the income per family member during the last 3 months did not exceed 320 EUR per month, 
or there is one person in your family, who receives an old-age or disability benefit up to 400 EUR per month.''\\
\end{quote}
Express your answer in terms of $3$ atomic propositions\\
$A$: ``You are living in a family where the average income per family member does not exceed 320 EUR per month during
the last $3$ months.''\\
$B$: ``You are a single who receives an old-age or disability benefit not exceeding 400 EUR per month.'' and\\
$C$: ``You can get low income status.''

\vspace{3pt}
\noindent
{\em (Write your expression here)} $\ldots$


\vspace{10pt}
{\bf Question 7.} 
({\em Note:} In this problem ``knights'' always tell the truth and ``knaves'' always lie.)\\
Person $A$ says ``B is a knave.''
Person $B$ says ``We are both knights.'' Determine whether each person is a knight or a knave.

\vspace{3pt}
\noindent
Is this situation possible: \hspace{5ex} YES \hspace{5ex} NO\\
If the situation is possible, who are $A,B$: $\ldots$





\end{document}



%%%%%%%%%%%%%%%%%%%%%%%%%%%%%%%%%%%%%%%%
%% End of RBS stuff
%%%%%%%%%%%%%%%%%%%%%%%%%%%%%%%%%%%%%%%%


%% 
%% Copyright (C) 2011-2017 by Brian D. Beitzel <brian at beitzel.com>
%% 
%% This work may be distributed and/or modified under the
%% conditions of the LaTeX Project Public License (LPPL), either
%% version 1.3c of this license or (at your option) any later
%% version.  The latest version of this license is in the file:
%% 
%% http://www.latex-project.org/lppl.txt
%% 
%% Users may freely modify these files without permission, as long as the
%% copyright line and this statement are maintained intact.
%% 
%% This work is not endorsed by, affiliated with, or probably even known
%% by, the American Psychological Association.
%% 
%% 
%% This work is "maintained" (as per LPPL maintenance status) by
%% Brian D. Beitzel.
%% 
%% This work consists of the file  apa6.dtx
%% and the derived files           apa6.ins,
%%                                 apa6.cls,
%%                                 apa6.pdf,
%%                                 README,
%%                                 APAamerican.txt,
%%                                 APAbritish.txt,
%%                                 APAdutch.txt,
%%                                 APAenglish.txt,
%%                                 APAgerman.txt,
%%                                 APAngerman.txt,
%%                                 APAgreek.txt,
%%                                 APAczech.txt,
%%                                 APAturkish.txt,
%%                                 APAendfloat.cfg,
%%                                 apa6.ptex,
%%                                 TeX2WordForapa6.bas,
%%                                 Figure1.pdf,
%%                                 shortsample.tex,
%%                                 longsample.tex, and
%%                                 bibliography.bib.
%% 
%%
%% End of file `./samples/shortsample.tex'.
