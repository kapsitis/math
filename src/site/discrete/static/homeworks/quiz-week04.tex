\documentclass[jou]{apa6}

\usepackage[american]{babel}

\usepackage{csquotes}
\usepackage[style=apa,sortcites=true,sorting=nyt,backend=biber]{biblatex}
\DeclareLanguageMapping{american}{american-apa}
\addbibresource{bibliography.bib}


%%%%%%%%%%%%%%%%%%%%%%%%%%%%%%%%%%%%%%%%
%% Discrete Structures
%% The start of RBS stuff
%%%%%%%%%%%%%%%%%%%%%%%%%%%%%%%%%%%%%%%%

% Working internal and external links in PDF
\usepackage{hyperref}
% Extra math symbols in LaTeX
\usepackage{amsmath}
\usepackage{gensymb}
\usepackage{amssymb}
% Enumerations with (a), (b), etc.
\usepackage{enumerate}

\let\OLDitemize\itemize
\renewcommand\itemize{\OLDitemize\addtolength{\itemsep}{-6pt}}

\usepackage{etoolbox}
\makeatletter
\preto{\@verbatim}{\topsep=3pt \partopsep=3pt }
\makeatother

% These sizes redefine APA for A4 paper size
\oddsidemargin 0.0in
\evensidemargin 0.0in
\textwidth 6.27in
\headheight 1.0in
\topmargin -24pt
\headheight 12pt
\headsep 12pt
\textheight 9.19in



\title{Sample Quiz 4}
\author{Discrete Structures, Fall 2020}
\affiliation{RBS}

\leftheader{Discrete Sample Quiz 4}

\abstract{%
}

%\keywords{}

\begin{document}

\thispagestyle{empty}

\twocolumn
{\Large Discrete Quiz 4}

\vspace{10pt}
{\bf Question 1.} Define the universe $U$ to be all possible remainders 
when we divide by $360$: $\{ 0, 1, 2, \ldots, 359 \}$. 
Also define $3$ subsets in this universe: 
$$\left\{ \begin{array}{rcl}
K_2 & = & \{ x \in U \,\mid\, x\;\text{divisible by}\;2 \},\\
K_3 & = & \{ x \in U \,\mid\, x\;\text{divisible by}\;3 \},\\
K_5 & = & \{ x \in U \,\mid\, x\;\text{divisible by}\;5 \},\\
\end{array} \right.$$

Denote by $\Phi$ the subset of $U$ containing all numbers
that are mutually prime with $360$ (no common divisors greater than $1$):
$\Phi = \{1,7,11,13,\ldots,359\}$.
Which set equality is valid regarding the subset $\Phi$:

\noindent
{\bf (A)} $\Phi = \left( K_2 \cup K_3 \cup K_5 \right)$\\
{\bf (B)} $\Phi = \left( K_2 \cap K_3 \cap K_5 \right)$\\
{\bf (C)} $\Phi = \left( \overline{K_2} \cup \overline{K_3} \cup \overline{K_5} \right)$\\
{\bf (D)} $\Phi = \left( \overline{K_2} \cap \overline{K_3} \cap \overline{K_5} \right)$\\
{\bf (E)} $\Phi = \left( \overline{K_2 \cap K_3} \cup \overline{K_2 \cap K_5} \cup \overline{K_3 \cap K_5} \right)$


Pick your answer as a single letter like this: {\tt G}

\vspace{6pt}
{\bf Question 2.}
Find the size of the set you constructed in the previous example. 

Write your answer as a single non-negative integer like this: {\tt 17}


\vspace{6pt}
{\bf Question 3.} We have the following sets:\\
$A$ is the set of all finite sequences of even positive positive numbers (such as $(6,22,10,14,2,6)$, and so on)\\
$B$ is the set of all infinite nondecreasing lists of even positive numbers (such as $(40 \leq 40 \leq 42 \leq 46 \leq \ldots)$, and so on)\\
$C$ is the set of all infinite nonincreasing lists of even positive numbers (such as $(64 \geq 58 \geq 58 \leq 54 \geq \ldots)$, and so on).\\
Clearly, all three sets are infinite. Determine their cardinalities - which list 
of cardinalities is equal to the list  $(|A|,|B|,|C|)$?


\noindent
{\bf (A)} $\left( \left| \mathbb{N} \right|, \left| \mathbb{N} \right|, \left| \mathbb{N} \right| \right)$. 
{\bf (B)} $\left( \left| \mathbb{N} \right|, \left| \mathbb{R} \right|, \left| \mathbb{N} \right| \right)$.
{\bf (C)} $\left( \left| \mathbb{N} \right|, \left| \mathbb{N} \right|, \left| \mathbb{R} \right| \right)$.
{\bf (D)} $\left( \left| \mathbb{N} \right|, \left| \mathbb{R} \right|, \left| \mathbb{R} \right| \right)$.
{\bf (E)} $\left( \left| \mathbb{R} \right|, \left| \mathbb{R} \right|, \left| \mathbb{R} \right| \right)$.


Pick your answer as a single letter like this: {\tt G}

\vspace{6pt}
{\bf Question 4.}
Let ${\displaystyle f(x) = (x^2)\;\mathbf{mod}\;11}$. Find the set $f(S)$ if $S = \{ 0,1,2,3,4,5,6,7,8,9,10 \}$. 

Write the list of elements of $f(S)$ as a sorted list like this: {\tt 1,2,3}




\vspace{6pt}
{\bf Question 5.}
How many 2-element sets are there in the powerset $\mathcal{P}\left( \{ \{ \mathtt{A}, \mathtt{B} \}, \mathtt{C}, \mathtt{D}, \mathtt{E} \} \right)$? 

Write your answer as a non-negative integer like this: {\tt 17}


\vspace{6pt}
{\bf Question 6.}
Given two sets $A = \{ x, y \}$ and $B = \{x, \{x \}\}$, check, if statements are true or false:\\
{\bf (A)} $x \subseteq B$.\\
{\bf (B)} $\emptyset \in \mathcal{P}(B)$.\\
{\bf (C)} $\{x\} \subseteq A - B$.\\
{\bf (D)} $|\mathcal{P}(A)| = 4$.

Write your answer as a sorted list of letters (which are true) like this: {\tt A,B,C,D}

\vspace{6pt}
{\bf Question 7.}
We define functions $g\,:\, A \rightarrow A$ and $f : A \rightarrow A$, where $A \{1, 2, 3, 4\}$ by 
listing all argument-value pairs: 
$f = \{(1, 2), (2, 3), (3, 4), (4, 1)\}$, $g = \{(1, 3), (2, 1), (3, 4), (4, 2)\}$.
Find the value pairs for the function $(f \circ g)^{-1}$. 

Write your answer as a comma-separated list like this: {\tt (1,1),(2,2),(3,3),(4,4)}


\vspace{6pt}
{\bf Question 8.} Find the value of this infinite sum:
$1 - 1/3 + 1/9 - 1/27 + 1/81 - \ldots$. 

Write your answer as a simple fraction: {\tt P/Q}


\vspace{6pt}
{\bf Question 9.} It is known that the function $f(n) = n^3 +88n^2 +3$ is in $O(n^3)$ \textendash{}
its asymptotic growth is as fast as the growth of the function $g(n) = n^3$. 
$\exists C \in \mathbb{Z}^{+}\;\exists n_0 \in \mathbb{Z}^{+}\;\forall n \in \mathbb{Z}^{+},$\\
$(n > n_0 \rightarrow |f(n)| \leq C\cdot{}|g(n)|)$
Find the smallest positive integer $C$ that would satisfy the above definition, 
and for your $C$ find the smallest possible $n_0$.

Write your answer $(C,n_0)$ as a pair of two numbers like this: {\tt 17,17}


\vspace{6pt}
{\bf Question 10.}
"Big O notation" allows to arrange functions according to the their
growth rate for large $n$. 
Identify, which list of functions is such that 
the first element of this list is in the big-O of the 
next element of that list and so on. (Intuitively, the first element
in the list is the slowest growing function, the last element is the
fastest growing one.)

{\bf (1)} $\log (n^{10})$, {\bf (2)} $(\log n)^2$, {\bf (3)} $\log \log n$,
{\bf (4)} $n\log n$, {\bf (5)} $\log(n!)$, {\bf (6)} $\log 2^n$.

Write your answer as a comma-separated list like this: {\tt 1,2,3,4,5,6}


\vspace{6pt}
{\bf Question 11.} Digits of all rational numbers $P/Q$ in $(0;1)$
are eventually periodic: they infinitely repeat some group of digits (the period)
starting from some place. For example,
the fraction $11/205 = 0.05(36585)$ has period of 5 digits and a pre-period "05"
of just two digits.
Find the predicate logic expression that tells
that sequence of digits $d(1),d(2),d(3),\ldots$ is eventually periodic (it may have
pre-period of any length, including length zero).

\noindent
{\bf (A)} $\exists N \in \mathbb{Z}^{+}\;\exists T \in \mathbb{Z}^{+}\;\forall n \in \mathbb{Z}^{+},$\\
$\left(n \geq N - 1 \rightarrow d(n) = d(n+T)\right)$.\\
{\bf (B)} $\exists N \in \mathbb{Z}^{+}\;\forall n \in \mathbb{Z}^{+}\;\exists T \in \mathbb{Z}^{+},$\\
$\left(n \geq N - 1 \rightarrow d(n) = d(n+T)\right)$.\\
{\bf (C)} $\forall n \in \mathbb{Z}^{+}\;\exists N \in \mathbb{Z}^{+}\;\exists T \in \mathbb{Z}^{+},$\\
$\left(n \geq N - 1 \rightarrow d(n) = d(n+T)\right)$.\\
{\bf (D)} $\forall n \in \mathbb{Z}^{+}\;\forall N \in \mathbb{Z}^{+}\;\exists T \in \mathbb{Z}^{+},$\\
$\left(n \geq N - 1 \rightarrow d(n) = d(n+T)\right)$. 

Pick your answer as a single letter like this: {\tt G}



\end{document}

