%%
%% This is file `./samples/shortsample.tex',
%% generated with the docstrip utility.
%%
%% The original source files were:
%%
%% apa6.dtx  (with options: `shortsample')
%% ----------------------------------------------------------------------
%% 
%% apa6 - A LaTeX class for formatting documents in compliance with the
%% American Psychological Association's Publication Manual, 6th edition
%% 
%% Copyright (C) 2011-2017 by Brian D. Beitzel <brian at beitzel.com>
%% 
%% This work may be distributed and/or modified under the
%% conditions of the LaTeX Project Public License (LPPL), either
%% version 1.3c of this license or (at your option) any later
%% version.  The latest version of this license is in the file:
%% 
%% http://www.latex-project.org/lppl.txt
%% 
%% Users may freely modify these files without permission, as long as the
%% copyright line and this statement are maintained intact.
%% 
%% This work is not endorsed by, affiliated with, or probably even known
%% by, the American Psychological Association.
%% 
%% ----------------------------------------------------------------------
%% 
\documentclass[jou]{apa6}

\usepackage[american]{babel}

\usepackage{csquotes}
\usepackage[style=apa,sortcites=true,sorting=nyt,backend=biber]{biblatex}
\DeclareLanguageMapping{american}{american-apa}
\addbibresource{bibliography.bib}


%%%%%%%%%%%%%%%%%%%%%%%%%%%%%%%%%%%%%%%%
%% Discrete Structures
%% The start of RBS stuff
%%%%%%%%%%%%%%%%%%%%%%%%%%%%%%%%%%%%%%%%

% Working internal and external links in PDF
\usepackage{hyperref}
% Extra math symbols in LaTeX
\usepackage{amsmath}
\usepackage{gensymb}
\usepackage{amssymb}
% Enumerations with (a), (b), etc.
\usepackage{enumerate}

\let\OLDitemize\itemize
\renewcommand\itemize{\OLDitemize\addtolength{\itemsep}{-6pt}}

\usepackage{etoolbox}
\makeatletter
\preto{\@verbatim}{\topsep=3pt \partopsep=3pt }
\makeatother

% These sizes redefine APA for A4 paper size
\oddsidemargin 0.0in
\evensidemargin 0.0in
\textwidth 6.27in
\headheight 1.0in
\topmargin -24pt
\headheight 12pt
\headsep 12pt
\textheight 9.19in



\title{Quiz for Week02}
\author{Discrete Structures, Fall 2020}
\affiliation{RBS}

\leftheader{Discrete Structures Lab01: Hints}

\abstract{%
}

%\keywords{}

\begin{document}
%\maketitle

\twocolumn
{\Large Discrete Structures Lab01: Hints}

\thispagestyle{empty}


{\bf Lemma 1:} For all propositions $a$, $\neg \neg a \rightarrow a$. 

{\bf Proof:}

\begin{itemize}
\item Assume that $\neg \neg a$ is true.
\item We sort two cases by ``classic'' axiom (Excluded Middle): either $a$ or $\neg a$ must be true. 
\item If $a$ is true, we are happy. 
\item Otherwise $\neg a$ is true (along with $\neg \neg a$ obtained before). This is a contradiction. 
\item Therefore $a$ must be true in all cases (it is either trivial, or a contradiction).
\end{itemize}

{\bf Lemma 2:} For all propositions $a$ and $b$, $\neg(a \rightarrow b) \rightarrow a$. 

{\bf Proof:}

\begin{itemize}
\item Assume that $\neg (a \rightarrow b)$ is true. 
\item We have to prove that $a$ is true. By Lemma 1, we will prove instead that $\neg neg a$ is true, 
then it will also imply $a$. 
\item We will assume that $\neg a$ is true, and attempt to get a contradiction (this means that $\neg \neg a$ must
be true). 
\item Let's prove now that $a \rightarrow b$ is true - this would be an immediate 
contradiction with $\neg (a \rightarrow b)$. 
\item To prove $a \rightarrow b$, assume that $a$ is true and let's prove $b$. But earlier
we assumed that $\neg a$. 
\item $a$ and $\neg a$ cannot be simultaneously true. This is a contradiction.
\end{itemize}

{\bf Peirce Lemma:} For all propositions $a$ and $b$, $((a \rightarrow b) \rightarrow a) \rightarrow a$.

{\bf Hint.} Just use Lemma 1 and 2 for this. And also the ``classic'' axiom: Sort 2 cases when $(a \rightarrow b)$
or $\neg (a \rightarrow b)$ are true.


{\bf Lemma 4:} For all propositions $a$ and $b$, $(\neg b \rightarrow \neg a) \rightarrow (a \rightarrow b)$.\\
This is the opposite direction from a well-known contrapositive ($\neg b \rightarrow a$ and $a \rightarrow b$
mean the same thing.)

{\bf Hint.} Use ``classic'' axiom (Excluded middle) on $b$. 

{\bf Lemma 5:} For all propositions $a,b,c,d,e$, 
$$((((a \rightarrow b) \rightarrow  (\neg c \rightarrow  \neg d)) \rightarrow  c) \rightarrow  e) \rightarrow \neg  a \rightarrow  (d -> e).$$

{\bf Hint.} Indeed, assume that $(((a \rightarrow  b) \rightarrow  \neg c \rightarrow  \neg d)\rightarrow  c) \rightarrow  e$; 
also assume $\neg a$ and $d$. Then you can prove $(((a \rightarrow b) \rightarrow \neg c \rightarrow \neg d) \rightarrow c)$ which is similar to what you need.

After all this, you can do Sample20 from the Coq lab.
