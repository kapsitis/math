\documentclass[jou]{apa6}
%\documentclass[11pt]{article}
\usepackage{ucs}
\usepackage[utf8x]{inputenc}
\usepackage{changepage}
\usepackage{graphicx}
\usepackage{amsmath}
\usepackage{gensymb}
\usepackage{amssymb}
\usepackage{enumerate}
\usepackage{tabularx}
\usepackage{lipsum}
\usepackage{hyperref}

\oddsidemargin 0.0in
\evensidemargin 0.0in
\textwidth 6.27in
\headheight 1.0in
\topmargin -0.1in
\headheight 0.0in
\headsep 0.0in
\textheight 9.0in

\usepackage{xcolor}

\setlength\parindent{0pt}

\newenvironment{myenv}{\begin{adjustwidth}{0.4in}{0.4in}}{\end{adjustwidth}}
\renewcommand{\abstractname}{Anotācija}
\renewcommand\refname{Atsauces}



\newcounter{alphnum}
\newenvironment{alphlist}{\begin{list}{(\Alph{alphnum})}{\usecounter{alphnum}\setlength{\leftmargin}{2.5em}} \rm}{\end{list}}


%16.3-6

\makeatletter
\let\saved@bibitem\@bibitem
\makeatother

\usepackage{bibentry}
%\usepackage{hyperref}


\title{Homework 1: Grading Criteria}
\author{Kalvis}
\affiliation{RBS}



\begin{document}
\thispagestyle{empty}

%\twocolumn

\begin{center}
{\Large RBS: Discrete Structures}\\
{\Large RBS: Individual Presentations Part 2: Draft Presentation}
\end{center}

The structure of your draft presentation is largely defined by the topic of your presentation
\textendash{} {\em ``Content is the king!''}. Your story and your form should follow the 
content, the objectives that you want to achieve by your presentation.
Below are a few guidelines as you prepare the draft presentation.


\section{Choosing the Title}

\begin{itemize}
\item Presentation title should not be identical 
to the name of your topic assigned at the beggining of the semester. 
If after some planning and anlysis it turned out that 
your topic becomes narrower or otherwise differs from what the instructor provided, 
your title should change as well.
\item Good titles should be short (2-5 words are usually fine), but they should
summarize what is in the presentation.
\item It may help to formulate a few candidate titles 
and select the final version after you have studied the topic carefully and
have done most of the content. (Same as authors and movie directors that 
name their work just before release.)
\end{itemize}


\section{Choosing the Layout}

During the draft presentation you can already start experimenting with the slide layout options. 
\begin{itemize}
\item Depending on your context analysis, choose the slide aspect ratio to be 
3:4 or 9:16. On most projectors or screen sharing environments 
the ``wide-screen'' (9:16) slides look best, but make sure 
that you have tested. If you plan to record your presentation as a video 
(and or publish it on YouTube), check what are the requirements there regarding
the aspect ratio. Switching to other slide layout may require for you to re-do lots of things.
\item Pay separate attention to your title slide. Does it have the required information
(the title, author, current/delivery date); it may have 
optional information (logo of RBS or other affiliation); preferrably keep it free from clutter
(unrelated logos or messages). 
\item Pay attention to the layout options. The theme should avoid background patterns; it should be
as clean as possible (black+colored text on a white background or colored text on black background). 
\item Pay attention to the font family you will be using. 
Experiment with several serif fonts (not just "Time New Roman", but also 
the old-style serifs - say, "Garamond", "Palatino") and with some sans-serif fonts
(not just "Arial", but also "Calibri", "Helvetica", "Tahoma"). 
You can use sans-serif in titles and serif in the slide text, but DON'T use more than 2 font families.
Check, how well they go together with your math formulas.
\end{itemize}


\section{Starting with ``Why''}

\begin{itemize}
\item Your 2nd slide should contain some motivation.
You can name it ``Objectives'' or ``WHY should I care'' or whatever. But your story should 
{\bf not} start with Agenda or Key Concepts or Table of Content (all of them 
are useful, and can be used later).
\item Your story should start by relating to the existing knowledge, assumptions, experiences of your 
listeners; it should motivate them to learn your new stuff. 
Good motivation never jumps to your {\bf solution} at once. You should first outline the 
{\bf problem}; and then your solution(s) can be explained as a response to that problem.
\end{itemize}

\section{How many slides will you need?}

\begin{itemize}
\item The number of slides depends on your style. If you are not overusing animation,
having one or two slides per each minute of presentation is usually the best practice. 
\item Each slide should contain 1 big picture (or several small ones); and usually some text. 
1--2 full sentences or up to 5-8 bullet points is usually the upper limit.
Do not overload slides with text. Slidedecks are not appropriate for writing prose.
Leave longer chunks of prose to the ``Notes'' sections.
\end{itemize}



\section{Using ``Notes'' Sections}

\begin{itemize}
\item As you prepare your presentation, please add the Notes sections to your slides. 
\item Ideally your slides themselves should have very limited verbal information. 
In this case ``Notes'' section should rhyme with your image: It should be accompanying 
text that explains, what happens in the slide. It should give another dimension to your message.
\item Typically you should {\bf not} use the Notes sections to write the sentences that you 
will pronounce aloud. Spoken language differs from the written language; your audience
will immediately tell, if you read some text rather than speak naturally.
\end{itemize}


\section{No regrets when you remove something}

\begin{itemize}
\item Do not make your slides too elaborate too early. Draft presentation may change many times.
In particular, it is OK to have very rough images/diagrams 
(far from the ``production quality'' ones you will insert later).
Doing too detailed slides will distract you from the content planning. 
\item Wireframes (very approximate layouts for diagrams or UI); images that 
need to be redesigned are still OK.
\item Can add more text than will be in your final presentation. 
In this case you have something to work on as you convert 
your draft into the final presentation.
\end{itemize}


\section{Summarizing the Objectives}

\begin{itemize}
\item Right before the ``References'' you should provide a finishing slide with 
restated objectives. This time you do not need to motivate or answer any ``why'' questions, 
but rather recap the important takeaway moments. 
\item The summary can closely mirror the objectives at the beginning. 
For example, if you have 3 related objectives or motivations, then your summary 
can restate them (this time in a summary form).
\end{itemize}


\section{Documenting Key References Only}

\begin{itemize}
\item As you prepare your presentation, you may need dozens of Web links about 
the related topics (even if they contain just a little bit of what you need). 
Your listeners typically would not appreciate all that information. They need a more 
condensed version.
\item References part is a good practice \textendash{} typically 1-2 last slides.
In the References you still do not want to show all the URLs that you used. 
Only leave those URLs that are the ``best of the best'', and that address specific objectives 
included in your presentation.
\item Your presentations do not need to follow APA guidelines on bibliographies.
On the other hand, very brief summaries for each URL would be helpful 
(tell, why did you find that link useful). 
\end{itemize}




\end{document}



