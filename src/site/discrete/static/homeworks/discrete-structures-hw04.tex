\documentclass[jou]{apa6}
%\documentclass[11pt]{article}
\usepackage{ucs}
\usepackage[utf8x]{inputenc}
\usepackage{changepage}
\usepackage{graphicx}
\usepackage{amsmath}
\usepackage{gensymb}
\usepackage{amssymb}
\usepackage{enumerate}
\usepackage{tabularx}
\usepackage{lipsum}
\usepackage{hyperref}
\usepackage[framemethod=TikZ]{mdframed}

\oddsidemargin 0.0in
\evensidemargin 0.0in
\textwidth 6.27in
\headheight 1.0in
\topmargin -0.1in
\headheight 0.0in
\headsep 0.0in
\textheight 9.0in

\usepackage{xcolor}

\setlength\parindent{0pt}

\newenvironment{myenv}{\begin{adjustwidth}{0.4in}{0.4in}}{\end{adjustwidth}}
\renewcommand{\abstractname}{Anotācija}
\renewcommand\refname{Atsauces}



\newcounter{alphnum}
\newenvironment{alphlist}{\begin{list}{(\Alph{alphnum})}{\usecounter{alphnum}\setlength{\leftmargin}{2.5em}} \rm}{\end{list}}


%16.3-6

\makeatletter
\let\saved@bibitem\@bibitem
\makeatother

\usepackage{bibentry}

\title{Homework 4}
\author{Discrete Structures}
\affiliation{RBS}

\begin{document}

\thispagestyle{empty}

\twocolumn
{\Large Discrete Homework 4}


\vspace{4pt}
{\bf Problem 1 (Rosen2019, \#60, p.649)} \textendash{} {\em After 9.5.}\\
{\bf (A)} Let $R$ be the relation on the set of functions from $\mathbb{Z}^{+}$ to 
$\mathbb{Z}^{+}$ such that $(f,g)$ belongs to $R$ if and only if $f$ is
$\Theta(g)$ (see Section 3.2). Show that $R$ is an equivalence relation.\\
{\bf (B)} Describe the equivalence class containing $f(n) = n^2$ for this 
equivalence relation. (Your description could 
use predicate/quantifier expression satisfied by 
all functions $h\,:\,\mathbb{Z}^{+} \rightarrow \mathbb{Z}^{+}$ 
equivalent to $f(n) = n^2$.)

{\em Note.} Big-Theta Definition (Rosen2019): Function $f(x)$
is in $\Theta(g)$ iff there are positive real numbers $C_1$ and $C_2$ and a positive 
real number $k$ such that 
$$C_1\left| g(x) \right| \leq \left| f(x) \right| \leq C_2 \left| g(x) \right|$$
whenever $x > k$. (See Definition 3 on p.227.)\\
%{\em Note 2.} The usual Big-O notation is not an equivalence relation, because
%it is not symmetric. For example, function $f(n) = n$ is in $O(g(n))$, where
%$g(n) = n^2$. But $g(n)$ is not in $O(f(n))$. Namely, linear function $f(n) = n$ 
%is bounded from above by a quadratic function $g(n) = n^2$, but not vice versa.

\vspace{6pt}
{\bf Problem 2 (Rosen2019, \#64, p.649)} \textendash{} {\em After 9.5.}\\
Do we necessarily get an equivalence relation when we form the symmetric closure of 
the reflexive closure of the transitive closure of a relation?

{\em Note.} Terms {\em reflexive closure} and {\em symmetric closure} are defined in 
(Rosen2019, p.628). The reflexive closure of a binary relation $R$ is 
the smallest relation containing $R$ that is reflexive: 
$R_1$ is obtained from $R$ by adding to it all pairs $(a,a)$ (unless they 
are already in $R$). The {\em symmetric closure} of a binary 
relation $R$ is the smallest relation $R_2$ containing $R$ that is symmetric 
(if pair $(a,b)$ belongs to $R$, then both 
pairs $(a,b)$ and $(b,a)$ are added to $R_2$). 

\vspace{6pt}
{\bf Problem 3.}\\
Suppose that winners of some lottery make a set $X$. 
Each winner should receive two prizes from some prize collection $Y$.
For each subset of the set of winners $S \subseteq X$ the set 
of prizes $N(S) \subseteq Y$ wanted by one or more people $p \in S$ satisfy 
$$|N(S)| \geq 2|S|.$$
Show that every winner can be given two prizes that s/he wants.
({\em Inspired by (Rosen2019, \#33, p.701)}.)
%% https://www.epfl.ch/labs/dcg/wp-content/uploads/2018/10/GT-ProblemSet3-Solutions.pdf


\vspace{6pt}
{\bf Problem 4 (Rosen2019, \#66, p.728)} \textendash{} {\em After 10.4.}\\
Suppose that you have a three-gallon jug and a five-gallon jug. 
You may fill either jug with water, you may empty either 
jug, and you may transfer water from either jug into the other jug [until it is full].\\ 
{\bf (A)} Use a path in a directed graph to show that you can end up with a jug containing
exactly one gallon.\\
{\bf (B)} How many vertices and how many edges are there in this directed graph?\\
(In order to build this graph of available states
use an ordered pair $(a,b)$ to indicate how much water is in each jug. Represent these ordered
pairs by vertices. Add an edge for each operation with the jugs.)

\vspace{6pt}
{\bf Problem 5 (Rosen2019, \#22, p.792)} \textendash{} {\em After 11.1.}\\
A chain letter starts when a person sends a letter to five others. 
Each person who receives the letter either sends it to five other people
who have never received it or does not send it to anyone. Suppose that 
$10,000$ people send out the letter before the chain ends and that no one receives
more than one letter. How many people receive the letter, and how many 
do not send it out?


\vspace{6pt}
{\bf Problem 6 (Rosen2019, \#46, p.793)} \textendash{} {\em After 11.1.}\\
How many vertices, leaves, and internal vertices does the rooted Fibonacci tree
$F_n$ have, where $n$ is a positive integer? What is its height?

{\em Note} The {\em rooted Fibonacci trees} $T_n$ are defined recursively 
in the following way. $T_1$ and $T_2$ are both the rooted tree consisting
of a single vertex, and for $n=3,4,\ldots$, the rooted tree $T_n$ 
is constructed from a root with $T_{n-1}$ as its left subtree and $T_{n-2}$
as its right subtree.



\vspace{6pt}
{\bf Problem 7 (Rosen2019, \#25, p.820)} \textendash{} {\em After 11.3.}\\
Construct the ordered rooted tree whose preorder traversal is 
$a,b,f,c,g,h,i,d,e,j,k,l$, where $a$ has four children, $c$ has three children, 
$j$ has two children, $b$ and $e$ have one child each, and all other vertices are leaves.

\vspace{6pt}
{\bf Problem 8.}
Assume that somebody wants to solve the following olympiad problem using ``brute force'':
\begin{mdframed}[roundcorner=6pt]
{\em Insert any arithmetic operation symbols ($+$, $-$, $\cdot$ and $/$) and
parentheses to get a correct equality:}\\
{\bf (A)} $3\;\;\;3\;\;\;7\;\;\;7\,=\,14$,\\
{\bf (B)} $3\;\;\;3\;\;\;7\;\;\;7\,=\,24$.\\
({\em \url{https://bit.ly/2JsXH5P}; Pg.1, P3.})
\end{mdframed}
How many different rooted trees can be obtained?
In Grade 5 there is no ``unary minus'' such as $(-3)\cdot 3$;
all four arithmetic operations are binary.

{\em Note.} 
You do not need to solve the quoted olympiad 
problem itself. Just count the possible expressions on the left side
that differ either by the syntax tree or by operation(s).

\vspace{6pt}
{\bf Problem 9 (Rosen2019, \#56, p.834)} \textendash{} {\em After 11.4.}\\
Show that it is possible to find a sequence of spanning trees leading from any spanning tree
to any other by successively removing one edge and adding another.

\vspace{6pt}
{\bf Problem 10 (Rosen2019, \#33, p.840)} \textendash{} {\em After 11.5.}\\
Show that if $G$ is a weighted graph with distinct edge weights, then for every simple
circuit of $G$, the edge of maximum wieght in this circuit 
does not belong to any minimum spanning tree of $G$.


\end{document}



