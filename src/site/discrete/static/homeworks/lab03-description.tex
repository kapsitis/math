%%
%% This is file `./samples/shortsample.tex',
%% generated with the docstrip utility.
%%
%% The original source files were:
%%
%% apa6.dtx  (with options: `shortsample')
%% ----------------------------------------------------------------------
%% 
%% apa6 - A LaTeX class for formatting documents in compliance with the
%% American Psychological Association's Publication Manual, 6th edition
%% 
%% Copyright (C) 2011-2017 by Brian D. Beitzel <brian at beitzel.com>
%% 
%% This work may be distributed and/or modified under the
%% conditions of the LaTeX Project Public License (LPPL), either
%% version 1.3c of this license or (at your option) any later
%% version.  The latest version of this license is in the file:
%% 
%% http://www.latex-project.org/lppl.txt
%% 
%% Users may freely modify these files without permission, as long as the
%% copyright line and this statement are maintained intact.
%% 
%% This work is not endorsed by, affiliated with, or probably even known
%% by, the American Psychological Association.
%% 
%% ----------------------------------------------------------------------
%% 
\documentclass[jou]{apa6}

\usepackage[american]{babel}

\usepackage{csquotes}
\usepackage[style=apa,sortcites=true,sorting=nyt,backend=biber]{biblatex}
\DeclareLanguageMapping{american}{american-apa}
\addbibresource{bibliography.bib}


%%%%%%%%%%%%%%%%%%%%%%%%%%%%%%%%%%%%%%%%
%% Discrete Structures
%% The start of RBS stuff
%%%%%%%%%%%%%%%%%%%%%%%%%%%%%%%%%%%%%%%%

% Working internal and external links in PDF
\usepackage{hyperref}
% Extra math symbols in LaTeX
\usepackage{amsmath}
\usepackage{gensymb}
\usepackage{amssymb}
% Enumerations with (a), (b), etc.
\usepackage{enumerate}
\usepackage{fancyvrb}
\usepackage{xcolor}

\let\OLDitemize\itemize
\renewcommand\itemize{\OLDitemize\addtolength{\itemsep}{-6pt}}

\usepackage{etoolbox}
\makeatletter
\preto{\@verbatim}{\topsep=3pt \partopsep=3pt }
\makeatother

% These sizes redefine APA for A4 paper size
\oddsidemargin 0.0in
\evensidemargin 0.0in
\textwidth 6.27in
\headheight 1.0in
\topmargin -24pt
\headheight 12pt
\headsep 12pt
\textheight 9.19in

\setlength\parindent{0pt}

\title{Quiz for Week02}
\author{Discrete Structures, Fall 2020}
\affiliation{RBS}

\leftheader{Discrete Structures Lab01: Hints}

\abstract{%
}

%\keywords{}

\begin{document}
%\maketitle

\twocolumn
{\Large Lab03/Lab04: String Matching}

\thispagestyle{empty}

In these two labs you will create two Scala classes {\tt lv.rbs.ds.lab03.KMPmatcher}
and {\tt lv.rbs.ds.lab03.BMmatcher} that implement Knuth-Morris-Pratt and 
Boyer-Moore algorithms respectively and output the JSON data structures
that can be used to demonstrate step-by-step behavior of this algorithm. 

Both algorithms are widely known and practically important, and there are many implementations available
in the Internet (including in Scala), but in this lab assignment we add 
one more twist \textendash{} our goal is to show the ``debug-like'' behavior of 
both string matchers. See these animations:\\ {\footnotesize
\url{http://whocouldthat.be/visualizing-string-matching/}\\
\url{https://people.ok.ubc.ca/ylucet/DS/KnuthMorrisPratt.html}\\
\url{https://people.ok.ubc.ca/ylucet/DS/BoyerMoore.html}\\
\url{https://dwnusbaum.github.io/boyer-moore-demo/} }

Our goal is NOT a fully-functional Web application
that would perform these tasks, but only a simple backend to support such Web applications. 
Implement the following two classes with the following public API: 

\section{KMP Matcher (Lab03)}

\begin{verbatim}
class KMPmatcher:
  new KMPmatcher(pattern: String)
  def getPrefixFun(): List[Int]
  def findAllIn(text: CharSequence): 
        Iterator[Int]
  def toJson(text: CharSequence): String
\end{verbatim}

\begin{itemize}
\item {\tt KMPmatcher(pattern:String)} is a constructor to 
initialize a class instance. Since there are some initialization costs
related to the preprocessing of the searchable {\tt pattern}, 
it might be beneficial to reuse the same matcher with the same 
pattern to search multiple texts. We therefore avoid passing 
the target {\tt text} right away.
\item {\tt getPrefixFun()} returns the prefix function $\pi(j)$ (for $j=0,\ldots,m$), 
where $m$ is the length of the searchable pattern. 
Prefix function is the key data structure used by the KMP algorithm. 
It is returned as a list of integers. So the length of this
list is $m+1$.
\item {\tt findAllIn(text: CharSequence)} returns an iterator of 
{\bf all} offsets in the {\tt text} where the searchable {\tt pattern} is found. 
Please note that in the LAB03 you have to return {\bf all} offsets; 
it is not sufficient to find just the first instance and then give up.
\item {\tt toJson(text: CharSequence)} in this case we return a 
JSON data structure as in the provided samples \textendash{} they include
also the failed attempts to match. This will be used in animation.

\end{itemize}


\section{Boyer-Moore Matcher (Lab04)}

\begin{verbatim}
class BMmatcher:
  new BMmatcher(pattern: String)
  def getGoodSuffixFun(): List[Int]
  def getBadCharFun(): List[(Char,Int)]
  def findAllIn(text: CharSequence): 
        Iterator[Int]
  def toJson(text: CharSequence): String
\end{verbatim}

The only difference with KMP is that 
JSON has different structure; and BM algorithm has 
every step scanning backwards: $\mathtt{start} \geq \mathtt{end}$.
Two data structures (Good Suffix function and Bad Character function)
are relevant only to the BM algorithm.

\section{JSON sample for KMP (Lab03)}

\begin{verbatim}
{
  "algorithm": "KMP",
  "pattern": "ABCDABD",
  "text": "ABC ABCDAB ABCDABCDABDE",
  "prefixFun": [[0,-1],[1,0],[2,0],[3,0],
    [4,0],[5,1],[6,2],[7,0]],
  "steps": [
    { "offset": 0, "start": 0, "end": 3 },
    { "offset": 3, "start": 0, "end": 0 },
    { "offset": 4, "start": 0, "end": 6 },
    { "offset": 8, "start": 2, "end": 2 },
    { "offset": 10, "start": 0, "end": 0 },
    { "offset": 11, "start": 0, "end": 6 },
    { "offset": 15, "start": 2, "end": 6, 
	    "match": "true" },   
    { "offset": 22, "start": 0, "end": 0 }
  ]
  "comparisons": 27
}
\end{verbatim}

\section{JSON sample for BM (Lab04)}


\begin{Verbatim}[commandchars=\\\{\}]
\{
  "algorithm": "BM",
  "pattern": "ABCDABD",
  "text": "ABC ABCDAB ABCDABCDABDE",
  "goodSuffixFun": \textcolor{red}{[[0,7],[1,7],[2,7],}
\textcolor{red}{      [3,7],[4,7],[5,7],[6,3],[7,1]]},
  "badCharFun": [["A",4],["B",5],["C",2],["D",6]],	
  "steps": [ 
    \{ "offset": 0, "start": 6, "end": 6 \},
    \{ "offset": 4, "start": 6, "end": 6 \},
    \{ "offset": 11, "start": 6, "end": 6 \},
    \{ "offset": 15, "start": 6, "end": 0, 
        "match": "true" \}    
  ],
  "comparisons": 10
\}
\end{Verbatim}



\end{document}


https://www-igm.univ-mlv.fr/~lecroq/string/node14.html



