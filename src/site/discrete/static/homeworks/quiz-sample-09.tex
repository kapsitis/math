\documentclass[jou]{apa6}

\usepackage[american]{babel}

\usepackage{csquotes}
\usepackage[style=apa,sortcites=true,sorting=nyt,backend=biber]{biblatex}
\DeclareLanguageMapping{american}{american-apa}
\addbibresource{bibliography.bib}


%%%%%%%%%%%%%%%%%%%%%%%%%%%%%%%%%%%%%%%%
%% Discrete Structures
%% The start of RBS stuff
%%%%%%%%%%%%%%%%%%%%%%%%%%%%%%%%%%%%%%%%

% Working internal and external links in PDF
\usepackage{hyperref}
% Extra math symbols in LaTeX
\usepackage{amsmath}
\usepackage{gensymb}
\usepackage{amssymb}
% Enumerations with (a), (b), etc.
\usepackage{enumerate}

\let\OLDitemize\itemize
\renewcommand\itemize{\OLDitemize\addtolength{\itemsep}{-6pt}}

\usepackage{etoolbox}
\makeatletter
\preto{\@verbatim}{\topsep=3pt \partopsep=3pt }
\makeatother

% These sizes redefine APA for A4 paper size
\oddsidemargin 0.0in
\evensidemargin 0.0in
\textwidth 6.27in
\headheight 1.0in
\topmargin -24pt
\headheight 12pt
\headsep 12pt
\textheight 9.19in



\title{Sample Quiz 8}
\author{Discrete Structures, Spring 2020}
\affiliation{RBS}

\leftheader{Discrete Sample Quiz 8}

\abstract{%
}

%\keywords{}

\setlength\parindent{0pt}

\begin{document}

\thispagestyle{empty}

\twocolumn
{\Large Discrete Sample Quiz 9}

\vspace{10pt}
{\bf Question 1 (Rosen7e, Ch.8, Q4).} 
Denote by $a_n$ the number of ways to go down an $n$-step staircase if you go down $1$, $2$, or $3$ steps at a time.
(In other words, $a_n$ denotes the number of ways how $n$ can be expressed as a sum of $1,2,3$, if the
order in the sum matters.)\\
Define $a_n$ as a recurrent sequence. Assume that the sequence starts from $a_1$. 

\vspace{10pt}
{\bf Question 2 (Rosen7e, Ch.8, Q11).}
A vending machine accepts only \$1 coins, \$1 bills, and \$2 bills. Let $a_n$ denote the
number of ways of depositing $n$ dollars in the vending machine, where the order in which the coins and bills
are deposited matters.\\
{\bf (A)} Find a recurrence relation for $a_n$ and give the necessary initial condition(s).\\
{\bf (B)} Find an explicit formula for $a_n$ by solving the recurrence relation in part {\bf (A)}.

\vspace{10pt}
{\bf Question 3 (Rosen7e, Ch.8, Q16-Q20).}
For each item solve the recurrence relation (characteristic equation 
or simply guess the pattern for the terms):\\
{\bf (A)} $a_n = a_{n-2}$, $a_0 = 2$, $a_1 = -1$.\\
{\bf (B)} $a_n = 2a_{n-1} + 2a_{n-2}$, $a_0 = 0$, $a_1 = 1$.\\
{\bf (C)} $a_n = 3na_{n-1}$, $a_0 = 2$.\\
{\bf (D)} $a_n = a_{n-1} + 3n$, $a_0 = 5$.\\
{\bf (E)} $a_n = 2a_{n-1} + 5$, $a_0 = 3$.

\vspace{10pt}
{\bf Question 4 (Rosen7e, Ch.8, Q24)}
Assume that the characteristic equation for a homogeneous 
linear recurrence relation with constant coefficients
is $(r + 2)(r + 4)^2 = 0$.\\
{\bf (A)} Describe the form for the general solution to the recurrence relation.\\
{\bf (B)} Define a recurrent sequence that leads to this characteristic equation.


\vspace{10pt}
{\bf Question 5 (Rosen7e, Ch.8, Q27)} 
The Catalan numbers $C_n$ count the number of strings of $n$ pluses ($+$) and $n$ minuses ($-$)
with the following property: as each
string is read from left to right, the number of pluses encountered is always 
at least as large as the number of minuses.\\
{\bf (A)} Verify this by listing these strings of lengths $2$, 
$4$, and $6$ and showing that there are $C_1$, $C_2$, and $C_3$ of
these, respectively.\\
{\bf (B)} Explain how counting these strings is the same as counting 
the number of ways to correctly parenthesize strings of variables.

{\em Note.} Catalan numbers 
$$C_0=1,\; C_1=1,\; C_2=2,\; C_3=5,\; C_4=14,\ldots$$
are defined as ${\displaystyle C_n = \frac{1}{n+1}{2n \choose n}}$.

\vspace{10pt}
{\bf Question 6  (Rosen7e, Ch.8, Q28, Q29)}\\
{\bf (A)} What form does a particular solution of the linear nonhomogeneous 
recurrence relation $a_n = 4a_{n-1} - 4a_{n-2} + F(n)$ have when $F(n) = 2^n$?\\
{\bf (B)} What form does a particular solution of the linear nonhomogeneous 
recurrence relation $a_n = 4a_{n-1} - 4a_{n-2} + F(n)$ have when $F(n) = n2^n$?


 

\vspace{10pt}
{\bf Question 6 (Rosen7e, Ch.8, Q32).}
Consider the recurrence relation $a_n = 2a_{n-1} + 3n$.\\
{\bf (A)} Write the associated homogeneous recurrence relation.
{\bf (B)} Find the general solution to the associated homogeneous recurrence relation.
{\bf (C)} Find a particular solution to the given recurrence relation.
{\bf (D)} Write the general solution to the given recurrence relation.
{\bf (E)} Find the particular solution to the given recurrence relation when $a_0 = 1$.


\vspace{10pt}
{\bf Question 8 (Rosen7e, Ch.8, Q37).} 
Suppose $f(n) = f(n/3) + 2n$, $f(1) = 1$. Find $f(27)$.


\vspace{10pt}
{\bf Question 9  (Rosen7e, Ch.8, Q53-Q63)} 
Find the coefficient of $x^8$ in the power series of each of the function:\\
{\bf (A)} $(1 + x^2 + x^4)^3$.\\
{\bf (B)} $(1 + x^2 + x^4 + x^6)^3$.\\
{\bf (C)} $(1 + x^2 + x^4 + x^6 + x^8)^3$.\\
{\bf (D)} $(1 + x^2 + x^4 + x^6 + x^8 + x^{10})^3$.\\
{\bf (E)} $(1 + x^3)^{12}$.\\
{\bf (F)} $(1 + x)(1 + x^2)(1 + x^3)(1 + x^4)(1 + x^5)$.\\
{\bf (G)} $1/(1 - 2x)$.\\
{\bf (H)} $x^3/(1 - 3x)$.\\
{\bf (I)} $1/(1 - x)^2$.\\
{\bf (J)} $x^2/(1 + 2x)^{2}$.\\
{\bf (K)} $1/(1 - 3x^2)$.




\end{document}

