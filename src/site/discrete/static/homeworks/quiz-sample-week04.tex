\documentclass[jou]{apa6}

\usepackage[american]{babel}

\usepackage{csquotes}
\usepackage[style=apa,sortcites=true,sorting=nyt,backend=biber]{biblatex}
\DeclareLanguageMapping{american}{american-apa}
\addbibresource{bibliography.bib}


%%%%%%%%%%%%%%%%%%%%%%%%%%%%%%%%%%%%%%%%
%% Discrete Structures
%% The start of RBS stuff
%%%%%%%%%%%%%%%%%%%%%%%%%%%%%%%%%%%%%%%%

% Working internal and external links in PDF
\usepackage{hyperref}
% Extra math symbols in LaTeX
\usepackage{amsmath}
\usepackage{gensymb}
\usepackage{amssymb}
% Enumerations with (a), (b), etc.
\usepackage{enumerate}

\let\OLDitemize\itemize
\renewcommand\itemize{\OLDitemize\addtolength{\itemsep}{-6pt}}

\usepackage{etoolbox}
\makeatletter
\preto{\@verbatim}{\topsep=3pt \partopsep=3pt }
\makeatother

% These sizes redefine APA for A4 paper size
\oddsidemargin 0.0in
\evensidemargin 0.0in
\textwidth 6.27in
\headheight 1.0in
\topmargin -24pt
\headheight 12pt
\headsep 12pt
\textheight 9.19in



\title{Sample Quiz 4}
\author{Discrete Structures, Fall 2020}
\affiliation{RBS}

\leftheader{Discrete Sample Quiz 4}

\abstract{%
}

%\keywords{}

\begin{document}

\thispagestyle{empty}

\twocolumn
{\Large Discrete Sample Quiz 4}

\vspace{10pt}
{\bf Question 1.} We define three sets in the universe $U$ of integer numbers
between $1$ and $70$ (inclusive):
$$\left\{ \begin{array}{rcl}
K_2 & = & \{ x \in U \,\mid\, 2 \mid x \},\\
K_5 & = & \{ x \in U \,\mid\, 5 \mid x \},\\
K_7 & = & \{ x \in U \,\mid\, 7 \mid x \},
\end{array} \right.$$

Find the size of the following sets.
Here $|X|$ denotes the number of elements in a finite set
(BTW, $|X|$ is also the notation for the cardinality of an infinite set).\\
{\em Note.} It was not intentional, but vertical bar: $\mid$
in this exercise happens to be used in three different ways: 
It is a separator when defining sets $K_2$; it is used to denote
divisibility; it also denotes set cardinality.

\begin{tabular}{ll} 
$\left| K_2 \cup K_5 \right|$ & $\ldots$ \\ 
$\left| K_2 \cap K_7 \right|$ & $\ldots$ \\ 
$\left| \overline{K_7} \right|$ & $\ldots$ \\
$\left| \overline{K_2 \cap K_5} \right|$ & $\ldots$ \\
$\left| K_2 \cap K_5 \cap K_7 \right|$ & $\ldots$ \\ 
$\left| K_2 \cap K_5 \cap \overline{K_7} \right|$ & $\ldots$ \\ 
$\left| K_2 \cup K_5 \cup K_7 \right|$ & $\ldots$ \\ 
$\left| \overline{K_2 \cup K_5 \cup K_7} \right|$ & $\ldots$ \\ 
\end{tabular}


\vspace{6pt}
{\bf Question 2.}
Find a counterexample to refute the following predicate expression:\\
$(\exists x \in U,\;P(x)) \wedge (\exists x\in U,\;Q(x)) \rightarrow$\\
$\rightarrow \exists x\in U,\;(P(x) \wedge Q(x))$.\\
Here $P(x)$ is true iff $P$ is a full square (a square of some integer number), 
$Q(x)$ is true iff $x$ is divisible by $5$, and $U$ is the set of all integers 
from the interval $[120;130]$.\\
{\em Note.} The three $x$'s in this formula refer to 
three unrelated (local) variables. 
If it looks confusing, you can rewrite it like this:\\
$(\exists x_1 \in U,\;P(x_1)) \wedge (\exists x_2\in U,\;Q(x_2)) \rightarrow$\\
$\rightarrow \exists x_3\in U,\;(P(x_3) \wedge Q(x_3))$.

\noindent
{\bf (A)} Identify the variables which you need to pick for your counter-example.\\
{\bf (B)} Pick the values for these variables to make the above statement false.


\vspace{6pt}
{\bf Question 3.} Determine the cardinality of the following sets (some finite number? 
equal to $|\mathbb{N}|$? equal to $|\mathcal{P}(\mathbb{N})| = |\mathbb{R}|$? equal to $|\mathcal{P}(\mathbb{R})|$?)
\begin{enumerate}[(A)]
\item The set of positive real numbers from $(0;1)$ with decimal representation 
containing only digits $0$ and $1$? 
\item The set of positive real numbers from $(0;1)$ with decimal representation 
containing only digits $0$ and $1$ (and it is known that the number of $0$s is finite)? 
\item The set of positive real numbers from $(0;1)$ that are fully periodic decimal fractions with 
a period of $2020$ digits? 
\item The set of positive real numbers from $(0;1)$ that have decimal representation 
without any digits "9"?
\item The set of all irrational $x \in (0;1)$ such that $x^3$ is rational?
\item Ordered pairs of real numbers $x_1,x_2$ such that $x_1,x_2 \in (0;1)$. 
\item Finite sequences of real numbers: $x_1,\ldots,x_n$, and all $x_i \in (0;1)$? (Here $n$ can 
be any positive integer)? 
\item Infinite sequences of real numbers from $(0;1)$: $\{ x_n \}$: $x_1,x_2,x_3,\ldots$.
\end{enumerate}


\vspace{6pt}
{\bf Question 4.}
Let ${\displaystyle f(x) = \left\lfloor \frac{x^3}{3} \right\rfloor}$. Find $f(S)$ if $S$ is:
\begin{enumerate}[(A)]
\item $S = \{ −2, −1, 0, 1, 2, 3 \}$.
\item $S = \{0, 1, 2, 3, 4, 5 \}$.
\item $S = \{1, 5, 7, 11 \}$.
\end{enumerate}
Is function $f: \mathbb{Z} \rightarrow \mathbb{Z}$ injective? Is it surjective? 
If it is not, mention counterexamples to show this. 

\vspace{6pt}
{\bf Question 5.}
Determine, if the given set is a powerset of some other set. If yes, which one?
\begin{enumerate}[(A)]
\item $\{\emptyset,\; \{\emptyset\},\; \{a\},\; \{\{a\}\},\; \{\{\{a\}\}\},\; \{\emptyset, a\},\; \{\emptyset, \{a\}\},\; \{\emptyset, \{\{a\}\}\},$\\
\mbox{}$\;\{a, \{a\}\},\; \{a, \{\{a\}\}\},\; \{\{a\}, \{\{a\}\}\},\; \{\emptyset, a, \{a\}\},\; \{\emptyset, a, \{\{a\}\}\},$\\
\mbox{}$\;\{\emptyset, \{a\}, \{\{a\}\}\},\; \{a, \{a\}, \{\{a\}\}\},\; \{\emptyset, a, \{a\}, \{\{a\}\}\}\}$.
\item $\{\emptyset, \{a\}\}$.
\item $\{\emptyset, \{a\}, \{\emptyset, a\}\}$.
\item $\{\emptyset, \{a\}, \{\emptyset\}, \{a, \emptyset\}\}$.
\item $\{\emptyset, \{a, \emptyset\}\}$.
\end{enumerate}

\vspace{6pt}
{\bf Question 6.}
Given two sets $A = \{ x, y \}$ and $B = \{x, \{x \}\}$, check, if statements are true or false:
\begin{enumerate}[(A)]
\item $x \subseteq B$.
\item $\emptyset \in \mathcal{P}(B)$.
\item $\{x\} \subseteq A - B$.
\item $|\mathcal{P}(A)| = 4$.
\end{enumerate}


\vspace{6pt}
{\bf Question 7.}
We define functions $g\,:\, A \rightarrow A$ and $f : A \rightarrow A$, where $A \{1, 2, 3, 4\}$ by 
listing all argument-value pairs: 
$$g = \{(1, 4), (2, 1), (3, 1), (4, 2)\},\;\;f = \{(1, 3), (2, 2), (3, 4), (4, 1)\}.$$
Find these functions by listing their argument/value pairs (or establish that they do not exist).
\begin{enumerate}[(A)]
\item Find $f \circ g$.
\item Find $g \circ f$.
\item Find $g \circ g$.
\item Find $g \circ (g \circ g)$.
\item Find $f^{-1}$.
\item Find $g^{-1}$.
\end{enumerate}


\vspace{6pt}
{\bf Question 8.} Find these sums:
\begin{enumerate}[(A)]
\item $1/4 + 1/8 + 1/16 + 1/32 + \ldots$.
\item $2 + 4 + 8 + 16 + 32 + \ldots + 2^{28}$.
\item $2 - 4 + 8 - 16 + 32 - \ldots - 2^{28}$.
\item $1 - 1/2 + 1/4 - 1/8 + 1/16 - \ldots$.
\end{enumerate}

\vspace{6pt}
{\bf Question 9.} Find an appropriate $O(g(n))$ for each function $f(n)$ defined below 
(pick your $g(n)$ to be the slowest growing among the functions such that $f(n)$ is in $O(g(n))$). 
\begin{enumerate}[(A)]
\item $f(n) = 1^2 + 2^2 + \ldots + n^2$.
\item ${\displaystyle f(n) = \frac{3n - 8 - 4n^3}{2n - 1}}$. 
\item ${\displaystyle f(n) = \sum\limits_{k=1}^{n} k^3}$.
\item ${\displaystyle f(n) = \frac{6n + 4n^5 - 4}{7n^2 - 3}}$. 
\item ${\displaystyle f(n) = \sum\limits_{k=2}^{n} k\cdot(k-1)}$.
\item ${\displaystyle f(n) = 3n^2 + 8n + 7}$
\end{enumerate}

\vspace{6pt}
{\bf Question 10.}
For the given functions, find an optimal $O(g(n))$; find
$C$ and $n_0$ (from the definition $|f(n)| < C\cdot |g(n)|$ as long as $n > n_0$). 
\begin{enumerate}[(A)]
\item ${\displaystyle f(n) = 3n^4 + \log_2 n^8}$. 
\item ${\displaystyle f(n) = \sum\limits_{k=1}^{n} (k^3 + k)}$.
\item ${\displaystyle f(n) = (n + 2)\log_2 (n^2 + 1) + \log_2 (n^3 + 1)}$.
\item ${\displaystyle f(n) = n^3 + \sin n^7}$.
\end{enumerate}


\vspace{6pt}
{\bf Question 11.} This is a Python fragment; variable {\tt n} can become very large; {\tt t} is some
fixed parameter. Denote by $f(n)$ the number of operations depending on the variable {\tt n}, 
where an operation is an addition or a multiplication, or raising to the power 2.
Find the slowest growing $g(n)$ so that $f(n)$ is in $O(g(n))$. 
\begin{verbatim}
sum = 0
for i in range(1,n+1):
    for j in range(1,n+1):
        sum += (i*t + j*t + 1)**2
\end{verbatim}

\vspace{6pt}
{\bf Question 12.} There are two functions $f,g: \mathbb{R} \rightarrow \mathbb{R}$ defined
for all real numbers and taking real values.
Find, which predicate logic expressions describe a statement that is logically
equivalent to the English sentence ``The function $f(n)$ is in $O(g(n))$''.\\
{\em Note.} There may be multiple correct answers. 

\begin{enumerate}[(A)]
\item $\forall n \in \mathbb{R}\;\exists n_0 \in \mathbb{R}\;\exists C \in \mathbb{R},$\\
$\left(n > n_0 \rightarrow |f(n)| \leq C\cdot{}|g(n)|\right)$. 
\item $\exists n_0 \in \mathbb{R}\;\forall n \in \mathbb{R}\;\exists C \in \mathbb{R},$\\
$(n > n_0 \rightarrow |f(n)| \leq C\cdot{}|g(n)|)$. 
\item $\exists n_0 \in \mathbb{R}\;\exists C \in \mathbb{R}\;\forall n \in \mathbb{R},$\\
$(n > n_0 \rightarrow |f(n)| \leq C\cdot{}|g(n)|)$. 
\item $\exists n_0 \in \mathbb{R}\;\exists C \in \mathbb{R}\;\forall n \in \mathbb{R},$\\
$(n > n_0 \rightarrow f(n) \leq C\cdot{}|g(n)|)$. 
\item $\exists n_0 \in \mathbb{R}\;\exists C \in \mathbb{R}\;\forall n \in \mathbb{R},$\\
$(n > n_0 \rightarrow |f(n)| \leq C\cdot{}g(n))$. 
\item $\exists n_0 \in \mathbb{R}\;\exists C \in \mathbb{R}\;\forall n \in \mathbb{R},$\\
$(n \geq n_0 \rightarrow |f(n)| < C\cdot{}|g(n)|)$. 
\item $\exists n_0 \in \mathbb{Z}^{+}\;\exists C \in \mathbb{Z}^{+}\;\forall n \in \mathbb{R},$\\
$(n > n_0 \rightarrow |f(n)| \leq C\cdot{}|g(n)|)$. 
\end{enumerate}




\end{document}

