%%
%% This is file `./samples/shortsample.tex',
%% generated with the docstrip utility.
%%
%% The original source files were:
%%
%% apa6.dtx  (with options: `shortsample')
%% ----------------------------------------------------------------------
%% 
%% apa6 - A LaTeX class for formatting documents in compliance with the
%% American Psychological Association's Publication Manual, 6th edition
%% 
%% Copyright (C) 2011-2017 by Brian D. Beitzel <brian at beitzel.com>
%% 
%% This work may be distributed and/or modified under the
%% conditions of the LaTeX Project Public License (LPPL), either
%% version 1.3c of this license or (at your option) any later
%% version.  The latest version of this license is in the file:
%% 
%% http://www.latex-project.org/lppl.txt
%% 
%% Users may freely modify these files without permission, as long as the
%% copyright line and this statement are maintained intact.
%% 
%% This work is not endorsed by, affiliated with, or probably even known
%% by, the American Psychological Association.
%% 
%% ----------------------------------------------------------------------
%% 
\documentclass[jou]{apa6}

\usepackage[american]{babel}

\usepackage{csquotes}
\usepackage[style=apa,sortcites=true,sorting=nyt,backend=biber]{biblatex}
\DeclareLanguageMapping{american}{american-apa}
\addbibresource{bibliography.bib}


%%%%%%%%%%%%%%%%%%%%%%%%%%%%%%%%%%%%%%%%
%% Discrete Structures
%% The start of RBS stuff
%%%%%%%%%%%%%%%%%%%%%%%%%%%%%%%%%%%%%%%%

% Working internal and external links in PDF
\usepackage{hyperref}
% Extra math symbols in LaTeX
\usepackage{amsmath}
\usepackage{gensymb}
\usepackage{amssymb}
% Enumerations with (a), (b), etc.
\usepackage{enumerate}

\let\OLDitemize\itemize
\renewcommand\itemize{\OLDitemize\addtolength{\itemsep}{-6pt}}

\usepackage{etoolbox}
\makeatletter
\preto{\@verbatim}{\topsep=3pt \partopsep=3pt }
\makeatother

% These sizes redefine APA for A4 paper size
\oddsidemargin 0.0in
\evensidemargin 0.0in
\textwidth 6.27in
\headheight 1.0in
\topmargin -24pt
\headheight 12pt
\headsep 12pt
\textheight 9.19in



\title{Discrete Structures (W1): Sample Quiz}
\author{Kalvis}
\affiliation{RBS}

\leftheader{Discrete Structures (W1)}

\abstract{These are sample questions \textendash{} they are similar, but not identical 
to the actual quiz planned for January 9, 2020. 
The actual quiz will have $7$ questions, its time is $15$ minutes; 
it accounts for 7~\textperthousand{} of your total grade.
% 7*8 + 11*4
}

%\keywords{}

\begin{document}
\maketitle

{\bf Question 1.} Fill in the missing entries in the truth table of this proposition:
$$E = \neg(r \rightarrow \neg q) \vee (p \wedge \neg r).$$

\begin{tabular}{ c | c | c | c }
$p$ & $q$ & $r$ & $E$ \\ \hline
{\tt T} & {\tt T} & {\tt T} & {\tt T} \\ \hline
{\tt T} & {\tt T} & {\tt F} & $\ldots$ \\ \hline
{\tt T} & {\tt F} & {\tt T} & {\tt F} \\ \hline
{\tt T} & {\tt F} & {\tt F} & $\ldots$ \\ \hline
{\tt F} & {\tt T} & {\tt T} & {\tt T} \\ \hline
{\tt F} & {\tt T} & {\tt F} & $\ldots$ \\ \hline
{\tt F} & {\tt F} & {\tt T} & {\tt F} \\ \hline
{\tt F} & {\tt F} & {\tt F} & $\ldots$ \\ \hline
\end{tabular}

\vspace{10pt}
{\bf Question 2.} Find the Boolean expression that has this truth table:

\begin{tabular}{ c | c | c }
$p$ & $q$ & ? \\ \hline
{\tt T} & {\tt T} & {\tt F} \\ \hline
{\tt T} & {\tt F} & {\tt T} \\ \hline
{\tt F} & {\tt T} & {\tt T} \\ \hline
{\tt F} & {\tt F} & {\tt F} \\ \hline
\end{tabular}

\noindent
\vspace{3pt}
{\em (Select 1 answer):}\\
{\bf (A)} $\neg (\neg ( p \wedge \neg p) \wedge \neg (\neg q \wedge q))$,\\
{\bf (B)} $\neg (\neg ( p \wedge q) \wedge \neg (\neg p \wedge \neg q))$,\\
{\bf (C)} $\neg (\neg ( p \wedge \neg q) \wedge \neg (\neg p \wedge q))$,\\
{\bf (D)} $\neg (( \neg p \wedge \neg q) \wedge \neg (p \wedge q))$.

\vspace{10pt}
{\bf Question 3.} Determine whether the following proposition is {\em satisfiable}:
$(\neg p \vee \neg q) \wedge (p \rightarrow q)$. If it is satisfiable, what are the 
truth values for $p$ and $q$ that makes it {\tt true}.\\
{\em Reminder.} A Boolean expression is called {\em satisfiable}, 
if you can assign its variables to make it evaluate to {\tt true}. In other words,
it is not always {\tt false}. 

\vspace{10pt}
{\bf Question 4.}
Consider the following proposition: ``Not eating vegetables is sufficient for not getting ice cream.''
Express it as a Boolean expression, if there are two atomic propositions:\\
$A$: ``Person $x$ eats vegetables.''\\
$B$: ``Person $x$ gets ice cream.''

\vspace{10pt}
{\bf Question 5.} 
Determine whether the following two propositions are logically equivalent:
$E_1 = p \rightarrow (\neg q \wedge r)$ and $E_2 = \neg p \vee \neg(r \rightarrow q)$.\\
If they are not equivalent, find some values $p,q,r$ that makes $E_1$ different 
from $E_2$.

\vspace{10pt}
{\bf Question 6.} 
Translate the given statement into propositional logic using the propositions provided:
\begin{quote}
``On certain highways in the Washington, DC metro area you are allowed to travel
on high occupancy lanes during rush hour only if there are at least three passengers in the vehicle.''
\end{quote}
Express your answer in terms of $3$ atomic propositions\\
$r$: ``You are traveling during rush hour.''\\
$t$: ``You are riding in a car with at least three passengers.'' and\\
$h$: ``You can travel on a high occupancy lane.''

\vspace{10pt}
{\bf Question 7a.} 
({\em Note:} In this problem ``knights'' always tell the truth and ``knaves'' always lie.)\\
On the island of knights and knaves you encounter two people, $A$ and $B$ (each of them knows
everything about himself and the other person).\\
Person $A$ says ``B is a knave.''\\
Person $B$ says ``At least one of us is a knight.''\\
Determine whether each person is a knight or a knave.

\vspace{10pt}
{\bf Question 7b.} 
An island has three kinds of people: knights who always
tell the truth, knaves who always lie, and normals who can either tell the truth or lie. 
You encounter three people, $A$, $B$, and $C$. 
You know one of the three people is a knight, one is a knave, and one is a normal. Each of the three
people knows the type of person each of the other two is.\\
$A$ says ``I am not a knight,''\\
$B$ says ``I am not a normal,'' and 
$C$ says ``I am not a knave.''\\
Write a possible solution (for example, $(A,B,C)=({\tt knight},{\tt knave},{\tt normal})$ or 
some other permutation), or state that there are no solutions.


\newpage 

\section{Answers}


{\bf Question 1:} Answer:

\begin{tabular}{ c | c | c | c }
$p$ & $q$ & $r$ & ? \\ \hline
{\tt T} & {\tt T} & {\tt T} & {\tt T} \\ \hline
{\tt T} & {\tt T} & {\tt F} & $\boxed{\mathtt{T}}$ \\ \hline
{\tt T} & {\tt F} & {\tt T} & {\tt F} \\ \hline
{\tt T} & {\tt F} & {\tt F} & $\boxed{\mathtt{T}}$ \\ \hline
{\tt F} & {\tt T} & {\tt T} & {\tt T} \\ \hline
{\tt F} & {\tt T} & {\tt F} & $\boxed{\mathtt{F}}$ \\ \hline
{\tt F} & {\tt F} & {\tt T} & {\tt F} \\ \hline
{\tt F} & {\tt F} & {\tt F} & $\boxed{\mathtt{F}}$ \\ \hline
\end{tabular}

\vspace{3pt}
\noindent
Note that $r=\mathtt{false}$ in all the unknown slots. 
We can simplify the Boolean expression:
\begin{align}
E & \equiv \neg(r \rightarrow \neg q) \vee (p \wedge \neg r) \equiv \nonumber \\
  & \equiv \neg(\mathtt{false} \rightarrow \neg q) \vee 
(p \wedge \mathtt{true}) \equiv  \nonumber \\
  & \equiv \neg(\mathtt{true}) \vee p \equiv \nonumber \\
  & \equiv \mathtt{false} \vee p \equiv p. \nonumber
\end{align}
Therefore we simply copy the value of $p$ in all four places.

\vspace{10pt}
{\bf Question 2:} Answer: {\bf (C)}.\\
The truth table is identical to $p \oplus q$ (exclusive OR). 
To see that answer {\bf (C)} is correct, apply De Morgan's law multiple times:
\begin{align}
\neg (\neg ( p \wedge \neg q) \wedge \neg (\neg p \wedge q)) & \equiv \nonumber \\
\neg \neg (p \wedge \neg q) \vee \neg \neg (\neg p \wedge q) & \equiv \nonumber \\
(p \wedge \neg q) \vee (\neg p \wedge q). & \nonumber
\end{align}

The last expression is exactly the exclusive OR (either $p$ is {\tt true} and
$q$ is {\tt false} or vice versa).

\vspace{10pt}
{\bf Question 3:} Answer: Yes,
$(\neg p \vee \neg q) \wedge (p \rightarrow q)$ is 
satisfiable.\\
Subexpression $(\neg p \vee \neg q)$ tells that either $p$ 
or $q$ (or both) should be {\tt false}. So, you should take
$p = \mathtt{false}$ to make $p \rightarrow q$ {\tt true} as well. 


\vspace{10pt}
{\bf Question 4.} Answer: $\neg A \rightarrow \neg B$ (or $B \rightarrow A$).\\
Condition $C$ is called {\em sufficient} for the result $R$, if 
$R$ is {\tt true} whenever $C$ is {\tt true}. 
This means that {\bf not} eating vegetables logically implies {\bf not} 
getting ice cream. It is exactly $\neg A \rightarrow \neg B$.\\
You can rewrite it as a contrapositive, if you like: $B \rightarrow A$:
getting ice cream implies having eaten vegetables. Both answers are
equivalent.

\vspace{10pt}
{\bf Question 5.} Answer: Yes, $E_1$ and $E_2$ are logically equivalent 
(they mean the same thing).
Consequently, there is no way to assign $p,q,r$ to make their truth values different.

\begin{itemize}
\item $p \rightarrow (\neg q \wedge r)$ means that either condition ($p$) is {\tt false} or 
the conclusion ($\neg q \wedge r$) is {\tt true}. So, $E_1$ can be rewritten as 
$\neg p \vee (\neg q \wedge r)$.
\item In $E_2$ the subexpression 
$\neg(r \rightarrow q)$ can be rewritten as $(r \wedge \neg q)$
(for the implication $r \rightarrow q$ to be {\tt false}, you should have
both $r = \mathtt{true}$ and $q = \mathtt{false}$, so it 
means $(r \wedge \neg q)$. And that is exactly the same expression we got 
from $E_1$, since $(r \wedge \neg q) = (\neg q \wedge r)$
\end{itemize}



\vspace{10pt}
{\bf Question 6.} Answer: $(r \wedge h) \rightarrow t$\\
"Only if" means {\bf necessity}. Namely, $(r \wedge h)$ implies $t$, 
or $(r \wedge h) \rightarrow t$ (Ability to use a high occupancy road during a rush 
hour implies at least three people in the car.)\\
You can rewrite this as a contrapositive (and expand $(r \wedge h)$ 
with De Morgan's law). Say, "If you do not have three people
in your car, then it is either not a high occupancy road, or it is not a rush hour."
This is an equivalent statement $\neg t \rightarrow (\neg r \vee \neg h)$. 


\vspace{10pt}
{\bf Question 7a.} Answer: $A$ is a knave; $B$ is a knight.\\
We sort cases.\\
(1) Assume that $A$ is a knight. He says that $B$ is a knave (and that must be true, 
since $A$ does not lie). 
But then $B$ should lie, when he says "At least one of us is a knight". 
The opposite of "At least one" ($\geq 1$) is "less than one" ($<1$); so in this
case there should be no knights at all. This is a contradiction. Therefore $A$ cannot be a knight.\\
(2) Assume that $A$ is a knave. Since he says that $B$ is a knave, $B$ should be a knight. 
And moreover, since $B$ says that "at least one of us is a knight", then 
it is still consistent, because he is a knight himself. 

\vspace{10pt}
{\bf Question 7b.} TBD (I can explain this in the class, if you wish.)



\end{document}



%%%%%%%%%%%%%%%%%%%%%%%%%%%%%%%%%%%%%%%%
%% End of RBS stuff
%%%%%%%%%%%%%%%%%%%%%%%%%%%%%%%%%%%%%%%%


%% 
%% Copyright (C) 2011-2017 by Brian D. Beitzel <brian at beitzel.com>
%% 
%% This work may be distributed and/or modified under the
%% conditions of the LaTeX Project Public License (LPPL), either
%% version 1.3c of this license or (at your option) any later
%% version.  The latest version of this license is in the file:
%% 
%% http://www.latex-project.org/lppl.txt
%% 
%% Users may freely modify these files without permission, as long as the
%% copyright line and this statement are maintained intact.
%% 
%% This work is not endorsed by, affiliated with, or probably even known
%% by, the American Psychological Association.
%% 
%% 
%% This work is "maintained" (as per LPPL maintenance status) by
%% Brian D. Beitzel.
%% 
%% This work consists of the file  apa6.dtx
%% and the derived files           apa6.ins,
%%                                 apa6.cls,
%%                                 apa6.pdf,
%%                                 README,
%%                                 APAamerican.txt,
%%                                 APAbritish.txt,
%%                                 APAdutch.txt,
%%                                 APAenglish.txt,
%%                                 APAgerman.txt,
%%                                 APAngerman.txt,
%%                                 APAgreek.txt,
%%                                 APAczech.txt,
%%                                 APAturkish.txt,
%%                                 APAendfloat.cfg,
%%                                 apa6.ptex,
%%                                 TeX2WordForapa6.bas,
%%                                 Figure1.pdf,
%%                                 shortsample.tex,
%%                                 longsample.tex, and
%%                                 bibliography.bib.
%% 
%%
%% End of file `./samples/shortsample.tex'.
