\documentclass[jou]{apa6}
%\documentclass[11pt]{article}
\usepackage{ucs}
\usepackage[utf8x]{inputenc}
\usepackage{changepage}
\usepackage{graphicx}
\usepackage{amsmath}
\usepackage{gensymb}
\usepackage{amssymb}
\usepackage{enumerate}
\usepackage{tabularx}
\usepackage{lipsum}
\usepackage{hyperref}

\oddsidemargin 0.0in
\evensidemargin 0.0in
\textwidth 6.27in
\headheight 1.0in
\topmargin -0.1in
\headheight 0.0in
\headsep 0.0in
\textheight 9.0in

\usepackage{xcolor}

\setlength\parindent{0pt}

\newenvironment{myenv}{\begin{adjustwidth}{0.4in}{0.4in}}{\end{adjustwidth}}
\renewcommand{\abstractname}{Anotācija}
\renewcommand\refname{Atsauces}



\newcounter{alphnum}
\newenvironment{alphlist}{\begin{list}{(\Alph{alphnum})}{\usecounter{alphnum}\setlength{\leftmargin}{2.5em}} \rm}{\end{list}}


%16.3-6

\makeatletter
\let\saved@bibitem\@bibitem
\makeatother

\usepackage{bibentry}
%\usepackage{hyperref}


\title{Homework 3}
\author{Discrete Structures}
\affiliation{RBS}



\begin{document}
%\maketitle


\thispagestyle{empty}

\twocolumn
{\Large Discrete Structures: Homework 3 Hints}

You are not required to read or to do anything with these hints. 
They only address one problem (HW3, P4). 


\vspace{8pt}
{\bf Problem 4 (Rosen2019, \#33, p.465)} \textendash{} {\em After Ch.6.}\\
How many bit strings of length $n$, where $n \geq 4$, contain exactly two 
occurrences of {\tt 01}.

In order to develop our counting techniques, we can consider a simpler problem: 
Find those bit strings of length $n$ (where $n \geq 2$) containing exactly one 
occurrence of {\tt 01}. 

Such numbers can be represented like this: 
\begin{verbatim}
.......01.........
\end{verbatim}
There {\tt 01} denotes the only occurrence of "01", but the sequences of 
dots represent (possibly empty) non-increasing sequences of $k$ and $n-k-2$ bits. 
Here $k = 0,1,2,\ldots,n-2$. 
(A sequence of bits is non-increasing, if it is constantly equal to "1" or to "0", or switches from 
value "1" to "0", but never from "0" to "1"). 

{\bf Statement.} 
There are altogether $k+1$ non-increasing sequences of length $k$. 
You can enumerate them (they may contain no more than one location, where the value switches from "1" to "0"). 

Consequently, as you move the only pattern {\tt 01} into various positions $k$, where 
($k = 0,1,2,\ldots,n-2$ denotes the {\em offset} \textendash{} how many symbols are to the left 
of this pattern. 
The total number of such strings is equal to this sum:

\begin{equation}
\label{eq1}
S_n = \sum\limits_{k=0}^{n-2} (k+1)\cdot(n-k-1).
\end{equation}


Indeed, there are two non-increasing sequences of length $k$ and $n-k-2$. 
They can be independently selected in $k+1$ and $(n-k-2)+1$ ways respectively. 

For example, if $n=5$, we get $S = 1 \cdot 4 + 2 \cdot 3 + 3 \cdot 2 + 4 \cdot 1$. 

$\mathtt{}\textcolor{red}{\mathtt{01}}\mathtt{000}$\\
$\mathtt{}\textcolor{red}{\mathtt{01}}\mathtt{100}$\\
$\mathtt{}\textcolor{red}{\mathtt{01}}\mathtt{110}$\\
$\mathtt{}\textcolor{red}{\mathtt{01}}\mathtt{111}$\\
$\mathtt{0}\textcolor{red}{\mathtt{01}}\mathtt{00}$\\
$\mathtt{0}\textcolor{red}{\mathtt{01}}\mathtt{10}$\\
$\mathtt{0}\textcolor{red}{\mathtt{01}}\mathtt{11}$\\
$\mathtt{1}\textcolor{red}{\mathtt{01}}\mathtt{00}$\\
$\mathtt{1}\textcolor{red}{\mathtt{01}}\mathtt{10}$\\
$\mathtt{1}\textcolor{red}{\mathtt{01}}\mathtt{11}$\\
$\mathtt{00}\textcolor{red}{\mathtt{01}}\mathtt{0}$\\
$\mathtt{00}\textcolor{red}{\mathtt{01}}\mathtt{1}$\\
$\mathtt{10}\textcolor{red}{\mathtt{01}}\mathtt{0}$\\
$\mathtt{10}\textcolor{red}{\mathtt{01}}\mathtt{1}$\\
$\mathtt{11}\textcolor{red}{\mathtt{01}}\mathtt{0}$\\
$\mathtt{11}\textcolor{red}{\mathtt{01}}\mathtt{1}$\\
$\mathtt{000}\textcolor{red}{\mathtt{01}}\mathtt{}$\\
$\mathtt{100}\textcolor{red}{\mathtt{01}}\mathtt{}$\\
$\mathtt{110}\textcolor{red}{\mathtt{01}}\mathtt{}$\\
$\mathtt{111}\textcolor{red}{\mathtt{01}}\mathtt{}$\\
 
Once we know, how to express $S_n$ (the number of bit strings
of length $n$ that contain the pattern 01 only once), we can calculate its values: 

\begin{tabular}{|r|l|}\hline
$n$ & $S_n$ \\ \hline
2 & $1 = 1 \cdot 1$ \\ \hline
3 & $4 = 1 \cdot 2 + 2 \cdot 1$ \\ \hline
4 & $10 = 1 \cdot 3 + 2 \cdot 2 + 3 \cdot 1$ \\ \hline
5 & $20$ \\ \hline
6 & $35$ \\ \hline
7 & $56$ \\ \hline
8 & $84$ \\ \hline
9 & $120$ \\ \hline
10 & $165$ \\ \hline
\end{tabular}

{\bf Statement:} If $S_n$ expresses the number of bit strings of length $n$ with exactly 
one pattern {\tt 01}, then the following statement is true:
$$S_{n+1} = S_{n-1} + n^2.$$
We can prove this identity by plugging into \ref{eq1}

Summation of squares has this formula (can be proven by induction):
$$1^2 + 2^2 + \ldots + n^2 = \frac{n(n+1)(2n+1)}{6}$$
Squares of even numbers have a similar formula (four times bigger):
$$2^2 + 4^2 + \ldots + (2n)^2 = \frac{2n(2n+2)(2n+1)}{6}$$

Therefore, for odd $n$ we can add together expressions like these: 
$$S_9 = S_7 + 8^2 = (S_5 + 6^2) + 8^2 = \ldots = $$
$$ = 2^2 + 4^2 + 6^2 + 8^2$$.

For odd $n$ we have a sum of all squares:
$$S_{2n+1} = \frac{2n(2n+2)(2n+1)}{6}.$$
But a similar formula applies to even $n$ as well:
$$S_{2n} = \frac{(2n-1)(2n+1)2n}{6}.$$

So the formula that is valid for any $n$: 
$$S_n = \frac{(n-1)n(n+1)}{6} = {{n+1} \choose 3}.$$



\end{document}



