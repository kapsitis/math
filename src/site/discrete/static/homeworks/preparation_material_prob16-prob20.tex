\documentclass[jou]{apa6}

\usepackage[american]{babel}

\usepackage{csquotes}
\usepackage[style=apa,sortcites=true,sorting=nyt,backend=biber]{biblatex}
\DeclareLanguageMapping{american}{american-apa}
\addbibresource{bibliography.bib}


%%%%%%%%%%%%%%%%%%%%%%%%%%%%%%%%%%%%%%%%
%% Discrete Structures
%% The start of RBS stuff
%%%%%%%%%%%%%%%%%%%%%%%%%%%%%%%%%%%%%%%%

% Working internal and external links in PDF
\usepackage{hyperref}
% Extra math symbols in LaTeX
\usepackage{amsmath}
\usepackage{gensymb}
\usepackage{amssymb}
% Enumerations with (a), (b), etc.
\usepackage{enumerate}

\let\OLDitemize\itemize
\renewcommand\itemize{\OLDitemize\addtolength{\itemsep}{-6pt}}

\usepackage{etoolbox}
\makeatletter
\preto{\@verbatim}{\topsep=3pt \partopsep=3pt }
\makeatother

% These sizes redefine APA for A4 paper size
\oddsidemargin 0.0in
\evensidemargin 0.0in
\textwidth 6.27in
\headheight 1.0in
\topmargin -24pt
\headheight 12pt
\headsep 12pt
\textheight 9.19in



\setlength\parindent{0pt}

\title{Midterm, 2020-02-17}
\author{Discrete Structures, Fall 2020}
\affiliation{RBS}

\leftheader{Midterm, 2020-02-17}

\abstract{%
}

%\keywords{}

\begin{document}

\thispagestyle{empty}

\twocolumn
{\Large LAB02 Preparation (P16-P20)}

All problems (P16-P20) use integer arithmetic (set $\mathbb{Z}$). 
They should be enclosed in a Z-scope. And some imports and definitions
are also assumed: 

\begin{verbatim}
Require Import ZArith.
Require Import Znumtheory.
Require Import BinInt.
Require Import Znumtheory.

Open Scope Z_scope.

Lemma sample2_16 ... 
Lemma sample2_17 ... 
Lemma sample2_18 ... 
Lemma sample2_19 ... 
Lemma sample2_20 ... 

Close Scope Z_scope
\end{verbatim}

\section{Reusing Existing Work}


In many cases you have an option \textendash{} 
prove simple lemmas on your own, or find them to reuse. 

Here is a useful lemma: number $1$ divides any other integer number $a \in \mathbb{Z}$.

\begin{verbatim}
Lemma sample2_16_helper: 
    forall a: Z, (1 | a). 
Proof.
  intros a. 
  unfold Z.divide.
  exists a.
  ring.
Qed.
\end{verbatim}

If you want to find lemmas containing subexpresison $(1 \,\mid\, \ldots)$, 
you can run search:

\begin{verbatim}
Locate "|".
\end{verbatim}

This should return this answer:

\begin{verbatim}
"( x | y )" := Z.divide x y : Z_scope
        (default interpretation)
\end{verbatim}

This immediately suggests that you search for {\tt Z.divide}:

\begin{verbatim}
Search Z.divide.
\end{verbatim}

It will print out all the lemmas containing divisibility relation on 
integers (the set $\mathbb{Z}$). Beware of similar lemmas in other number
sets (e.g. "nat" - nonnegative integers). It is a different set, and 
these theorems are not applicable for your problems. 

\begin{verbatim}
Z.divide_1_l: forall n : Z, (1 | n)
\end{verbatim}

This lemma shows that each integer number is divisible by $1$. 
In our case it was easy to prove it independently, but usually 
searching for an existing result saves your time.


\section{Look Around Before you Prove}

{\bf (A)} These will print out specific lemmas or definitions:

\begin{verbatim}
Print rel_prime.
Print Z.add_simpl_l.
Print Z.mul_add_distr_r.
Print Z.pow_pos.
\end{verbatim}

\vspace{6pt}
{\bf (B)} Search all lemmas containing some concept:

\begin{verbatim}
Search Zis_gcd.
Search Z.divide.
Search Z.divide.
\end{verbatim}

\vspace{6pt}
{\bf (C)} Search algebraic patterns with wildcards. 
The first two patterns might display some lemmas how to 
open parentheses using distributivity.

\begin{verbatim}
SearchRewrite (_ * (_ + _)).
SearchRewrite ((_ + _) * _).
SearchRewrite (_ * (_ * _)).
SearchRewrite (_ + _ - _).
SearchRewrite (_ - _).
SearchRewrite (_ + _).
SearchRewrite (_ * _).
SearchRewrite (1 * _).
\end{verbatim}

If you run {\tt SearchRewrite} for a single "minus": 

\begin{verbatim}
Zminus_0_l_reverse: 
    forall n : Z, n = n - 0
Zminus_diag_reverse: 
    forall n : Z, 0 = n - n
Zplus_minus_eq: 
    forall n m p : Z, n = m + p -> p = n - m
Zminus_plus_simpl_l: 
    forall n m p : Z, p + n - (p + m) = n - m
Zminus_plus_simpl_l_reverse: 
    forall n m p : Z, n - m = p + n - (p + m)
Zminus_plus_simpl_r: 
    forall n m p : Z, n + p - (m + p) = n - m
Zeq_minus: forall n m : Z, 
    n = m -> n - m = 0
Lemma rel_prime_bezout : 
  forall a b:Z, rel_prime a b -> Bezout a b 1.
\end{verbatim}


\vspace{6pt}
{\bf (D)} Locate some notations: 

\begin{verbatim}
Locate "|".
Locate "^".
\end{verbatim}




\vspace{20pt}
{\bf Problem 16.} If $\text{gcd}(a,b)=1$ and $c$ divides $a$, then $\text{gcd}(b,c)=1$.

{\bf Hints:} Use the definition of GCD (greatest common divisor) in your hypothesis: We know that every $k$ that is a divisor of 
both $a$ and $b$ is equal to $1$. 
Now assume that there is some number (possibly a different $k'$) that divides both $b$ and $c$. 
You must show that it also must be equal to $1$. 

You may need to prove or find the following lemma:\\
$\forall a,b,c \in \mathbb{Z},\; (a \,\mid\, b) \rightarrow (b \,\mid\,  c) \rightarrow (a \,\mid\, c)$.
Namely, the divisibility relation is {\em transitive}: If $a$ divides $b$ and $b$ divides $c$, then 
$a$ divides $c$. 

If you see {\tt Zis\textunderscore{}gcd} as one of the hypotheses, you can destruct the hypothesis:
\begin{verbatim}
destruct H as [_ _ H1].
\end{verbatim}
Here the underscores are placeholders (they tell that you will not use these hypotheses, so you are not
giving them any names). 

If you see {\tt Zis\textunderscore{}gcd} in your goal, then type this tactic: 
\begin{verbatim}
apply Zis_gcd_intro.
\end{verbatim}
It will create a different goal - one that actually describes the meaning of the
greatest common divisor. 





\vspace{20pt}
{\bf Problem 17.} If $\text{gcd}(a,b)=1$, then 
$\text{gcd}(ac,b)=\text{gcd}(c,b)$. 

{\bf Hints:} This theorem is easiest to solve using Bezout's indentity. 
You may need a warm-up \textendash{} several lemmas:

\begin{enumerate}[(A)]
\item $\forall n \in \mathbb{Z},\; 1 \cdot n = n$.
\item $\forall a, b, c \in \mathbb{Z},\; (a \,\mid\, b) \rightarrow (a \,\mid\, c \cdot b)$.
\item $\forall a,b,c,k \in \mathbb{Z},\; 
\text{gcd}(a,b) = 1 \rightarrow (k \,\mid\, a \cdot c) \rightarrow (k \,\mid\, b) \rightarrow (k \,\mid\, c)$.
\end{enumerate}

The first two lemmas are very easy. The last lemma may need to use Bezout's identity: Once $a$ and $b$ are mutual 
primes, there should be integers $x,y$ such that $ax + by = 1$. This can be used to prove that $k$
must divide $c$ (if we already know that $k$ divides $a \cdot c$ and $b$. 

In order to prove the last lemma (and also the Problem 17 itself) you may need to use
several lemmas (that already exist in Coq). Please take a look at these:

\begin{verbatim}
rel_prime_bezout
mul_1_left
Z.mul_add_distr
Z.mul_assoc
div_multiple_left
\end{verbatim}


\vspace{20pt}
{\bf Problem 18.} If $\text{gcd}(a,b)=1$ and $c$ divides $(a+b)$, then $\text{gcd}(a,c)=\text{gcd}(b,c)=1$.

{\bf Hints.} This theorem may need a few easy lemmas:

\begin{enumerate}[(A)]
\item If $a = b$ and $c = d$ then $a-c = b-d$.
\item If $\text{gcd}(a,b) = 1$ and $c$ divides $a+b$, then  $\text{gcd}(a,c) = 1$. 
\end{enumerate}

A few more predefined lemmas may be useful. For example, 

\begin{verbatim}
Z.divide_1_l
Z.add_comm
Zis_gcd_sym
\end{verbatim}

\vspace{20pt}
{\bf Problem 19.}
If $\text{gcd}(a,b)=1$; $d$ divides $ac$; $d$ divides $bc$, then $d$ divides $c$.

{\bf Hints.} You may need to state the Bezout's identity, then "destruct" it \textendash{} 
find those specific $x,y$ which satisfy $ax+by=1$. Then you can 
find out about divisors of $c$ as well. 

\vspace{20pt}
{\bf Problem 20.} If $\text{gcd}(a,b)=1$, then $\text{gcd}(a^2,b^2)=1$.

{\bf Hints.} Bezout's identity may work. But you may also want to prove a chain 
of equalities: $\text{gcd}(a,b)= \text{gcd}(a,b^2)= \text{gcd}(a^2,b^2)$. 
In order to prove this, you need to use the fact that $a$,$b$ are mutually prime.


\vspace{20pt}
{\bf Problem 15.} In this task you need to define a function (using recursive "fixpoint") to 
compute the result of some procedure where we remove numbers from a list. 
See \url{https://bit.ly/2VhNikC} for details. 



\end{document}

