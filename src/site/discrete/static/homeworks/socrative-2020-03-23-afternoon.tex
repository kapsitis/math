\documentclass[jou]{apa6}

\usepackage[american]{babel}

\usepackage{csquotes}
\usepackage[style=apa,sortcites=true,sorting=nyt,backend=biber]{biblatex}
\DeclareLanguageMapping{american}{american-apa}
\addbibresource{bibliography.bib}


%%%%%%%%%%%%%%%%%%%%%%%%%%%%%%%%%%%%%%%%
%% Discrete Structures
%% The start of RBS stuff
%%%%%%%%%%%%%%%%%%%%%%%%%%%%%%%%%%%%%%%%

% Working internal and external links in PDF
\usepackage{hyperref}
% Extra math symbols in LaTeX
\usepackage{amsmath}
\usepackage{gensymb}
\usepackage{amssymb}
% Enumerations with (a), (b), etc.
\usepackage{enumerate}

\let\OLDitemize\itemize
\renewcommand\itemize{\OLDitemize\addtolength{\itemsep}{-6pt}}

\usepackage{etoolbox}
\makeatletter
\preto{\@verbatim}{\topsep=3pt \partopsep=3pt }
\makeatother

% These sizes redefine APA for A4 paper size
\oddsidemargin 0.0in
\evensidemargin 0.0in
\textwidth 6.27in
\headheight 1.0in
\topmargin -24pt
\headheight 12pt
\headsep 12pt
\textheight 9.19in



\title{Sample Quiz 8}
\author{Discrete Structures, Spring 2020}
\affiliation{RBS}

\leftheader{Discrete Sample Quiz 8}

\abstract{%
}

%\keywords{}

\setlength\parindent{0pt}

\begin{document}

\thispagestyle{empty}

\twocolumn
{\Large Socrative, 2020-03-23 Afternoon}

\vspace{6pt}
{\bf Question 1}\\
We have a set of $5$ numbers: $A = \{ 1,2,3,4,5 \}$. 
Let $R$ be a binary relation on $A$ such that $aRb$
iff $|a - b| = 1$ (namely, the distance between the numbers
$a$ and $b$ is exactly $1$ \textendash{} both numbers
are next to each other).  The matrix of this relation is shown 
in the figure.
$$M_R = \left( \begin{array}{ccccc}
0 & 1 & 0 & 0 & 0 \\
1 & 0 & 1 & 0 & 0 \\
0 & 1 & 0 & 1 & 0 \\
0 & 0 & 1 & 0 & 1 \\
0 & 0 & 0 & 1 & 0 \\
\end{array} \right)$$
Now consider the 2nd power of the relation $R$: $R^2$. 
How many 1's are there in the matrix of $R^2$?


\vspace{6pt}
{\bf Question 2}\\
Consider the following binary relation $R$ in a set of $4$ elements that 
is defined by the following matrix: 
$$M_R = \left( \begin{array}{cccc}
0 & 0 & 0 & 1 \\
1 & 0 & 1 & 0 \\
1 & 0 & 0 & 1 \\
0 & 0 & 1 & 0 \\
\end{array} \right)$$
How many 1's are there in the transitive closure $R^{\ast}$? 


\vspace{6pt}
{\bf Question 3}\\
Hasse diagram showing the divisibility relations for numbers 
$\{ 0,1,\ldots,9 \}$ is shown in the figure. 
What is the total number of valid divisibility relations 
in this set? In other words, how many links you get, if you 
compute transitive closure of this Hasse diagram (and also 
add a reflexive loop from each number to itself, except $0$
does not have a loop to itself). 

%% 13 original links in the Hasse diagram
%% 9 loops
%% 8 new


\end{document}

