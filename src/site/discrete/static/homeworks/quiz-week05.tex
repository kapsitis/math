\documentclass[jou]{apa6}

\usepackage[american]{babel}

\usepackage{csquotes}
\usepackage[style=apa,sortcites=true,sorting=nyt,backend=biber]{biblatex}
\DeclareLanguageMapping{american}{american-apa}
\addbibresource{bibliography.bib}


%%%%%%%%%%%%%%%%%%%%%%%%%%%%%%%%%%%%%%%%
%% Discrete Structures
%% The start of RBS stuff
%%%%%%%%%%%%%%%%%%%%%%%%%%%%%%%%%%%%%%%%

% Working internal and external links in PDF
\usepackage{hyperref}
% Extra math symbols in LaTeX
\usepackage{amsmath}
\usepackage{gensymb}
\usepackage{amssymb}
% Enumerations with (a), (b), etc.
\usepackage{enumerate}

\let\OLDitemize\itemize
\renewcommand\itemize{\OLDitemize\addtolength{\itemsep}{-6pt}}

\usepackage{etoolbox}
\makeatletter
\preto{\@verbatim}{\topsep=3pt \partopsep=3pt }
\makeatother

% These sizes redefine APA for A4 paper size
\oddsidemargin 0.0in
\evensidemargin 0.0in
\textwidth 6.27in
\headheight 1.0in
\topmargin -24pt
\headheight 12pt
\headsep 12pt
\textheight 9.19in



\title{Sample Quiz 5}
\author{Discrete Structures, Fall 2020}
\affiliation{RBS}

\leftheader{Discrete Sample Quiz 5}

\abstract{%
}

%\keywords{}

\begin{document}

\thispagestyle{empty}

\twocolumn
{\Large Discrete Quiz 5}

\vspace{6pt}
{\bf Question 1.} Do the prime factorization for this 6-digit positive integer: 
$510,510$. 

Write your answer as a product of prime powers {\tt p\^{}a*q\^{}b*r\^{}c} or similar. 
Numbers $p,q,r$ etc. should be in increasing order. All exponents (even those equal to $1$) should be written explicitly.


\vspace{6pt}
{\bf Question 2.} Are these statements true or false (do not forget that ``all integers'' include also negative numbers):
\begin{enumerate}[(A)] 
\item For all integers $a,b,c$, if $a|c$ and $b|c$, then $(a + b)|c$.
\item For all integers $a,b,c,d$, if $a|b$ and $c|d$, then $(ac)|(b + d)$.
\item For all integers $a,b$, if $a|b$ and $b|a$, then $a = b$.
\item For all integers $a,b,c$, if $a|(b + c)$, then $a|b$ and $a|c$.
\item For all integers $a,b,c$, if $a|bc$, then $a|b$ or $a|c$.
\item If $p$ and $q$ are primes ($> 2$), then $pq + 1$ is never prime.
\end{enumerate}

Write your answer as a comma-separated string of T/F. For example, {\tt T,T,T,T,T,T}.\\
{\em Note.} Even though you only write the answers, 
make sure that you are able to justify your answer. For true statements you should be able to 
find a reasoning; for false ones \textendash{} a counterexample. 


\vspace{6pt}
{\bf Question 3.} Find $\text{lcm}(24^{75},75^{24})$ 

Write your answer as a product of prime powers {\tt p\^{}a*q\^{}b*r\^{}c} or similar. 
Numbers $p,q,r$ etc. should be in increasing order. All exponents (even those equal to $1$) should be written explicitly.


\vspace{6pt}
{\bf Question 4.}
Convert $(100\,1100\,0011)_2$ to base $16$, base $8$ and base $7$. 

Write your answer as $3$ comma-separated numbers. For the hexadecimal notation use capital letters A,B,C,D,E,F.


\vspace{6pt}
{\bf Question 5.}
Express the infinite periodic binary fraction $0.0001100110011\ldots_2 = 0.0(0011)$ as a rational number.\\
{\em Note 1.} In binary fractions a digit that is $k$ places after the point is multiplied by $2^k$. 
For example, $0.1_2$ means $1/2$; $0.01_2$ means $1/4$ and so on.\\
{\em Note 2.} You may need to use infinite geometric progression to find its value.

Write your answer as {\tt P/Q}, where $P,Q$ are both in decimal notation. 


\vspace{6pt}
{\bf Question 6.} Find the sum and the product of these two integers written in binary: 
$(101011)_2$, $(1101011)_2$.\\
{\em Note.} You may want to try the addition and multiplication algorithm directly in 
binary (without converting them into the decimal and back to the binary).

Write your answer as two comma-separated numbers (both written in binary). 


\vspace{6pt}
{\bf Question 7.} Write the first 10 powers of number $3$ modulo $11$: $3^1,3^2,3^3,\ldots,3^{10}$. 

Write your answer as a comma-separated list of ten remainders (mod $11$), \textendash{} numbers between $0$ and $10$.


\vspace{6pt}
{\bf Question 8.} Alice has only 21-cent coins, 
Bob has 34-cent coins. Alice wants to pay Bob exactly $1$ cent. 
Find two non-negative integers $s,t$ that satisfy $21s - 34t = 1$. 

Write your answer as two comma-separated integers. 

\vspace{6pt}
{\bf Question 9.} Consider the opposite situation: Alice has only 34-cent coins, 
Bob has only 21-cent coins. Find two non-negative integers 
$s,t$ that satisfy $34s - 21t = 1$. 

Write your answer as two comma-separated integers. 

\vspace{6pt}
{\bf Question 10.} Find $21^{-1}$ modulo $34$. (This is a number $z$ between $0$ and $33$ such that
$21z \equiv 1\;(\text{mod}\,34)$.)

Write your answer as a number modulo $34$ (i.e. between $0$ and $33$). 

\vspace{6pt}
{\bf Question 11.} Solve the congruence equation $21x \equiv 11\;(\text{mod}\,34)$. 

Write your answer as a number modulo $34$ (i.e. between $0$ and $33$). 

\vspace{6pt}
{\bf Question 12.} Solve the Bezout identity for the numbers $a=390$,$b=72$: Find any integers $x,y$
satisfying the equation $390x + 72y = \text{gcd}(390,72)$. 

Write your answer as two comma-separated integers.


\newpage

\section{Answers}

{\bf Question 1.}\\ 
Answer: {\tt 2\^{}1*3\^{}1*5\^{}1*7\^{}1*11\^{}1*13\^{}1*17\^{}1}\\
When factorizing a number with computer, try dividing it by small primes $2,3,5,7,\ldots$
(if divisible, then divide by that prime, and try to divide by that prime again, and by all 
larger primes). Do so, until you reacn $\sqrt{n}$.

If you wish to factorize large ``symmetric-looking'' numbers even faster (where the same 
fragment repeats itself many times), notice
that $510510 = 510 \cdot 1001$. Then 
factorize each of these numbers separately. 

\vspace{6pt}
{\bf Question 2.} Answer: {\tt F,F,F,F,F,T}\\
{\bf (A)} $2 \mid 6$ and $3 \mid 6$, but $(2+3) \not\mid 6$ (i.e.\ $5$ does not divide $6$).\\
{\bf (B)} $2 \mid 2$ and $3 \mid 3$, but $(2 \cdot 3) \not\mid (2+3)$ (i.e.\ $6$ does not divide $5$).\\
{\bf (C)} For positive $a,b$ this would be true, but for $\mathbb{Z}$ it is not: $2 \mid (-2)$ and $(-2) \mid 2$, 
but $2 \neq -2$. 
{\bf (D)} $5 \mid (3+7)$, but $5 \not\mid 3$ and also $5 \not\mid 7$; so you can make even both conclusions fail, 
not just one of them.
{\bf (E)} $6 \mid 2 \cdot 15$, but $6 \not\mid 2$ and also $6 \not\mid 15$. (This statement 
$(a | bc) \rightarrow ((a | b) \vee (a | c))$ would be true iff $a$ is a prime number. Then it is called
{\em Euclid's Lemma}. But it is false for all non-prime $a$).
{\bf (F)} The last statement is true, because $pq+1$ would be an even number bigger than $2$, so it cannot be prime.

We summarize, that all statements can have counterexamples, but the last one is always true.

\vspace{6pt}
{\bf Question 3.} Answer: {\tt 2\^{}225*3\^{}75*5\^{}48}\\
$\text{lcm}(24^{75},75^{24}) = \text{lcm}(2^{225}\cdot{}3^{75},3^{24}\cdot{}5^{48})$. Then take the maximal 
values for each prime power $2^a$, $3^b$, $5^c$ in both numbers.

\vspace{6pt} 
{\bf Question 4.} Answer: {\tt 4C3,2303,3361}\\
Hexadecimal notation can be obtained, if we group digits by four
(from the end of the number): {\tt 100:1100:0011}. Encode each group of digits: {\tt 4C3}.\\
Octal notation can be obtained as we group digits by three: 
{\tt 10:011:000:011}. Encode each group: {\tt 2303}.\\
Decimal notation is $\mathtt{4} \cdot 16^2 + \mathtt{C} \cdot 16^1 + \mathtt{3} = 
4 \cdot 256 + 12 \cdot 16 + 3 = 1219$. We can divide $1219$ by $7$ and each time record the remainder:
\begin{verbatim}
1219:7 = 174, R.1
174:7 = 24, R.6
24:7 = 3, R.3
3:7 = 0, R.3
\end{verbatim}
Write all the remainders from right to left: {\tt 3361}. This is the 
representation of $1219$ in base $7$. You can check this by Horner's scheme:
$$((((\mathtt{3}) \cdot 7 + \mathtt{3})\cdot 7 + \mathtt{6})\cdot 7 + \mathtt{1}) = 1219.$$

\vspace{6pt}
{\bf Question 5.} Answer: {\tt 1/10}\\
$$\alpha = 0.0001100110011\ldots_2 = \frac{1}{2}\cdot 0.001100110011\ldots_2 = $$
$$= \frac{1}{2} \cdot 3 \cdot 0.000100010001\ldots_2 = $$
$$= \frac{3}{2} \cdot \left( \frac{1}{16} + \frac{1}{16^2} + \frac{1}{16^3} + \ldots \right).$$
Apply the formula of infinite geometric progression:
$$\left( \frac{1}{16} + \frac{1}{16^2} + \frac{1}{16^3} + \ldots \right) = \frac{\frac{1}{16}}{1-\frac{1}{16}} = \frac{1}{15}.$$
We get $\alpha = \frac{3}{2} \cdot \frac{1}{15} = \frac{3}{30} = \frac{1}{10}$.

{\em Note.} This example has some practical implications: the floating point number $0.1$ 
looks short and simple in decimal system. But it is an infinite periodic fraction when 
written in binary. Therefore it is stored with a rounding error in 
computer memory; doing arithmetic on such numbers may cause these errors to accumulate.

\vspace{6pt}
{\bf Question 6.} Answer: {\tt 10010110,1000111111001}\\
Fastest way is adding (or multiplying) in binary notation using grid paper. 
If we want to double-check the result, we can convert each number into decimal:
$$101011_2 = 43_{10};\;\;1101011_2 = 107_{10}.$$
Sum is $150$ and the product is $4601$. 
Then convert back both numbers $(150, 4601)$ into binary.

\vspace{6pt}
{\bf Question 7.} Answer: {\tt 3,9,5,4,1,3,9,5,4,1}\\
Every next remainder is the previous remainder multiplied by $3$
modulo $11$. 
To avoid operating with large numbers, we immediately reduce
each power $3^{k+1} = 3^k \cdot 3$ as a remainder (between $0$ and $10$).

These remainders form a period; after a period of $5$, the remainders
repeat indefinitely. (Accordingly to the {\em Little Fermat Theorem}, 
$a^{10}$ should be congruent $1$ (mod $11$) 
for any $a$ not divisible by $11$; and after that the remainders
will be periodic.
So, even for other $a \neq 3$ something similar should happen.)

\vspace{6pt}
{\bf Question 8.} Answer: {\tt 13,8}\\
Observe that $\text{gcd}(21,34)=1$, so the numbers $21$ and $34$ are mutual primes. 
We can guess these coefficients $s,t$ (or try out various $21s$ until 
it gives the remainder $1$, when divided by $34$). If we want to proceed by 
Blankinship's algorithm, it will work efficiently for any numbers:

$$\left( \begin{array}{ccc}
21 & 1 & 0 \\
34 & 0 & 1 \end{array} \right) 
\leadsto
\left( \begin{array}{ccc}
21 & 1 & 0 \\
13 & -1 & 1 \end{array} \right) 
\leadsto$$

$$
\leadsto
\left( \begin{array}{ccc}
8 & 2 & -1 \\
13 & -1 & 1 \end{array} \right) 
\leadsto
\left( \begin{array}{ccc}
8 & 2 & -1 \\
5 & -3 & 2 \end{array} \right) 
\leadsto
$$

$$
\leadsto
\left( \begin{array}{ccc}
3 & 5 & -3 \\
5 & -3 & 2 \end{array} \right) 
\leadsto
\left( \begin{array}{ccc}
3 & 5 & -3 \\
2 & -8 & 5 \end{array} \right) 
\leadsto
$$

$$\leadsto
\left( \begin{array}{ccc}
1 & 13 & -8 \\
2 & -8 & 5 \end{array} \right) 
\leadsto
\left( \begin{array}{ccc}
1 & 13 & -8 \\
0 & -34 & 21 \end{array} \right).$$

The last two tables have one of the two rows:
$1$, $13$, $-8$. This means that we are able
to get number $1$, by using coefficients $13$ 
and $-8$ respectively: 
$$21 \cdot (13) + 34 \cdot (-8).$$
Therefore $s = 13; t = -8$. 

There are infinitely many other answers (you can get 
all of them, if you increment $s$ by $34k$ and 
$t$ by $21k$. Then both changes will cancel out: 
$$(13,8);\;\;(47,29);\;\;(81,50);\ldots$$


\vspace{6pt}
{\bf Question 9.} Answer: {\tt 13,21} or {\tt 34,55} etc.\\
From the previous question we already know that $21s + 34t = 1$
for $(s,t) = (13,-8)$. In fact, there are infinitely 
many pairs $(s,t)$ satisfying that equation $21s + 34t = 1$. 
If we decrease $s$ by $34$ and increase 
$t$ by $21$, then the expression does not change; then 
we can decrease/increase again, and so on.

Let us perform that step once: $(s_2,t_2) = (13-34,-8+21) = (-21,13)$.
Therefore $21s_2 + 34t_2 = 1$ or $21\cdot (-21) + 34 \cdot 13 = 1$. 
Therefore Alice can pay with $13$ 34-cent coins and get back 
$21$ 21-cent coins. This would also let her pay $1$ cent.

\vspace{6pt}
{\bf Question 10.} Answer: {\tt 13}\\
We rewrite the solution obtained from Question 8.
Since $21s - 34t = 1$ (for $s=13$, $t=8$), 
we can compute remainders
from both sides:
$$1 = 21s - 34t \;\equiv\; 21s\;(\text{mod}\,34).$$
Therefore $s=13$ satisfies $21 \cdot s \equiv 1$.  

\vspace{6pt}
{\bf Question 11.} Answer: {\tt 7}\\
We know that $21^{-1} = 13$ (mod $34$) from the 
previous exercise. Now we can solve the 
congruence equation: 
$21x \equiv 11\;(\text{mod}\;34)$;\\
$21^{-1} \cdot 21x \equiv 21^{-1} \cdot 11\;(\text{mod}\;34)$;\\
$1x \equiv 13 \cdot 11\;(\text{mod}\;34)$;\\
$x \equiv 143 \equiv 7\;(\text{mod}\;34)$.

\vspace{6pt}
{\bf Question 12.} Answer: {\tt 5,-27}\\
We can easily guess that $\text{gcd}(390,72)=6$ and
with some trial and error we can find that
$$390 \cdot (5) + 72\cdot (-27) = 6.$$

To show an algorithmic way, 
we could do the row operations like in Blankinship's algorithm again: 
start from the table:
$$\left( \begin{array}{ccc}
390 & 1 & 0 \\
72 & 0 & 1 \end{array} \right) 
\leadsto \ldots$$
But let us show another (less formal) method: write
the regular Euclidean algorithm (and preserve 
coefficients before $390$ and $72$):
\begin{itemize}
\item $30 = 390 - 5 \cdot 72$.
\item $12 = 72 - 2 \cdot 30 = $\\
$= 72 - 2 \cdot (390 - 5 \cdot 72) =$\\
$= 11 \cdot 72 - 2 \cdot 390$;
\item $6 = 30 - 2 \cdot 12 = $\\
$(390 - 5 \cdot 72) - 2 \cdot (11 \cdot 72 - 2 \cdot 390) =$\\
$=5 \cdot 390 -27 \cdot 72$.
\end{itemize}


\end{document}

