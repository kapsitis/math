\documentclass[jou]{apa6}

\usepackage[american]{babel}

\usepackage{csquotes}
\usepackage[style=apa,sortcites=true,sorting=nyt,backend=biber]{biblatex}
\DeclareLanguageMapping{american}{american-apa}
\addbibresource{bibliography.bib}


%%%%%%%%%%%%%%%%%%%%%%%%%%%%%%%%%%%%%%%%
%% Discrete Structures
%% The start of RBS stuff
%%%%%%%%%%%%%%%%%%%%%%%%%%%%%%%%%%%%%%%%

% Working internal and external links in PDF
\usepackage{hyperref}
% Extra math symbols in LaTeX
\usepackage{amsmath}
\usepackage{gensymb}
\usepackage{amssymb}
% Enumerations with (a), (b), etc.
\usepackage{enumerate}

\let\OLDitemize\itemize
\renewcommand\itemize{\OLDitemize\addtolength{\itemsep}{-6pt}}

\usepackage{etoolbox}
\makeatletter
\preto{\@verbatim}{\topsep=3pt \partopsep=3pt }
\makeatother

% These sizes redefine APA for A4 paper size
\oddsidemargin 0.0in
\evensidemargin 0.0in
\textwidth 6.27in
\headheight 1.0in
\topmargin -24pt
\headheight 12pt
\headsep 12pt
\textheight 9.19in



\title{Sample Quiz 8}
\author{Discrete Structures, Spring 2020}
\affiliation{RBS}

\leftheader{Discrete Sample Quiz 8}

\abstract{%
}

%\keywords{}

\setlength\parindent{0pt}

\begin{document}

%\thispagestyle{empty}

\twocolumn
\section{Quiz 8: Probabilities}

\vspace{10pt}
{\bf Question 1} 
Two chess queens are placed on two different places in a $4 \times 4$ chess-board.
Assume that all the ${16 \choose 2}$ possibilities how they are placed have equal probabilities.
Find a probability that one queen {\em attacks} the other. (Two queens attack each other, if
they are located on the same horizontal, vertical or diagonal).

Write your answer as a rational fraction: {\tt P/Q}

\vspace{10pt}
{\bf Question 2} Assume that you are generating 10-bit sequences (a string of 0's and 1's). 
All the $2^{10}$ sequences have equal probabilities. 
Find the probability of an event that the 10-bit sequence does NOT contain two consecutive 0's anywhere.

Write your answer as a rational fraction: {\tt P/Q}

\vspace{10pt}
{\bf Question 3}
Two players have 4 cubic dices. Instead of the usual numbers, their faces have the following numbers on their faces:\\
{\bf Dice A:} 4, 4, 4, 4, 0, 0;\\
{\bf Dice B:} 3, 3, 3, 3, 3, 3;\\
{\bf Dice C:} 6, 6, 2, 2, 2, 2;\\
{\bf Dice D:} 5, 5, 5, 1, 1, 1.

In a single round Players Alice and Bob randomly select two of the dices (with equal probabilities they can 
select any of the six pairs - $(A,B)$, or $(A,C)$, or $(A,D)$, or $(B,C)$, or $(B,D)$, or $(C,D)$).
There can be three outcomes:\\
{\bf Outcome 1:} They have selected "opposite dices" - pairs $(A,C)$ or $(B,D)$. In this case the payoff
is zero (nobody pays anything to the other).\\
{\bf Outcome 2:} The dices win in the "clockwise manner" ($A>B$ or $B>C$ or $C>D$ or $D>A$) - then Alice
wins 1 euro.\\
{\bf Outcome 3:} The dices win in the "counter-clockwise manner" ($B>A$ or $C>B$ or $D>C$ or $A>D$) - then Bob
wins 1 euro.\\
({\em Note.} The expression $A>B$ means that the number that rolled out on the the dice $A$ was larger than the number on dice $B$; 
but $B>A$ denotes the opposite event.)

Find the expected value - how much money Alice is expected to win in a single round of such a game.

Write your answer as a rational fraction: {\tt P/Q}\\
For example, if the expected win for Alice is 0.10 EUR, then write {\tt 1/10}. If Alice is expected on average to lose 0.10 EUR
per one round of this game, then write {\tt -1/10}. 




\vspace{10pt}
{\bf Question 4} For every year we count the number of Friday's that fall on the 13th date of some month
(such as Friday, March 13, 2020). Denote this count by $X$ \textendash{} it is your random variable.
Find the expected value and the variance of $X$. Round them to the nearest thousandth. 

Write your answer as two comma-separated numbers: {\tt D.DDD,D.DDD}.

\vspace{10pt}
{\bf Question 5} What is the
probability that a randomly chosen positive integer between $1$ and $600$ is not divisible 
by either $6$ or $10$?

Write your answer as a rational fraction: {\tt P/Q}


\vspace{10pt}
{\bf Question 6} A chip factory {\em Intel} adds one toy animal to every bag of chips.
There are three sorts of animals - Aligators, Bears or Cats (each one appears with probability $p=1/3$). 
Find the expected number of the chip bags one needs to purchase to collect all three animals.

Write your answer as a rational fraction: {\tt P/Q}

\vspace{10pt}
{\bf Question 7} You create a random bit string of length five (all $32$ bit strings are equally probable). Consider
these events:\\
$E_1$: the bit string chosen begins with $1$;\\
$E_3$: the bit string chosen has exactly three $1$’s.\\
{\bf (A)} Find $p(E_1 \,\mid\, E_3)$.\\
{\bf (B)} Find $p(E_3 \,\mid\, E_1)$.

Write your answer as a comma-separated rational fractions {\tt P1/Q1,P2/Q2}






\newpage

\subsection{Answers}

\vspace{10pt}
{\bf Question 1.} Answer: {\tt 19/30}

To make the counting easier, let us assume that both queens are distinguishable 
(we can have $Q_1$ and $Q_2$ as white and black queen). 
Then the ways to place them are $16 \cdot 15$ \textendash{} which is twice the number 
of combinations ${16 \choose 2} = \frac{15 \cdot 16}{1 \cdot 2}$. 

\begin{verbatim}
Q x x x    x Q x x    x x x .
x x . .    x x x .    x Q x x
x . x .    . x . x    x x x .
x . . x    . x . .    . x . x
\end{verbatim}

Queen $Q_1$ can be placed in a corner of the $4 \times 4$ square ($4$ chances out of $16$), 
on an edge ($8$ chances out of $16$) or near the center ($4$ chances out of $16$)
\textendash{} see the pictures above.
In the first two cases queen $Q_1$ attacks $9$ places (out of $15$). In the last case
it attacks $11$ places (out of $15$). The ultimate probability that $Q_1$ attacks $Q_2$ 
is
$$\frac{4}{16} \cdot \frac{9}{15} + \frac{8}{16} \cdot \frac{9}{15} + \frac{4}{16} \cdot \frac{11}{15} = \frac{152}{16 \cdot 15} = 
\frac{19}{30}.$$

We could also count the mutual attack positions in the original problem (when $Q_1$ and $Q_2$ 
are indistinguishable). But we would get the same result, since attacking positions
are symmetric (if $Q_1$ attacks $Q_2$, then $Q_2$ attacks $Q_1$).





\vspace{10pt}
{\bf Question 2.} Answer: {\tt 9/64}\\
We can count all the sequences that do not contain two consecutive 0's: \url{https://bit.ly/2TPwMaj}.
If we denote by $f_0(n)$ the count of all $n$-bit sequences that do not contain two consecutive 0s, 
we can prove that 
$$f_0(0)=1,\;f_0(1)=2,\;f_0(2)=3,\;f_0(3)=5,\ldots$$
For arbitrary $n$ we get $f_0(n) = F_{n+2}$, where $F_{n}$ ($0,1,1,2,3,\ldots$) is the 
Fibonacci sequence. 

The $12$th member of the Fibonacci sequence is $F_{12}=144$. 
Therefore the proportion of $10$-bit sequences equals $\frac{144}{1024}=\frac{9}{64}$.






\vspace{10pt}
{\bf Question 3.} Answer: {\tt 2/9}\\
There is $1/3$ probability to pick $(A,C)$ or $(B,D)$ (payoff is $0$ in this case).\\
Regarding the other four pairs, the probabilities are the following:
\begin{itemize}
\item $A$ wins $B$ with probability $2/3$.
\item $B$ wins $C$ with probability $2/3$.
\item $C$ wins $D$ with probability $\frac{1}{3} \cdot \frac{1}{1} + \frac{2}{3} \cdot \frac{1}{2}=\frac{2}{3}$:
{\bf Either} we have $C=6$ (probability $1/3$); and it wins with probability $1$. {\bf Or} we have $C=2$ (probability $2/3$) 
and it wins with probability $1/2$ (whenever $D=1$).
\item $D$ wins $A$ with probability $\frac{1}{2} \cdot \frac{1}{1} + \frac{1}{2} \cdot \frac{1}{3}=\frac{2}{3}$:
{\bf Either} we have $D=5$ (probability $1/2$); and it wins with probability $1$. {\bf Or} we have $D=1$ (probability $1/2$) 
and it wins with probability $1/3$ (whenever $A=0$).
\end{itemize}

For each pair $(A,B)$, $(B,C)$, $(C,D)$, or $(D,A)$ the expected payoff for Alice is
$$\frac{2}{3}(+1\,\text{EUR}) + \frac{1}{3}(-1\,\text{EUR}) = \frac{1}{3}\,\text{EUR}.$$

The expected probability for Alice to win in a single round can be obtained 
as a sum, where we multiply probabilities for each pair  
($(A,C)$, $(B,D)$, $(A,B)$, $(B,C)$, $(C,D)$, or $(D,A)$) with their respective payoffs (either 0 or 1/3 euro):
$$\frac{1}{6} \cdot 0 + \frac{1}{6} \cdot 0 + \frac{1}{6} \cdot \frac{1}{3}
+ \frac{1}{6} \cdot \frac{1}{3} + \frac{1}{6} \cdot \frac{1}{3} + \frac{1}{6} \cdot \frac{1}{3} = \frac{2}{9}.$$


\vspace{10pt}
{\bf Question 4.} Answer: {\tt 1.714,0.490}\\
Assume that each year (and therefore each month) can start on 
each weekday with exactly the same probability (that is $1/7$). 
Every month has date $13$, so there is a $1/7$ probability that 
this date will happen on Friday. 

{\bf (A)} If $X$ is the number of months having 13th date on Friday, 
then $E(X)$ must be $12/7 = 1.714$. It is not because the
weekdays that start months in a single year are independent (which they are not!), 
but rather because every $28$-year cycle 
should have the same number of each type of month (January, February, etc.) 
starting on each weekday: exactly $4$ Januaries starting on Monday, 
exactly $4$ Januaries starting on Tuesday, etc. ($4+4+\ldots+4 = 28$).
Only those Januaries that start on Sundays will have $13$th date on Friday. 

{\bf (B)} Let us create a table reflecting the number of Fridays on 13th 
depending on whether the year is the leap year and the weekday it starts. 

\begin{tabular}{|l|l|l|} \hline
Year & Non-leap & Leap \\ \hline
Sun & 2 (Jan,Oct) & 3 (Jan,Apr,Jul) \\ \hline
Mon & 2 (Apr,Jul) & 2 (Sep,Dec) \\ \hline
Tue & 2 (Sep,Dec) & 1 (Jun) \\ \hline
Wed & 1 (Jun) & 2 (Mar,Nov) \\ \hline
Thu & 3 (Feb,Mar,Nov) & 2 (Feb,Aug) \\ \hline
Fri & 1 (Aug) & 1 (May) \\ \hline
Sat & 1 (May) &  1 (Oct) \\ \hline
\end{tabular}

The probability to get non-leap year starting on a certain 
weekday is $\frac{3}{4}\cdot\frac{1}{7} = \frac{3}{28}$. 
The probability to get leap year starting on a certain weekday
is $\frac{1}{28}$.

We find the variance $V(X)$ using its definition:
$$V(X) = \sum (x_i - E(X))^2 p(x_i) = $$
$$ = \frac{3}{28}\left( \left( 3 - \frac{12}{7} \right)^2 + 
3\left( 2 - \frac{12}{7} \right)^2 + 3\left( 1 - \frac{12}{7} \right)^2 \right) + $$ 
$$ + \frac{1}{28}\left( \left( 3 - \frac{12}{7} \right)^2 + 
3\left( 2 - \frac{12}{7} \right)^2 + 3\left( 1 - \frac{12}{7} \right)^2 \right) = $$ 
$$ = \frac{1}{7}\cdot\frac{9^2}{7^2} + \frac{3}{7}\cdot\frac{2^2}{7^2} + 
\frac{3}{7}\cdot\frac{(-5)^2}{7^2} = \frac{168}{7^3} = \frac{24}{49} 
\approx 0.490.$$ 






{\em Note.} In fact, the assumption that a year starts with each 
weekday with a probability $1/7$ is (slightly) false.
\url{https://bit.ly/2vo0YQt} explains that each $400$ year cycle in Gregorian calendar
repeats the same weekdays. Namely, the calendar for year $1600$ is identical to 
the calendar of year $2000$; year $1620$ starts on the same weekday as 
year $2020$, and so on.

To verify this, notice that exactly $97$ of all $400$ years are leap years:
({\bf either} years divisible by $4$, but not with $100$, {\bf or} years divisible
by $400$. During this $400$ year cycle the number of days:
$$303 \cdot 365 + 97 \cdot 366 \equiv 303 \cdot 1 + 97 \cdot 2 \equiv 497 \equiv 0\;(\text{mod}\,7).$$
Since this number is divisible by $7$, we should count the proportions 
of the weekdays within one $400$ year cycle. 




\vspace{10pt}
{\bf Question 5.} Answer: {\tt 23/30}\\
Let $U$ be the (universe) set of all numbers between $1$ and $600$. 
Define three more sets:
\begin{align}
A & = \{ i \in U \,\mid\, i\;\text{is divisible by}\;2 \}, \nonumber \\
B & = \{ i \in U \,\mid\, i\;\text{is divisible by}\;3 \}, \nonumber \\
C & = \{ i \in U \,\mid\, i\;\text{is divisible by}\;5 \}, \nonumber 
\end{align}

Numbers $x \in \overline{A \cap B} = \overline{A} \cup \overline{B}$ 
are not divisible by $6$. Numbers 
$x \in \overline{A \cap C} = \overline{A} \cup \overline{C}$
are not divisible by $10$. Numbers in the intersection are not divisible
either by $6$ or by $10$:
$$(\overline{A} \cup \overline{B}) \cap (\overline{A} \cup \overline{C}) = 
\overline{A} \cup (\overline{B} \cap \overline{C}) = 
\overline{A} \cup \overline{B \cup C}.$$

Inclusion-exclusion principle tells that the number of elements in this union:
$$|\overline{A} \cup \overline{B \cup C}| = |\overline{A}| + 
|\overline{B \cup C}| - |\overline{A} \cap \overline{B \cup C}|.$$
We can compute:
\begin{itemize}
\item $|\overline{A}| = (1/2) \cdot 600 = 300$ ... not divisible by $2$.
\item $|\overline{B \cup C}| = 600 - |B \cup C| = 600 - (|B| + |C| - |B \cap C|) = $
$= 600 - (200 + 120 - 40) = 320$ ... not divisible either by $3$ or $5$.
\item $|\overline{A} \cap \overline{B \cup C}| = 160$.
\end{itemize}

Let us return to the original question: counting the elements 
in $\overline{A} \cup \overline{B \cup C}$:
$$|\overline{A} \cup \overline{B \cup C}| = 300 + 320 - 160 = 460.$$
All these numbers are not divisible either by $10$ or by $6$, so 
their proportion is $\frac{460}{600} = \frac{23}{30}$.


\vspace{10pt}
{\bf Question 6.} Answer: {\tt 11/2}\\

When you buy the first bag of chips, you necessarily get
a toy animal you did not have before. Waiting time for this is always $1$. 

Assume that you already have one toy; then it might take $x_1 = 1$, 
$x_2 = 2$, $x_3 = 3$, etc. more bags to find a different toy animal. 
The respective probabilities of these events are $p_1 = \frac{2}{3}$, 
$p_2 = \frac{1}{3}\cdot\frac{2}{3}$, $p_3 = \frac{1}{3}\cdot\frac{1}{3}\cdot\frac{2}{3}$, etc.
These probabilities make a {\em geometric distribution}. 
The expected waiting time is the sum $x_1p_1 + x_2p_2 + x_3p_3 + \ldots$: 
$$S = 1 \cdot \frac{2}{3} + 2 \cdot \frac{1}{3} \cdot \frac{2}{3} + 3 \cdot \frac{1^2}{3^2} \cdot \frac{2}{3} + \ldots$$
To find the value of $S$, we multiply it by $3$: 
$$3S = 3 \cdot \frac{2}{3} + 2 \cdot \frac{2}{3} + 3 \cdot \frac{1}{3} \cdot \frac{2}{3} + 4 \cdot \frac{1^2}{3^2} \cdot \frac{2}{3}\ldots$$
If we subtract 2nd from the 1st and bring the factor $2/3$ to the front: 
$$3S - S = \frac{2}{3} \left( 4  + \left(3 \cdot \frac{1}{3} - 2 \cdot \frac{1}{3} \right)
+  \left(4 \cdot \frac{1^2}{3^2} - 3 \cdot \frac{1}{3^2} \right) + \ldots \right).$$
$$2S = \frac{2}{3} \left( 4 + \frac{1}{3} + \frac{1}{3^2} + \ldots \right)$$
We get that $3S = 4\frac{1}{2}$ and $S = \frac{3}{2}$. I.e. we expect to get the
second toy after $1.5$ bags of chips.

Once we have 2 toys, the final toy can also be found by summation and infinite
geometrical progression. The expected waiting time is $3$ (bags of chips). 

The total waiting time for all $3$ toys is $\frac{3}{3} + \frac{3}{2} + \frac{3}{1} = \frac{11}{2}$.


\vspace{10pt}
{\bf Question 7.} Answer: {\tt 3/5,3/8}\\
The conditional probability of $E_1$ given $E_3$ is defined like this:
$$p(E_1 \,\mid\, E_3) = \frac{p(E_1 \cap E_3)}{p(E_3)}.$$
The conditional probability of $E_3$ given $E_1$ is defined similarly:
$$p(E_3 \,\mid\, E_1) = \frac{p(E_1 \cap E_3)}{p(E_1)}.$$

\begin{itemize}
\item Out of $32$ bit sequences there are $6$ sequences that
are in $E_1 \cap E_3$: They start with $1$ and also contain exactly three $1$'s. 
Indeed, the first bit is $1$, and there are ${4 \choose 2}$ 
ways to select the remaining two bits that equal $1$. 
\item Out of $32$ bit sequences there are $16$ sequences that 
are in $E_1$. They start with $1$.
\item Out of $32$ bit sequences there are $10$ sequences that 
are in $E_3$. There are ${5 \choose 3}$ ways to select 
three bits that equal $1$.
\end{itemize}

Compute both conditional probabilities:
$$p(E_1 \,\mid\, E_3) = \frac{p(E_1 \cap E_3)}{p(E_3)} = \frac{6/32}{10/32} = \frac{3}{5}.$$
$$p(E_3 \,\mid\, E_1) = \frac{p(E_1 \cap E_3)}{p(E_1)} = \frac{6/32}{16/32} = \frac{3}{8}.$$



\end{document}



