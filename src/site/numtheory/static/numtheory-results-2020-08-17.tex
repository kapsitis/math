\documentclass[11pt]{article}
\usepackage{ucs}
\usepackage[utf8x]{inputenc}
\usepackage{changepage}
\usepackage{graphicx}
\usepackage{amsmath}
\usepackage{gensymb}
\usepackage{amssymb}
\usepackage{enumerate}
\usepackage{tabularx}
\usepackage{lipsum}

\oddsidemargin 0.0in
\evensidemargin 0.0in
\textwidth 6.27in
\headheight 1.0in
\topmargin -0.1in
\headheight 0.0in
\headsep 0.0in
%\textheight 9.69in
\textheight 9.50in

\setlength\parindent{0pt}

\newenvironment{myenv}{\begin{adjustwidth}{0.4in}{0.4in}}{\end{adjustwidth}}
\renewcommand{\abstractname}{Anotācija}
\renewcommand\refname{Atsauces}

\newenvironment{uzdevums}[1][\unskip]{%
\vspace{3mm}
\noindent
\textbf{#1:}
\noindent}
{}

\newcommand{\subf}[2]{%
  {\small\begin{tabular}[t]{@{}c@{}}
  #1\\#2
  \end{tabular}}%
}



\newcounter{alphnum}
\newenvironment{alphlist}{\begin{list}{(\Alph{alphnum})}{\usecounter{alphnum}\setlength{\leftmargin}{2.5em}} \rm}{\end{list}}


\makeatletter
\let\saved@bibitem\@bibitem
\makeatother

\usepackage{bibentry}
%\usepackage{hyperref}


\begin{document}

\thispagestyle{empty}

{\Large \bf Skaitļu teorijas rezultāti, 2020-08-17}


\vspace{6ex}
{\bf Ķīniešu atlikumu teorēma}\\
Ja $m_1,m_2,\ldots,m_k$ ir pa pāriem savstarpēji pirmskaitļi, 
bet $a_1,a_2,\ldots,a_k$ ir jebkādi veseli skaitļi, tad eksistē
vesels atrisinājums $x \in \mathbb{Z}$ šādai kongruenču 
sistēmai 
$$\left\{ \begin{array}{l}
x \equiv a_1\;(\text{mod}\,m_1)\\
x \equiv a_2\;(\text{mod}\,m_2)\\
\vdots\\
x \equiv a_k\;(\text{mod}\,m_k)
\end{array} \right.$$
Turklāt visi šīs sistēmas atrisinājumi ir savstarpēji kongruenti 
pēc moduļa $M = m_1\cdot m_2 \cdot \ldots \cdot m_k$.


\vspace{6ex}
{\bf Kāpinātāja pacelšanas lemma (Lifting the Exponent Lemma)}\\

Dots $p$ ir pirmskaitlis, $x$ un $y$ ir veseli skaitļi, 
kuri ar $p$ nedalās ($p \not\,\,\mid x$ un $p \not\,\,\mid y$), 
bet to starpība $x-y$ ar $p$ dalās ($p \mid x-y$).

{\bf (A)} ja $p$ ir nepāru, tad 
$$\nu_p(x^n - y^n) = \nu_p(x-y) + \nu_p(n).$$

{\bf (B)} ja $p=2$ un $n$ ir pāru skaitlis, tad 
$$\nu_2(x^n - y^n) = \nu_2(x-y) + \nu_2(n) + \nu_2(x+y) - 1.$$



\end{document}
