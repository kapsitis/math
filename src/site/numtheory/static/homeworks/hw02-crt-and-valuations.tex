\documentclass[11pt]{article}
\usepackage{ucs}
\usepackage[utf8x]{inputenc}
\usepackage{changepage}
\usepackage{graphicx}
\usepackage{amsmath}
\usepackage{gensymb}
\usepackage{amssymb}
\usepackage{enumerate}
\usepackage{tabularx}
\usepackage{lipsum}
\usepackage{amsthm}
\usepackage{thmtools}
\usepackage{hyperref}

\usepackage{fontspec} % loaded by polyglossia, but included here for transparency 
\usepackage{polyglossia}

\usepackage{xeCJK}
\setCJKmainfont{SimSun}
\setmainlanguage{russian} 
\setotherlanguage{english}

\newfontfamily\cyrillicfont[Script=Cyrillic]{Times New Roman}
\newfontfamily\cyrillicfontsf[Script=Cyrillic]{Arial}
\newfontfamily\cyrillicfonttt[Script=Cyrillic]{Courier New}

\oddsidemargin 0.0in
\evensidemargin 0.0in
\textwidth 6.27in
\headheight 1.0in
\topmargin 0.0in
\headheight 0.0in
\headsep 0.0in
%\textheight 9.69in
\textheight 9.00in
 
\setlength\parindent{0pt}

\newenvironment{myenv}{\begin{adjustwidth}{0.4in}{0.4in}}{\end{adjustwidth}}
\renewcommand{\abstractname}{Anotācija}
\renewcommand\refname{Atsauces}

%\newenvironment{uzdevums}[1][\unskip]{%
%\vspace{3mm}
%\noindent
%\textbf{#1:}
%\noindent}
%{}

% (4;10;12;17)
% (p1.19;5;15;20)

% http://tex.stackexchange.com/questions/196961/thmtools-declaration-for-theorem-and-proof
\declaretheoremstyle[headfont=\normalfont\bfseries,notefont=\mdseries\bfseries,bodyfont = \normalfont,headpunct={:}]{normalhead}
\declaretheorem[name={Uzdevums}, style=normalhead,numberwithin=section]{problem}

%\def\changemargin#1#2{\list{}{\rightmargin#2\leftmargin#1}\item[]}
\def\changemargin#1#2{\list{}\item[]}
\let\endchangemargin=\endlist 


\newcommand{\subf}[2]{%
  {\small\begin{tabular}[t]{@{}c@{}}
  #1\\#2
  \end{tabular}}%
}



\newcounter{alphnum}
\newenvironment{alphlist}{\begin{list}{(\Alph{alphnum})}{\usecounter{alphnum}\setlength{\leftmargin}{2.5em}} \rm}{\end{list}}

\newenvironment{zhtext}{\fontfamily{MS PGothic}\selectfont}{\par}


\makeatletter
\let\saved@bibitem\@bibitem
\makeatother

\usepackage{bibentry}
%\usepackage{hyperref}

\newenvironment{tulkojums}[1][\unskip]{%
\begin{changemargin}{8mm}{8mm}
\fontsize{9}{11}
\selectfont
\textbf{#1:}
}
{ 
\fontsize{12}{14}
\selectfont
\end{changemargin}
}

\setcounter{section}{2}


\begin{document}

\begin{center}
{\Large \bf NMS Izlase junioriem: 2.nodarbība skaitļu teorijā}\\
{\bf Ieteicams izvēlēties un rakstiski noformēt 
5 no 8 uzdevumiem līdz 2019.g. 30.decembrim.}\\
{Var risināt uz papīra vai iesūtīt elektroniski: "kalvis.apsitis", domēns "gmail.com"}
\end{center}

\vspace{10pt}
{\bf \large 3.nodaļa: Ķīniešu atlikumu teorēma}

\begin{problem}
% Yuhong; p55, ex3
Pierādīt, ka jebkuram naturālam skaitlim $n$, ir $n$ pēc kārtas sekojoši 
naturāli skaitļi, ka jebkurm no tiem ir dalītājs, kas ir pilns kvadrāts, kas lielāks par $1$. 
\end{problem}

\begin{problem}
Katram naturālam skaitlim $n$, ir $n$ pēc kārtas sekojoši naturāli skaitļi, no kuriem neviens
nav {\em potents skaitlis}.\\
{\em Piezīme:} Par potentu saucam naturālu skaitli $n$, ka jebkuram pirmskaitlim $p$: ja $n$ dalās 
ar $p$, tad $n$ dalās arī ar $p^2$. Sk.\ \url{https://en.wikipedia.org/wiki/Powerful\%5Fnumber}.
\end{problem}

\begin{problem}
Dotajam naturālam skaitlim $n$, ar $f(n)$ apzīmējam mazāko naturālo skaitli, ka 
${\displaystyle \sum\limits_{k=1}^{f(n)} k}$ dalās ar $n$. 
Pierādīt, ka $f(n) = 2n-1$ tad un tikai tad, ja $n$ ir skaitļa $2$ pakāpe.
\end{problem}

\begin{problem}
Ar $n$ un $k$ apzīmējam veselus skaitļus, ka $n>0$ un skaitlis $k(n-1)$ ir pāra skaitlis. 
Pierādīt, ka eksistē skaitļi $x$ un $y$, ka $\text{gcd}(x,n) = \text{gcd}(y,n) = 1$ un 
$x + y \equiv k\;(\text{mod}\,n)$. 
\end{problem}

\begin{problem}
Dots naturāls skaitlis $x$. Pierādīt, ka ir $n$ pēc kārtas sekojoši naturāli skaitļi, 
no kuriem neviens nav pirmskaitļa pakāpe. 
\end{problem}


\begin{problem}
Ar $m, n$ apzīmēti naturāli skaitļi, kas apmierina šādu īpašību:
$$ \text{gcd}(11k-1,m) = \text{gcd}(11k-1,n)$$
ir spēkā visiem naturāliem skaitļiem $k$. Pierādīt, ka $m = 11^rn$ kādam veselam skaitlim $r$. 
\end{problem}



\vspace{10pt}
{\bf \large 4.nodaļa: Valuācijas}
% AoPS; 4,5

\begin{problem}
Dots naturāls skaitlis $k > 1$. Pierādiet, ka eksistē bezgalīgi daudzi naturāli skaitļi $n$, kuriem 
$$n \,\mid\, 1^n + 2^n + 3^n + \ldots + k^n.$$
\end{problem}

\begin{problem}
Dots naturāls skaitlis $n > 1$.
Pierādiet, ka skaitlim $a^n - b^n$ ir pirmreizinātājs, kurš nav skaitļa $a-b$ dalītājs.
\end{problem}




\end{document}


