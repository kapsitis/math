\documentclass[11pt]{article}
\usepackage{ucs}
\usepackage[utf8x]{inputenc}
\usepackage{changepage}
\usepackage{graphicx}
\usepackage{amsmath}
\usepackage{gensymb}
\usepackage{amssymb}
\usepackage{enumerate}
\usepackage{tabularx}
\usepackage{lipsum}
\usepackage{amsthm}
\usepackage{thmtools}


\usepackage{fontspec} % loaded by polyglossia, but included here for transparency 
\usepackage{polyglossia}

\usepackage{xeCJK}
\setCJKmainfont{SimSun}
\setmainlanguage{russian} 
\setotherlanguage{english}

\newfontfamily\cyrillicfont[Script=Cyrillic]{Times New Roman}
\newfontfamily\cyrillicfontsf[Script=Cyrillic]{Arial}
\newfontfamily\cyrillicfonttt[Script=Cyrillic]{Courier New}

\oddsidemargin 0.0in
\evensidemargin 0.0in
\textwidth 6.27in
\headheight 1.0in
\topmargin 0.0in
\headheight 0.0in
\headsep 0.0in
%\textheight 9.69in
\textheight 9.00in
 
\setlength\parindent{0pt}

\newenvironment{myenv}{\begin{adjustwidth}{0.4in}{0.4in}}{\end{adjustwidth}}
\renewcommand{\abstractname}{Anotācija}
\renewcommand\refname{Atsauces}

%\newenvironment{uzdevums}[1][\unskip]{%
%\vspace{3mm}
%\noindent
%\textbf{#1:}
%\noindent}
%{}


% http://tex.stackexchange.com/questions/196961/thmtools-declaration-for-theorem-and-proof
\declaretheoremstyle[headfont=\normalfont\bfseries,notefont=\mdseries\bfseries,bodyfont = \normalfont,headpunct={:}]{normalhead}
\declaretheorem[name={Uzdevums}, style=normalhead,numberwithin=section]{problem}

\def\changemargin#1#2{\list{}{\rightmargin#2\leftmargin#1}\item[]}
\let\endchangemargin=\endlist 


\newcommand{\subf}[2]{%
  {\small\begin{tabular}[t]{@{}c@{}}
  #1\\#2
  \end{tabular}}%
}



\newcounter{alphnum}
\newenvironment{alphlist}{\begin{list}{(\Alph{alphnum})}{\usecounter{alphnum}\setlength{\leftmargin}{2.5em}} \rm}{\end{list}}

\newenvironment{zhtext}{\fontfamily{MS PGothic}\selectfont}{\par}


\makeatletter
\let\saved@bibitem\@bibitem
\makeatother

\usepackage{bibentry}
%\usepackage{hyperref}

\newenvironment{tulkojums}[1][\unskip]{%
\begin{changemargin}{8mm}{8mm}
\fontsize{9}{11}
\selectfont
\textbf{#1:}
}
{ 
\fontsize{12}{14}
\selectfont
\end{changemargin}
}

\setcounter{section}{0}


\begin{document}

\begin{center}
{\Large \bf Atlases sacensības komandu olimpiādei ``Baltijas ceļš''}\\
{\bf 2018.\ gada 22.\ septembris, Rīga (1.\ diena)}
\end{center}

{\footnotesize \em 
Risināšanas laiks: 4 stundas un 30 minūtes.\\
Jautājumus drīkst uzdot pirmo 30 minūšu laikā.\\ 
Atļauts izmantot tikai rakstāmpiederumus, lineālu un cirkuli.
}

\begin{problem}[BW.TST.2018.1]
Doti $n \geq 2$ pozitīvi reāli skaitļi $p_1,p_2,\ldots,p_n$. Kādu vislielāko un 
vismazāko vērtību var pieņemt izteiksme $x_1^2 + x_2^2 + \ldots + x_n^2$, ja 
zināms, ka $x_1,x_2,\ldots,x_n$ ir tādi nenegatīvi reāli skaitļi, ka 
$x_1p_1 + x_2p_2 + \ldots + x_np_n = 1$? 
\end{problem}

\begin{problem}[BW.TST.2018.2]
Atrast visus reālu skaitļu pārus $(x,y)$, kas apmierina vienādojumu sistēmu: 
\[  \left\{ \begin{array}{l}
y(x+y)^2 = 2,\\
8y(x^3 - y^3) = 13.
\end{array} \right. \]
\end{problem}

\begin{problem}[BW.TST.2018.3]
Veselu skaitļu virknē $a_1,a_2,\ldots$ visiem $n \geq 1$ izpildās $a_{n+2} = a_{n+1} + a_n$. 
Pierādīt, ka, ja eksistē tāds $k$, ka $a_k = a_{k+2018}$, tad kāds no šīs virknes locekļiem 
ir vienāds ar nulli. 
\end{problem}

\begin{problem}[BW.TST.2018.4]
Funkcija $f(x)$ definēta reāliem skaitļiem un pieņem reālas vērtības, pie tam visiem reāliem $x$ ir spēkā
nevienādība
\[ \sqrt{2f(x)} - \sqrt{2f(x) - f(2x)} \geq 2. \]
Pierādīt, ka visiem reāliem $x$ ir spēkā nevienādība
\begin{enumerate}[(a)]
\item $f(x) \geq 4$.
\item $f(x) \geq 7$.
\end{enumerate}
\end{problem}


\begin{problem}[BW.TST.2018.5]
Kārlis un Laila spēlē spēli uz galdiņa, kas sastāv no $n \geq 5$ rindā
novietotiem lauciņiem. Uz pirmā lauciņa tiek novietots kauliņš un 
spēlētāji pēc kārtas izdara gājienus, sāk Kārlis.
Vienā gājienā var pārvietot kauliņu vienu lauciņu uz priekšu, četrus lauciņus uz 
priekšu vai divus lauciņus atpakaļ. 
Jebkuru no šīm darbībām drīkst veikt tikai tādā gadījumā, ja pēc tās izpildes 
kauliņš joprojām atrdīsies uz kāda no galdiņa lauciņiem. Uzvar tas spēlētājs, kurš pārvieto
kauliņu uz pēdējā lauciņa. 
Noskaidrojiet, kurām $n$ vērtībām katram no spēlētājiem eksistē uzvaroša stratēģija.
\end{problem}

\begin{problem}[BW.TST.2018.6]
Uz rūtiņu lapas pa rūtiņu līnijām uzzīmēts taisnstūris $ABCD$. Visas rūtiņu virsotnes, kas 
atrodas taisnstūra iekšienē vai uz tā malām, nokrāsotas četrās krāsās tā, ka 
\begin{itemize}
\item visas virsotnes, kas atrodas uz malas $AB$, nokrāsotas vai nu 1.\ vai 2.\ krāsā,
\item visas virsotnes, kas atrodas uz malas $BC$, nokrāsotas vai nu 2.\ vai 3.\ krāsā, 
\item visas virsotnes, kas atrodas uz malas $CD$, nokrāsotas vai nu 3.\ vai 4.\ krāsā,
\item visas virsotnes, kas atrodas uz malas $DA$, nokrāsotas vai nu 4.\ vai 1.\ krāsā,
\item nekādas divas blakus virsotnes nav nokrāsotas 1.\ un 3.\ vai 2.\ un 4.\ krāsā.
\end{itemize}
Ievērojiet, ka no šiem nosacījumiem izriet, ka virsotne $A$ ir nokrāsota 1.\ krāsā, utt. 
Pierādīt, ka var atrast rūtiņu, kurai visas četras virsotnes ir nokrāsotas dažādās krāsās.
\end{problem}

\begin{problem}[BW.TST.2018.7]
Plaknē atlikti $n \geq 3$ punkti, nekādi $3$ no kuriem neatrodas uz vienas taisnes. Vai noteikti
iespējams uzzīmēt $n$-stūri, kura virsotnes atrodas dotajos punktos un kura malas nekrustojas?
\end{problem}

\begin{problem}[BW.TST.2018.8]
Dots naturāls skaitlis $n \geq 2$. Lauras klasē ir vairāk nekā $n+2$ skolēni, un katrs no tiem atrisināja
dažus uzdevumus. Ir zināms, ka:
\begin{itemize}
\item katriem diviem skolēniem ir tieši viens uzdevums, kuru viņi abi ir atrisinājuši,
\item katriem diviem uzdevumiem ir tieši viens skolēns, kurš tos abus ir atrisinājis,
\item vienu uzdevumu atrisināja Laura un vēl tieši $n$ citi skolēni.
\end{itemize}
Cik skolēnu ir Lauras klasē?
\end{problem}




\begin{center}
{\bf 2018.\ gada 23.\ septembris, Rīga (2.\ diena)}
\end{center}

\begin{problem}[BW.TST.2018.9]
Šaurleņķa trijstūrim $ABC$, kuram $AB < AC$, apvilkta riņķa līnija $\Gamma$, kuras centrs ir $O$. 
Malas $AB$ viduspunkts ir $D$, uz malas $AC$ atlikts punkts $E$, tā, ka $BE = CE$. 
Trijstūrim $BDE$ apvilktā riņķa līnija krusto $\Gamma$ punktā $F$ (kas nesakrīt ar $B$). 
No punkta $B$ pret $AO$ novilkts perpendikuls $BK$, zināms, ka $A$ un $K$
atrodas dažādās pusēs taisnei $BE$. Pierādīt, ka taisnes $DF$ un $CK$ krustojas punktā, kas atrodas
uz riņķa līnijas $\Gamma$. 
\end{problem}

\begin{problem}[BW.TST.2018.10]
Platleņķa trijstūrī $ABC$, kura platais leņķis ir $B$, novilkti augstumi $AD$, $BE$ un $CF$. Punkti 
$T$ un $S$ ir attiecīgi $AD$ un $CF$ viduspunkti. Punkti $M$ un $N$ ir simetriski punktam $T$ attiecīgi pret
taisnēm $BE$ un $BD$. Pierādīt, ka trijstūrim $BMN$ apvilktā riņķa līnija iet caur punktu $S$. 
\end{problem}

\begin{problem}[BW.TST.2018.11]
Trijstūra $ABC$ leņķi ir $\sphericalangle A = 80^{\circ}$, $\sphericalangle B = 70^{\circ}$, $\sphericalangle C = 30^{\circ}$. 
No virsotnes $A$ novilkta bisektrise $AD$, un uz tās (trijstūra iekšienē) atzīmēts punkts $P$ tā, ka 
$\sphericalangle BPC = 130^{\circ}$. No punkta $P$ novilkti perpendikuli $PX$, $PY$ un $PZ$ attiecīgi pret 
malām $BC$, $AC$ un $AB$. Pierādīt, ka 
\[ AY^3 + BZ^3 + CX^3 = AZ^3 + BX^3 + CY^3. \] 
\end{problem}

\begin{problem}[BW.TST.2018.12]
Uz paralelograma $ABCD$ malas $BC$ atzīmēts punkts $X$, bet uz malas $CD$ -- punkts $Y$, nogriežņi $BY$ un $DX$ krustojas punktā $P$. 
Pierādīt, ka taisne, kas iet caur nogriežņu $BD$ un $XY$ viduspunktiem, ir paralēla taisnei $AP$ (vai sakrīt ar to). 
\end{problem}

\begin{problem}[BW.TST.2018.13]
Vai eksistē tāds pirmskaitlis $p$, ka nevienam pirmskaitlim $p$ skaitlis
\[ \sqrt[3]{p^2 + q} \]
nav naturāls?
\end{problem}

\begin{problem}[BW.TST.2018.14]
Par naturālu skaitļu virkni $a_1,a_2,\ldots$ zināms, ka $a_1 = 2$ un visiem $n \geq 1$ skaitlis 
$a_{n+1}$ ir lielākais pirmskaitlis, ar ko dalās skaitlis $a_1a_2\ldots{}a_n + 1$. Pierādīt, ka neviens no 
šīs virknes nlocekļiem nav vienāds ne ar $5$, ne ar $11$.
\end{problem}

\begin{problem}[BW.TST.2018.15]
Vai eksistē tāds naturāls skaitlis $n$, ka var atrast vismaz $2018$ naturālu skaitļu četriniekus $(x,y,z,t)$, 
kas apmierina vienādojumu sistēmu
\[ \left\{ \begin{array}{l}
x + y + z = n\\
xyz = 2t^3. 
\end{array} \right. \]
\end{problem}

\begin{problem}[BW.TST.2018.16]
Sauksim naturālu skaitli par {\em vienkāršu}, ja tas nedalās ne ar viena pirmskaitļa kvadrātu. 
Pierādīt, ka ir bezgalīgi daudz tādu $n$, ka gan $n$, gan $n+1$ ir vienkārši skaitļi.
\end{problem}


\end{document}


