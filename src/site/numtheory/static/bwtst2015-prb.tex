\documentclass[11pt]{article}
\usepackage{ucs}
\usepackage[utf8x]{inputenc}
\usepackage{changepage}
\usepackage{graphicx}
\usepackage{amsmath}
\usepackage{gensymb}
\usepackage{amssymb}
\usepackage{enumerate}
\usepackage{tabularx}
\usepackage{lipsum}
\usepackage{amsthm}
\usepackage{thmtools}


\usepackage{fontspec} % loaded by polyglossia, but included here for transparency 
\usepackage{polyglossia}

\usepackage{xeCJK}
\setCJKmainfont{SimSun}
\setmainlanguage{russian} 
\setotherlanguage{english}

\newfontfamily\cyrillicfont[Script=Cyrillic]{Times New Roman}
\newfontfamily\cyrillicfontsf[Script=Cyrillic]{Arial}
\newfontfamily\cyrillicfonttt[Script=Cyrillic]{Courier New}

\oddsidemargin 0.0in
\evensidemargin 0.0in
\textwidth 6.27in
\headheight 1.0in
\topmargin 0.0in
\headheight 0.0in
\headsep 0.0in
%\textheight 9.69in
\textheight 9.00in
 
\setlength\parindent{0pt}

\newenvironment{myenv}{\begin{adjustwidth}{0.4in}{0.4in}}{\end{adjustwidth}}
\renewcommand{\abstractname}{Anotācija}
\renewcommand\refname{Atsauces}

%\newenvironment{uzdevums}[1][\unskip]{%
%\vspace{3mm}
%\noindent
%\textbf{#1:}
%\noindent}
%{}


% http://tex.stackexchange.com/questions/196961/thmtools-declaration-for-theorem-and-proof
\declaretheoremstyle[headfont=\normalfont\bfseries,notefont=\mdseries\bfseries,bodyfont = \normalfont,headpunct={:}]{normalhead}
\declaretheorem[name={Uzdevums}, style=normalhead,numberwithin=section]{problem}

\def\changemargin#1#2{\list{}{\rightmargin#2\leftmargin#1}\item[]}
\let\endchangemargin=\endlist 


\newcommand{\subf}[2]{%
  {\small\begin{tabular}[t]{@{}c@{}}
  #1\\#2
  \end{tabular}}%
}



\newcounter{alphnum}
\newenvironment{alphlist}{\begin{list}{(\Alph{alphnum})}{\usecounter{alphnum}\setlength{\leftmargin}{2.5em}} \rm}{\end{list}}

\newenvironment{zhtext}{\fontfamily{MS PGothic}\selectfont}{\par}


\makeatletter
\let\saved@bibitem\@bibitem
\makeatother

\usepackage{bibentry}
%\usepackage{hyperref}

\newenvironment{tulkojums}[1][\unskip]{%
\begin{changemargin}{8mm}{8mm}
\fontsize{9}{11}
\selectfont
\textbf{#1:}
}
{ 
\fontsize{12}{14}
\selectfont
\end{changemargin}
}

\setcounter{section}{0}



\begin{document}

\begin{center}
{\Large \bf Atlases sacensības komandu olimpiādei ``Baltijas ceļš''}\\
{\bf 2015.\ gada 19.\ septembris, Rīga (1.\ diena)}
\end{center}


\begin{problem}[BW.TST.2015.1]
Doti reāli skaitļi $x$ un $y$, tādi, ka
\[ x^4y^2 + y^4 + 2x^3y + 6x^2y + x^2 + 8 \leq 0. \]
Pierādīt, ka $x \geq -\frac{1}{6}$. 
\end{problem}

\begin{problem}[BW.TST.2015.2]
Par funkciju $f: \mathbb{R} \rightarrow \mathbb{R}$ zināms, ka 
\begin{itemize}
\item $f(x) > f(y)$ visiem reāliem $x > y$; 
\item $f(x) > x$ visiem reāliem $x$; 
\item $f(2x - f(x)) = x$ visiem reāliem $x$. 
\end{itemize}
Pierādīt, ka $f(x) = x + f(0)$ visiem reāliem skaitļiem $x$. 
\end{problem}

\begin{problem}[BW.TST.2015.3]
Pierādīt, ka neeksistē polinoms $P(x)$ ar veseliem koeficientiem un naturāls skaitlis $m$, tādi, ka 
\[ x^m + x + 2 = P(P(x)) \]
izpildās visiem veseliem skaitļiem $x$. 
\end{problem}

\begin{problem}[BW.TST.2015.4]
Vai izliektā 2014-stūrī var novilkt dažas diagonāles tā, ka tās nekrustojas, 
viss 2014-stūris ir sadalīts trijstūros un katra virsotne pieder nepāra skaitam šo trijstūru?
\end{problem}

\begin{center}
{\bf 2015.\ gada 20.\ septembris, Rīga (2.\ diena)}
\end{center}

\begin{problem}[BW.TST.2015.5]
$BE$ ir šaurleņķa trijstūra $ABC$ augstums. Taisne $l$ pieskaras trijstūra $ABC$ apvilktajai 
riņķa līnijai punktā $B$, no $C$ pret $l$ novilkts perpendikuls $CF$. Pierādīt, ka 
taisnes $EF$ un $AB$ ir paralēlas. 
\end{problem}

\begin{problem}[BW.TST.2015.6]
$AM$ ir trijstūra $ABC$ mediāna. No punkta $C$ pret leņķa $CMA$ bisektrisi novilkts perpendikuls $CC_1$, 
no punkta $B$ pret leņķa $BMA$ bisektrisi novilkts perpendikuls $BB_1$. Pierādīt, ka taisne 
$AM$ krusto nogriezni $B_1C_1$ tā viduspunktā. 
\end{problem}

\begin{problem}[BW.TST.2015.7]
Divas riņķa līnijas $\Gamma_1$ un $\Gamma_2$ krustojas punktos $A$ un $B$, punkts $P$ neatrodas uz taisnes $AB$. 
Taisne $AP$ vēlreiz krusto $\Gamma_1$ un $\Gamma_2$ attiecīgi punktos $K$ un $L$, taisne $BP$ vēlreiz krusto 
$\Gamma_1$ un $\Gamma_2$ attiecīgi punktos $M$ un $N$ un visi līdz šim minētie punkti ir dažādi. 
Ap trijstūriem $KMP$ un $LNP$ apvilkto riņķa līniju 
centri ir $O_1$ un $O_2$. Pierādīt, ka $O_1O_2$ ir perpendikulārs $AB$. 
\end{problem}

\begin{problem}[BW.TST.2015.8]
Dots fiksēts racionāls skaitlis $q$. Skaitli $x$ sauksim par {\em harizmātisku}, ja var atrast tādu naturālu 
skaitli $n$ un veselus skaitļus $\alpha_1, \alpha_2, \ldots, \alpha_n$, ka 
\[ x = (q + 1)^{\alpha_1} \cdot (q+2)^{\alpha_2} \cdot \ldots \cdot (q + n)^{\alpha_n}. \]
\begin{alphlist}
\item Pierādīt, ka var atrast tādu $q$, kam visi pozitīvie racionālie skaitļi ir harizmātiski. 
\item Vai tiesa, ka visiem $q$, ja skaitlis $x$ ir harizmātisks, tad arī $x + 1$ ir harizmātisks?
\end{alphlist}
\end{problem}


\begin{center}
{\bf 2015.\ gada 26.\ septembris, Rīga (3.\ diena)}
\end{center}

\begin{problem}[BW.TST.2015.9]
Divi spēlētāji uz $N \times N$ rūtiņu laukuma spēlē sekojošu spēli. Tie pēc kārtas iekrāso pa vienai rūtiņai tā, 
lai nekādas divas iekrāsotās rūtiņas neatrastos uz vienas diagonāles. Spēlētājs, kurš nevar izdarīt
gājienu, zaudē. Kādām $N$ vērtībām pirmajam spēlētājam eksistē uzvaroša stratē\v{g}ija?
\end{problem}


\begin{problem}[BW.TST.2015.10]
Vai tiesa, ka visiem naturāliem $n$ vienmēr iespējams katram no $n$ bērniem iedot pa cepurei, kas 
nokrāsota vienā no 100 krāsām tā, ka, ja kāda meitene ir pazīstama ar vairāk nekā 2015 zēniem, tad 
ne visiem šiem zēniem cepures ir vienā krāsā, un, ja kāds zēns ir pazīštams ar vairāk nekā 2015 meitenēm, 
tad ne visām šīm meitenēm cepures ir vienā krāsā?
\end{problem}

\begin{problem}[BW.TST.2015.11]
Par figūru uz rūtiņu lapas sauksim patvaļīgu galīgu saistītu rūtiņu kopu, t.i. tādu rūtiņu kopu, 
kurai no jebkuras rūtiņas uz jebkuru citu iespējams aiziet, ejot tikai pa šīs figūras rtiņām. 
Pierādīt, ka katram naturālam $n$ eksistē tāda figūra uz rūtiņu lapas, ka to var sagriezt ``stūrīšos''
(1.\ zīm.) tieši $F_n$ veidos, kur $F_n$ ir $n$-tais Fibonači skaitlis (Fibonači skaitļu virknē 
$F_1 = F_2 = 1$ un katram $i \geq 1$ izpildās $F_{i+2} = F_{i+1} + F_i$). Piemēram, 
$2 \times 3$ rūtiņu taisnstūri iespējams sagriezt stūrīšos tieši divos veidos (2.\ zīm.). 

\begin{center}
\begin{tabular}{cc}
$\vcenter{\hbox{\includegraphics[width=0.5in]{BwTst2015-11-fig1.png}}}$ &
$\vcenter{\hbox{\includegraphics[width=1.65in]{BwTst2015-11-fig2.png}}}$  \\ 
1.\ zīm.: Stūrītis & 
\begin{minipage}{3in}
\begin{center}
2.\ zīm.: Divi veidi\\ kā sagriezt $2 \times 3$ taisnstūri
\end{center}
\end{minipage}\\
\end{tabular}
\end{center}
\end{problem}

\begin{problem}[BW.TST.2015.12]
Reāliem pozitīviem skaitļiem $a,b,c$ izpildās vienādība $abc = 1$. Pierādīt, ka 
\[ \frac{a^{2014}}{1 + 2bc} + \frac{b^{2014}}{1 + 2ac} + \frac{c^{2014}}{1 + 2ab} \geq 
\frac{3}{ab + bc + ac}. \]
\end{problem}


\begin{center}
{\bf 2015.\ gada 26.\ septembris, Rīga (4.\ diena)}
\end{center}

\begin{problem}[BW.TST.2015.13]
Vai eksistē tādi pozitīvi reāli skaitļi $a$ un $b$, ka $[ an + b ]$ ir pirmskaitlis visām 
naturālām $n$ vērtībām. Ar $[ x ]$ apzīmē skaitļa veselo daļu -- lielāko veselo skaitli, kas
nepārsniedz $x$. 
\end{problem}

\begin{problem}[BW.TST.2015.14]
Ar $S(a)$ apzīmēsim skaitļa $a$ ciparu summu. Kādām naturālām $R$ vērtībām var 
atrast tādu naturālu $n$, ka 
\[ \frac{S(n^2)}{S(n)} = R? \]
\end{problem}

\begin{problem}[BW.TST.2015.15]
Ar $\omega(n)$ apzīmēsim dažādo pirmskaitļu skaitu, ar ko dalās $n$. Pierādīt, ka ir bezgalīgi 
daudz tādu naturālu skaitļu $n$, kuriem $\omega(n) < \omega(n+1) < \omega(n+2)$. 
\end{problem}

\begin{problem}[BW.TST.2015.16]
Punkti $X$, $Y$ un $Z$ atrodas uz taisnes $k$ tieši šādā secībā. Uz nogriežņiem 
$XZ$, $XY$, $YZ$ kā diametriem konstruētas riņķa līnijas 
$\omega_1$, $\omega_2$, $\omega_3$. Taisne $l$, kas iet caur punktu $Y$, krusto 
$\omega_1$ punktos $A$ un $D$, $\omega_2$ punktā $B$ un $\omega_3$ punktā $C$ tā, 
ka punkti $A,B,Y,C,D$ atrodas uz $l$ tieši šādā secībā. 
Pierādīt, ka $AB = CD$. 
\end{problem}



\end{document}


