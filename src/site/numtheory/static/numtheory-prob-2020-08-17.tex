\documentclass[a4paper]{article}
\usepackage{ucs}
\usepackage[utf8x]{inputenc}
\usepackage{changepage}
\usepackage{graphicx}
\usepackage{amsmath}
\usepackage{gensymb}
\usepackage{amssymb}
\usepackage{enumerate}
\usepackage{tabularx}
\usepackage{lipsum}
\usepackage{amsthm}
\usepackage{thmtools}
\usepackage{xcolor}



%\documentclass[jou]{apa6}
%\usepackage[american]{babel}

%\usepackage{csquotes}
%\usepackage[style=apa,sortcites=true,sorting=nyt,backend=biber]{biblatex}
%\DeclareLanguageMapping{american}{american-apa}
%\addbibresource{bibliography.bib}


%%%%%%%%%%%%%%%%%%%%%%%%%%%%%%%%%%%%%%%%
%% Discrete Structures
%% The start of RBS stuff
%%%%%%%%%%%%%%%%%%%%%%%%%%%%%%%%%%%%%%%%

% Working internal and external links in PDF
\usepackage{hyperref}
% Extra math symbols in LaTeX
\usepackage{amsmath}
\usepackage{gensymb}
\usepackage{amssymb}
% Enumerations with (a), (b), etc.
\usepackage{enumerate}
\usepackage[framemethod=TikZ]{mdframed}
\usepackage{xcolor}

\let\OLDitemize\itemize
\renewcommand\itemize{\OLDitemize\addtolength{\itemsep}{-6pt}}

\usepackage{etoolbox}
\makeatletter
\preto{\@verbatim}{\topsep=3pt \partopsep=3pt }
\makeatother

% These sizes redefine APA for A4 paper size
\oddsidemargin 0.0in
\evensidemargin 0.0in
\textwidth 6.27in
\headheight 1.0in
\topmargin -24pt
\headheight 12pt
\headsep 12pt
\textheight 9.19in



%\title{Sample Quiz 8}
%\author{Discrete Structures, Spring 2020}
%\affiliation{RBS}

%\leftheader{Discrete Sample Quiz 8}

%\abstract

%\keywords{}

\setlength\parindent{0pt}

\begin{document}

%\thispagestyle{empty}

\twocolumn

{\Large \bf NMS izlase, 2020-08-17}

\vspace{4pt}
{\bf 1.uzdevums.}\\
Pierādīt, ka jebkuram naturālam $n$, eksistē $n+1$ 
savstarpēji pirmskaitļi $k_0,k_1,\ldots,k_n$, kas visi lielāki par $1$, kuriem 
$k_0 \cdot k_1 \cdot \ldots \cdot k_n - 1$
ir divu pēc kārtas sekojošu naturālu skaitļu reizinājums.



\vspace{4pt}
{\bf 2.uzdevums.}\\
Definējam polinomu $\textcolor{red}{P(n)=n^2+n+1}$. 
Aplūkojam $m$ pēc kārtas ņemtas šī polinoma vērtības:
$$S = \{P(n),P(n + 1),P(n + 2),\ldots,P(n+(m-1))\}.$$ 
Kādai mazākajai $m$ vērtībai var atrast tādu $n$, ka 
ikvienam skaitlim $a \in S$ atrodas $b \in S$, ka $a \neq b$ 
un $a,b$ nav savstarpēji pirmskaitļi. 



\vspace{10pt}
{\bf 3.uzdevums.}\\
Veselu skaitļu kopu $S$ sauksim par {\em saknisku}, ja
kat\-ram naturālam skaitlim $n$ un koeficientiem 
$$a_0,a_1,\ldots,a_n \in S,$$ 
polinoma $a_0 + a_1 + \ldots + a_n x^n$
visas tās saknes, kas ir veseli skaitļi, arī pieder $S$. 
Atrast visas sakniskās veselu skaitļu kopas, kas satur visus skaitļus
formā $2^a - 2^b$ ($a$ un $b$ ir jebkuras naturālas vērtības).


\vspace{10pt}
{\bf 4.uzdevums.}\\
Apzīmējam virkni: 
$$x_n = \prod\limits_{j=1}^{n} \left(2^j - 1 \right).$$
{\bf (A)} Atrast $\nu_5 \left( x_{100} \right)$ \textendash{} lielāko 
kāpinātāju $m$, kuram $x_{100}$ dalās ar $5^m$.\\
{\bf (B)} Atrast $\nu_5 (100!)$.\\
{\bf (C)} Atrast $\nu_7 (x_{100})$.\\
{\bf (D)} Atrast $\nu_7 (100!)$.\\
{\bf (E)} Atrast visus naturālo skaitļu $(k,n)$ pārus, kuriem izpildās
$$k! = (2^n - 1)(2^n - 2)...(2^n - 2^{n-1}).$$


\vspace{10pt}
{\bf 5.uzdevums.}\\
Ar $n$ apzīmējam naturālu skaitli. 
Katrā tabulas $n \times n$ šūnā ierakstīts pa veselam skaitlim. 
Pieņemsim, ka izpildās divi nosacījumi:
\begin{itemize}
\item 
Katrs skaitlis tabulā ir kongruents ar $1$ pēc $n$ moduļa. 
\item 
Skaitļu summa katrā rindiņā un arī katrā kolon\-nā ir kongruenta $n$ pēc
$n^2$ moduļa. 
\end{itemize}
Ar $R_i$ apzīmējam $i$-tās rindas skaitļu reizinājumu, un 
$C_j$ apzīmē $j$-tās kolonnas skaitļu reizinājumu. 
Pierādīt, ka summas $R_1 + \ldots + R_n$ un 
$C_1 + \ldots + C_n$ ir kongruentas pēc moduļa $n^4$.




\vspace{10pt}
{\bf 6.uzdevums.}\\
Ar $a_1 < a_2 < ... < a_n$ apzīmējam naturālus skaitļus, kuri ir savstarpēji pirmskaitļi. 
Zināms, ka $a_1$ ir pirmskaitlis un $a_1 \geq n+2$. 
Reālu skaitļu intervālā $I = \left[0;\,a_1\cdot{}a_2\cdot\ldots\cdot a_n\right]$ 
atzīmējam visus veselos skaitļus, kuri dalās vismaz ar vienu no skaitļiem 
$a_1,\ldots,a_n$. Šie punkti sadala $I$ vairākos mazākos nogriežņos. 
Pierādīt, ka visu šo nogriežņu garumu kvadrātu summa dalās ar $a_1$. 



\end{document}

