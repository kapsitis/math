\documentclass[11pt]{article}
\usepackage{ucs}
\usepackage[utf8x]{inputenc}
\usepackage{changepage}
\usepackage{graphicx}
\usepackage{amsmath}
\usepackage{gensymb}
\usepackage{amssymb}
\usepackage{enumerate}
\usepackage{tabularx}
\usepackage{lipsum}
\usepackage{amsthm}
\usepackage{thmtools}


\usepackage{fontspec} % loaded by polyglossia, but included here for transparency 
\usepackage{polyglossia}

\usepackage{xeCJK}
\setCJKmainfont{SimSun}
\setmainlanguage{russian} 
\setotherlanguage{english}

\newfontfamily\cyrillicfont[Script=Cyrillic]{Times New Roman}
\newfontfamily\cyrillicfontsf[Script=Cyrillic]{Arial}
\newfontfamily\cyrillicfonttt[Script=Cyrillic]{Courier New}

\oddsidemargin 0.0in
\evensidemargin 0.0in
\textwidth 6.27in
\headheight 1.0in
\topmargin 0.0in
\headheight 0.0in
\headsep 0.0in
%\textheight 9.69in
\textheight 9.00in
 
\setlength\parindent{0pt}

\newenvironment{myenv}{\begin{adjustwidth}{0.4in}{0.4in}}{\end{adjustwidth}}
\renewcommand{\abstractname}{Anotācija}
\renewcommand\refname{Atsauces}

%\newenvironment{uzdevums}[1][\unskip]{%
%\vspace{3mm}
%\noindent
%\textbf{#1:}
%\noindent}
%{}


% http://tex.stackexchange.com/questions/196961/thmtools-declaration-for-theorem-and-proof
\declaretheoremstyle[headfont=\normalfont\bfseries,notefont=\mdseries\bfseries,bodyfont = \normalfont,headpunct={:}]{normalhead}
\declaretheorem[name={Uzdevums}, style=normalhead,numberwithin=section]{problem}

\def\changemargin#1#2{\list{}{\rightmargin#2\leftmargin#1}\item[]}
\let\endchangemargin=\endlist 


\newcommand{\subf}[2]{%
  {\small\begin{tabular}[t]{@{}c@{}}
  #1\\#2
  \end{tabular}}%
}



\newcounter{alphnum}
\newenvironment{alphlist}{\begin{list}{(\Alph{alphnum})}{\usecounter{alphnum}\setlength{\leftmargin}{2.5em}} \rm}{\end{list}}

\newenvironment{zhtext}{\fontfamily{MS PGothic}\selectfont}{\par}


\makeatletter
\let\saved@bibitem\@bibitem
\makeatother

\usepackage{bibentry}
%\usepackage{hyperref}

\newenvironment{tulkojums}[1][\unskip]{%
\begin{changemargin}{8mm}{8mm}
\fontsize{9}{11}
\selectfont
\textbf{#1:}
}
{ 
\fontsize{12}{14}
\selectfont
\end{changemargin}
}

\setcounter{section}{0}


\begin{document}

\begin{center}
{\Large \bf Atlases sacensības komandu olimpiādei ``Baltijas ceļš''}\\
{\bf 2016.\ gada 17.\ septembris, Rīga (1.\ diena)}
\end{center}


\begin{problem}[BW.TST.2016.1]
Uz tāfeles uzrakstīti $2016$ skaitļi: $\frac{1}{2016},\frac{2}{2016},\frac{3}{2016},\ldots,\frac{2016}{2016}$. 
Vienā gājienā atļauts izvēlēties jebkurus divus uz tāfeles uzrakstītos skaitļus $a$ un $b$, tos nodzēst, 
un to vietā uzrakstīt skaitli $3ab - 2a - 2b + 2$. Noteikt, 
kāds skaitlis būs palicis uzrakstīts uz tāfeles pēc $2015$ gājieniem!
\end{problem}

\begin{problem}[BW.TST.2016.2]
Doti tādi naturāli skaitļi $m$, $n$ un $X$, ka $X \geq m$ un $X \geq n$. Pierādīt, ka var atrast
tādus divus veselus skaitļus $u$ un $v$, ka $|u| + |v| > 0$, $|u| \leq \sqrt{X}$, $|v| \leq \sqrt{X}$ un 
\[ 0 \leq mu  + nv \leq 2\sqrt{X}. \]
\end{problem}

\begin{problem}[BW.TST.2016.3]
Dots $2016$-ās pakāpes polinoms $P$ ar reāliem koeficientiem un 
kvadrātlisks polinoms $Q$ ar reāliem koeficientiem. Vai iespējams, ka 
polinoma $P(Q(x))$ saknes ir tieši šie visi skaitļi:
\[ -2015,\;-2014,\;\ldots,\;-2,\;-1,\;1,\;2,\;\ldots,\;2016,\;2017? \]
\end{problem}

\begin{problem}[BW.TST.2016.4]
Atrast visas funkcijas $f\;:\;\mathbb{R} \rightarrow \mathbb{R}$, kas definētas 
reāliem skaitļiem, pieņem reālas vērtības un kurām visiem reāliem $x$ un $y$ izpildās 
vienādība: 
\[ f\left( 2^x + 2y \right)  = 2^y f(f(x))f(y). \]
\end{problem}


\begin{problem}[BW.TST.2016.5]
Doti reāli pozitīvi skaitļi $a$, $b$, $c$ un $d$, kuriem izpildās vienādības 
\[ a^2 + ab + b^2 = 3c^2 \;\; \mbox{un} \;\; a^3 + a^2b + ab^2 + b^3 = 4d^3. \]
Pierādīt, ka 
\[ a + b + d \leq 3c. \]
\end{problem}

\begin{problem}[BW.TST.2016.6]
Dots naturāls skaitlis $n$, kuram var atrast pirmskaitli, kas ir mazāks nekā 
$\sqrt{n}$ un kas nav $n$ dalītājs. Virkne 
$a_1,a_2,\ldots,a_n$ ir skaitļi $1,2,\ldots,n$ sakārtoti kaut kādā secībā. 
Šai virknei atradīsim garāko augošo apakšvirkni $a_{i_1} < a_{i_2} < \cdots < a_{i_k}$, 
($i_1 < \cdots < i_k$) un garāko dilstošo apakšvirkni 
$a_{j_1} > \cdots > a_{j_l}$, ($j_1 < \cdots < j_l$). 
Pierādīt, ka vismaz viena no šīm divām apakšvirknēm 
$a_{i_1},\ldots,a_{i_k}$ un $a_{j_1},\ldots,a_{j_l}$ satur skaitli, kas nav $n$ dalītājs.  
\end{problem}

\begin{problem}[BW.TST.2016.7]
Nekurnekadzemes parlamentā visa darbība notiek komisijās, kuru 
sastāvā ir tieši trīs deputāti. Konstitūcija nosaka, ka jebkuru triju 
komisiju apvienojumā jābūt vismaz pieciem deputātiem. 
Komisiju saimi sauksim par {\em kliķi}, ja katrām divām no tām 
ir tieši divi kopīgi deputāti, 
bet, pievienojot šai saimei jebkuru citu komisiju, šis nosacījums
vairs neizpildās. 
Pierādīt, ka divām dažādām kliķēm nevar būt vairāk par vienu kopīgu komisiju. 
\end{problem}

\begin{problem}[BW.TST.2016.8]
Šaha festivālā piedalījās $3n - 2$ dalībnieki, daži no tiem savā starpā izspēlēja vienu 
šaha partiju. Pierādīt, ka izpildās vismaz viens no šiem apgalvojumiem:
\begin{alphlist}
\item Var atrast tādus $n$ šahistus $A_1,A_2,\ldots,A_n$, ka $A_i$ ir izspēlējis partiju ar 
$A_{i+1}$ visiem $i = 1,\ldots,n-1$. 
\item Var atrast septiņus šahistus $B_1,\ldots,B_7$, kuri cits ar citu nav spēlējuši, izņemot 
varbūt trīs pārus $(B_1,B_2)$, $(B_3,B_4)$ un $(B_5,B_6)$, kas katrs var gan būt, 
gan nebūt izspēlējuši šaha partiju. 
\end{alphlist}
\end{problem}

\begin{problem}[BW.TST.2016.9]
Skaitļi no $1$ līdz $2016$ ir sadalīti trīs (nešķeļošās) apakškopās $A$, $B$ un $C$, 
katra no tām satur tieši $672$ skaitļus. 
Pierādīt, ka var atrast trīs tādus skaitļus, katru no citas apakškopas, ka divu 
no tiem summa ir vienāda ar trešo. 
\end{problem}

\begin{problem}[BW.TST.2016.10]
Uz bezgalīgas rūtiņu lapas ir novietoti bezgaglīgi daudzi $1 \times 2$ rūtiņu taisnstūri, 
to malas iet pa rūtiņu līnijām, un tie nesaskaras cits ar citu pat ne ar stūriem. 
Vai tiesa, ka atlikušo rūtiņu lapu var pilnībā noklāt ar $1 \times 2$ rūtiņu tainstūriem?
\end{problem}


\begin{center}
{\bf 2016.\ gada 18.\ septembris, Rīga (2.\ diena)}
\end{center}

\begin{problem}[BW.TST.2016.11]
Vai kvadrātu, kura laukums ir $2015$, var sagriezt ne vairāk kā piecos
daudzstūros tā, lai no šiem daudzstūriem pēc tam varētu salikt taisnstūri, kura malu garumi ir naturāli skaitļi? 
\end{problem}


\begin{problem}[BW.TST.2016.12]
Kādiem naturāliem skaitļiem $m$ un $n$ plaknē var atrast punktus $A_1, A_2,\ldots,A_n$ un 
$B_1,B_2,\ldots,B_m$, kuri visi ir dažādi un jebkuram plaknes punktam $P$ izpildās vienādība
\[ |PA_1|^2 + |PA_2|^2 + \cdots + |PA_n|^2 = 
|PB_1|^2 + |PB_2|^2 + \cdots + |PB_m|^2 ? \]
\end{problem}

\begin{problem}[BW.TST.2016.13]
Plaknē izvēlēti $4$ punkti $A$, $B$, $C$ un $X$ tā, ka izpildās nevienādības 
$AX \leq BC$, $BX \leq AX$ un $CX \leq AX$. Pierādīt, ka 
$\sphericalangle BAC \leq 150^{\circ}$. 
\end{problem}

\begin{problem}[BW.TST.2016.14]
$AK$ ir dažādmalu trijstūra $ABC$ bisektrise, tajā ievilktā riņķa līnija pieskaras 
tā malām $BC$ un $AC$ attiecīgi punktos $D$ un $E$. $AD$ un $EK$ krustojas punktā $P$. 
Pierādīt, ka taisnes $PC$ un $AK$ ir perpendikulāras. 
\end{problem}

\begin{problem}[BW.TST.2016.15]
Trijstūrī $ABC$ $\sphericalangle BAC = 60^{\circ}$, tā augstumi $AD$, $BE$ un $CF$ 
krustojas punktā $H$. Malu $BC$, $CA$ un $AB$ viduspunkti ir attiecīgi $K$, $L$ un $M$. 
Pierādīt, ka nogriežņu $AH$, $DK$, $EL$ un $FM$ viduspunkti atrodas uz vienas riņķa līnijas. 
\end{problem}

\begin{problem}[BW.TST.2016.16]
Kāda ir izteiksmes
\[ \operatorname{LKD}\left( n^2 + 3, (n+1)^2 + 3 \right) \]
lielākā iespējamā vērtība naturāliem $n$?  
\end{problem}

\begin{problem}[BW.TST.2016.17]
Vai var atrast piecus tādus pirmskaitļus $p, q, r, s, t$, ka 
\[ p^3 + q^3 + r^3 + s^3 = t^3 ? \] 
\end{problem}

\begin{problem}[BW.TST.2016.18]
Atrisināt veselos skaitļos vienādojumu sistēmu:
\[ 
\left\{ \begin{array}{ccl}
a^3 & = & abc + 2a + 2c \\
b^3 & = & abc - c\\
c^3 & = & abc - a + b\\
\end{array} \right. 
\]
\end{problem}

\begin{problem}[BW.TST.2016.19]
Pierādīt, ka vienādojumam 
\[ x^{2015} + y^{2015} = z^{2016} \]
ir bezgalīgi daudz atrisinājumu, kur $x$,$y$ un $z$ ir dažādi naturāli skaitļi. 
\end{problem}

\begin{problem}[BW.TST.2016.20]
Kādiem naturālu skaitļu pāriem $(a,b)$ izteiksmes 
\[ \left(a^6 + 21a^4b^2 + 35a^2 b^4 + 7b^6 \right) \left( b^6 + 21 b^4 a^2 + 35 b^2 a^4 + 7a^6 \right) \]
vērtība ir pirmskaitļa pakāpe?
\end{problem}


\end{document}


