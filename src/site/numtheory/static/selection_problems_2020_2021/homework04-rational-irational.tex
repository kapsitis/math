\documentclass[a4paper,12pt]{article}

%\usepackage{amsmath,amssymb,multicol,tikz,enumitem}
\usepackage[margin=2cm]{geometry}
%\usetikzlibrary{calc}
\usepackage{amsmath}
\usepackage{amsthm}
\usepackage{thmtools}
\usepackage{hyperref}
\usepackage{enumerate}
\usepackage{xcolor}

\pagestyle{empty}

\newcommand\N{\mathbf{N}}
\newcommand\Q{\mathbf{Q}}
\newcommand\R{\mathbf{R}}
\newcommand\Z{\mathbf{Z}}

\usepackage{array}
\newcolumntype{P}[1]{>{\centering\arraybackslash}p{#1}}

\newcommand\indd{${}$\hspace{20pt}}

\declaretheoremstyle[headfont=\normalfont\bfseries,notefont=\mdseries\bfseries,bodyfont = \normalfont,headpunct={:}]{normalhead}
\declaretheorem[name={Uzdevums}, style=normalhead,numberwithin=section]{problem}

\setcounter{section}{4}

\setlength\parindent{0pt}

\begin{document}

\begin{center}
\parbox{3.5cm}{\flushleft\bf Skaitļu teorija\linebreak NMS juniori} \hfill {\bf\LARGE Mājasdarbs \#4} \hfill \parbox{3.5cm}{\flushright\bf 2020./2021.m.g.} \\[2pt]
\rm\small 2021.gada 27.marts
\end{center}

%\hrule\vspace{2pt}\hrule
\hrule

\vspace{10pt}
{\bf Iesniegšanas termiņš:} 2021.g.\ 17.aprīlis\\
{\bf Kam iesūtīt:} {\tt kalvis.apsitis}, domēns {\tt gmail.com}


\vspace{10pt}
\begin{problem}
Polinomā $f(n) = n^2 + n + 41$ pēc kārtas ievieto veselos nenegatīvos skaitļus $n=0,1,2,3,\ldots$ un iegūst sekojošas vērtības:
\[  41, 43, 47, 53, 61, 71, 83, 97, 113, 131, 151, 173, 197, 223, 251, 281, 313, 347, 383, 421, 461, \ldots \]
Izrakstām šīs virknes locekļu ciparus un iegūstam skaitli $\alpha = 0.4143475361718397113\ldots$.
Pierādīt, ka skaitlis $\alpha$ ir iracionāls.
\end{problem}

\vspace{10pt}
\begin{problem}
Definējam virkni $L_0,L_1,L_2,\ldots$ ar šādu formulu: 
\[ L_n = \left( \frac{1 + \sqrt{5}}{2} \right)^n + \left( \frac{1 - \sqrt{5}}{2} \right)^n. \]

% https://artofproblemsolving.com/community/c6t566324f6h2360868_lucas_sequence_primes_divisibility
\begin{enumerate}[(a)]
\item Pierādīt, ka virknes $L_n$ locekļi apmierina sakarību: 
\begin{equation}
\label{eq:lucas}
L_n^2 = 5F_n^2 + 4(-1)^n\;\;\text{visiem $n = 0,1,2,\ldots$.}
\end{equation}
$F_n$ apzīmē Fibonači skaitļus ($F_0 = 0$, $F_1 = 1$, un $F_n = F_{n-1} + F_{n-2}$, ja $n \geq 2$). 
\item Pierādīt, ka eksistē bezgalīgi daudzi pirmskaitļi $p$, kuri dala kādu virknes $L_k$ locekli.
\item Pierādīt, ka $L_k$ nedalās ar pirmskaitļiem $p = 20m + 13$
(piemēram $p = 13,53,73,113,\ldots$)\\
{\em Ieteikums.} Var izmantot formulu (\ref{eq:lucas}).
\end{enumerate}
\end{problem}

\vspace{10pt}
\begin{problem}
Ar $a_n$ apzīmējam $n$-to locekli virknē\\ $1, 2,2, 3,3,3, 4,4,4,4, 5,5,5,5,5, 6,6,6,6,6,6,\ldots$, 
ko veido, atkārtojot katru naturālu skaitli $k$ tieši $k$ reizes. Pierādīt, ka
\[ a_n = \left\lfloor \sqrt{2n} + \frac{1}{2} \right\rfloor. \]
\end{problem}

% https://artofproblemsolving.com/community/c6t269f6h495739_floor_function
% https://en.wikipedia.org/wiki/Beatty_sequence#Rayleigh_theorem
\vspace{10pt}
\begin{problem}
Pieņemsim, ka  $\gamma, \delta$ ir pozitīvi iracionāli skaiļi, turklāt ${\displaystyle \frac{1}{\gamma}+\frac{1}{\delta}=1}$. Definējam divas virknes:
\[ a_n=\lfloor n\gamma \rfloor,\;\;\; b_n =\lfloor n\delta\rfloor. \] 
Pierādīt, ka ikviens naturāls skaitlis parādās tieši vienā no abām virknēm $a_n$ vai $b_n$ (bet ne abās virknēs).
\end{problem}

\vspace{10pt}
\begin{problem} Kopu $A$ sauksim par {\em sanumurējamu}, ja $|A| \leq |\N|$, t.i.\ eksistē injektīva funkcija $f: A \rightarrow \N$. 
(Citiem vārdiem, kopas $A$ elementiem var piekārtot numurus, kas ir naturāli skaitļi, tā, lai neviens numurs netiktu izmantots divreiz). 
\begin{enumerate}[(a)]
\item Ar $S$ apzīmējam visu naturālo skaitļu virkņu kopu (tās elementi ir bezgalīgas virknes no naturāliem skaitļiem, kur locekļi var arī atkārtoties). 
Vai kopa $S$ ir sanumurējama?
\item Ar $S_1$ apzīmējam visu {\em nedilstošo} naturālo skaitļu virkņu kopu (tā satur bezgalīgas virknes $x_1,x_2,x_3,\ldots$, kurām 
$x_1 \leq x_2 \leq x_3 \leq \ldots$). Vai kopa $S_1$ ir sanumurējama?
\item Ar $S_2$ apzīmējam visu {\em neaugošo} naturālo skaitļu virkņu kopu (tā satur bezgalīgas virknes $x_1,x_2,x_3,\ldots$, kurām 
$x_1 \geq x_2 \geq x_3 \geq \ldots$). Vai kopa $S_2$ ir sanumurējama?
\end{enumerate}
{\em Ieteikums.} Tām kopām, kuras nav sanumurējamas, var izmantot Kantora diagonalizāciju (sk. \url{https://bit.ly/3lYe8cy}); 
sanumurējamām kopām pietiek atrast veidu, kā sanumurēt.
\end{problem}



\end{document}









