\documentclass[11pt]{article}
\usepackage{ucs}
\usepackage[utf8x]{inputenc}
\usepackage{changepage}
\usepackage{graphicx}
\usepackage{amsmath}
\usepackage{gensymb}
\usepackage{amssymb}
\usepackage{enumerate}
\usepackage{tabularx}
\usepackage{lipsum}
\usepackage{amsthm}
\usepackage{thmtools}


\usepackage{fontspec} % loaded by polyglossia, but included here for transparency 
\usepackage{polyglossia}

\usepackage{xeCJK}
\setCJKmainfont{SimSun}
\setmainlanguage{russian} 
\setotherlanguage{english}

\newfontfamily\cyrillicfont[Script=Cyrillic]{Times New Roman}
\newfontfamily\cyrillicfontsf[Script=Cyrillic]{Arial}
\newfontfamily\cyrillicfonttt[Script=Cyrillic]{Courier New}

\oddsidemargin 0.0in
\evensidemargin 0.0in
\textwidth 6.27in
\headheight 1.0in
\topmargin 0.0in
\headheight 0.0in
\headsep 0.0in
%\textheight 9.69in
\textheight 9.00in
 
\setlength\parindent{0pt}

\newenvironment{myenv}{\begin{adjustwidth}{0.4in}{0.4in}}{\end{adjustwidth}}
\renewcommand{\abstractname}{Anotācija}
\renewcommand\refname{Atsauces}

%\newenvironment{uzdevums}[1][\unskip]{%
%\vspace{3mm}
%\noindent
%\textbf{#1:}
%\noindent}
%{}


% http://tex.stackexchange.com/questions/196961/thmtools-declaration-for-theorem-and-proof
\declaretheoremstyle[headfont=\normalfont\bfseries,notefont=\mdseries\bfseries,bodyfont = \normalfont,headpunct={:}]{normalhead}
\declaretheorem[name={Uzdevums}, style=normalhead,numberwithin=section]{problem}

\def\changemargin#1#2{\list{}{\rightmargin#2\leftmargin#1}\item[]}
\let\endchangemargin=\endlist 


\newcommand{\subf}[2]{%
  {\small\begin{tabular}[t]{@{}c@{}}
  #1\\#2
  \end{tabular}}%
}



\newcounter{alphnum}
\newenvironment{alphlist}{\begin{list}{(\Alph{alphnum})}{\usecounter{alphnum}\setlength{\leftmargin}{2.5em}} \rm}{\end{list}}

\newenvironment{zhtext}{\fontfamily{MS PGothic}\selectfont}{\par}


\makeatletter
\let\saved@bibitem\@bibitem
\makeatother

\usepackage{bibentry}
%\usepackage{hyperref}

\newenvironment{tulkojums}[1][\unskip]{%
\begin{changemargin}{8mm}{8mm}
\fontsize{9}{11}
\selectfont
\textbf{#1:}
}
{ 
\fontsize{12}{14}
\selectfont
\end{changemargin}
}

\setcounter{section}{0}


\begin{document}

\begin{center}
{\Large \bf Atlases sacensības komandu olimpiādei ``Baltijas ceļš''}\\
{\bf 2017.\ gada 23.\ septembris, Rīga (1.\ diena)}
\end{center}

{\footnotesize \em 
Risināšanas laiks: 4 stundas un 30 minūtes.\\
Jautājumus drīkst uzdot pirmo 30 minūšu laikā.\\ 
Atļauts izmantot tikai rakstāmpiederumus, lineālu un cirkuli.
}

\begin{problem}[BwTst2017.1]
Pierādīt, ka visiem reāliem $x>0$ ir spēkā vienādība
\[ \sqrt{\frac{1}{3x+1}} + \sqrt{\frac{x}{x+3}} \geq 1. \]
Kurām $x$ vērtībām ir spēkā vienādība?
\end{problem}

\begin{problem}[BwTst2017.2]
Atrast visus reālu skaitļu pārus $(x,y)$, kas apmierina vienādojumu
\[ \frac{(x+y)(2 - \sin(x+y))}{4 \sin^2(x+y)} = \frac{xy}{x+y}. \]
\end{problem}

\begin{problem}[BwTst2017.3]
Atrast visas funkcijas $f(x): \mathbb{Z} \rightarrow \mathbb{Z}$, kas
definētas veseliem skaitļiem, pieņem veselas vērtības un visiem 
$x,y \in \mathbb{Z}$ izpildās
\[ f(x+y) + f(xy) = f(x)f(y) + 1. \]
\end{problem}

\begin{problem}[BwTst2017.4]
Polinoma $P(x) = 2x^3 - 30x^2 + cx$ vērtības pie kādiem trīs pēc kārtas 
sekojošiem veseliem skaitļiem arī ir attiecīgi trīs pēc kārtas sekojoši veseli skaitļi. 
Nosakiet šīs vērtības!
\end{problem}


\begin{problem}[BwTst2017.5]
Burvju astoņstūris ar astoņstūris, kura malas iet pa rūtiņu lapas rūtiņu 
līnijām un malu garumi ir $1,2,3,4,5,6,7,8$ (jebkādā secībā). 
Kāds ir lielākais iespējamais burvju astoņstūra laukums?
\end{problem}

\begin{problem}[BwTst2017.6]
$13 \times 13$ rūtiņu laukuma katrā rūtiņā ierakstīts naturāls skaitlis. 
Pierādīt, ka var izvēlēties $2$ rindas un $4$ kolonnas tā, ka to $8$ krustpunktos 
ierakstīto skaitļu summa dalās ar $8$.  
\end{problem}

\begin{problem}[BwTst2017.7]
Uz lapas viens aiz otra augošā secībā bez atstarpēm uzrakstīti visi sešciparu 
naturālie skaitļi no $100000$ līdz $999999$. Kāda ir lielākā $k$ vērtība, kurai 
šajā virknē vismaz divās dažādās vietās iespējams atrast vienu un to pašu 
$k$-ciparu skaitli?
\end{problem}

\begin{problem}[BwTst2017.8]
Šaha turnīrā piedalījās $2017$ šahisti, katrs ar katru izspēlēja tieši vienu šaha
partiju. Sauksim šahistu trijotni $A,B,C$ par principiālu, ja $A$ uzvavrēja $B$, 
$B$ uzvarēja $C$, bet $C$ uzvarēja $A$. Kāds ir lielākais iespējamais principiālu 
šahistu trijotņu skaits? 
\end{problem}




\begin{center}
{\bf 2017.\ gada 24.\ septembris, Rīga (2.\ diena)}
\end{center}

\begin{problem}[BwTst2017.9]
Vienādsānu trijstūrī $ABC$, kurā $AC=BC$ un $\sphericalangle{}ABC < 60^{\circ}$, 
$I$ un $O$ ir attiecīgi ievilktās un apvilktās riņķa līniju centri. 
Trijstūrim $BIO$ apvilktā riņķa līnija vēlreiz krusto malu $BC$ punktā $D$. 
Pierādīt, ka 
\begin{enumerate}
\item taisnes $AC$ un $DI$ ir paralēlas, 
\item taisnes $OD$ un $IB$ ir perpendikulāras.
\end{enumerate}
\end{problem}


\begin{problem}[BwTst2017.10]
Šaurleņķa trijstūra $ABC$, kuram $AC < AB$, apvilktās riņķa līnijas rādiuss ir $R$, 
tās loka $BC$ (kurš nesatur $A$) viduspunkts ir $S$. Uz augstuma $AD$ pagarinājuma 
atlikts punkts $T$ tā, ka $D$ atrodas starp $A$ un $T$ un $AT = 2R$. 
Pierādīt, ka $\sphericalangle{} AST = 90^{\circ}$. 
\end{problem}

\begin{problem}[BwTst2017.11]
Uz trijstūra $ABC$ bisektrises $AL$ pagarinājuma atlikts punkts $P$ tā, ka 
$PL = AL$. Pierādīt, ka trijstūra $PBC$ perimetrs nepārsniedz trijstūra $ABC$ perimetru. 
\end{problem}

\begin{problem}[BwTst2017.12]
Šaurleņķa trijstūrim $ABC$ apvilktajai riņķa līnijai $\omega$ novilkts diametrs $AK$, 
uz nogriežņa $BC$ izvēlēts patvaļīgs punkts $M$, taisne $AM$ krusto $\omega$ punktā $Q$. 
Perpendikula, kas no $M$ vilkts pret $AK$, pamats ir $D$, pieskare, kas riņķa līnijai 
$\omega$ novilkta caur punktu $Q$, krusto taisni $MD$ punktā $P$. 
Uz $\omega$ izvēlēts punkts $L$ (atšķirīgs no $Q$) tā, ka $PL$ ir $\omega$ pieskare. 
Pierādīt, ka punkti $L$, $M$ un $K$ atrodas uz vienas taisnes. 
\end{problem}

\begin{problem}[BwTst2017.13]
Pierādīt, ka skaitlis 
\[ \sqrt{1 + \frac{1}{n^2} + \frac{1}{(n+1)^2}} \]
ir racionāls visiem naturāliem $n$. 
\end{problem}

\begin{problem}[BwTst2017.14]
Vai var atrast trīs naturālus skaitļus $a,b,c$, kuru lielākais kopīgais dalītājs ir $1$
un kuriem izpildās vienādība
\[ ab + bc+ac = (a+b-c)(b+c-a)(c+a-b)? \]
\end{problem}

\begin{problem}[BwTst2017.15]
Ciparu virkni $D = d_{n-1}d_{n-2}\ldots{}d_0$ sauksim par stabilu skaitļa nobeigumu, 
ja jebkuram naturālam skaitlim $m$, kas beidzas ar $D$, arī jebkura tā naturāla pakāpe $m^k$
beidzas ar $D$. Pierādīt, ka katram naturālam $n$ ir tieši četri stabili skaitļa nobeigumi, 
kuru garums ir $n$. 
\end{problem}

\begin{problem}[BwTst2017.16]
Virknes $a_1,a_2,\ldots,a_{2016}$ un $b_1,b_2,\ldots,b_{2016}$ katra satur visus naturālos 
skaitļus no $1$ līdz $2016$ katru tieši vienu reizi 
(citiem vārdiem sakot tās abas ir skaitļu $1,2,\ldots,2016$ permutācijas). 
Pierādīt, ka var atrast tādus dažādus indeksus $i$ un $j$, ka $a_ib_i - a_jb_j$ dalās ar $2017$.
\end{problem}


\end{document}
