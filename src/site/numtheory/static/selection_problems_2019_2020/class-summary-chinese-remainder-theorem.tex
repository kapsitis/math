\documentclass[jou]{apa6}

\usepackage[american]{babel}
\usepackage{hyperref}
\usepackage{amsthm}
\usepackage{thmtools}

\usepackage{csquotes}
\usepackage[style=apa,sortcites=true,sorting=nyt,backend=biber]{biblatex}
\DeclareLanguageMapping{american}{american-apa}
\addbibresource{bibliography.bib}

\title{2019-12-14 Class Summary: Chinese Remainder Theorem}

\author{Kalvis Aps\={\i}tis, {\small \tt kalvis.apsitis}{\small \tt @gmail.com}}
\affiliation{Riga Business School (RBS), University of Latvia (LU)}


\leftheader{NMS Selection Training in Number Theory: 2019-12-14 Class Summary}

\declaretheoremstyle[headfont=\normalfont\bfseries,notefont=\mdseries\bfseries,bodyfont = \normalfont,headpunct={:}]{normalhead}
\declaretheorem[name={Example}, style=normalhead,numberwithin=section]{problem}

\setcounter{section}{2}

\abstract{This document lists the key results from the December 14, 2019 class in 
Number Theory. This training for competition math is aimed at 16\textendash{}18 year 
olds (typically, Grades 10\textendash{}12).
}

\keywords{Chinese remainder theorem, Bezout's identity, Modular arithmetic.}

\begin{document}
\maketitle


\section{Examples}

These are {\bf not} the homework problems; it is just a supplementary study material 
with examples taken from the lecture; most of them were analyzed during the lecture.
In case you have forgotten something, hints for these examples are given at the end of this document.
Full notes and solutions in Latvian: \url{http://linen-tracer-682.appspot.com/numtheory-tales/tale-numtheory-jun03-crt/content.html#/section}

\begin{problem}
Find integers $x,y$ such that $18x + 42y = 6$. 
(Here we have chosen $6 = \mathit{gcd}(18,42)$.)
\end{problem}

\begin{problem} 
Prove that the sequence $1,11,111,\ldots$ contains
an infinite subsequence such that any two members of that subsequence
are mutually prime.
\end{problem}

{\bf Blankinship Algorithm:} Blankinship Algorithm can be 
used to find solutions for the 
Bezout's identity: the integers $x,y$ such that $ax+by=d$.
It is explained here: \url{http://mathworld.wolfram.com/BlankinshipAlgorithm.html}.
It applies Gaussian row operations to a $2 \times 3$ matrix: you 
should know how to subtract one row from another.

{\bf Inverse Congruence Class:} For a congruence class $a$ 
that is mutually prime with modulo $m$, denote by $a^{-1}$ 
a congruence class such that $a^{-1}\cdot a \equiv 1$ modulo $m$. 
(In other words: Given a number $a < m$, find some number $b$
such that $a \cdot b$ gives remainder $1$ when divided by $m$.)


\begin{problem} 
Find inverses $1^{-1}$, $3^{-1}$, $5^{-1}$, $7^{-1}$, $9^{-1}$, $11^{-1}$, $13^{-1}$, $15^{-1}$
(all modulo $16$).
\end{problem}

{\bf Chinese Remainder Theorem:}
For multiple mutually prime modulos $m_1,m_2,\ldots,m_k$ one can find $x$ that is 
congruent to any numbers $a_1,a_2,\ldots,a_k$ with respect to those modulos.

\begin{problem} 
Find a natural number $x$ that is a solution to this system of congruences:
$$\left\{ \begin{array}{l}
x \equiv 1\;(\mathit{mod}\,3)\\
x \equiv 2\;(\mathit{mod}\,5)\\
x \equiv 3\;(\mathit{mod}\,7)
\end{array} \right.$$
\end{problem}


\begin{problem} 
Assume that you want to find number $x$ that satisfies both congruences:
$$\left\{ \begin{array}{l}
x \equiv 4\;(\mathit{mod}\,5)\\
x \equiv 6\;(\mathit{mod}\,11)
\end{array} \right.$$
You can solve this "graphically" - build a $5 \times 11$ table representing
all possible pairs of remainders, when you divide numbers by $5$ and by $11$. 
Fill in this table by choosing $x=1,2,3,\ldots$ until you find the necessary 
combination of remainders $(4;6)$. 
\end{problem}

\begin{problem} 
Find the smallest positive integer $n$, such that 
numbers $\sqrt[5]{5n}$, $\sqrt[6]{6n}$, $\sqrt[7]{7n}$ are all positive integers.\\
(From {\em Vilniaus universiteto Matematikos ir informatikos fakulteto olimpiadas} - 
a Lithuanian olympiad for high school students by Vilnius university; 2016, 
Grade 10, P3.)
\end{problem}

\begin{problem} 
Prove that for each positive integer n, there are pairwise relatively prime integers
$k_0, k_1, \ldots, k_n$ , all strictly greater than $1$, 
such that $k_0 k_1 \ldots k_n - 1$ is the product of
two consecutive integers.
\end{problem}

\begin{problem} 
Prove that for every positive integer $n$, there exist integers $a$ and $b$ such that $4a^2 + 9b^2 - 1$ is divisible by $n$.\\
({\em Math Prize for Girls Olympiad, 2010, P2}). 
\end{problem}

\begin{problem}
Are there infinitely many Fibonacci numbers that give the following remainders when divided by $1001$:\\
{\bf (a)} remainder $0$; {\bf (b)} remainder $900$; {\bf (c)} remainder $1000$.
\end{problem}

\begin{problem}
Prove or disprove the following hypotheses.\\
{\bf (a)} For all $k \geq  2,$ each sequence of $k$ consecutive positive integers contains a number that is not divisible by any prime number less than $k$.\\
{\bf (a)} For all $k\geq 2,$ each sequence of $k$ consecutive positive integers contains a number that is relatively prime to all other members of the sequence.\\
({\em Baltic Way, 2016, P2}). 
\end{problem}

\newpage

\section{Hints for Some Examples}

{\bf Hint 2.1:} You can find this by trial and error for small numbers. 
Or you can run Euclidean Algorithm to find the GCD (greatest common divisor) 
of numbers $18$ and $42$. It will tell you, how many times you should add 
or subtract $18$ and $42$ to get number $6$. (Blaninship's method is essentially 
the same thing.)

{\bf Hint 2.2:} If you build the following sequence of mutual primes:
$2,3,7,43,\ldots$ (every next number equals the product of all the previous ones plus $1$), 
then the corresponding numbers $11$, $111$, $1111111$, and so on will give remainder $1$
every time you divide them one by another.

{\bf Hint 2.3:} In order to find, say, $9^{-1}$ (modulo $16$), you can 
try out all the odd remainders ($1,3,\ldots,13,15$). Or you can solve the
Bezout's identity $9x - 16y = 1$. Blankinship's algorithm again.

{\bf Hint 2.4:} One can build such a number step by step - 
first write all the numbers congruent to $1$ (modulo $3$): 
$1,4,7,\ldots$ until you find one that gives remainder $2$ when 
divided by $5$, and so on. (Since $3,5,7$ are mutually prime, 
Chinese remainder theorem promises that you will succeed.)

{\bf Hint 2.5:} Solution is shown in the table: \url{https://bit.ly/2MCzMmf}. 
Try to locate numbers
$0,1,2,3,\ldots$ in this table and see the sequence how they fill up the table. 
Similar ideas are used by problems that ask you to "Measure exactly $4$ liters
of water, given two jugs with volumes $5L$ and $11L$ respectively". 

{\bf Hint 2.6:} Search for $n$ in the form $n = 2^a3^b5^c7^d$. 
Then write the necessary conditions (as modular congruences) for all 
the unknown powers $a,b,c,d$. 

{\bf Hint 2.7:} Look at the polynomial $F(t) = t^2 + t + 1$ (it is 
a product of two consecutive numbers $t$ and $t+1$ plus $1$).\\
Note that all the remainders it gives, when divided by $2$, by $3$, etc. 
are periodic. And if $F(t)$ sometimes is divisible by a prime $p_1$ and
sometimes by a prime $p_2$, then eventually $F(t)$ (for some special 
arguments $t$) will be divisible by them both: $p_1 \cdot p_2$.\\
Now, all you need to show that there are infinitely many primes that 
sometimes divide the values of $F(t)$. At this point remember
the proof that there are infinitely many primes. Assume that this is not true - 
i.e. $F(t)$ is divisible by only finitely many primes. Then plug into 
$F(t)=t^2 + t + 1$ the number 
$t = p_1\cdot{}p_2\ldots{}\cdot{}p_k+1$ their product plus $1$.

{\bf Hint 2.8:} Use Chinese remainder theorem to avoid looking at {\em all} possible
$n$. Just look at the prime powers $p^k$ (and all the remaining $n$ can be 
obtained by combining the solutions for $p^k$ in a certain way).\\
Next, consider two separate cases: $n = 2^k$ (you can now pick $b$
so that it is inverse of $3$ modulo $2^k$ - so that the term $9b^2 - 1$
is congruent to $0$). On the other hand, if $n=p^k$ for some other $p \neq 2$, 
then pick $a$ equal to inverse of $2$ modulo $p^k$ for similar reasons.

{\bf Hint 2.9:} The remainders of Fibonacci numbers when divided by any fixed $d$
are periodic (because the pairs of neighboring remainders eventually 
start to repeat). What is more interesting: All the remainders of Fibonacci sequence
are "clean periodic" (not just "eventually periodic") - every remainder belongs to 
the period. If $F_0 = 0$ (divisible by any $d$), then it means that infinitely 
often $F_n$ will be divisible by that $d$.\\
{\bf (a)} is simple - since $F_0$ is divisible by $1001$, then the remainder $0$
is clearly in the period (modulo $1001$).\\
For {\bf (b)} and {\bf (c)} you need to factorize $1001$ as a product of three prime
factors and search for the combinations of remainders (as per Chinese remainder
theorem). 

{\bf Hint 2.10:} Statement {\bf (a)} is clearly false. Just start from $2$ and you
will find a sequence, where every member is divisible by some small prime.\\
For {\bf (b)} you need to express some segment of $k$ subsequent numbers as an overlap 
of several arithmetic progressions with prime differences $d < k$
(so that every progression contains at least two members among these $k$ subsequent 
numbers and all the $k$ numbers are covered at least by one sequence).\\
This is doable when $k=17$. See \url{https://bit.ly/2Q2XASA}. Finally - use 
Chinese remainder theorem to find an actual value $N$ such that the numbers
from $N$ to $N+16$ (inclusive) give the remainders you need.

\end{document}


