\documentclass[11pt]{article}
\usepackage{ucs}
\usepackage[utf8x]{inputenc}
\usepackage{changepage}
\usepackage{graphicx}
\usepackage{amsmath}
\usepackage{gensymb}
\usepackage{amssymb}
\usepackage{enumerate}
\usepackage{tabularx}
\usepackage{lipsum}

\oddsidemargin 0.0in
\evensidemargin 0.0in
\textwidth 6.27in
\headheight 1.0in
\topmargin -0.1in
\headheight 0.0in
\headsep 0.0in
%\textheight 9.69in
\textheight 9.50in

\setlength\parindent{0pt}

\newenvironment{myenv}{\begin{adjustwidth}{0.4in}{0.4in}}{\end{adjustwidth}}
\renewcommand{\abstractname}{Anotācija}
\renewcommand\refname{Atsauces}

\newenvironment{uzdevums}[1][\unskip]{%
\vspace{3mm}
\noindent
\textbf{#1:}
\noindent}
{}

\newcommand{\subf}[2]{%
  {\small\begin{tabular}[t]{@{}c@{}}
  #1\\#2
  \end{tabular}}%
}



\newcounter{alphnum}
\newenvironment{alphlist}{\begin{list}{(\Alph{alphnum})}{\usecounter{alphnum}\setlength{\leftmargin}{2.5em}} \rm}{\end{list}}


\makeatletter
\let\saved@bibitem\@bibitem
\makeatother

\usepackage{bibentry}
%\usepackage{hyperref}


\begin{document}

\thispagestyle{empty}

{\Large \bf NMS izlases nodarbība, 2019-06-21}

{\small
\begin{uzdevums}[IMO.SHL.2014.N6]
Ar $a_1 < a_2 <  \cdots <a_n$ apzīmējam naturālus skaitļus, kas ir 
savstarpēji pirmskaitļi. Turklāt $a_1$ ir pirmskaitlis un
$a_1 \geq n + 2$. Reālās taisnes nogrieznī 
$I = [0, a_1 a_2  \cdots a_n ]$ 
atzīmējam visus veselos skaitļus, kas dalās ar vismaz vienu no 
skaitļiem
$a_1, \ldots, a_n$. Šie punkti sadala $I$ mazākos nogriežņos.
Pierādīt, ka šo nogriežņu garumu kvadrātu summa dalās ar $a_1$. 
\end{uzdevums}


\begin{uzdevums}[IMO.SHL.2014.N7]
Dots naturāls skaitlis $c \ge 1$. Definējam naturālu skaitļu 
virkni ar vienādībām $a_1 = c$ un
$$a_{n+1}=a_n^3-4c\cdot a_n^2+5c^2\cdot a_n+c$$ 
visiem $n \geq 1$. 
Pierādīt, ka jebkuram naturālam $n \geq 2$ eksistē
pirmskaitlis $p$, ar kuru dalās $a_n$, bet nedalās 
neviens no skaitļiem $a_1,\ldots,a_{n-1}$.
\end{uzdevums}

\begin{uzdevums}[IMO.SHL.2014.N8]
Katram reālam skaitlim $x$, ar $||x||$ apzīmējam 
attālumu starp $x$ un tuvāko veselo skaitli. 
Pierādīt, ka jebkuram naturālu skaitļu pārim $(a,b)$
eksistē nepāru pirmskaitlis $p$ un naturāls skaitlis $k$, 
kas apmierina sakarību:
$\left|\left|\frac{a}{p^k}\right|\right| + 
\left|\left|\frac{b}{p^k}\right|\right| + 
\left|\left|\frac{a+b}{p^k}\right|\right|=1.$
\end{uzdevums}


\begin{uzdevums}[IMO.SHL.2015.N8]
Katram naturālam skaitlim $n$, kura sadalījums pirmreizinātājos ir 
$n = \prod_{n=1}^k p_i^{\alpha_i}$, definējam
$$\mho(n) = \sum_{i:p_i > 10^{100}} \alpha_i.$$
Tātad, $\mho(n)$ ir skaitļa pirmreizinātāju skaits, kuri lielāki par
$10^{100}$, kas summēti, ņemot vērā atkārtojumus.
Atrast visas stingri augošas funkcijas $f:\mathbb{Z} \rightarrow \mathbb{Z}$, 
ka visiem veseliem $a$ un $b$, kam $a>b$, izpildās sakarība:
$\mho\left( f(a) - f(b) \right) \leq \mho(a-b)$
\end{uzdevums}

\begin{uzdevums}[IMO.SHL.2016.N7]
Ar $n$ apzīmēts nepāru naturāls skaitlis. Dekarta plaknē 
izraudzīts daudzstūris (vienkārša, slēgta lauzta līnija) $P$, 
kura laukums ir $S$. Visām tā virsotnēm abas koordinātes
ir veseli skaitļi, un visu tā malu garumu kvadrāti dalās ar $n$. 
Pierādīt, ka $2S$ ir vesels skaitlis, kas dalās ar $n$.
\end{uzdevums}


\begin{uzdevums}[IMO.SHL.2016.N8]
Atrast visus polinomus $P(x)$ ar nepāru pakāpi $d$ un 
veseliem koeficientiem, kas apmierina sekojošu īpašību: 
Katram naturālam skaitlim $n$ eksistē $n$ naturāli 
skaitļi $x_1,x_2,\ldots,x_n$, ka 
${\displaystyle \frac{1}{2} < \frac{P(x_i)}{P(x_j)} < 2}$ 
un ${\displaystyle \frac{P(x_i)}{P(x_j)}}$
vienāds ar racionālu skaitli kāpinātu pakāpē $d$ 
(visiem indeksu pāriem $i$ un $j$, kur $1 \leq i,j \leq n$). 
\end{uzdevums}


\begin{uzdevums}[IMO.SHL.2017.N6]
Atrast mazāko naturālo skaitli $n$ vai pierādīt, ka tāds neeksistē, 
kam būtu sekojoša īpašība: Ir bezgalīgi daudz tādu 
pozitīvu racionālu skaitļu komplektu 
$(a_1,a_2,\cdots,a_n)$, kuriem abi skaitļi
$a_1 + a_2 + \cdots + a_n$ un
$\frac{1}{a_1} + \frac{1}{a_2} + \cdots + \frac{1}{a_n}$
ir veseli.
\end{uzdevums}

\begin{uzdevums}[IMO.SHL.2017.N7]
Sakārtots veselu skaitļu pāris $(x, y)$ ir primitīvs punkts, 
ja $x$ un $y$ lielākais kopīgais
dalītājs ir $1$. Pierādiet, ka katrai galīgai primitīvu 
punktu kopai $S$ eksistē vesels pozitīvs skaitlis
$n$ un tādi veseli skaitļi $a_0,a_1,\ldots,a_n$, 
ka katram $(x, y)$ pārim no $S$ izpildās:\\
$a_0x^n + a_1x^{n-1}y + a_2x^{n-2}y^2 + 
\cdots + a_{n-1}xy^{n-1} + a_n y^n = 1.$
\end{uzdevums}


\begin{uzdevums}[IMO.SHL.2018.N3]
Definējam virkni $a_0, a_1, a_2,\ldots$
ar sakarību $a_n = 2^n + 2^{\lfloor n/2 \rfloor}$. 
Pierādīt, ka eksistē bezgalīgi daudzi šīs virknes locekļi, 
ko var izteikt kā (divu vai vairāku) šīs virknes 
locekļu summu. Kā arī bezgalīgi daudzi locekļi, 
kurus tādā veidā nevar izteikt.
\end{uzdevums}


\begin{uzdevums}[IMO.SHL.2018.N6]
Dota $f\,:\,\{1,2,3,\ldots\}\,\rightarrow\{2,3,\ldots\}$, 
funkcija, kas apmierina sakarību
$f(m+n)\,\mid\,f(m)+f(n)$ ($f(m+n)$ ir $f(m)+f(n)$ dalītājs)
visiem naturālu skaitļu pāriem $m,n$. Pierādīt, ka 
eksistē naturāls skaitlis $c>1$, kurš ir visu 
$f$ vērtību dalītājs.
\end{uzdevums}

\begin{uzdevums}[IMO.SHL.2018.N7]
Dots vesels skaitlis $n \geq 2018$ un 
$a_1,a_2,\ldots,a_n,b_1,b_2,\ldots,b_n$
ir pa pāriem dažādi naturāli skaitļi, kas 
nepārsniedz $5n$. Pieņemsim, ka virkne
$$\frac{a_1}{b_1},\frac{a_2}{b_2},\ldots,\frac{a_n}{b_n}$$
veido aritmētisku progresiju. Pierādīt, ka visi virknes locekļi 
ir savā starpā vienādi.
\end{uzdevums}
}

\end{document}
