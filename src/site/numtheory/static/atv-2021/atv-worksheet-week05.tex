\documentclass[a4paper,12pt]{article}

%\usepackage{amsmath,amssymb,multicol,tikz,enumitem}
\usepackage[margin=2cm]{geometry}
%\usetikzlibrary{calc}
\usepackage{amsmath}
\usepackage{amsthm}
\usepackage{thmtools}
\usepackage{hyperref}
\usepackage{enumerate}
\usepackage{xcolor}
\usepackage{fancyvrb}

\pagestyle{empty}

\newcommand\Q{\mathbf{Q}}
\newcommand\R{\mathbf{R}}
\newcommand\Z{\mathbf{Z}}

\usepackage{array}
\newcolumntype{P}[1]{>{\centering\arraybackslash}p{#1}}

\newcommand\indd{${}$\hspace{20pt}}

\declaretheoremstyle[headfont=\normalfont\bfseries,notefont=\mdseries\bfseries,bodyfont = \normalfont,headpunct={:}]{normalhead}
\declaretheorem[name={Uzdevums}, style=normalhead,numberwithin=section]{problem}

\setcounter{section}{100}
\setcounter{problem}{4}

\setlength\parindent{0pt}

\begin{document}

\begin{center}
\parbox{3.5cm}{\flushleft\bf Skaitļu teorija \newline ATV} \hfill {\bf\LARGE Uzdevumu lapa \#5} \hfill \parbox{3.5cm}{\flushright\bf 2021-02-03} %\\[2pt]
%\rm\small 2020.gada 15.novembris
\end{center}

%\hrule\vspace{2pt}\hrule
\hrule

%\vspace{10pt}
%{\bf Iesniegšanas termiņš:} 2020.g.\ 2.janvāris\\
%{\bf Kam iesūtīt:} {\tt kalvis.apsitis}, domēns {\tt gmail.com}

%\vspace{5pt}
%{\em (Par racionāliem un iracionāliem skaitļiem, skaitļu pierakstu un periodiskām virknēm.)}

Šie uzdevumi no nesenām ``Baltic Way'' olimpiādēm 
izmanto {\em Kāpinātāja pacelšanas lemmas} ({\em Lifting the Exponent Lemmas}).


\vspace{10pt}
\begin{problem}
%BW.2015.16.SOL
Ar $P(n)$ apzīmējam lielāko pirmskaitli, ar ko dalās $n$. Atrast visus
naturālos skaitļus $n \geq 2$, kam
\[ P(n) + \lfloor \sqrt{n} \rfloor = P(n+1) + \lfloor \sqrt{n+1} \rfloor. \]
(Piezīme: $\lfloor x \rfloor$ apzīmē lielāko veselo skaitli, kas nepārsniedz $x$.)
\end{problem}


\vspace{10pt}
\begin{problem}
%BW.2015.17.SOL
Atrast visus naturālos skaitļus $n$, kuriem $n^{n-1} - 1$ dalās ar $2^{2015}$,
bet nedalās ar $2^{2016}$.
\end{problem}


\vspace{10pt}
\begin{problem}
%BW.2016.5.SOL
Dots pirmskaitlis $p > 3$, kuram $p \equiv 3\;(\operatorname{mod}\,4)$.
Dotam naturālam skaitlim $a_0$ virkni
$a_0, a_1, \ldots$ definē kā $a_n = a^{2^n}_{n-1}$ visiem
$n = 1, 2, \ldots$. Pierādīt, ka $a_0$ var izvēlēties tā, ka
apakšvirkne $a_N, a_{N+1}, a_{N+2}, \ldots$
nav konstanta pēc moduļa $p$ nevienam naturālam $N$.
\end{problem}


\vspace{10pt}
%BW.TST.2015.15.SOL
\begin{problem}
Ar $\omega(n)$ apzīmēsim dažādo pirmskaitļu skaitu, ar ko dalās $n$. Pierādīt, ka ir bezgalīgi
daudz tādu naturālu skaitļu $n$, kuriem $\omega(n) < \omega(n+1) < \omega(n+2)$.
\end{problem}


\vspace{10pt}
%BW.2015.17.SOL
\begin{problem}
Pirmskaitlim $p$ un naturālam skaitlim $n$ apzīmējam 
ar $f(p, n)$ lielāko veselo skaitli $k$, kuram $p^k \mid n!$. 
Dots fiksēts pirmskaitlis $p$, bet $m$ un $c$ ir jebkādi naturāli skaitļi. 
Pierādīt, ka eksistē bezgalīgi daudzi tādi naturāli skaitļi $n$, kuriem 
$f(p, n) \equiv c \pmod m$.
\end{problem}


\end{document}