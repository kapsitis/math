\documentclass[a4paper,12pt]{article}

%\usepackage{amsmath,amssymb,multicol,tikz,enumitem}
\usepackage[margin=2cm]{geometry}
%\usetikzlibrary{calc}
\usepackage{amsmath}
\usepackage{amsthm}
\usepackage{thmtools}
\usepackage{hyperref}
\usepackage{enumerate}
\usepackage{xcolor}
\usepackage{fancyvrb}
\usepackage{MnSymbol}

\pagestyle{empty}

\newcommand\Q{\mathbf{Q}}
\newcommand\R{\mathbf{R}}
\newcommand\Z{\mathbf{Z}}

\usepackage{array}
\newcolumntype{P}[1]{>{\centering\arraybackslash}p{#1}}

\newcommand\indd{${}$\hspace{20pt}}

\declaretheoremstyle[headfont=\normalfont\bfseries,notefont=\mdseries\bfseries,bodyfont = \normalfont,headpunct={:}]{normalhead}
\declaretheorem[name={Uzdevums}, style=normalhead,numberwithin=section]{problem}

\setcounter{section}{101}
\setcounter{problem}{38}

\setlength\parindent{0pt}

\begin{document}

\clearpage
\begin{center}
\parbox{3.5cm}{\flushleft\bf Skaitļu teorija \newline ATV} \hfill {\bf\LARGE Uzdevumu lapa \#4} \hfill \parbox{3.5cm}{\flushright\bf 2021-01-27} %\\[2pt]
\end{center}

\hrule

%\vspace{10pt}
%{\em (Kļūdas labojums).} Iepriekšējā uzdevumu lapā nebija līdz galam ierakstīts, ka
%reizinājums $a_ia_{i+1}$ ir $0$ (t.i. blakusesošie koeficienti nevar abi vienlaikus būt 
%atšķirīgi no nulles):

%\vspace{10pt}
%{\bf Uzdevums 101.35:} Pierādiet, ka jebkuru veselu skaitli n var izteikt formā
%$$n = a_0 + a_1 2^1 + \cdots  + a_m2^m,$$
%kur $a_i \in \{ -1,0,1\}$, un $a_ia_{i+1} = 0$ visiem $i$, $0 \leq i < m$.
%Uzrakstiet šādā formā skaitli $1985$.


\vspace{20pt}
Turpmākajos uzdevumos, iespējams, nepieciešams 
lietot matemātisko indukciju ar induktīvās hipotēzes pastiprināšanu. 
Tas nozīmē, ka uzdevumā pierādāmo apgalvojumu tīšām pārraksta nedaudz ``stiprāku'', 
lai varētu normāli veikt induktīvo pāreju.\\
Sk. diskusiju \url{https://bit.ly/3o3OBOA}.


\vspace{10pt}
\begin{problem}
%9.39. 
Pierādīt, ka jebkuram naturālam skaitlim $n$ eksistē tāds naturāls skaitlis $m$, kuram
\[ \left( \sqrt{2} - 1 \right)^n = \sqrt{m} - \sqrt{m-1}. \]
\end{problem}


\vspace{10pt}
\begin{problem}
%9.40. 
Pierādiet, ka eksistē bezgalīgi daudz naturālu skaitļu $n$, kuriem skaitlis $2^n+2$  dalās ar $n$.
\end{problem}



\vspace{10pt}
\begin{problem}
%9.41. 
Pierādiet, ka eksistē bezgalīgi daudz naturālu skaitļu $n$, kuriem skaitlis $n!$ dalās ar $n^2 + 1$.
\end{problem}



\vspace{10pt}
\begin{problem}
%9.42. 
Dots nepāra pirmskaitlis $p$ un veseli skaitļi $a_1,a_2,\ldots,a_{p-1}$, kuri nedalās ar $p$. 
Pierādiet, ka, aizstājot dažus no šiem skaitļiem ar pretējiem, var iegūt $p-1$ skaitļus, kuru summa dalās ar $p$.
\end{problem}



\vspace{10pt}
\begin{problem}
%9.43. 
Doti tādi naturāli skaitļi $a_1,a_2,\ldots,a_n$, ka $a_k \leq k$, un visu šo $n$ skaitļu summa ir pāra skaitlis. 
Pierādiet, ka, aizvietojot dažus no tiem ar pretējiem, var iegūt $n$ skaitļus, kuru summa ir $0$.
\end{problem}


\vspace{10pt}
\begin{problem}
%9.44. 
Doti veseli skaitļi $a_1 = 1,a_2,\ldots,a_n$, kuriem $a_i \leq a_{i+1} \leq 2a_i$ visiem $i \in \{ 1,2,\ldots,n-1 \}$. 
Zināms, ka šo skaitļu summa ir pāra skaitlis. Pierādiet, ka šos skaitļus var sadalīt divās grupās tā, 
ka skaitļu summas abās grupās ir vienādas.
\end{problem}


\vspace{10pt}
\begin{problem}
%9.45. 
Pierādiet, ka jebkuram naturālam skaitlim $s$ vienādojumam
\[ \frac{1}{x_1} + \frac{1}{x_2} + \cdots + \frac{1}{x_s} = 1 \]
eksistē galīgs, lielāks par nulli, atrisinājumu skaits.
\end{problem}


\vspace{10pt}
\begin{problem}
%9.46. 
Kurš no skaitļiem 
\[ \underbrace{2^{3^{2^{3^{\udots}}}}}_{\mbox{\footnotesize $n$ simboli}} \;\;\text{un} \underbrace{3^{2^{3^{2^{\udots}}}}}_{\mbox{\footnotesize $n$ simboli}} \]
ir lielāks?
\end{problem}


\vspace{10pt}
\begin{problem}
%9.47. 
Pierādiet, ka jebkuram naturālam skaitlim $n$ eksistē naturāls skaitlis, 
kuru var uzrakstīt kā divu kvadrātu summu tieši n dažādos veidos. 
(Izteiksmes $a^2 + b^2$ un $b^2 + a^2$ uzskatīsim par vienādām).
\end{problem}


\vspace{10pt}
\begin{problem}
%9.48. 
Dota virkne $a_1 \in \mathbf{N}$, $a_{n+1} = a_n + 2^{a_n}$, ja $n \geq 1$. 
Pierādiet, ka šajā virknē ir bezgalīgi daudz locekļu, kuri
\begin{enumerate}[(a)]
\item nedalās ar $3$,
\item dalās ar $3$.
\end{enumerate}
\end{problem}


\vspace{10pt}
\begin{problem}
%9.49. 
Kādiem naturāliem skaitļiem $n$ visi skaitļi 
\[ C_n^2,C_n^2,\ldots,C_n^n \]	 
ir nepāra skaitļi?\\
{\em Piezīme.} Ar $C_n^k$ apzīmē kombināciju skaitu pa $k$ no $n$, t.i. $\displaystyle{ \frac{n!}{k!(n-k)!}}$.
\end{problem}


\vspace{10pt}
\begin{problem}
%9.50. 
Dota funkcija  $f(x,y)$, kura definēta visiem nenegatīviem pozitīviem skaitļiem. 
Dots, ka visiem nenegatīviem pozitīviem skaitļiem $x$ un $y$ izpildās vienādības
\begin{enumerate}[(a)]
\item $f(0,y) = y+1$,
\item $f(x+1,0) = f(x,1)$,
\item $f(x+1,y+1) = f(x,f(x+1,y))$.
\end{enumerate}
Aprēķināt vērtību $f(4,1980)$.
\end{problem}


\vspace{10pt}
\begin{problem}
%9.51. 
Funkcija  $f(x)$ definēta veselām pozitīvām $x$ vērtībām, un tās vērtības arī ir veseli pozitīvi skaitļi. 
Zināms, ka vienlaikus ir spēkā šādas trīs īpašības:
\begin{enumerate}[(1)]
\item $f(1) < f(2) < f(3) < \cdots < f(n) < f(n+1) < \cdots$ t.i., funkcija $f(x)$ ir stingri augoša;
\item $f(985) = 1985$;
\item ja veseliem pozitīviem skaitļiem $m$ un $k$ lielākais kopīgais dalītājs ir $1$, tad 
$f(m \cdot k) = f(m) \cdot f(k)$.
\end{enumerate}
Aprēķināt 
\begin{enumerate}[(a)]
\item $f(1000)$;
\item $f(3599)$;
\item $f(n)$ patvaļīgam pozitīvam $n$.
\end{enumerate}
\end{problem}



\vspace{10pt}
\begin{problem}
%9.52. 
Ar $a_n$ apzīmējam to dažādo veidu skaitu, kuros $n$ var izsacīt kā tādu saskaitāmo summu, 
kas nepieņem citas vērtības kā $1; 3; 4$. Pieļaujamas arī summas, kas sastāv no viena saskaitāmā. Veidus, kas atšķiras tikai ar saskaitāmo kārtību, uzskatām par dažādiem.\\
Piemēram, $a_1=1$; $a_2=1$; $a_3=2$; $a_4=4$.
Pierādīt: ja $n$ \textendash{}  pāra skaitlis, tad $a_n$ ir naturāla skaitļa kvadrāts.
\end{problem}






\end{document}