\documentclass[a4paper,12pt]{article}

%\usepackage{amsmath,amssymb,multicol,tikz,enumitem}
\usepackage[margin=2cm]{geometry}
%\usetikzlibrary{calc}
\usepackage{amsmath}
\usepackage{amsthm}
\usepackage{thmtools}
\usepackage{hyperref}
\usepackage{enumerate}
\usepackage{xcolor}
\usepackage{fancyvrb}

\pagestyle{empty}

\newcommand\Q{\mathbf{Q}}
\newcommand\R{\mathbf{R}}
\newcommand\Z{\mathbf{Z}}

\usepackage{array}
\newcolumntype{P}[1]{>{\centering\arraybackslash}p{#1}}

\newcommand\indd{${}$\hspace{20pt}}

\declaretheoremstyle[headfont=\normalfont\bfseries,notefont=\mdseries\bfseries,bodyfont = \normalfont,headpunct={:}]{normalhead}
\declaretheorem[name={Uzdevums}, style=normalhead,numberwithin=section]{problem}

\setcounter{section}{101}
\setcounter{problem}{22}

\setlength\parindent{0pt}

\begin{document}

\clearpage
\begin{center}
\parbox{3.5cm}{\flushleft\bf Skaitļu teorija \newline ATV} \hfill {\bf\LARGE Uzdevumu lapa \#3} \hfill \parbox{3.5cm}{\flushright\bf 2021-01-20} %\\[2pt]
\end{center}

\hrule



\vspace{10pt}
\begin{problem}
%9.23. 
Pierādiet, ka jebkuru naudas summu, kas lielāka par $7$ kapeikām var nomaksāt ar $3$ un $5$ kapeiku monētām.
\end{problem}



\vspace{10pt}
\begin{problem}
%9.24. 
Pierādiet, ka jebkuru naturālu skaitli, kurš lielāks par $11$, var izteikt kā divu saliktu skaitļu summu.
\end{problem}



\vspace{10pt}
\begin{problem}
%9.25. 
Pierādīt, ka jebkuru naturālu skaitli $k$ var bezgalīgi daudzos veidos izteikt formā
$$k = 1^2 \pm 2^2 \pm  \cdots \pm m^2 ,$$
kur $m$ ir naturāls skaitlis, un zīmes "$\pm$" izvēlamies patvaļīgi.
\end{problem}



\vspace{10pt}
\begin{problem}
%9.26.
Atrodiet visus tādus naturālus skaitļus $s$, kuriem vienādojumam
$$\frac{1}{x_1^2} + \frac{1}{x_2^2} + \cdots + \frac{1}{x_s^2} = 1$$	 
ir vismaz viens atrisinājums naturālos skaitļos  .
\end{problem}

\vspace{10pt}
\begin{problem}
%9.27. 
Pierādiet, ka katram naturālam skaitlim   vienādojumam
$$\frac{1}{x_1^3} + \frac{1}{x_2^3} + \cdots + \frac{1}{x_s^3} = \frac{1}{x_{s+1}^3}$$
ir bezgalīgi daudz atrisinājumu naturālos skaitļos $x_1,x_2,\ldots,x_{s+1}$.
\end{problem}

\vspace{10pt}
\begin{problem}
%9.28. 
Pierādiet, ka jebkuram naturālam skaitlim $m$, ja $s$ ir pietiekami liels naturāls skaitlis, vienādojumam
$$\frac{1}{x_1^m} + \frac{1}{x_2^m} + \cdots + \frac{1}{x_s^m} = 1$$	 
eksistē atrisinājums naturālos skaitļos $x_1,x_2,\ldots,x_{s}$.
\end{problem}

\vspace{10pt}
\begin{problem}
%9.29. 
Dotas $555$ lodes, kuru masas ir $1, 2, \ldots, 555$ grami. Sadaliet tās trijās pēc skaita un masas vienādās grupās.
\end{problem}

\vspace{10pt}
\begin{problem}
%9.30. 
Dotas $n$ lodes, kuru masas ir $1, 2, \ldots, n$ grami. Kādiem naturāliem skaitļiem tās var sadalīt trīs pēc masas vienādās grupās?
\end{problem}

\vspace{10pt}
\begin{problem}
%9.31. 
Dots septiņpadsmitciparu skaitlis $A$. Skaitlis $B$ ir iegūts no skaitļa $A$, uzrakstot tā ciparus pretējā secībā. 
Pierādiet, ka skaitlis $A+B$ satur vismaz vienu pāra ciparu.
\end{problem}

\vspace{10pt}
\begin{problem}
%9.32. 
Kādas vērtības var pieņemt ciparu summa naturālam skaitlim, kas dalās ar $7$?
\end{problem}

\vspace{10pt}
\begin{problem}
%9.33.
Kādas funkcijas $f(t)$ vienlaicīgi apmierina sekojošas īpašības:
\begin{enumerate}[(a)]
\item $f(t)$ definēta visiem veseliem skaitļiem, un tās vērtības ir veseli skaitļi,
\item $f(0)=1$,
\item katram veselam $n$ ir spēkā $f(f(n))=n$,
\item katram veselam $n$ ir spēkā $f(f(n+2)+2) = n$?
\end{enumerate}
\end{problem}


\vspace{10pt}
\begin{problem}
%9.34.
Dota Fibonači virkne:
$$a_1 = 1,\;a_2 = 2,\;a_{n+1}=a_n + a_{n-1},\;\mbox{ja $n \geq 2$}.$$
Pierādiet, ka jebkuru naturālu skaitli var uzrakstīt kā dažu (varbūt viena) atšķirīgu šīs virknes locekļu summu.
\end{problem}


\vspace{10pt}
\begin{problem}
%9.35.
Pierādiet, ka jebkuru veselu skaitli n var izteikt formā
$$n = a_0 + a_1 2^1 + \cdots  + a_m2^m,$$
kur $a_i \in \{ -1,0,1\}$, un $a_ia_{i+1} = 0$ visiem $i$, $0 \leq i < m$.
Uzrakstiet šādā formā skaitli $1985$.
\end{problem}


\vspace{10pt}
\begin{problem}
%9.36.
Pierādiet, ka patvaļīgu pozitīvu daļskaitli $\frac{m}{n}$ var izteikt kā dažādu naturālu skaitļu apgriezto lielumu summu.
\end{problem}


\vspace{10pt}
\begin{problem}
%9.37.
Pierādiet, ka patvaļīgu īstu daļskaitli $\frac{m}{n}$ var izteikt formā
$$\frac{m}{n} = \frac{1}{q_1} + \frac{1}{q_1q_2} + \cdots + \frac{1}{q_1q_2\cdots{}q_r}.$$
Šeit $q_1,q_2,\ldots,q_r$ ir naturāli skaitļi, turklāt $q_1 \leq q_2 \leq \cdots \leq q_r$.
\end{problem}


\vspace{10pt}
\begin{problem}
%9.38.
$A$ ir bezgalīga naturālu skaitļu kopa. Katrs $A$ elements ir ne vairāk kā $1990$ 
dažādu pirmskaitļu reizinājums. Pierādīt, ka eksistē tāds skaitlis $p$ un tāda kopas 
$A$ bezgalīga apakškopa $B$, ka katru divu dažādu $B$ elementu lielākais kopīgais dalītājs ir $p$.
\end{problem}



\end{document}