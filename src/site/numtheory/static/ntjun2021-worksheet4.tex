\documentclass[a4paper,12pt]{article}
\usepackage[table]{xcolor}
\usepackage{array,amsmath,amssymb,multicol,tikz}
\usepackage{hyperref,colonequals}
\usepackage{fancyvrb}
\usepackage[margin=2cm]{geometry}
\usetikzlibrary{calc,arrows.meta}

\usepackage{xcolor}

\pagestyle{empty}

\newcommand\N{\mathbf{N}}
\newcommand\Q{\mathbf{Q}}
\newcommand\R{\mathbf{R}}
\newcommand\Z{\mathbf{Z}}

\newcommand\rem{\textup{rem}}

\begin{document}
\renewcommand\arraystretch{1.5}

\begin{center}
\parbox{3cm}{\flushleft\bf Skaitļu Teorija}
\hfill
\parbox{7cm}{\centering {\bf\Huge Darba lapa \#4}}
\hfill
\parbox{3cm}{\flushright\bf 2020/2021 \linebreak 27.marts}
\end{center}

\hrule\vspace{2pt}\hrule

\hrule

\begin{enumerate}


\item \textbf{Iesildīšanās (Logaritmi):} 
\begin{enumerate}
\item No desmitciparu skaitļa desmit reizes vilka kvadrātsakni. Kuram veselam skaitlim vistuvākais rezultāts?
\item Kurā pakāpē jākāpina $2$, lai iegūtu vērtību $2\sqrt{2}$? $0.125$? $1/1024$?
\item Cik dažādām $n$ vērtībām var sakrist ciparu skaits skaitļu $2^n$ decimālpierakstos? (Atrast visas iespējas.)
\item Datora atmiņā dots skaitlis $n$. Kā noskaidrot, cik šī skaitļa decimālpierakstā ir ciparu?
\end{enumerate}


\item \textbf{Iesildīšanās (Saknes):}
\begin{enumerate}
\item Vai $\sqrt{2}$, $3 + \sqrt{2}$, $2\sqrt{2}$, $\sqrt{2} + \sqrt{3}$ ir racionāli vai iracionāli?
\item $ax^2 + bx + c = 0$ ir kvadrātvienādojums ar racionāliem koeficientiem. Viena no tā saknēm 
ir $x_1 = 2 + \sqrt{5}$. Vai otra sakne var būt $x_2 = 2 - \sqrt{5}$? Vai otra sakne var būt $x_2 = 3 - \sqrt{3}$? 
\end{enumerate}

\item Pierādīt vai apgāzt apgalvojumus:
\begin{enumerate}
\item Ja $p + q$ ir racionāls, tad vai nu abi $p,q$ ir racionāli vai arī abi ir iracionāli.
\item Ja $pq$ ir racionāls, tad vai nu abi $p,q$ ir racionāli vai arī abi ir iracionāli.
\item Ja $p^2$ un $q^2$ ir abi racionāli, tad arī reizinājums  $(p+q)(p-q)$ ir racionāls.
\item Ja $p^3$ un $p^5$ ir racionāli, tad arī $p$ ir racionāls.
\item Ja $p^6$ un $p^{15}$ ir racionāli, tad arī $p$ ir racionāls.
\item Ja $pq$ un $p+q$ abi ir racionāli, tad $p$ un $q$ ir racionāli.
\end{enumerate}



\item 
Pierādīt vai apgāzt apgalvojumu: Ja $x^2$ ir iracionāls, tad $x^3$ ir iracionāls.

\item 
Ar $a_n$ apzīmējam $n$-to locekli virknē $1, 2,2, 3,3,3, 4,4,4,4, 5,5,5,5,5, 6,6,6,6,6,6,\ldots$, 
ko veido, atkārtojot katru naturālu skaitli $k$ tieši $k$ reizes. Pierādīt, ka
\[ a_n = \left\lfloor \sqrt{2n} + \frac{1}{2} \right\rfloor. \]


\item 
Pierādīt, ka $\sqrt[3]{2}$ nevar izteikt formā $a + b\sqrt{r}$, kur $a,b,r$ ir racionāli skaitļi. 

\item 
Pierādīt, ka $(\sqrt{2} - 1)^n$, $n \in \N$ ir izsakāms formā $\sqrt{m} - \sqrt{m-1}$, kur $m \in \N$.


\item {\bf Veselā daļa.} 
Ar $x$ apzīmēts jebkurš reāls skaitlis. Pierādīt, ka 
$\lfloor 3x \rfloor = \lfloor x \rfloor + \lfloor x + \frac{1}{3} \rfloor +
\lfloor x + \frac{2}{3} \rfloor$.





\end{enumerate}




\end{document}

