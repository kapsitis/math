\documentclass[11pt]{article}
\usepackage{ucs}
\usepackage[utf8x]{inputenc}
\usepackage{changepage}
\usepackage{graphicx}
\usepackage{amsmath}
\usepackage{gensymb}
\usepackage{amssymb}
\usepackage{enumerate}
\usepackage{tabularx}
\usepackage{lipsum}
\usepackage{amsthm}
\usepackage{thmtools}


\usepackage{fontspec} % loaded by polyglossia, but included here for transparency 
\usepackage{polyglossia}

\usepackage{xeCJK}
\setCJKmainfont{SimSun}
\setmainlanguage{russian} 
\setotherlanguage{english}

\newfontfamily\cyrillicfont[Script=Cyrillic]{Times New Roman}
\newfontfamily\cyrillicfontsf[Script=Cyrillic]{Arial}
\newfontfamily\cyrillicfonttt[Script=Cyrillic]{Courier New}

\oddsidemargin 0.0in
\evensidemargin 0.0in
\textwidth 6.27in
\headheight 1.0in
\topmargin 0.0in
\headheight 0.0in
\headsep 0.0in
%\textheight 9.69in
\textheight 9.00in
 
\setlength\parindent{0pt}

\newenvironment{myenv}{\begin{adjustwidth}{0.4in}{0.4in}}{\end{adjustwidth}}
\renewcommand{\abstractname}{Anotācija}
\renewcommand\refname{Atsauces}

%\newenvironment{uzdevums}[1][\unskip]{%
%\vspace{3mm}
%\noindent
%\textbf{#1:}
%\noindent}
%{}

% (4;10;12;17)
% (p1.19;5;15;20)

% http://tex.stackexchange.com/questions/196961/thmtools-declaration-for-theorem-and-proof
\declaretheoremstyle[headfont=\normalfont\bfseries,notefont=\mdseries\bfseries,bodyfont = \normalfont,headpunct={:}]{normalhead}
\declaretheorem[name={Uzdevums}, style=normalhead,numberwithin=section]{problem}

%\def\changemargin#1#2{\list{}{\rightmargin#2\leftmargin#1}\item[]}
\def\changemargin#1#2{\list{}\item[]}
\let\endchangemargin=\endlist 


\newcommand{\subf}[2]{%
  {\small\begin{tabular}[t]{@{}c@{}}
  #1\\#2
  \end{tabular}}%
}



\newcounter{alphnum}
\newenvironment{alphlist}{\begin{list}{(\Alph{alphnum})}{\usecounter{alphnum}\setlength{\leftmargin}{2.5em}} \rm}{\end{list}}

\newenvironment{zhtext}{\fontfamily{MS PGothic}\selectfont}{\par}


\makeatletter
\let\saved@bibitem\@bibitem
\makeatother

\usepackage{bibentry}
%\usepackage{hyperref}

\newenvironment{tulkojums}[1][\unskip]{%
\begin{changemargin}{8mm}{8mm}
\fontsize{9}{11}
\selectfont
\textbf{#1:}
}
{ 
\fontsize{12}{14}
\selectfont
\end{changemargin}
}

\setcounter{section}{1}


\begin{document}

\begin{center}
{\Large \bf Rudens brīvdienu uzdevumi Martai}
\end{center}

\vspace{10pt}
{\bf \large 11.klase}

\begin{problem}
Atrast vienādojuma
$$|x - 2| + |x - 3| = 5$$
visus atrisinājumus reālos skaitļos.
\end{problem}

\begin{problem}
Kādā secībā skaitļi 
${\displaystyle 2^{3^4}}$, ${\displaystyle 2^{4^3}}$, 
${\displaystyle 3^{2^4}}$, ${\displaystyle 3^{4^2}}$, 
${\displaystyle 4^{2^3}}$, ${\displaystyle 4^{3^2}}$ 
izvietoti uz skaitļu ass?\\
{\em Piezīme:} Pieraksts ${\displaystyle a^{b^c}}$
apzīmē ${\displaystyle a^{\left( b^c \right)}}$, nevis 
${\displaystyle \left( a^b \right)^c}$.
\end{problem}

\begin{problem}
Ar skaitļiem no $1$ līdz $2n$, izmantojot tos katru 
tieši vienu reizi bez atkārtojumiem, 
pierakstīti $n$ parastu daļskaitļu skaitītāji un saucēji. 
Vai iespējams, ka iegūto daļu summa ir
vesels skaitlis, ja
\begin{enumerate}[(a)]
\item $n = 4$;
\item $n = 6$?
\end{enumerate}
\end{problem}

\begin{problem}
Dots leņkis $\alpha$, kam izpildās $0^{\circ} < \alpha < 90^{\circ}$. 
Pierādiet nevienādību
$$\left( \frac{1}{2\cos \frac{\alpha}{2}} \right)^2 + 
\left( \frac{1}{4\cos \frac{\alpha}{4}} \right)^2 + 
\left( \frac{1}{8\cos \frac{\alpha}{8}} \right)^2 <
\left( \frac{1}{\sin \alpha} \right)^2.$$
\end{problem}

\begin{problem}
Mediāna, kas vilkta no kādas vienādsānu trīsstūra pamata virsotnes,
ir tikpat gara cik pati pamata mala. 
Atrodiet šī trijstūra laukumu, 
ja tā pamata malas garums ir 1 metrs.
\end{problem}



\begin{problem}
Atrodiet visus naturālos $n$, kuriem izteiksmē
$$0 + 1 + 2 + 3 + \ldots + n$$
var nomainiet daļu plus zīmju ar mīnus zīmēm tā, 
lai iegūtās izteiksmes vērtība būtu $0$.
\end{problem}




\end{document}


