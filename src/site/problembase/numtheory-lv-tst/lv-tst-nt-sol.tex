\documentclass[11pt]{article}

\usepackage{standalone}
\usepackage{hyperref}
\usepackage{longtable}

\usepackage{ucs}
\usepackage[utf8x]{inputenc}
\usepackage{changepage}
\usepackage{graphicx}
\usepackage{amsmath}
\usepackage{gensymb}
\usepackage{enumerate}
\usepackage{tabularx}
\usepackage{lipsum}
\usepackage{thmtools}
\usepackage{color}

\usepackage{fontspec} % loaded by polyglossia, but included here for transparency
\usepackage{polyglossia}


\usepackage{amsmath,amssymb}  
\usepackage{etoolbox}  
\usepackage[amsmath,framed,thmmarks]{ntheorem}  
\usepackage{thmtools}

\usepackage{xeCJK}
\setCJKmainfont{SimSun}

\newfontfamily\cyrillicfont[Script=Cyrillic]{Times New Roman}
\newfontfamily\cyrillicfontsf[Script=Cyrillic]{Arial}
\newfontfamily\cyrillicfonttt[Script=Cyrillic]{Courier New}

\oddsidemargin 0.0in
\evensidemargin 0.0in
\textwidth 6.27in
\headheight 1.0in
\topmargin 0.0in
\headheight 0.0in
\headsep 0.0in
%\textheight 9.69in
\textheight 9.00in

\setlength\parindent{0pt}

\newenvironment{myenv}{\begin{adjustwidth}{0.4in}{0.4in}}{\end{adjustwidth}}
\renewcommand{\abstractname}{Anotācija}
\renewcommand\refname{Atsauces}

%% KAP
\declaretheoremstyle[headfont=\normalfont\bfseries,notefont=\mdseries\bfseries,bodyfont = \normalfont,headpunct={:}]{normalhead}

\theorembodyfont{\normalfont}
\theoremheaderfont{\normalfont\bfseries}
\theoremseparator{.}
\theoremprework{\setlength
\theorempreskipamount{0 pt}\setlength\theorempostskipamount{0 pt}}


\newtheorem{innercustomthm}{Uzd.}
\newenvironment{problem}[1]
  {\renewcommand\theinnercustomthm{#1}\innercustomthm}
  {\endinnercustomthm}
  

\def\changemargin#1#2{\list{}{\rightmargin#2\leftmargin#1}\item[]}
\let\endchangemargin=\endlist


\newcommand{\subf}[2]{%
  {\small\begin{tabular}[t]{@{}c@{}}
  #1\\#2
  \end{tabular}}%
}

\newcounter{alphnum}
\newenvironment{alphlist}{\begin{list}{(\Alph{alphnum})}{\usecounter{alphnum}\setlength{\leftmargin}{2.5em}} \rm}{\end{list}}

\newenvironment{zhtext}{\fontfamily{MS PGothic}\selectfont}{\par}


\makeatletter
\let\saved@bibitem\@bibitem
\makeatother

\usepackage{bibentry}
%\usepackage{hyperref}

\newenvironment{tulkojums}[1][\unskip]{%
\begin{changemargin}{8mm}{8mm}
\fontsize{9}{11}
\selectfont
\textbf{#1:}
}
{
\fontsize{12}{14}
\selectfont
\end{changemargin}
}

\renewcommand\refname{References}

\setcounter{section}{0}

\begin{document}

\begin{center}
{\LARGE Latvijas papildsacensības, Skaitļu teorija: Atrisinājumi}
\end{center}



\begin{problem}{LV.TST.2012.9-12.1.SOL} \label{LV.TST.2012.9-12.1.SOL}
Ar $S(x)$ apzīmēsim skaitļa $x$ ciparu summu.
Aprēķināt\\ $S(S(S(2012^{2012})))$.
\end{problem}

{\small
{\bf Atrisinājums.} 
Kā zināms no dalāmības pazīmes ar $9$, ir spēkā
sakarība
\[ n \equiv S(n)\;(\operatorname{mod} 9)\]
t.i.\ jebkurš naturāls skaitlis $n$
un tā ciparu summa pieder tai pašai kongruences klasei $(\operatorname{mod} 9)$.
No\-skaid\-ro\-sim, kādai kongruences klasei pieder $2012^{2012}$, jeb
kāds ir šī skaitļa atlikums, to dalot ar $9$.

\[ 2012^{2012} \equiv 5^{2012} \equiv 5^2 5^{2010} \equiv 25 \cdot (5^6)^{335} \equiv 25 \equiv 7\;(\operatorname{mod} 9) \]

Mēs izmantojām Eilera teorēmu: ja $\operatorname{gcd}(a,9) = 1$, tad $a^{\varphi(9)} \equiv 1\;(\operatorname{mod} 9)$,
kur $\varphi(9)$ ir Eilera funkcija (cik daudzi no skaitļiem $\{ 1,\ldots,9 \}$ ir savstarpēji
pirmskaitļi ar $9$). Viegli redzēt, ka $\varphi(9) = 6$, tādēļ $5^6 \equiv 1\;(\operatorname{mod} 9)$.

Kad esam uzzinājuši, ka $2012^{2012}$ ir kongruents ar $7$ pēc $9$ moduļa,
dalāmības pazīme skaitlim $9$ ļauj secināt, ka arī $s_1 = S(2012^{2012})$, $s_2 = S(S(2012^{2012}))$
un $s_3 = S(S(S(2012^{2012})))$ pieder šai pašai kongruences klasei. Citiem vārdiem,
skaitlis $s_3 = S(S(S(2012^{2012})))$ dod atlikumu $7$, dalot ar $9$ jeb
$s_3$ ciparu summa dod atlikumu $7$, dalot ar $9$.
Pamatosim, ka $s_3 = S(S(S(2012^{2012})))$ vienāds ar $7$.

{\bf Apgalvojums A:}
Mazākais naturālais skaitlis $n \equiv 7\;(\operatorname{mod} 9)$, kam
$S(n) > 7$, ir 79.\\
{\bf Pamatojums:} Izrakstām dažus mazākos naturālos skaitļus, kuri dod atlikumu $7$,
dalot ar $9$:
\[ 7, 16, 25, 34, 43, 52, 61, 70, 79, \ldots \]
Viegli redzēt, ka $79$ ir pirmais skaitlis šajā virknē, kuram ciparu summa ir nevis $7$,
bet $16$. $\square$

{\bf Apgalvojums B:}
Mazākais naturālais skaitlis $n \equiv 7\;(\operatorname{mod} 9)$, kam
$S(S(n)) > 7$ ir $799999999$\\
{\bf Pamatojums:} Mums jāatrod mazākais $n$, kuram $S(n) \equiv 7\;(\operatorname{mod} 9)$,
bet $S(n) > 7$. Saskaņā ar iepriekšējo apgalvojumu, mazākā ciparu summa $S(n)$ ar šādu
īpašību ir $79$. Mazākais ciparu skaits skaitlī, kura ciparu summa ir $79$ ir deviņi cipari.
Summu $79$ var iegūt divos veidos:
\begin{align}
79 &= 9+9+9+9+9+9+9+9+7 \nonumber \\
79 &= 9+9+9+9+9+9+9+8+8 \nonumber
\end{align}
Vismazākais deviņciparu skaitlis sanāks tad, ja tā pirmais cipars būs mazākais iespējamais,
t.i. "7".

Tā kā, skaitlim pieaugot par $1$, tā ciparu summa var palielināties ne vairāk kā par $1$,
nevarēs gadīties tā, ka mazākais naturālais skaitlis $n$
(kam $n \equiv 7\;(\operatorname{mod} 9)$ un $S(S(n)) > 7$)
būs ar ciparu summu, kas lielāka par $79$, jo ciparu summa $79$ tiks sasniegta pirms
jebkuras lielākas ciparu summas. $\square$

{\bf Apgalvojums C:}
Ir spēkā nevienādība $S(2012^{2012}) < 8 \cdot 10^8 - 1$. \\
{\bf Pamatojums:}
Novērtēsim skaitli $2012^{2012}$ no augšas:
\begin{align}
2012^{2012} &< 2100^{2012} = 2100^2 2100^{2010} = 2100^2 (2100^3)^{670} = \nonumber \\
            &= 441 \cdot 100^2 \cdot (9261 \cdot 10^6)^{670} < 441 \cdot 10^4 \cdot (10^{10})^{670} = \nonumber \\
            &= 441 \cdot 10^4 \cdot 10^{6700} < 1000 \cdot 10^{6704} = 10^{6707} \nonumber
\end{align}
Esam ieguvuši, ka skaitļa $2012^{2012}$ decimālpierakstā ir ne vairāk kā $6707$ cipari,
jo šis skaitlis ir mazāks par $10^{6707}$. Tādēļ šī skaitļa ciparu summa
nevar pārsniegt $9 	\cdot 6707 = 60363$, jo neviens cipars nav lielāks par $9$.
Šis ciparu summas novērtējums ir daudz mazāks nekā $8 \cdot 10^8 - 1$. $\square$

Saskaņā ar Apgalvojumu C,
$S(2012^{2012}) < 799999999$.\\
Saskaņā ar Apgalvojumu B, $S(S(2012^{2012})) < 79$.\\
Saskaņā ar Apgalvojumu A, $S(S(S(2012^{2012}))) = 7$.\\
{\bf Atbilde.} $S(S(S(2012^{2012}))) = 7$.

{\bf Piezīme.} Izmantojot valodu R un \texttt{gmp} (Gnu multiple precision) bibliotēku,
atbildi $7$ var pārbaudīt ar tiešu skaitļošanu:

\fbox{\includegraphics[width=2.77in]{images/LV.TST2012.9-12.1.SOL.png}}
}

\end{document}

