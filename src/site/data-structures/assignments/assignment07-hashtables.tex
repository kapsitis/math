\documentclass[a4paper,12pt]{article}

\usepackage{amsmath,amssymb,amsthm,tikz}
\usetikzlibrary{calc,arrows.meta}
\usepackage[margin=20mm]{geometry}
\usepackage{hyperref}

\setlength{\parindent}{0pt}
\setlength{\columnsep}{1cm}

\begin{document}

%\twocolumn

\thispagestyle{empty}

\begin{center}
{\Large Assignment 7, 2020-11-05,}\\
{\em 12 minutes} 
\end{center}

\noindent


\vspace{10pt}
{\bf Question 1 (``Folding HashCodes'' for Strings).}\\
You have $3$ different string keys (representing for example car registration numbers): 
$\textcolor{blue}{\mathtt{K}abc}$, $\textcolor{blue}{\mathtt{K}bca}$, 
$\textcolor{blue}{\mathtt{K}bc}$, and you want to compute 
a hash function using the ``mod two folding'' (also known as XOR), and 
will insert your results in a hash table $H$ with $7$ slots: $H[0],\ldots,H[6]$.\\
(Here $a,b,c$ denote the last $3$ digits from your Student ID. 
For example, if $abc = 789$, then the strings to encode are 
$\textcolor{blue}{\mathtt{K789}}$, 
$\textcolor{blue}{\mathtt{K897}}$, 
$\textcolor{blue}{\mathtt{K89}}$. They all start with the same ASCII letter {\tt K}. 

{\bf (A)} Compute the (uncompressed) hashcode values for all the $3$ string values.  
Your hashcode function $h_1(s)$ is defined as follows:
$$h_1(s) = \bigoplus\limits_{i=0}^{L-1} \operatorname{ord}(c[i]) = c[0] \oplus \ldots \oplus c[L-1].$$
Here $L$ is the length of the string $s$.
By $c[i]$ we denote the $i$th character of the input string $s$ ($i = 0,1,\ldots,L-1$).
All the hash values computed by this function are 1 byte long, they are 
integers in $[0;255]$.  

{\em Note.} By $\operatorname{ord}(c)$ we denote
the ASCII code of some character {\tt c}; these ASCII codes are also integers in $[0;255]$. 
(See the ASCII codes: \url{http://www.asciitable.com/}.) 

\vspace{10pt}
{\bf (B)} Represent the bits of all four characters in 
$\textcolor{blue}{\mathtt{K}abc}$ (one capital letter "K" and the three digits from your ID). 
Check the computation of $h_1(\textcolor{blue}{\mathtt{K}abc})$ by writing these bits 
aligned in columns and add them up modulo $2$. 
(To see how a character is represented as a sequence of bits, use ASCII code in hexadecimal. 
For example, character "A" has hex code {\tt 0x41}, i.e. it is represented by 
these eight bits: {\tt 01000001} (since hex "4" converts into {\tt 0100}, but "1" converts into 
{\tt 0001}). 


\vspace{10pt}
{\bf (C)} Compute the compressed hash values for the same 3 strings
modulo $7$. Namely, 
the compressed hash value is 
$$h_2(h_1(s)) = h_1(s)\;\text{mod}\;7.$$

\vspace{10pt}
{\bf (D)} Draw the four string objects in a hashtable $H$ with $7$ cells
($H[0],\ldots,H[6]$). Are there any collisions?

\vspace{30pt}
{\em Note.} 
Here is the pseudocode of the abovementioned string hashing function $h_1(s)$ (in Python): 
\begin{verbatim}
def h1(s):
    h = 0
    for c in s:
        h = h ^ ord(c)
    return h
\end{verbatim}



% https://stackoverflow.com/questions/17016175/c-unordered-map-using-a-custom-class-type-as-the-key




\end{document}



