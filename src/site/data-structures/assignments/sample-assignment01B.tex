%\documentclass[jou]{apa6}
\documentclass[11pt]{article}
\usepackage{ucs}
\usepackage[utf8x]{inputenc}
\usepackage{changepage}
\usepackage{graphicx}
\usepackage{amsmath}
\usepackage{gensymb}
\usepackage{amssymb}
\usepackage{enumerate}
\usepackage{tabularx}
\usepackage{lipsum}
\usepackage{hyperref}
\usepackage{fancyvrb}

\oddsidemargin 0.0in
\evensidemargin 0.0in
\textwidth 6.27in
\headheight 1.0in
\topmargin -0.1in
\headheight 0.0in
\headsep 0.0in
\textheight 9.0in

\usepackage{xcolor}

\setlength\parindent{0pt}

\newenvironment{myenv}{\begin{adjustwidth}{0.4in}{0.4in}}{\end{adjustwidth}}
\renewcommand{\abstractname}{Anotācija}
\renewcommand\refname{Atsauces}



\newcounter{alphnum}
\newenvironment{alphlist}{\begin{list}{(\Alph{alphnum})}{\usecounter{alphnum}\setlength{\leftmargin}{2.5em}} \rm}{\end{list}}


%16.3-6

\makeatletter
\let\saved@bibitem\@bibitem
\makeatother

\usepackage{bibentry}
%\usepackage{hyperref}


%\title{Homework 1: Grading Criteria}
%\author{Kalvis}
%\affiliation{RBS}



\begin{document}
\thispagestyle{empty}

%\twocolumn


\begin{center}
{\Large Sample Assignment 1B, 2020-09-10}
\end{center}

{\bf Total time:} $12$ minutes.
Write your solutions on the handout in free spaces.

\vspace{10pt}
{\bf Question 1 (Bitwise Operations).} Consider the following variable assignments:
\begin{Verbatim}[frame=single]
long long int a = -6;  
long b = 0347;  // Reminder: Octal notation starts with "0"
char c = '\t';  // TAB character
\end{Verbatim}

Draw the memory content of these variables in {\em hexadecimal notation}. Ensure that the hex notation
length equals their actual length. You may need to add leading zeroes.

\vspace{10pt}
Memory content of {\tt a}: \rule{4cm}{0.4pt} 

\vspace{10pt}
Memory content of {\tt b}: \rule{4cm}{0.4pt} 

\vspace{10pt}
Memory content of {\tt c}: \rule{4cm}{0.4pt} 




\vspace{20pt}
{\bf Question 2 (Flowchart).} 
Draw a flowchart that shows how the following {\tt switch-case} statement is evaluated. 
Please draw the flowchart nodes accurately: Use only $5$ kinds of nodes:\\
{\bf (1)} Start node (oval: one outgoing arrow).\\
{\bf (2)} Stop node (oval: one incoming arrow).\\
{\bf (3)} Conditional statement (diamond: one incoming and two outgoing arrows). Also mark the branch taken on {\tt true}. \\
{\bf (4)} Regular statement (rectangle: one incoming and one outgoing arrow).\\
{\bf (5)} Merging two branches (black dot: two incoming arrows, one outgoing arrow).


\vspace{5mm}
\begin{tabular}{@{}ll@{}} 
\begin{minipage}{0.48\columnwidth}
\begin{Verbatim}[frame=single]
int a = 0;
char c; 
cin >> c; 
switch( c ) {
    case 'A':
        a += 1;
    case 'B':
        a += 2;
		break;
    default :
        a += 4;
}
\end{Verbatim}
\end{minipage} &
\begin{minipage}{0.5\columnwidth}
\mbox{}
\end{minipage}
\end{tabular}










\end{document}



