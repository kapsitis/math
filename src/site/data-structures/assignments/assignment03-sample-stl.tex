%\documentclass[jou]{apa6}
\documentclass[11pt]{article}
\usepackage{ucs}
\usepackage[utf8x]{inputenc}
\usepackage{changepage}
\usepackage{graphicx}
\usepackage{amsmath}
\usepackage{gensymb}
\usepackage{amssymb}
\usepackage{enumerate}
\usepackage{tabularx}
\usepackage{lipsum}
\usepackage{hyperref}
\usepackage{fancyvrb}
\usepackage{mathtools}
\usepackage{color}

\oddsidemargin 0.0in
\evensidemargin 0.0in
\textwidth 6.27in
\headheight 1.0in
\topmargin -0.1in
\headheight 0.0in
\headsep 0.0in
\textheight 9.0in

\usepackage{xcolor}


\setlength{\parindent}{0pt}
\setlength{\columnsep}{1cm}

\newenvironment{myenv}{\begin{adjustwidth}{0.4in}{0.4in}}{\end{adjustwidth}}
\renewcommand{\abstractname}{Anotācija}
\renewcommand\refname{Atsauces}



\newcounter{alphnum}
\newenvironment{alphlist}{\begin{list}{(\Alph{alphnum})}{\usecounter{alphnum}\setlength{\leftmargin}{2.5em}} \rm}{\end{list}}


%16.3-6

\makeatletter
\let\saved@bibitem\@bibitem
\makeatother

\usepackage{bibentry}
%\usepackage{hyperref}


%\title{Homework 1: Grading Criteria}
%\author{Kalvis}
%\affiliation{RBS}



\begin{document}
\thispagestyle{empty}

\twocolumn


\begin{center}
{\Large Sample Assignment 3}\\
{\em Discussed on 2020-09-24}\\
{\em Not graded}
\end{center}

\vspace{10pt}
{\bf Question 1 (Storing numbers in a queue)} Data structure {\tt std::vector} 
is very popular; it allows {\tt push\textunderscore{}back(...)} method:
you can add new elements at the end, but not at the beginning. 

Double ended queue ({\em deque}) is another data structure (as fast and efficient as vector); 
it allows pushing elements at either end:\\
{\tt push\textunderscore{}front(...)} and {\tt push\textunderscore{}back(...)}.

\vspace{10pt}
Source file {\bf File1.cpp}:
{\footnotesize
\begin{center}
\begin{minipage}{.85\columnwidth}
\begin{Verbatim}[frame=single,numbers=left]
#include <iostream>
#include <string>
#include <deque>

using namespace std;
int main() {
  deque<int> queue1;
  int n; 

  while (cin >> n) { 
    queue1.push_front(n); 
  }
  
  string sep; // initialized as empty
  for (int m : queue1) { 
      cout << sep << m; sep = ","; 
  } 
  cout << endl;

  int head = queue1.at(0);
  queue1.pop_front();
  cout << head << " -> ";
  for (int m : queue1) { 
      cout << m << ",";
  }
}  
\end{Verbatim}
\end{minipage}
\end{center}
}

\vspace{10pt}
Input file {\bf input.txt}:
{\footnotesize
\begin{center}
\begin{minipage}{.85\columnwidth}
\begin{Verbatim}[frame=single,numbers=left]
1 3 5
2 4 6
\end{Verbatim}
\end{minipage}
\end{center}
}

What is the output from {\bf File1.cpp} on this input?

\vspace{10pt}
{\bf (A)}
{\footnotesize
\begin{Verbatim}[frame=single]
1,3,5,2,4,6
6 -> 4,2,5,3,1,
\end{Verbatim}
}

\vspace{10pt}
{\bf (B)}
{\footnotesize
\begin{Verbatim}[frame=single]
1,3,5,2,4,6
1 -> 3,5,2,4,6,
\end{Verbatim}
}

\vspace{10pt}
{\bf (C)}
{\footnotesize
\begin{Verbatim}[frame=single]
6,4,2,5,3,1
1 -> 3,5,2,4,6,
\end{Verbatim}
}

\vspace{10pt}
{\bf (D)}
{\footnotesize
\begin{Verbatim}[frame=single]
6,4,2,5,3,1
6 -> 4,2,5,3,1,
\end{Verbatim}
}




\newpage
\vspace{20pt}
{\bf Question 2 (Reading Line by Line)}

\vspace{10pt}
Source file {\bf File2.cpp}:
{\footnotesize
\begin{center}
\begin{minipage}{.85\columnwidth}
\begin{Verbatim}[frame=single,numbers=left]
#include <iostream>
#include <sstream>
#include <map>
#include <string>
#include <vector>

using namespace std;
int main() {
  map<int, vector<int>> mymap;

  vector<int> vect;
  string line;
  while (getline(cin, line)) {
    istringstream sstr(line);
    int n;        
    while (sstr >> n) {
      vect.push_back(n);
    }
	
    mymap.insert(
      make_pair(vect.at(0), vect));
    vect.clear();
  }

  map<int,vector<int>>::iterator it= 
      mymap.begin();
  while (it != mymap.end()) {
    cout << it ->first << ": ";
    for (int m : it -> second) { 
      cout << m << ",";
    }
    it++; cout << endl;
  }
} 
\end{Verbatim}
\end{minipage}
\end{center}
}

\vspace{10pt}
Input file {\bf input.txt}:
{\footnotesize
\begin{center}
\begin{minipage}{.85\columnwidth}
\begin{Verbatim}[frame=single,numbers=left]
1 3 5
7 8 9
2 4 6
\end{Verbatim}
\end{minipage}
\end{center}
}

Mark true/false statements about this code:

\begin{enumerate}[(A)]
\item The output is $3$ lines
\item The output is $1$ line
\item Iterator {\tt it} visits pairs from {\tt mymap} in the same order they were inserted.
\item Iterator {\tt it} visits pairs from {\tt mymap} in a random order. 
\item Iterator {\tt it} visits pairs from {\tt mymap} in increasing order 
of the key ({\tt it -> first}). 
\end{enumerate}



%%%%%%%%%%%%%%%%%%%%%%%%%%%%%%%%%%%%%%%%
%% Stuff where tree building does not happen
%%%%%%%%%%%%%%%%%%%%%%%%%%%%%%%%%%%%%%%%

%% SET data struct.


\newpage
\vspace{20pt}
{\bf Question 3:} This code inserts some elements in 
a {\tt std::set}, then tries to find 
(element {\tt 5} is there, but {\tt 4} is not). 
Finally, we iterate over the set in two different ways: 
as in C++11 (for-loop syntax for an iterator) or 
an older construct with an explicit iterator.


\vspace{10pt}
Source file {\bf File3.cpp}:
{\footnotesize
\begin{center}
\begin{minipage}{.85\columnwidth}
\begin{Verbatim}[frame=single,numbers=left]
#include <iostream>
#include <set>

using namespace std;
int main() {
  set<int> myset;
  myset.insert(11);
  myset.insert(13);
  myset.insert(5);
  myset.insert(7);
  myset.insert(5);
  bool b4,b5;
  b4 = myset.find(4) == myset.end();
  b5 = myset.find(5) == myset.end();
  cout << "(b4,b5)=(" << b4 << 
    "," << b5 << ")" << endl;
  for (int u: myset) {
      cout << u << "; ";
  } 
  cout << endl;
  set<int>::iterator it;
  it = myset.begin();
  while (it != myset.end()) {
    cout << (*it) << "; ";
    it++;
  }
}
\end{Verbatim}
\end{minipage}
\end{center}
}


Mark which statements about this code are true/false.

\begin{enumerate}[(A)]
\item 1st line in output is {\tt (b3,b4) = (0,1)}.
\item Lines 17--19 and 23--26 iterate over {\tt myset} in the same way. 
\item Iterator on Lines 17--19 visits elements in increasing order.
\item Iterator on Lines 17--19 visits elements in random order.
\end{enumerate}


\newpage

\mbox{}
\newpage
\vspace{20pt}
{\bf Question 4 (Sets/Vectors with Custom Classes)}\\
The code below does the task of EX02: it reads student data from STDIN;
outputs the smallest and the largest student compared by age; 
or by height (if ages are equal). 


\vspace{10pt}
Source file {\bf File4.cpp}:
{\footnotesize
\begin{center}
\begin{minipage}{.85\columnwidth}
\begin{Verbatim}[frame=single,numbers=left]
#include <iostream>
#include <iomanip> 
#include <set>

using namespace std;
struct Student {
  int age;
  double height;
  Student(int aa = 1, double hh = 1): 
          age(aa), height(hh) {}

  friend istream &operator>>(
    istream  &input, Student &S ) { 
      input >> S.age >> S.height;
      return input;            
  }

  friend ostream &operator<<( 
    ostream &output, 
	const Student &S ) { 
      output << "Student(" << 
        S.age << "," << std::fixed << 
        std::setprecision(5) << 
		S.height << ")";
      return output;            
  }

  friend bool operator<(
    const Student &left, 
    const Student &right) {
      return (left.age<right.age) || 
      (left.age == right.height && 
        left.height<right.height);
    }
};

int main() {
    int n; cin >> n;
    set<Student> myset;
    for (int i = 0; i < n; i++) {
        Student student; 
        cin >> student;
        myset.insert(student);
    }
    cout << *(myset.begin()) << endl;
    cout << *(--myset.end());
    return 0;
}
\end{Verbatim}
\end{minipage}
\end{center}
}

\newpage
Write short answers to these questions:

\begin{enumerate}[(A)]
\item What happens, if Line 43 
is rewritten as follows:
{\tt cout << *(myset.end());}
\item How would you overload the comparison operation, 
if you only look at {\tt age}, but order by age alphabetically (as in a dictionary). 
Namely, age "17" comes before age "2" (since digit "1" alphabetically 
precedes digit "2"). 
{\em (Just show how the Lines 31--33 would look, if you order alphabetically.)}
\item Assume that the class {\tt Student} had a destructor.
On which line of your code (if at all) is it called, when you read the input containing data for a few students and insert them all in a set.
\item Assume that you want to use {\tt std::vector} instead of 
a set; and output the first and the last element you inserted into a vector. 
What would you write on Line 43, Line 45 and 46? (How to add something to an end of a vector? 
How to get the first element? The last element?)
\end{enumerate}




\newpage
{\Large \bf Solutions}

\vspace{10pt}
{\bf Question 1.} Answer: D

\vspace{10pt}
{\bf Question 2.} Answer: {\tt true}, {\tt false}, {\tt false}, {\tt false}, {\tt true}.

\vspace{10pt}
{\bf Question 3.} Answer: {\tt false}, {\tt true}, {\tt true}, {\tt false}.

\vspace{10pt}
{\bf Question 4.} TBD 





\end{document}




#include <iostream>
#include <vector>
#include <string>
using namespace std;

struct Folder {
    string name;
    vector<Folder> children;
    Folder(string nn) {name = nn;}
    void makeSub(Folder ff) {
        children.push_back(ff);
    }
};

int main() {    
    Folder f1("f1");
    Folder f2("f2");
    Folder f3("f3"); 

    f2.makeSub(f3);
    f1.makeSub(f2);
    cout << f2.children.size() << endl;
}



============================

/*
    Node n1(1);
    Node n2(2);
    Node n3(3);
    Node n4(4);
    Node n5(5);
    Node n6(6);
    Node n7(7);
    
    nodeMap.insert(pair<int, Node>(1, n1));
    n1.addChild(&n2);
    n1.addChild(&n3);
    n1.addChild(&n4);
    n1.addChild(&n5);
	nodeMap.insert(pair<int, Node>(1, n2));
    n2.addChild(&n6);
    n2.addChild(&n7);
    n1.printPreorder();
	
	pazaudē n2 bērnus
	
*/
