%\documentclass[jou]{apa6}
\documentclass[11pt]{article}
\usepackage{ucs}
\usepackage[utf8x]{inputenc}
\usepackage{changepage}
\usepackage{graphicx}
\usepackage{amsmath}
\usepackage{gensymb}
\usepackage{amssymb}
\usepackage{enumerate}
\usepackage{tabularx}
\usepackage{lipsum}
\usepackage{hyperref}
\usepackage{fancyvrb}

\oddsidemargin 0.0in
\evensidemargin 0.0in
\textwidth 6.27in
\headheight 1.0in
\topmargin -0.1in
\headheight 0.0in
\headsep 0.0in
\textheight 9.0in

\usepackage{xcolor}

\setlength\parindent{0pt}

\newenvironment{myenv}{\begin{adjustwidth}{0.4in}{0.4in}}{\end{adjustwidth}}
\renewcommand{\abstractname}{Anotācija}
\renewcommand\refname{Atsauces}



\newcounter{alphnum}
\newenvironment{alphlist}{\begin{list}{(\Alph{alphnum})}{\usecounter{alphnum}\setlength{\leftmargin}{2.5em}} \rm}{\end{list}}


%16.3-6

\makeatletter
\let\saved@bibitem\@bibitem
\makeatother

\usepackage{bibentry}
%\usepackage{hyperref}


%\title{Homework 1: Grading Criteria}
%\author{Kalvis}
%\affiliation{RBS}



\begin{document}
\thispagestyle{empty}

%\twocolumn


\begin{center}
{\Large Sample Assignment 1, 2020-09-14}
\end{center}


{\bf Question 1 (Bitwise Operations).} Write the output
(and the content of variables {\tt a,b,c} in hexadecimal notation),
after this snipped is executed:
\begin{Verbatim}[frame=single]
unsigned int a = 0xACE02468;
unsigned int b = (a << 12) & (a >> 20);
unsigned int c = (a << 12) | (a >> 20);
std::cout << std::hex << "a = " << a << std::endl;
std::cout << std::hex << "b = " << b << std::endl;
std::cout << std::hex << "c = " << c << std::endl;
\end{Verbatim}


\vspace{10pt}
Hexadecimal memory content of {\tt a}: \rule{4cm}{0.4pt} 

\vspace{10pt}
Hexadecimal memory content of {\tt b}: \rule{4cm}{0.4pt} 

\vspace{10pt}
Hexadecimal memory content of {\tt c}: \rule{4cm}{0.4pt} 

{\em Please note that unsigned ints are $4$ bytes long. If you do left shift (respectively, a right shift) on 
such variables, the bits on the right (respectively, on the left) are filled with zeroes.}









\vspace{10pt}
{\bf Question 2.} 
Draw a flowchart for this {\tt switch-case} statement. Use only $5$ kinds of nodes:\\
{\footnotesize
{\bf (1)} Start node (oval: one outgoing arrow).\\
{\bf (2)} Stop node (oval: one incoming arrow).\\
{\bf (3)} Conditional statement (diamond: one incoming and two outgoing arrows). Mark the ``true'' branch.\\
{\bf (4)} Regular statement (rectangle: one incoming and one outgoing arrow).\\
{\bf (5)} Merging two branches (black dot: two incoming arrows, one outgoing arrow).
}

\vspace{5mm}
\begin{tabular}{@{}ll@{}} 
\begin{minipage}{0.38\columnwidth}
\begin{Verbatim}[frame=single]
int x = 0;
char c; 
cin >> c; 
switch( c ) {
    case 'A':
        x += 1;
    case 'B':
        x += 2;
        break;
    default :
        x += 4;
}
cout << "x= " << x << endl;
\end{Verbatim}



 

\end{minipage} &
\begin{minipage}{0.5\columnwidth}
\mbox{}

\end{minipage}
\end{tabular}


\vspace{20pt}
{\bf Question 3 (Side Effects).}\\
What is the value of {\tt x} output by the code snippet above, 
if {\tt cin} inputs letter {\tt 'A'}?

\vspace{10pt}
{\tt x=}: \rule{4cm}{0.4pt}


\newpage





\end{document}



