\documentclass[a4paper,12pt]{article}
\usepackage[table]{xcolor}
\usepackage{array,amsmath,amssymb,multicol,tikz}
\usepackage{hyperref,colonequals}
\usepackage[margin=2cm]{geometry}
\usepackage{fancyvrb}
\usetikzlibrary{calc,arrows.meta}
\usepackage{xcolor}

\tikzset{
arr/.style={line width=1pt,-{Straight Barb[width=6pt, length=3pt]}, shorten >=2pt}
}

\pagestyle{empty}

\newcommand\N{\mathbf{N}}
\newcommand\Q{\mathbf{Q}}
\newcommand\R{\mathbf{R}}
\newcommand\Z{\mathbf{Z}}

\newcommand\rem{\textup{rem}}

\begin{document}

\begin{center}
\parbox{3cm}{\flushleft\bf Discrete\linebreak Structures}
\hfill
\parbox{7cm}{\centering {\bf\Huge Samples, Part 1}}
\hfill
\parbox{3cm}{\flushright\bf Spring 2021 \linebreak May 21}
\end{center}

\hrule\vspace{2pt}\hrule

\hrule


\vspace{10pt}
{\bf Boolean expressions.}

\begin{enumerate}

% 1.(a)
\item {\small \em
 Express an English sentence as a Boolean expression.}\\
In each of the following sentences underline
up to three atomic propositions and write them as
compound propositions using logical connectives.
\begin{enumerate}
\item For walks in a forrest to be safe, it is necessary, but not
sufficient that there are are no poisonous mushrooms along your path
and no active ticks.
\item There are no active ticks and walks in a forrest are safe, but
there are poisonous mushrooms along your path.
\item Walks in a forrest are not safe iff there are no
active ticks.
\item There are either
active ticks or poisonous mushrooms along your path or both,
whenever the walks in the forrest are not safe.
\end{enumerate}

% 1.(b)
\item {\small \em
Build a truth table for a Boolean Expression.}\\
$((p \oplus q) \wedge (p \oplus \neg q)) \oplus (\neg q \oplus \neg r)$.

% 1.(c)
\item {\small \em
Build a DNF or a CNF for a given truth table.}\\
Fill in the truth table for the logical expression $p \rightarrow (\neg q \wedge r)$.
Create either a DNF or a CNF for this
truth table (which one \textendash{} is up to you).

% 1.(d)
\item {\small \em
Simplify a Boolean Expression using identities; prove, disprove tautologies.}\\
Simplify the following Boolean expressions:
\begin{enumerate}
\item $(p \oplus q) \oplus (p \oplus r)$
\item $(p \wedge q) \wedge (p \rightarrow q \rightarrow r)$
\end{enumerate}


% 1.(e)
\item {\small \em
Shade regions in a Venn diagram corresponding to a Boolean Expression.}\\
Draw a Venn diagram (the ovals correspond to the regions where
the corresponding Boolean variables are true).
Shade the region, where the compound Boolean expression is true.
\begin{enumerate}
\item $(p \rightarrow \neg r) \vee (q \rightarrow \neg r)$
\item $(p \vee q) \oplus r$
\item $(p \wedge q) \vee (\neg q \wedge r)$
\end{enumerate}
\end{enumerate}



\vspace{10pt}
{\bf Quantifiers.}

\begin{enumerate}
%2.(a)
\item {\small \em
Express math and programming concepts with predicates.}\\
Write a logical expression for a predicate $P(x,y)$ which has
value true iff the point $(x,y)$ is inside a unit circle (with radius $1$
and center $(0,0)$).\\
{\em Note 1.} If $(x,y)$ is exactly on the border of the unit circle,
$P(x,y)$ should evaluate to false.\\
{\em Note 2.} Express similar predicates for points belonging to
the 1st (or 2nd, 3rd, 4th) quadrant, or belonging to some rectangle, etc.

%2.(b)
\item {\small \em
Express an English sentence as a predicate logic expression.}\\
Introduce 1 or 2-argument predicates and express the following sentences in the predicate logic
\begin{enumerate}
\item Some student in this class has visited Mexico, but has not visited Argentina.
\item All students in this class have learned at least one programming language.
\item There is a student in this class who has taken every course offered  by
one of the departments of this university.
\item Some student in this class is a citizen of the same country as exactly one other
student in this class.
\end{enumerate}

%2.(c)
\item {\small \em
Restore parentheses, identify free/bound variables in a predicate expression.}\\
For every variable occurrence determine its scope (and what quantifier bounds it, if any
such quantifier exists). Draw an arrow from a variable to its quantifier.
\[ \forall x \in S\  \left( \exists x \in S\ R(x) \vee P(x) \right) \wedge Q(x). \]
Does this expression depend on some parameter (and if so, underline that parameter).

%2.(d)
\item {\small \em
Write the negation for a predicate expression; simplify using De Morgan’s laws.}\\
Let $U$ be a set universe and $P(x,y,z)$ be some predicate with three variables $x,y,z \in U$.
Rewrite the statement so that negations are applied only to the predicates directly
(no negation is outside a quantifier or applied to an expression containing logical connectives).
\[ \neg \forall x \in U\
\left( \exists y \in U\ \forall z \in U\ P(x,y,z) \wedge \exists z \in U\ \forall y \in U\ P(x,y,z)  \right). \]

%2.(e)
\item {\small \em Read and write set-builder notation.}\\
Use the set-builder notation to describe the set $S$ of all those positive integers $k$ that can be expressed
as the product of two positive integers $a,b$ in just one way: Neither $a$, nor $b$ equals $1$ or $k$
(the order of multiplication does not matter; i.e.\ product $a \cdot b$ and $b \cdot a$ counts as the same way).
\end{enumerate}



\vspace{10pt}
{\bf Functions and Relations}

\begin{enumerate}
%3.(a)
\item {\small \em Given a function, prove/disprove that it is injective, surjective or bijective.}\\
A function $f$ switches the places of two digits in a positive integer. For example:\\
$f(123) = 213$ and $f(123456) = 123546$.\\
(In general the 2nd and the 3rd digits from the right switch their places; no other changes are done.
For numbers $n < 100$, $f(n) = n$ as there is no 3rd digit from the right in this case.)\\
Find if $f$ is injective, surjective and/or bijective function.

%3.(b)
\item {\small \em Given function definitions, evaluate their compositions and inverses.}\\
Find the inverse function for $f$ defined in the previous problem.
Find the composition $f \circ f \circ f$ for this function.


%3.(c)
\item {\small \em Given a sequence, identify its properties, is it (eventually) constant/periodic, etc.}\\
Let $a_1 = 7$ and every next member $a_{n+1}$ is the last digit of $3a_n +2$.
Is the sequence $(a_n)$ constant? Eventually constant? Periodic? Eventually periodic?


%3.(d)
\item {\small \em Convert a description of a binary relation into another form.}\\
Write a matrix representation for a binary relation $R \subseteq \{ 0,1,2,3,4,5 \}^2$ such
that $(a,b) \in R$ iff $|a + b|$ or $|a - b|$ is divisible by $3$.

%3.(e)
\item {\small \em Given a binary relation determine if it is a (injective, surjective, bijective) function.}\\
Let $S = \{ 1,2,3,4,5,6 \}$. Let $R \subseteq S \times S$ be a relation such that
$(a,b) \in R$ iff $|a - b| = 3$.
\begin{enumerate}
\item
Prove that the relation $R$ is a function relation.
\item
Is this function injectiive, surjective and/or bijective?
\end{enumerate}

%3.(f)
\item {\small \em Given a binary relation determine if it is reflexive, symmetric, antisymmetric or transitive.}\\
Let $\Z$ be the set of all integers, and $R \subseteq \Z^2$ is the relation
such that $(a,b) \in R$ iff $|a - b|$ is divisible by $7$.
Is the relation $R$ a reflexive, symmetric, antisymmetric, transitive and/or equivalence relation?



%3.(g)
\item {\small \em Compute compositions and powers for relations, find transitive closures.}\\
The set $S$ consists of $8 \times 8$ squares; it is the regular chess board.
Two squares $a,b \in S$ are in relation $B$ (written as $(a,b) \in B$) iff
it is possible to go from $a$ to $b$ in a single step using a bishop.
(Bishops are going diagonally \textendash{} they can cross as many squares as they like as long as they
stay on the same diagonal.) Let $B^t$ be the transitive closure of the relation $B$.
How many squares are in relation $B^t$ with $c \in S$ (where $c$ is the
square located on the bottom-left corner of the chess board).
\end{enumerate}



\vspace{10pt}
{\bf Proofs.}

\begin{enumerate}
%4.(a)
\item {\small \em Prove an implication directly.}\\
Prove that for any integer $x$ that has decimal notation ending with digit $7$,
the cube $x^3$ has the decimal notation that always ends with the same digit
(no matter which $x$ you pick).


%4.(b)
\item {\small \em Prove an implication by contradiction.}\\
Assume that $\alpha + \beta$ is a rational number and $\alpha^2 + \beta^2$ is
an irrational number. Prove that $\alpha \cdot \beta$ is an irrational number.


%4.(c)
\item {\small \em  Prove a logical equivalence (if and only if).}\\
Prove that an integer number $n$ belongs to the interval $[10000,999999]$
iff $\lfloor \sqrt{n} \rfloor$ is an integer with exactly $3$ digits in
its decimal notation.


%4.(d)
\item {\small \em Prove by counterexample.}\\
Prove that there exists a prime number $p$ such that the congruence equation
\[ x^2 \equiv 2 \pmod{p} \]
has an integer solution $x$.


%4.(e)
% 1*1 + 2*2 + 3*6 + ...
\item {\small \em
Prove by mathematical induction.}\\
Prove the following identity for every positive integer $n$:
\[ 1 \cdot 1! + 2 \cdot 2! + 3 \cdot 3! + \ldots + n \cdot n! = (n+1)! - 1. \]




%4.(f)
\item {\small \em Prove or disprove equality of two numbers or sets.}\\
Prove that a connected graph with $n$ vertices is a tree iff it has $n-1$ edges.
\end{enumerate}

\end{document}
