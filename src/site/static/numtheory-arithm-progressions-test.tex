\documentclass[11pt]{article}
\usepackage{ucs}
\usepackage[utf8x]{inputenc}
\usepackage{changepage}
\usepackage{graphicx}
\usepackage{amsmath}
\usepackage{gensymb}
\usepackage{amssymb}
\usepackage{enumerate}
\usepackage{tabularx}
\usepackage{lipsum}

\oddsidemargin 0.0in
\evensidemargin 0.0in
\textwidth 6.27in
\headheight 1.0in
\topmargin -0.1in
\headheight 0.0in
\headsep 0.0in
%\textheight 9.69in
\textheight 9.50in

\setlength\parindent{0pt}

\newenvironment{myenv}{\begin{adjustwidth}{0.4in}{0.4in}}{\end{adjustwidth}}
\renewcommand{\abstractname}{Anotācija}
\renewcommand\refname{Atsauces}

\newenvironment{uzdevums}[1][\unskip]{%
\vspace{3mm}
\noindent
\textbf{#1:}
\noindent}
{}

\newcommand{\subf}[2]{%
  {\small\begin{tabular}[t]{@{}c@{}}
  #1\\#2
  \end{tabular}}%
}



\newcounter{alphnum}
\newenvironment{alphlist}{\begin{list}{(\Alph{alphnum})}{\usecounter{alphnum}\setlength{\leftmargin}{2.5em}} \rm}{\end{list}}


\makeatletter
\let\saved@bibitem\@bibitem
\makeatother

\usepackage{bibentry}
%\usepackage{hyperref}


\begin{document}

\thispagestyle{empty}

{\Large \bf Aritmētiskas progresijas: Rudzātu vidusskola, 2019-07-30}


\begin{uzdevums}[1.jautājums]
Nosaukt piecus mazākos kopīgos dalītājus skaitļiem 
$8$ un $18$. 
\end{uzdevums}

\vspace{2ex}
{\em Ierakstiet 5 dalītājus augošā secībā, atdalot ar komatiem:} \_\_\_\_\_\_\_\_\_\_\_\_\_\_\_

\vspace{6ex}
\begin{uzdevums}[2.jautājums]
Atrast $\mbox{MKD}(6,7,8)$ \textendash{} visu trīs skaitļu mazāko kopīgo dalītāju.
\end{uzdevums}

\vspace{2ex}
{\em Ierakstiet MKD:} \_\_\_\_\_

\vspace{6ex}
\begin{uzdevums}[3.jautājums]
\begin{enumerate}[(a)]
\item Dota aritmētiska progresija $(a_n)$, kam $a_1 = 12$, $d = 29$. 
Atrast, cik daudzi tās locekļi ir trīsciparu skaitļi.  
\item Kādu $a_1$ jāizvēlas, lai progresijā ar $d=29$ būtu iespējami 
daudz trīsciparu skaitļu?
\end{enumerate}
\end{uzdevums}

\vspace{2ex}
{\em (a) Ierakstiet trīsciparu locekļu skaitu:} \_\_\_\_\_\\

\noindent
{\em (b) Ierakstiet $a_1$, lai trīsciparu locekļu būtu visvairāk:} \_\_\_\_\_

\vspace{6ex}
\begin{uzdevums}[4.jautājums]
Kāds ir mazākais naturālais skaitlis, kuru, dalot ar $20$, atlikumā iegūst $13$, 
bet, dalot ar $21$, atlikumā iegūst $3$.
\end{uzdevums}

\vspace{2ex}
{\em Ierakstiet naturālo skaitli:} \_\_\_\_\_

\vspace{6ex}
\begin{uzdevums}[5.jautājums]
Karlsons sev pusdienām nopirka $8$ pīrādziņus un $15$ magoņmaizītes, bet
Brālītis \textendash{} vienu pīrādziņu un vienu magoņmaizīti. Karlsons par savām
pusdienām samaksāja tieši divus eiro (katra maizīte un pīrādziņš maksā veselu
skaitu centu). Cik samaksāja Brālītis?
\end{uzdevums}

\vspace{2ex}
{\em Ierakstiet, cik samaksāja Brālītis eirocentos:} \_\_\_\_\_

\end{document}
