%\documentclass[jou]{apa6}
\documentclass[11pt]{article}
\usepackage{ucs}
\usepackage[utf8x]{inputenc}
\usepackage{changepage}
\usepackage{graphicx}
\usepackage{amsmath}
\usepackage{gensymb}
\usepackage{amssymb}
\usepackage{enumerate}
\usepackage{tabularx}
\usepackage{lipsum}
\usepackage{hyperref}
\usepackage{fancyvrb}
\usepackage{mathtools}

\oddsidemargin 0.0in
\evensidemargin 0.0in
\textwidth 6.27in
\headheight 1.0in
\topmargin -0.1in
\headheight 0.0in
\headsep 0.0in
\textheight 9.0in

\usepackage{xcolor}

\setlength\parindent{0pt}

\newenvironment{myenv}{\begin{adjustwidth}{0.4in}{0.4in}}{\end{adjustwidth}}
\renewcommand{\abstractname}{Anotācija}
\renewcommand\refname{Atsauces}



\newcounter{alphnum}
\newenvironment{alphlist}{\begin{list}{(\Alph{alphnum})}{\usecounter{alphnum}\setlength{\leftmargin}{2.5em}} \rm}{\end{list}}


%16.3-6

\makeatletter
\let\saved@bibitem\@bibitem
\makeatother

\usepackage{bibentry}
%\usepackage{hyperref}


%\title{Homework 1: Grading Criteria}
%\author{Kalvis}
%\affiliation{RBS}



\begin{document}
\thispagestyle{empty}

\twocolumn



\begin{center}
{\Large Sample Assignment 2, 2020-09-17},
{\em (Not graded)}
\end{center}

\vspace{10pt}
{\bf Question 1 (Passing parameters):} 


{\footnotesize
\begin{center}
\begin{minipage}{.85\columnwidth}
\begin{Verbatim}[frame=single,numbers=left]
#include <iostream>

void swap(int* a, int b, int c) {
  int temp = a[0];
  a[0] = a[1];
  a[1] = temp;
  temp = b; 
  b = c;
  c = temp;
}

using namespace std;
int main() {
  int arr[] = {1,2,3,4,5,6,7,8,9,10};
  int b = 4, c = 5, d = 6;
  swap(arr, b, c);
  arr[++d] = d++;
  for (int i = 0; i < 10; i++)
  cout << arr[i] << " ";
  cout << endl;
  cout << b << " " << c << " " << d;
}
\end{Verbatim}
\end{minipage}
\end{center}
}

Please draw the memory content of the array {\tt arr}
and variables {\tt b,c,d} 
after running the code.


\newpage

{\bf \Large Solutions} Answers:\\[10pt]

\begin{tabular}{|c|r|} \hline
{\bf Variable} & {\bf Hex value} \\ \hline
{\tt arr[0..9]} & $2,1,3,4,5,6,7,7,9,10$ \\ \hline
{\tt b} & $4$ \\ \hline
{\tt c} & $5$ \\ \hline
{\tt c} & $8$ \\ \hline
\end{tabular}

{\tt swap()} can exchange two values {\tt arr[0]} and {\tt arr[1]}, because it receives the pointer 
to the whole array. (Therefore, {\tt arr[0]=2}, {\tt arr[1]=1}.)

On the other hand, very similar code in {\tt swap()} fails to swap variables {\tt b} and {\tt c}, 
since they are passed by value.

The line {\tt arr[++d] = d++;} starts evaluating the expression from the right-hand side
({\tt dd++} makes $d=7$). Then it performs the pre-increment on $d$ (it becomes $d=8$) and assigns $arr[7]$ to be $7$.  it just assigns incremented value $d$ (it is $7$)
to the element {\tt arr[6]}, which is already equal to $7$. 








\end{document}


