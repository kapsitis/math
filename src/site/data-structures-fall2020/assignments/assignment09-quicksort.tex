\documentclass[a4paper,12pt]{article}

\usepackage{amsmath,amssymb,amsthm,tikz}
\usetikzlibrary{calc,arrows.meta}
\usepackage[margin=20mm]{geometry}
\usepackage{hyperref}
\usepackage{xcolor}

\setlength{\parindent}{0pt}
\setlength{\columnsep}{1cm}



\begin{document}

%\twocolumn

\thispagestyle{empty}

\begin{center}
{\Large Assignment 9}\\
%{\Large Published on 2020-11-08,}\\
{\em 12 minutes} 
\end{center}

\noindent


\vspace{10pt}
{\bf Quicksort Pseudocode}. 

\[
\begin{array}{rl}
 & \text{\textsc{Quicksort}}(A[\ell\;\ldots\;r]):\\
1 & \text{\textbf{if\ }} l<r:\\
2 & \hspace{.5cm} i = \ell \;\;\;\;\;\;\;\;\; \textcolor{teal}{\text{\em ($i$ increases from the left and searches elements $\geq$ than pivot)}}\\
3 & \hspace{.5cm} j = r+1	\;\; \textcolor{teal}{\text{\em ($j$ decreases from the right and searches elements $\leq$ than pivot.)}}\\	
4 & \hspace{.5cm} v = A[\ell] \;\;\;\; \textcolor{teal}{\text{\em ($v$ is the pivot.)}}\\
5 & \hspace{.5cm} \text{\textbf{while\ }} i<j:\\
6 & \hspace{1.0cm} i = i+1\\
7 & \hspace{1.0cm} \text{\textbf{while\ }} \textcolor{red}{i<r} \text{\textbf{\ and\ }} A[i]<v:\\
8 & \hspace{1.5cm} i = i+1\\
9 & \hspace{1.0cm} j = j-1\\
10 & \hspace{1.0cm} \text{\textbf{while\ }} \textcolor{red}{j>\ell} \text{\textbf{\ and\ }} A[j]>v:\\
11 & \hspace{1.5cm} j = j-1\\
12 & \hspace{1.0cm} A[i] \leftrightarrow A[j] \;\; \textcolor{teal}{\text{\em (Undo the extra swap at the end)}}\\
13 & \hspace{0.5cm} A[i] \leftrightarrow A[j] \;\; \textcolor{teal}{\text{\em (Undo the extra swap at the end)}}\\
14 & \hspace{0.5cm} \textcolor{red}{A[j]} \leftrightarrow A[\ell] \;\; \textcolor{teal}{\text{\em (Move pivot to its proper place)}}\\
15 & \hspace{0.5cm} \text{\textsc{Quicksort}}(A[\ell\;\ldots\;j-1])\\
16 & \hspace{0.5cm} \text{\textsc{Quicksort}}(A[j+1\;\ldots\;r])\\
\end{array}
\]


\vspace{20pt}
{\bf Question 1.} 
You are given an array:

\vspace{5pt}
\begin{tabular}{|c|c|c|c|c|c|c|c|c|c|c|c|} \hline
$a+10$ & $\;\;c\;\;$ & $a+20$ & $\;\;a\;\;$ & $c+5$ & $\;\;b\;\;$ & $b+20$ & $a+15$ & $b+1$ & $b+15$ & $\;2\cdot c\;$ & $b+2$ \\ \hline
\end{tabular}

\vspace{5pt}
Here $a,b,c$ are the last three digits of your Student ID.\\
The pseudocode (same as in the sample) is used to sort it. Pivot is the leftmost element.

\vspace{5pt}
{\bf (A)} Run the initial call of $\text{\textsc{QuickSort}}(A[0..11])$.
Draw the state of the array every time you swap two 
elements. 

\vspace{5pt}
{\bf (B)}
Draw the content of the array immediately {\bf before} the second recursive 
call of $\text{\textsc{QuickSort}}()$.
(The original call $\text{\textsc{QuickSort}}(A[0..11])$ is assumed to be the
$0^{\text{th}}$ call of this function). 







\end{document}



