\documentclass[a4paper,12pt]{article}

\usepackage{amsmath,amssymb,amsthm,multicol,tikz,enumitem}
\usepackage{hyperref}
\usepackage[margin=2cm]{geometry}
\usepackage{fancyvrb}
\usetikzlibrary{calc}

\newcommand\N{\mathbf{N}}
\newcommand\Q{\mathbf{Q}}
\newcommand\R{\mathbf{R}}
\newcommand\Z{\mathbf{Z}}

\newcommand\rem{\textup{rem}}

% Comment out one or the other

\newcommand\answer[1]{}
\newcommand\ans[1]{}
%\newcommand\notanswer[1]{#1}
%\newcommand\answer[1]{\\[5pt]{\color{blue}{#1}}\hfill{\color{blue}$\qed$}\\[-5pt]} 
%\newcommand\ans[1]{{\color{blue}{#1}}}
%\newcommand\notanswer[1]{}

\begin{document}

\begin{center}
{\bf\Huge New Problems} \\[5pt]
Discrete Structures \\
Due Sometime\\[5pt]
\textit{*Submit each question separately in .pdf format (except question 5)*}
\end{center}

\hrule
\vspace{2pt}
\hrule
\vspace{12pt}

\begin{enumerate}
\item Draw an Euler diagram representing the classes of sequences 
represented in the homework 4, problem 4. 
Show examples of sequences that fall in various strange places on this diagram
(sequences that have unusual combinations of properties).
\item Suggest that the participants demonstrate some set identity that
can only be disproved by analyzing elements that are outside all the sets. 
\item Draw some diagrams in the Cartesian product between the sets A,B,C
(on a rectangular grid). Ask the participants to describe - what is inside or outside the shaded regions in the rectangle. 
\item Create functions (surjective? with infinite preimages?) between 
$\Z$ and some Cartesian products on remainder sets.
 
\end{enumerate}

\end{document}