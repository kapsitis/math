\documentclass[a4paper,12pt]{article}

\usepackage{amsmath,amssymb,amsthm,multicol,tikz,enumitem}
\usepackage{hyperref}
\usepackage[margin=2cm]{geometry}
\usepackage{fancyvrb}
\usetikzlibrary{calc}

\newcommand\N{\mathbf{N}}
\newcommand\Q{\mathbf{Q}}
\newcommand\R{\mathbf{R}}
\newcommand\Z{\mathbf{Z}}

\newcommand\rem{\textup{rem}}

% Comment out one or the other

\setlength{\parindent}{0pt}

\newcommand\answer[1]{}
\newcommand\ans[1]{}
%\newcommand\notanswer[1]{#1}
%\newcommand\answer[1]{\\[5pt]{\color{blue}{#1}}\hfill{\color{blue}$\qed$}\\[-5pt]}
%\newcommand\ans[1]{{\color{blue}{#1}}}
%\newcommand\notanswer[1]{}



\begin{document}

\begin{center}
{\bf\Huge Comprehensive Exam Topics} \\[5pt]
Discrete Structures \\
%(4 separate times)\\[5pt]
\textit{*You must justify all your answers to recieve full credit*}
\end{center}

\hrule
\vspace{2pt}
\hrule
\vspace{12pt}

Comprehensive Exam in Discrete Structures revisits the most essential theory material and tests your knowledge on it.
It is offered in four parts \textendash{} every part has mandatory office hours
followed by the exam on some following day.
Each part lasts $60$ minutes, it has $6$ questions (up to $10$ points each).
You will need to pass all four parts;
all four results should be at least $30$ points  \textendash{} half of the maximum.


\vspace{10pt}
{\bf Part 1: Logic and Proofs}


{\small
\begin{enumerate}

\item \textbf{Boolean expressions.} Truth tables, logical equivalences, Venn diagrams.
\begin{enumerate}
\item Express an English sentence as a Boolean expression.
\item Build a truth table for a Boolean Expression.
\item Build a DNF or a CNF for a given truth table.
\item Simplify a Boolean Expression using identities; prove, disprove tautologies.
\item Shade regions in a Venn diagram corresponding to a Boolean Expression.
\end{enumerate}
\item \textbf{Quantifiers.} Predicates, quantifiers.
\begin{enumerate}
\item Express math and programming concepts with predicates.
\item Express an English sentence as a predicate logic expression.
\item Restore parentheses, identify free/bound variables in a predicate expression.
\item Write the negation for a predicate expression; simplify using De Morgan's laws.
\item Read and write set-builder notation.
\end{enumerate}
\item \textbf{Functions and Relations} Injections, surjections, bijections, relations and their properties.
\begin{enumerate}
\item Given a function, prove/disprove that it is injective, surjective or bijective.
\item Given function definitions, evaluate their compositions and inverses.
\item Given a sequence, identify its properties, is it (eventually) constant/periodic, etc.
\item Convert a desription of a binary relation into another form.
\item Given a binary relation determine if it is a (injective, surjective, bijective) function.
\item Given a binary relation determine if it is reflexive, symmetric, antisymmetric or transitive.
\item Compute compositions and powers for relations, find transitive closures.
\end{enumerate}
\item \textbf{Proofs.} Simple statements grouped by the method of proof.
\begin{enumerate}
\item Prove an implication directly.
\item Prove an implication by contradiction.
\item Prove a logical equivalence (if and only if).
\item Prove by counterexample.
\item Prove by mathematical induction.
\item Prove or disprove equality of two numbers or sets.
\end{enumerate}
\end{enumerate}
}


\vspace{10pt}
{\bf Part 2: Structures}


{\small
\begin{enumerate}

\item \textbf{Number Theory.} Congruences. Bezout identity. Inverses. Number representations.
\begin{enumerate}
\item Factorize a number into a product of prime powers.
\item Divide numbers with remainders as in $n = qd + r$, express decimal digits.
\item Find members of arithmetic progressions, also by modulo $m$.
\item Given two integers, find their GCD (also LCM) by Euclidean algorithm.
\item Given integers, solve Bezout identity with Blankenship algorithm.
\item Given $m$ and $x$, compute
multiplicative inverse $\overline{x}$ modulo $m$; solve linear congruences.
\item Convert periodic decimal numbers to rational fractions.
\item Given a decimal integer, convert it to binary, hexadecimal (and vice versa).
\end{enumerate}

\item \textbf{Graphs.} Graph concepts, subgraphs, graph families and isomorphisms.
\begin{enumerate}
\item Count the number of graphs with a given property or parameter.
\item Find if some configuration (with vertex degrees, subgraphs, cycles, paths,
other graph properties) is possible.
\item Convert between graph representations.
Check if a graph is bipartite, complete, or connected.
\item Given a tree, check the condition for a Euler circuit (or path) and find it.
\item Given two graphs prove or disprove they are isomorphic.
\end{enumerate}

\item \textbf{Trees.} Tree concepts, traversing trees with BFS, DFS.
\begin{enumerate}
\item Given the count of vertices, edges, height or other parameter, estimate other parameters.
\item Given an $n$-ary tree and some parameters, estimate other parameters.
\item Convert between representations: tree diagrams, lists of edges, or traversals.
\item Given a prefix, infix or postfix notation, convert it into the syntax tree or other notation(s).
\item Given an undirected graph, do a DFS and BFS traversal, indicating all steps.
\end{enumerate}
\end{enumerate}
}





\vspace{10pt}
{\bf Part 3: Counting and Estimation}


{\small
\begin{enumerate}
\item \textbf{Combinatorics.} Permutations, combinations, binomial coefficients, pigeonhole principle.
\begin{enumerate}
\item Given a word problem, count variants using the product, sum, difference rules.
\item Given a set of restrictions and symmetries, count variants using the division rule.
\item Count variants using
combinations and permutation formulas with or without repetition.
\item Given a polynomial, find coefficients using binomial and multinomial rules.
\end{enumerate}

\item \textbf{Recurrent Sequences.} Periodicity, 1st and 2nd order recurrences, Master theorem.
\begin{enumerate}
\item Evaluate $\sum\limits_{i=0}^n \ldots$ and similar constructs.
\item Prove a property of a recurrent sequence by induction or using invariants.
\item Prove that a recurrent sequence has a closed formula using induction.
\item Given a 1st order non-homogeneous recurrence, solve it.
\item Given a 2nd order homogeneous recurrence, solve it.
\item Given a word problem (sets of strings, Tower of Hanoi, tilings, etc.) build recurrences.
\end{enumerate}


\item \textbf{Big-O notation.}
\begin{enumerate}
\item Given functions $f,g$, check by definition that $f(n)$ is in $O(g(n))$, $\Omega(g(n))$, $\Theta(g(n))$.
\item Given a function $f(x)$, simplify it to get its ``optimal'' $O(g(x))$ or $\Theta(g(x))$ class.
\item Given a collection of functions, arrange them by growth.
\item Given a pseudocode, basic operations and input length, estimate its time as $O(g(n))$.
\end{enumerate}
\end{enumerate}
}






\vspace{10pt}
{\bf Part 4: Probabilities}

{\small
\begin{enumerate}
\item \textbf{Events.} Events, complements, independence, conditional probability, Bernoulli trials.
\begin{enumerate}
\item Describe events in a sample space and compute probabilities using Laplace's definition.
\item Compute probabilities of derived events (complementary, intersection, union, etc.).
\item Find conditional probabilities for events and their combinations.
\item Prove or disprove pairwise and mutual independence of events.
\item Analyze the probabilities of the outcomes of a probabilistic 2-player game.
\item Express conditional probabilities using Bayes' theorem.
\end{enumerate}


\item \textbf{Random Variables.} Expected value, variance, distributions, Chebyshev's inequality.
\begin{enumerate}
\item Identify the geometric, binomial, and Bernoulli distributions.
\item Given a definition of a random variable, find its probability mass function.
\item Given a distribution for a discrete random variable $X$, compute $E(X)$ and $V(X)$.
\item Estimate the probability of $X$ being in an interval by Chebyshev's inequality.
\end{enumerate}
\end{enumerate}
}



\end{document}
